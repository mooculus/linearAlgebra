\documentclass{ximera}
\input{../preamble.tex}
\title{Properties of the determinant}
\author{Crichton Ogle}

\begin{document}
\begin{abstract}
  The determinant is connected to many of the key ideas in linear algebra.
\end{abstract}
\maketitle

The determinant satisfies a number of useful properties, among them
\begin{enumerate}
\item (Determinants commute with products) If $A,B$ are two square matrices of the same dimensions, then $Det(A*B) = Det(A)Det(B)$.
\item (Test for singularity) If $A$ is a numerical square matrix, then $A$ is singular iff $Det(A) = 0$. Alternatively, $A$ is invertible as a matrix iff $Det(A)$ is invertible as a number.
\item (Determinant of triangular matrices) If $A$ is triangular (either upper or lower), then $Det(A) = A(1,1)A(2,2)\dots A(n,n) =$ the product of the diagonal entries.
\item (Invariance under transposition) $Det(A) = Det(A^T)$.
\item (Determinants of elementary matrices)
\begin{enumerate}
\item For any type I elementary matrix $P_{ij}$, $Det(P_{ij}) = -1$.
\item For any type II elementary matrix $D_i(r)$, $Det(D_i(r)) = r$.
\item For any type III elementary matrix $E_{ij}(a)$, $Det(E_{ij}(a)) = 1$.
\end{enumerate}
\item (Cofactor expansions along rows or columns) If $A$ is an $n\times n$ matrix, then
\begin{enumerate}
\item (Rows) For any $1\le i\le n$, $Det(A) = \sum_{j=1}^n A(i,j)A_{ij}$. Moreover, if $1\le i\ne k\le n$, then $\sum_{j=1}^n A(i,j)A_{kj} = 0$.
\item (Columns) For any $1\le j\le n$, $Det(A) = \sum_{i=1}^n A(i,j)A_{ij}$. Moreover, if $1\le j\ne k\le n$, then $\sum_{i=1}^n A(i,j)A_{ik} = 0$.
\end{enumerate}
\end{enumerate}
\vskip.3in

These properties can be used to produce algorithms for computing the determinant of a numerical matrix that are considerably more efficient than direct application of the definition (either using cofactor expansion along the first row, or the combinatorial sum over all permutations).
\vskip.2in

We will say that two matrices of the same dimension are {\it type III row equivalent} if one can be derived from the other by using only type III row operations, and are {\it type I-III row equivalent} if one can be derived from the other by using only type I and type III row operations.
\vskip.2in

The following is not difficult to prove.

\begin{theorem} If $A$ is an $n\times n$ numerical matrix, then $A$ is I-III row equivalent to an upper triangular matrix.
\end{theorem}

\begin{exercise} Prove this result.
\end{exercise}
\vskip.2in

A useful consequence of this theorem is the following algorithm for computing the determinant of a purely numerical matrix $A$:

\begin{itemize}
\item Using type I and III row operations, convert $A$ into an upper triangular matrix $B$.
\item Compute $Det(B)$ as the product of its diagonal entries.
\item Count the number of type I operations you used in going from $A$ to $B$. If it was an even number then $Det(A) = Det(B)$. If it was an odd number then $Det(A) = -Det(B)$.
\end{itemize}
\vskip.2in

In practice the above procedure computes $Det(A)$ much more quickly than brute force application of either definition, and is used in many applications. However, it is only useful in the case $A$ is purely numerical (all of its entries are either real or complex numbers). For more general matrices that have non-numerical entries (for example, real-valued functions), Theorem 1 above no longer holds, and more sophisticated methods are needed for effectively computing the determinant.
\end{document}
