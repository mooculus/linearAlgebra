\documentclass{ximera}
\input{../preamble.tex}
\title{Cofactor expansion}
\author{Crichton Ogle}

\begin{document}
\begin{abstract}
  One method for computing the determinant is called cofactor expansion.
\end{abstract}
\maketitle

If $A$ is an $n\times n$ matrix, with $n > 1$, we define the $(i,j)^{th}$ minor of $A$ - denoted $M_{ij}(A)$ - to be the $(n-1)\times(n-1)$ matrix derived from $A$ by deleting the $i^{th}$ row and $j^{th}$ column.
\vskip.2in

\begin{example} Suppose A = $\begin{bmatrix} 2 & 3 & -1\\-2 & -4 & 5\\0 & -3 & 7\end{bmatrix}$. Then $M_{23}(A)$ is the $2\times 2$ matrix gotten by deleting the second row and third column: $M_{23}(A) = \begin{bmatrix} 2 & 3\\0 & -3\end{bmatrix}$.
\end{example}
\vskip.2in

\begin{exercise}
For the above matrix $A$, compute $M_{32}(A)$ and $M_{33}(A)$.
\end{exercise}
\vskip.2in


 To define the determinant in the framework of cofactors, one proceeds with an {\it inductive} or {\it recursive} definition. In such a definition, we give an explicit formula in the case $n=1$; then prior to defining the determinant for $n\times n$ matirices, we assume that the determinant {\it has already been given for $(n-1)\times(n-1)$ matrices}. Note that if we are given an $n\times n$ matrix $A$, all of its minors will be of dimensions $(n-1)\times (n-1)$, for which we can assume the determinant has already been defined. For indices $1\le i,j\le n$, define the $(i,j)^{th}$ cofactor of $A$ to be
\[
A_{ij} = (-1)^{i+j}Det(M_{ij}(A))
\]
Then

\begin{definition} If $A = [a]$ is a $1\times 1$ matrix, then $Det(A) = a$. For $n > 1$, $Det(A) = \sum_{j=1}^n A(1,j)A_{1j}$
\end{definition}

This is sometimes also referred to as {\it cofactor expansion along the first row}.
\vskip.2in
\begin{example} Suppose A = $\begin{bmatrix} 2 & 3\\1 & -4\end{bmatrix}$. Then $M_{11}(A) = [-4], M_{12}(A) = [1]$, so $A_{11} = (-1)^{1+1}(-4) = -4$ and $A_{12} = (-1)^{1+2}(1) = -1$. Then $Det(A) = A(1,1)A_{11} + A(1,2)A_{12} = (2)(-4) + (3)(-1) = -11$.
\end{example}
\vskip.2in

More generally, cofactor expansion can be easily applied to an arbitrary $2\times 2$ matrix to recover the usual expression for the determinant in that case. 

\begin{example} Suppose $A = \begin{bmatrix} a_{11} & a_{12}\\a_{21} & a_{22}\end{bmatrix}$. Then $M_{11}(A) = [a_{22}], M_{12}(A) = [a_{21}]$, so $A_{11} = (-1)^{1+1}(a_{22}) = a_{22}$ and $A_{12} = (-1)^{1+2}(a_{21}) = -a_{21}$. Then 
\[
Det(A) = A(1,1)A_{11} + A(1,2)A_{12} = a_{11}a_{22} - a_{12}a_{21}
\]
\end{example}
\vskip.2in

The following gives an example of how one would use the definition above to compute the determinant of a $3\times 3$ matrix.

\begin{example} Suppose A = $\begin{bmatrix} 2 & 3 & -1\\-2 & -4 & 5\\0 & -3 & 7\end{bmatrix}$. Then\newline
\[
M_{11}(A) = \begin{bmatrix} -4 & 5\\-3 & 7\end{bmatrix},\quad M_{12}(A) = \begin{bmatrix} -2 & 5\\0 & 7\end{bmatrix},\quad M_{13}(A) = \begin{bmatrix} -2 & -4\\0 & -3\end{bmatrix}.
\]
The corresponding cofactors are $A_{11} = (-1)^{1+1}((-4)*7 - 5*(-3)) = -13,\,\, A_{12} = (-1)^{1+2}((-2)*7 - 5*0) = 14,\,\, A_{13} = (-1)^{1+3}((-2)*(-3) - (-4)*0) = 6$.
\vskip.1in

So for this matrix $Det(A) = A(1,1)A_{11} + A(1,2)A_{12} + A(1,3)A_{13} = 2*(-13) + 3*14 + (-1)*6 = -26 + 42 - 6 = 10$.
\end{example}
\vskip.2in
 
 \begin{exercise} Suppose $A = \begin{bmatrix} a_{11} & a_{12} & a_{13}\\a_{21} & a_{22} & a_{23}\\ a_{31} & a_{32} & a_{33}\end{bmatrix}$. Use the above definition and the result of Example 3 above to express $Det(A)$ in the same manner as done in Example 3.
 \end{exercise}
\end{document}
