\documentclass{ximera}
\input{../preamble.tex}
\title{Cofactor expansion}
\author{Crichton Ogle}

\begin{document}
\begin{abstract}
  One method for computing the determinant is called cofactor expansion.
\end{abstract}
\maketitle

If $A$ is an $n\times n$ matrix, with $n > 1$, we define the $(i,j)^{th}$ minor of $A$ to be $M_{ij}(A)$ to be the $(n-1)\times(n-1)$ matrix derived from $A$ by deleting the $i^{th}$ row and $j^{th}$ column. For example,
\vskip.2in

% BADBAD: [include example here]

 In this framework one proceeds with an {\it inductive} or {\it recursive} definition. In such a definition, we give an explicit formula in the case $n=1$, and that prior to defining the determinant for $n\times n$ matirices, that the determinant {\it has already been given for $(n-1)\times(n-1)$ matrices}. For indices $1\le i,j\le n$, define the $(i,j)^{th}$ cofactor of $A$ to be
\[
A_{ij} = (-1)^{i+j}Det(M_{ij}(A))
\]
Then

\begin{definition} Id $A = [a]$ is a $1\times 1$ matrix, then $Det(A) = a$. For $n > 1$, $Det(A) = \sum_{j=1}^n A(1,j)A_{1j}$
\end{definition}

This is sometimes also referred to as {\it cofactor expansion along the first row}.
\vskip.2in
\begin{example} Suppose A = $\begin{bmatrix} 2 & 3\\1 & -4\end{bmatrix}$. Then $M_{11}(A) = [-4], M_{12}(A) = [1]$, so $A_{11} = (-1)^{1+1}(-4) = -4$ and $A_{12} = (-1)^{1+2}(1) = -1$. Then $Det(A) = A(1,1)A_{11} + A(1,2)A_{12} = (2)(-4) + (3)(-1) = -11$.
\end{example}
\vskip.2in

More generally, 

\begin{example} Suppose $A = \begin{bmatrix} a_{11} & a_{12}\\a_{21} & a_{22}\end{bmatrix}$. Then $M_{11}(A) = [a_{22}], M_{12}(A) = [a_{21}]$, so $A_{11} = (-1)^{1+1}(a_{22}) = a_{22}$ and $A_{12} = (-1)^{1+2}(a_{21}) = a_{21}$. Then 
\[
Det(A) = A(1,1)A_{11} + A(1,2)A_{12} = a_{11}a_{22} - a_{12}a_{21}
\]
\end{example}
\vskip.2in

\end{document}
