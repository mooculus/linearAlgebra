\documentclass{ximera}
\author{Rob Beezer}
\input{../../preamble.tex}

\title{Possibilities for solution sets}

\begin{document}
\begin{abstract}
  A system of linear equations may have zero, one, or many solutions.
\end{abstract}
\maketitle

\begin{definition}
A solution of a system of linear equations in $n$ variables, $\scalarlist{x}{n}$, is an ordered list of $n$ complex numbers, $\scalarlist{s}{n}$ such that if we substitute $s_1$ for $x_1$, $s_2$ for $x_2$, $s_3$ for $x_3$, \ldots, $s_n$ for $x_n$,  then for every equation of the system the left side will equal the right side, i.e., each equation is true simultaneously.
\end{definition}

More typically, we will write a solution in a form like $x_1=12$, $x_2=-7$, $x_3=2$ to mean that $s_1=12$, $s_2=-7$, $s_3=2$ in the above notation.

\begin{exercise}
  Consider the system of equations
  \begin{align*}
    x_1+2x_2 + x_4&= 7,\\
    x_1+x_2+x_3-x_4&=3,\\
    3x_1+x_2+5x_3-7x_4&=1.
  \end{align*}
  
  \begin{question}
    Which of the following are solutions to the system?
    \begin{multipleChoice}
      \choice[correct]{$x_{1}=-2$, $x_{2}=4$, $x_{3}=2$, $x_{4}=1$}
      \choice{$x_{1}=-2$, $x_{2}=5$, $x_{3}=3$, $x_{4}=4$}
    \end{multipleChoice}
  
    \begin{question}
      Is $x_{1}=-12$, $x_{2}=11$, $x_{3}=1$, $x_{4}=-3$ a solution to the system?
      \begin{multipleChoice}
        \choice[correct]{Yes}
        \choice{No}
      \end{multipleChoice}
      
      \begin{feedback}
        We have already seen that $x_{1}=-2$, $x_{2}=4$, $x_{3}=2$,
        $x_{4}=1$ is a solution, but no we also see that $x_{1}=-12$,
        $x_{2}=11$, $x_{3}=1$, $x_{4}=-3$ is a solution, too.
      \end{feedback}
    \end{question}

  \end{question}
  
\end{exercise}

To discuss \textit{all} of the possible solutions to a system of linear equations, we now define the set of all solutions.

\begin{definition}
  The \dfn{solution set} of a system of equations is the collection
  (or set) of all solutions.
\end{definition}

\begin{problem}
  Be aware that a solution set might consist of a single solution, of infinitely many solutions---or there may be no solutions, in which case we write the solution set as the empty set, $\varnothing$.  Let's explore some examples of these possibilities.
  
  \begin{example}
    Consider the system of two equations with two variables,
    \begin{align*}
      2x_1+3x_2&=3\\
      x_1-x_2&=4
    \end{align*}
    
    \begin{question}
      If we plot the solutions to each of these equations separately on the
      $x_{1}x_{2}$-plane, we get \wordChoice{\choice{one line}\choice[correct]{two lines}}.
      
      \begin{question}
        In which point do these two lines intersect?
        \begin{multipleChoice}
          \choice[correct]{$(x_1,\,x_2)=(3,\,-1)$}
          \choice{$(x_1,\,x_2)=(-3,\,1)$}
        \end{multipleChoice}
        
        \begin{feedback}
          These lines have exactly one point in common.  From the geometry, we believe that this is the only solution to the system of equations, and so we say the solution is unique.
        \end{feedback}
      \end{question}
    \end{question}

  \end{example}

  \begin{example}
    Now adjust the system with a different second equation,
    \begin{align*}
      2x_1+3x_2&=3,\\
      4x_1+6x_2&=6.
    \end{align*}

    \begin{question}
      A plot of the solutions to these equations individually results in two lines, one on top of the other!  How many pairs of points satisfy both these equations?
      \begin{multipleChoice}
        \choice{No points satisfy both equations}
        \choice{Exactly one point satisfies both equations}
        \choice[correct]{Infinitely many points satisfy both equations}
      \end{multipleChoice}    
    
      \begin{feedback}
        There are infinitely many pairs of points that make both equations true.  You will learn soon how to describe this infinite solution set precisely.
      \end{feedback}

      \begin{question}
        How does the second equation relate to the first equation?
        \begin{multipleChoice}
          \choice[correct]{The second equation is a twice the first equation.}
          \choice{The second equation is a three times the first equation.}
        \end{multipleChoice}
      \end{question}
    \end{question}
  \end{example}

  \begin{example}
    One more minor adjustment provides a third system of linear equations,
    \begin{align*}
      2x_1+3x_2&=3\\
      4x_1+6x_2&=10
    \end{align*}

    \begin{question}
      A plot now reveals \wordChoice{\choice{one line}\choice[correct]{two lines}} with \wordChoice{\choice{positive slope}\choice[correct]{identical slopes}}.

      \begin{question}
        Consequently, how many pairs of points satisfy both these equations?
        \begin{multipleChoice}
          \choice[correct]{No points satisfy both equations}
          \choice{Exactly one point satisfies both equations}
          \choice{Infinitely many points satisfy both equations}
        \end{multipleChoice}    
        
        \begin{feedback}
          They have no points in common, and so the system has a solution set that is empty, $S=\emptyset$.
        \end{feedback}
      \end{question}
    \end{question}

    This example exhibits all of the typical behaviors of a system of
    equations. A subsequent theorem will tell us that every system of
    linear equations has a solution set that is empty, contains a
    single solution or contains infinitely many solutions.
  \end{example}

\end{problem}

\end{document}
