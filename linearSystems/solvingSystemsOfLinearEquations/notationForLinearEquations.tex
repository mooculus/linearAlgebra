\documentclass{ximera}
\author{Rob Beezer}

\input{../../preamble.tex}

\title{Notation for systems of linear equations}

\begin{document}
\begin{abstract}
  The notation for systems of linear equations involving many subscripts.
\end{abstract}
\maketitle

To begin with, some terminology. 

\begin{definition}
  A \dfn{system of linear equations} is a collection of $m$ equations
  in the variable quantities $x_1,\,x_2,\,x_3,\ldots,x_n$ of the form,
  \begin{align*}
    a_{11}x_1+a_{12}x_2+a_{13}x_3+\dots+a_{1n}x_n&=b_1\\
    a_{21}x_1+a_{22}x_2+a_{23}x_3+\dots+a_{2n}x_n&=b_2\\
    a_{31}x_1+a_{32}x_2+a_{33}x_3+\dots+a_{3n}x_n&=b_3\\
                                                 &\vdots\\
    a_{m1}x_1+a_{m2}x_2+a_{m3}x_3+\dots+a_{mn}x_n&=b_m
  \end{align*}
  where the values of $a_{ij}$, $b_i$ and $x_j$, $1\leq i\leq m$,
  $1\leq j\leq n$, are from the set of complex numbers.

  A system consisting of $m$ equations in the same collection of $n$
  unknowns is referred to as an $m\times n$ system, which reads as:
  ``m-by-n system.''
\end{definition}

The number of rows $m$ and the number of columns $n$ are called the {\it dimensions} of the system. The system is said to be
\begin{itemize}
\item \dfn{underdetermined} if $m < n$ (less equations than unknowns);
\item \dfn{overdetermined} if $m > n$ (more equations than unknowns); and
\item \dfn{balanced} if $m = n$ (same number of equations as unknowns).
\end{itemize}

\begin{example}
  Given the system of linear equations,
  \begin{align*}
    x_1+2x_2 + x_4&= 7\\
    x_1+x_2+x_3-x_4&=3\\
    3x_1+x_2+5x_3-7x_4&=1
  \end{align*}
  we have $n=\answer{4}$ variables and $m=\answer{3}$ equations.  Also,
  \begin{align*}
    a_{11}&=1 & a_{12}&=2 & a_{13}&=\answer{0} & a_{14}&=1 & b_{1}&=7\\
    a_{21}&=1 & a_{22}&=1 & a_{23}&=\answer{1} & a_{24}&=-1 & b_{2}&=3\\
    a_{31}&=3 & a_{32}&=1 & a_{33}&=\answer{5} & a_{34}&=-7 & b_{3}&=1
  \end{align*}
\end{example}

\end{document}

%%% Local Variables:
%%% mode: latex
%%% TeX-master: t
%%% End:
