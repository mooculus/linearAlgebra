\documentclass{ximera}
\author{Rob Beezer}
\input{../../preamble.tex}

\title{Nonlinear Examples}

\begin{document}
\begin{abstract}
  Not every system of equations is a linear system.
\end{abstract}
\maketitle

\begin{example}
  Suppose we desire the simultaneous solutions of the two equations,
  \begin{align*}
    x^2+y^2&=1,\\
    -x+\sqrt{3}y&=0.
  \end{align*}
  
  You can check by substitution that $x=\tfrac{\sqrt{3}}{2},\;y=\tfrac{1}{2}$ and $x=-\tfrac{\sqrt{3}}{2},\;y=-\tfrac{1}{2}$ are both solutions.

  \begin{question}
    Are there any other solutions?
    \begin{multipleChoice}
      \choice[correct]{No}
      \choice{Yes}
    \end{multipleChoice}
    
    \begin{question}
      We need to also convince ourselves that these are the
      \textit{only} solutions.  To see this, plot each equation on the
      $xy$-plane, which means to plot $(x,\,y)$ pairs that make an
      individual equation true.  In this case we get a circle centered
      at the origin with radius 1 and a straight line through the origin
      with slope $\tfrac{1}{\sqrt{3}}$.  The intersections of these two
      curves are our desired simultaneous solutions, and so we believe
      from our plot that the two solutions we know already are indeed
      the only ones.  We like to write solutions as sets, so in this
      case we write the set of solutions as
      \[
        S=\set{\left(\tfrac{\sqrt{3}}{2},\,\answer{\frac{1}{2}}\right),\,\left(-\tfrac{\sqrt{3}}{2},\,-\tfrac{1}{2}\right)}
      \]
      
      \begin{question}
        This example yielded exactly two solutions, but earlier we asserted that a system of linear equations has either zero, one, or infinitely many solutions.
        
        The second equation is $-x+\sqrt{3}y=0$ so it is already in our desired form.  Is this system of equations a system of \textit{linear} equations?
        \begin{multipleChoice}
          \choice{Yes: we can write the first equation as $x \cdot x + y \cdot y = 1$.}
          \choice[correct]{No.}
        \end{multipleChoice}
        
        \begin{feedback}
          The above equations are not a linear system of equations because the coefficients are not fixed complex numbers.
        \end{feedback}
      \end{question}
      
    \end{question}
    
  \end{question}
  
\end{example}


\end{document}
