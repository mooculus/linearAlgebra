\documentclass{ximera}
\author{Rob Beezer}

\input{../../preamble.tex}

\title{Matrices and vectors}

\begin{document}
\begin{abstract}
  We will isolate the key bits of information about a system of
  equations into something called a matrix, and then use this matrix
  to systematically solve the equations.
\end{abstract}
\maketitle

After solving a few systems of equations, you will recognize that it does not matter so much what we call our variables, as opposed to what numbers act as their coefficients.  A system in the variables $x_1,\,x_2,\,x_3$  would behave the same if we changed the names of the variables to $a,\,b,\,c$ and kept all the constants the same and in the same places.  In this section, we will isolate the key bits of information about a system of equations into something called a matrix, and then use this matrix to systematically solve the equations.  Along the way we will obtain one of our most important and useful computational tools.


\begin{definition}
  An $m\times n$ \dfn{matrix} is a rectangular layout of numbers having $m$ rows and $n$ columns.
\end{definition}

We will use upper-case Latin letters ($A,\,B,\,C,\dotsc$) to denote
matrices and squared-off brackets to delimit the layout.  Many use
large parentheses instead of brackets---the distinction is not
important.  Rows of a matrix will be referenced starting at the top
and working down (i.e., row 1 is at the top) and columns will be
referenced starting from the left (i.e., column 1 is at the left).
For a matrix $A$, the notation $\matrixentry{A}{ij}$ will refer to the
complex number in row $i$ and column $j$ of $A$.

\begin{exercise}
  Consider the matrix
  \[
    A = \begin{bmatrix}
      5 & 3 & 5 \\
      2 & 6 & 9 \\
      1 & 8 & 0 \\
    \end{bmatrix}.
  \]
  Then $\matrixentry{A}{2,1} = \answer{2}$ and $\matrixentry{A}{1,3} = \answer{5}$.
\end{exercise}

\begin{question}
  Is $A_{2,1}$ a\ldots
  \begin{multipleChoice}
    \choice[correct]{number?}
    \choice{matrix?}
  \end{multipleChoice}
  
  \begin{feedback}
    Be careful with this notation for individual entries, since
    it is easy to think that $\matrixentry{A}{ij}$ refers to the
    \emph{whole} matrix.  It does not.  It is just a
    \emph{number,} but is a convenient way to talk about the
    individual entries simultaneously.
  \end{feedback}
\end{question}

\begin{definition}
A \dfn{column vector} of size $m$ is an ordered list of $m$ numbers, which is written in order vertically, starting at the top and proceeding to the bottom.  At times, we will refer to a column vector as simply a \dfn{vector}.  Column vectors will be written in bold, usually with lower case Latin letters from the end of the alphabet such as $\vect{u}$, $\vect{v}$, $\vect{w}$, $\vect{x}$, $\vect{y}$, $\vect{z}$.  Some books like to write vectors with arrows, such as $\vec{u}$.  Writing by hand, some like to put arrows on top of the symbol, or a tilde underneath the symbol, as in $\underset{\sim}{\textstyle u}$.  To refer to the \dfn{entry} or \dfn{component} of vector $\vect{v}$ in location $i$ of the list, we write $\vectorentry{\vect{v}}{i}$.
\end{definition}


\begin{question}
  Is $\vectorentry{\vect{v}}{i}$ a\ldots
  \begin{multipleChoice}
    \choice[correct]{number?}
    \choice{vector?}
  \end{multipleChoice}
  
  \begin{feedback}
    Be careful with this notation.  While the symbols
    $\vectorentry{\vect{v}}{i}$ might look somewhat substantial, as an
    object this represents just one entry of a vector, which is just a
    single number.
  \end{feedback}

\end{question}

\begin{definition}
The \dfn{zero vector} of size $m$ is the column vector of size $m$ where each entry is the number zero,
\[
\zerovector=
\colvector{0\\0\\0\\\vdots\\0}
\]
or defined much more compactly, $\vectorentry{\zerovector}{i}=0$ for $1\leq i\leq m$.
\end{definition}

\begin{definition}
For a system of linear equations,
\begin{align*}
a_{11}x_1+a_{12}x_2+a_{13}x_3+\dots+a_{1n}x_n&=b_1\\
a_{21}x_1+a_{22}x_2+a_{23}x_3+\dots+a_{2n}x_n&=b_2\\
a_{31}x_1+a_{32}x_2+a_{33}x_3+\dots+a_{3n}x_n&=b_3\\
\vdots&\\
a_{m1}x_1+a_{m2}x_2+a_{m3}x_3+\dots+a_{mn}x_n&=b_m
\end{align*}
the \dfn{coefficient matrix} is the $m\times n$ matrix
\begin{align*}
A=
\begin{bmatrix}
a_{11}&a_{12}&a_{13}&\dots&a_{1n}\\
a_{21}&a_{22}&a_{23}&\dots&a_{2n}\\
a_{31}&a_{32}&a_{33}&\dots&a_{3n}\\
\vdots&\\
a_{m1}&a_{m2}&a_{m3}&\dots&a_{mn}\\
\end{bmatrix}.
\end{align*}
\end{definition}

\begin{definition}
For a system of linear equations,
\begin{align*}
a_{11}x_1+a_{12}x_2+a_{13}x_3+\dots+a_{1n}x_n&=b_1\\
a_{21}x_1+a_{22}x_2+a_{23}x_3+\dots+a_{2n}x_n&=b_2\\
a_{31}x_1+a_{32}x_2+a_{33}x_3+\dots+a_{3n}x_n&=b_3\\
\vdots&\\
a_{m1}x_1+a_{m2}x_2+a_{m3}x_3+\dots+a_{mn}x_n&=b_m,
\end{align*}
the \dfn{vector of constants} is the column vector of size $m$
\[
\vect{b}=
\begin{bmatrix}
b_1\\
b_2\\
b_3\\
\vdots\\
b_m\\
\end{bmatrix}.
\]
\end{definition}

\begin{definition}
For a system of linear equations,
\begin{align*}
a_{11}x_1+a_{12}x_2+a_{13}x_3+\dots+a_{1n}x_n&=b_1\\
a_{21}x_1+a_{22}x_2+a_{23}x_3+\dots+a_{2n}x_n&=b_2\\
a_{31}x_1+a_{32}x_2+a_{33}x_3+\dots+a_{3n}x_n&=b_3\\
\vdots&\\
a_{m1}x_1+a_{m2}x_2+a_{m3}x_3+\dots+a_{mn}x_n&=b_m
\end{align*}
the \dfn{solution vector} is the column vector of size $n$
\[
\vect{x}=
\begin{bmatrix}
x_1\\
x_2\\
x_3\\
\vdots\\
x_n\\
\end{bmatrix}.\]
\end{definition}

The solution vector may do double-duty on occasion.  It might refer to
a list of variable quantities at one point, and subsequently refer to
values of those variables that actually form a particular solution to
that system.

\begin{definition}
If $A$ is the coefficient matrix of a system of linear equations and $\vect{b}$ is the vector of constants, then we will write $\linearsystem{A}{\vect{b}}$ as a shorthand expression for the  system of linear equations, which we will refer to as the \dfn{matrix representation} of the linear system.
\end{definition}

\begin{example}[Notation for systems of linear equations]
The system of linear equations
\begin{align*}
2x_1+\answer{4}x_2-3x_3+5x_4+x_5&=9\\
3x_1+x_2+\quad\quad x_4-3x_5&=0\\
-2x_1+\answer{7}x_2-5x_3+2x_4+2x_5&=-3
\end{align*}
has coefficient matrix
\[
A=
\begin{bmatrix}
2 & 4 & \answer{-3} & 5 & 1\\
3 & 1 & \answer{0} & 1 & -3\\
-2 & 7 & \answer{-5} & 2 & 2
\end{bmatrix}
\]
and vector of constants
\[
\vect{b}=\colvector{9\\\answer{0}\\\answer{-3}}
\]
and so will be referenced as $\linearsystem{A}{\vect{b}}$.
\end{example}

\begin{definition}
Suppose we have a system of $m$ equations in $n$ variables, with coefficient matrix $A$ and vector of constants $\vect{b}$.  Then the \dfn{augmented matrix} of the system of equations is the $m\times(n+1)$ matrix whose first $n$ columns are the columns of $A$ and whose last column ($n+1$) is the column vector $\vect{b}$.  This matrix will be written as $\augmented{A}{\vect{b}}$.
\end{definition}

The augmented matrix \textit{represents} all the important information in the system of equations, since the names of the variables have been ignored, and the only connection with the variables is the location of their coefficients in the matrix.  It is important to realize that the augmented matrix is just that, a matrix, and \emph{not} a system of equations.  In particular, the augmented matrix does not have any ``solutions,'' though it will be useful for finding solutions to the system of equations that it is associated with.  However, notice that an augmented matrix always belongs to some system of equations, and vice versa, so it is tempting to try and blur the distinction between the two.


\end{document}
