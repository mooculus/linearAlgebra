\documentclass{ximera}
\author{Rob Beezer}

\input{../../preamble.tex}

\title{Examples of Row-Echelon Form}

\begin{document}
\begin{abstract}
  We run through some examples of using these definitions and theorems
  to solve some systems of equations.
\end{abstract}
\maketitle

From now on, when we have a matrix in reduced row-echelon form, we
will mark the leading 1's with a small box.  This will help you count,
and identify, the pivot columns.  In your work, you can box 'em,
circle 'em or write 'em in a different color---just identify 'em
somehow.  This device will prove very useful later and is a
\textit{very good habit} to start developing \textit{right now}.

\begin{example}
  Let us find the solutions to the following system of equations,
  \begin{align*}
    -7x_1 -6 x_2 - 12x_3 &=-33\\
    5x_1  + 5x_2 + 7x_3 &=24\\
    x_1 +4x_3 &=5
  \end{align*}
  First, form the augmented matrix,
  \[
    \begin{bmatrix}
      -7&\answer{-6}&- 12&-33\\
      5&\answer{5}&7&24\\
      1&\answer{0}&4&5
    \end{bmatrix}
  \]
  and work to reduced row-echelon form, first with $j=1$,
  \begin{align*}
    \xrightarrow{\rowopswap{1}{3}}
    &
      \begin{bmatrix}
        1&0&4&5\\
        5&5&7&24\\
        -7&-6&-12&-33
      \end{bmatrix} \\
    \xrightarrow{\rowopadd{-5}{1}{2}}
    & \begin{bmatrix}
      1&0&4&5\\
      0&5&-13&-1\\
      -7&-6&-12&-33
    \end{bmatrix}\\
    \xrightarrow{\rowopadd{7}{1}{3}}
    &
      \begin{bmatrix}
        \leading{1}&0&4&5\\
        0&5&-13&-1\\
        0&-6&16&2
      \end{bmatrix}
  \end{align*}
  Now, with $j=2$,
  \begin{align*}
    \xrightarrow{\rowopmult{\frac{1}{5}}{2}}
    &
      \begin{bmatrix}
        \leading{1}&0&4&5\\
        0&1&\frac{-13}{5}&\frac{-1}{5}\\
        0&-6&16&2
      \end{bmatrix}
                 \xrightarrow{\rowopadd{6}{2}{3}}
                 \begin{bmatrix}
                   \leading{1}&0&4&5\\
                   0&\leading{1}&\frac{-13}{5}&\frac{-1}{5}\\
                   0&0&\frac{2}{5}&\frac{4}{5}
                 \end{bmatrix}
  \end{align*}
  And finally, with $j=3$,
  \begin{align*}
    \xrightarrow{\rowopmult{\frac{5}{2}}{3}}
    &
      \begin{bmatrix}
        \leading{1}&0&4&5\\
        0&\leading{1}&\frac{-13}{5}&\frac{-1}{5}\\
        0&0&1&2
      \end{bmatrix}
               \xrightarrow{\rowopadd{\frac{13}{5}}{3}{2}}
               \begin{bmatrix}
                 \leading{1}&0&4&5\\
                 0&\leading{1}&0&5\\
                 0&0&1&2
               \end{bmatrix}\\
    \xrightarrow{\rowopadd{-4}{3}{1}}
    &
      \begin{bmatrix}
        \leading{1}&0&0&\answer{-3}\\
        0&\leading{1}&0&\answer{5}\\
        0&0&\leading{1}&2
      \end{bmatrix}
  \end{align*}
  This is now the augmented matrix of a very simple system of equations, namely $x_1=\answer{-3}$, $x_2=\answer{5}$, $x_3=2$, which has an obvious solution.  Furthermore, we can see that this is \wordChoice{\choice[correct]{the only solution}\choice{one of many solutions}} to this system.

  \begin{feedback}[correct]
    Indeed, we have determined the entire solution set to be
    \[
      S=\set{\colvector{-3\\5\\2}}.
    \]
  \end{feedback}

\end{example}

\begin{example}
Let us find the solutions to the following system of equations,
\begin{align*}
x_1 -x_2 +2x_3 & =1\\
2x_1+ x_2 + x_3 & =8\\
x_1 + x_2 & =5
\end{align*}
First, form the augmented matrix,
\[
\begin{bmatrix}
1 & -1 & 2 & 1\\
2 & 1 & 1 & 8\\
1 & 1 & 0 & 5
\end{bmatrix}
\]
and work to reduced row-echelon form, first with $j=1$,
\begin{align*}
\xrightarrow{\rowopadd{-2}{1}{2}}
&
\begin{bmatrix}
1 & -1 & 2 & 1\\
0 & 3 & -3 & 6\\
1 & 1 & 0 & 5
\end{bmatrix}
\xrightarrow{\rowopadd{-1}{1}{3}}
\begin{bmatrix}
\leading{1} & -1 & 2 & 1\\
0 & 3 & -3 & 6\\
0 & 2 & -2 & 4
\end{bmatrix}
\end{align*}
Now, with $j=2$,
\begin{align*}
\xrightarrow{\rowopmult{\frac{1}{3}}{2}}
&
\begin{bmatrix}
\leading{1} & -1 & 2 & 1\\
0 & 1 & -1 & 2\\
0 & 2 & -2 & 4
\end{bmatrix}
\xrightarrow{\rowopadd{1}{2}{1}}
\begin{bmatrix}
\leading{1} & 0 & 1 & 3\\
0 & 1 & -1 & 2\\
0 & 2 & -2 & 4
\end{bmatrix}\\
\xrightarrow{\rowopadd{-2}{2}{3}}
&
\begin{bmatrix}
\leading{1} & 0 & 1 & 3\\
0 & \leading{1} & -1 & 2\\
0 & 0 & 0 & \answer{0}
\end{bmatrix}
\end{align*}
The system of equations represented by this augmented matrix needs to be considered a bit differently than the previous example.  First, the last row of the matrix is the equation $0=\answer{0}$, which is \textit{always} true, so it imposes no restrictions on our possible solutions and therefore we can safely ignore it as we analyze the other two equations.  These equations are,
\begin{align*}
x_1+x_3&=3\\
x_2-x_3&=2.
\end{align*}
While this system is fairly easy to solve, it also appears to have a multitude of solutions.  For example, choose $x_3=1$ and see that then $x_1=2$ and $x_2=\answer{3}$ will together form a solution.  Or choose $x_3=0$, and then discover that $x_1=3$ and $x_2=\answer{2}$ lead to a solution.  Try it yourself: pick \textit{any} value of $x_3$ you please, and figure out what $x_1$ and $x_2$ should be to make the first and second equations (respectively) true.  We'll wait while you do that.

Because of this behavior, we say that $x_3$ is a ``free'' or ``independent'' variable.  But why do we vary $x_3$ and not some other variable?  For now, notice that the third column of the augmented matrix is not a pivot column.  With this idea, we can rearrange the two equations, solving each for the variable whose index is the same as the column index of a pivot column.
\begin{align*}
x_1&=3-x_3\\
x_2&=2+x_3
\end{align*}

To write the set of solution vectors in set notation, we have
\begin{align*}
S&=\setparts{\colvector{3-x_3\\2+x_3\\x_3}}{x_3\in\complexes}
\end{align*}
\end{example}

\begin{example}
Let us find the solutions to the following system of equations,
\begin{align*}
2x_1  + x_2 + 7x_3 - 7x_4 &= 2 \\
-3x_1 + 4x_2 -5x_3 - 6x_4 &=  3 \\
x_1 +x_2 + 4x_3 - 5x_4 &=  2
\end{align*}
First, form the augmented matrix,
\[
\begin{bmatrix}
2 & 1 & 7 & -7 & 2\\
-3 & 4 &  -5 & -6 &  3\\
1 & 1 & 4 &  -5 & 2
\end{bmatrix}
\]
and work to reduced row-echelon form, first with $j=1$,
\begin{align*}
\xrightarrow{\rowopswap{1}{3}}
&
\begin{bmatrix}
1 & 1 & 4 &  -5 & 2\\
-3 & 4 &  -5 & -6 &  3\\
2 & 1 & 7 & -7 & 2
\end{bmatrix} \\
\xrightarrow{\rowopadd{3}{1}{2}} &
\begin{bmatrix}
1 & 1 & 4 &  -5 & 2\\
0 & 7 &  7 & -21 &  9\\
2 & 1 & 7 & -7 & 2
\end{bmatrix}\\
\xrightarrow{\rowopadd{-2}{1}{3}} &
\begin{bmatrix}
\leading{1} & 1 & 4 &  -5 & 2\\
0 & 7 &  7 & -21 &  9\\
0 & -1 & -1 & 3 & -2
\end{bmatrix} 
\end{align*}
Now, with $j=2$,
\begin{align*}
\xrightarrow{\rowopswap{2}{3}}
&
\begin{bmatrix}
\leading{1} & 1 & 4 &  -5 & 2\\
0 & -1 & -1 & 3 & -2\\
0 & 7 &  7 & -21 &  9
\end{bmatrix}
\xrightarrow{\rowopmult{-1}{2}}
\begin{bmatrix}
\leading{1} & 1 & 4 &  -5 & 2\\
0 & 1 & 1 & -3 & 2\\
0 & 7 &  7 & -21 &  9
\end{bmatrix}\\
\xrightarrow{\rowopadd{-1}{2}{1}}
&
\begin{bmatrix}
\leading{1} & 0 & 3 &  -2 & 0\\
0 & 1 & 1 & -3 & 2\\
0 & 7 &  7 & -21 &  9
\end{bmatrix}
\xrightarrow{\rowopadd{-7}{2}{3}}
\begin{bmatrix}
\leading{1} & 0 & 3 &  -2 & 0\\
0 & \leading{1} & 1 & -3 & 2\\
0 & 0 &  0 & 0 &  -5
\end{bmatrix}
\end{align*}
And finally, with $j=4$,
\begin{align*}
\xrightarrow{\rowopmult{-\frac{1}{5}}{3}}
&
\begin{bmatrix}
\leading{1} & 0 & 3 &  -2 & 0\\
0 & \leading{1} & 1 & -3 & 2\\
0 & 0 &  0 & 0 &  1
\end{bmatrix}
\xrightarrow{\rowopadd{-2}{3}{2}}
\begin{bmatrix}
\leading{1} & 0 & 3 &  -2 & 0\\
0 & \leading{1} & 1 & -3 & 0\\
0 & 0 &  0 & 0 &  \leading{1}
\end{bmatrix}
\end{align*}
Let us analyze the equations in the system represented by this augmented matrix.  The third equation will read $0=1$.  This is patently false.  No choice of values for our variables will ever make it true.  We are done.  Since we cannot even make the last equation true, we have no hope of making all of the equations simultaneously true.  So this system has no solutions, and its solution set is the empty set, $\emptyset=\set{\ }$.

Notice that we could have reached this conclusion sooner.  After performing the row operation
$\rowopadd{-7}{2}{3}$, we can see that the third equation reads $0=-5$, a false statement.  Since the system represented by this matrix has no solutions, none of the systems represented has any solutions.  However, for this example, we have chosen to bring the matrix all the way  to reduced row-echelon form as practice.
\end{example}

These three examples illustrate the full range of possibilities for a
system of linear equations---no solutions, one solution, or infinitely
many solutions.

We (and everybody else) will often speak of ``row-reducing'' a matrix.  This is an informal way of saying we begin with a matrix $A$ and then analyze \textit{the} matrix $B$ that is row-equivalent to $A$ and in reduced row-echelon form.  So the term \dfn{row-reduce} is used as a verb, but describes something a bit more complicated, since we do not really change $A$.   This process will always be successful and $B$ will be unambiguous.  Typically, an investigation of $A$ will proceed by analyzing $B$ and applying theorems whose hypotheses include the row-equivalence of $A$ and $B$, and usually the hypothesis that $B$ is in reduced row-echelon form.



\end{document}
