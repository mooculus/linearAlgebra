\documentclass{ximera}
\author{Rob Beezer}

\input{../../preamble.tex}

\title{Row Operations}

\begin{document}
\begin{abstract}
  We have seen how certain operations we can perform on equations will
  preserve their solutions.  The next two definitions and the
  following theorem carry over these ideas to augmented matrices.
\end{abstract}
\maketitle


\begin{definition}[Row Operations]
The following three operations will transform an $m\times n$ matrix into a different matrix of the same size, and each is known as a \dfn{row operation}.
\begin{enumerate}
\item Swap the locations of two rows.
\item Multiply each entry of a single row by a nonzero quantity.
\item Multiply each entry of one row by some quantity, and add these values to the entries in the same columns of a second row.  Leave the first row the same after this operation, but replace the second row by the new values.
\end{enumerate}
We will use a symbolic shorthand to describe these row operations:
\begin{enumerate}
\item $\rowopswap{i}{j}$: Swap the location of rows $i$ and $j$.
\item $\rowopmult{\alpha}{i}$: Multiply row $i$ by the nonzero scalar $\alpha$.
\item $\rowopadd{\alpha}{i}{j}$: Multiply row $i$ by the scalar $\alpha$ and add to row $j$.
\end{enumerate}
\end{definition}

\begin{definition}[Row-Equivalent Matrices]
Two matrices are \dfn{row-equivalent} if one can be obtained from the other by a sequence of row operations.
\end{definition}

\begin{example}
  Are the matrices
  \[
    A=\begin{bmatrix}
      2&-1&3&4\\
      5&2&-2&3\\
      1&1&0&6
    \end{bmatrix}
  \]
  and
  \[
    B=\begin{bmatrix}
      1&1&0&6\\
      5&2&-2&3\\
      2&-1&3&4
    \end{bmatrix}
  \]
  row-equivalent?
  
  \begin{multipleChoice}
    \choice[correct]{Yes.}
    \choice{No.}
  \end{multipleChoice}
  
  \begin{feedback}
    Yes, they are row-equivalent, as can be seen from
    \[
      \begin{bmatrix}
        2&-1&3&4\\
        5&2&-2&3\\
        1&1&0&6
      \end{bmatrix}
      \xrightarrow{\rowopswap{1}{3}}
      \begin{bmatrix}
        1&1&0&6\\
        5&2&-2&3\\
        2&-1&3&4
      \end{bmatrix}.
    \]
  \end{feedback}

  \begin{question}
    Are the matrices
    \[
      B= \begin{bmatrix}
        1&1&0&6\\
        5&2&-2&3\\
        2&-1&3&4
      \end{bmatrix}
    \]
    and
    \[
      C = \begin{bmatrix}
        1&1&0&6\\
        3&0&-2&-9\\
        2&-1&3&4
      \end{bmatrix}
    \]
    row-equivalent?
    
    \begin{multipleChoice}
      \choice[correct]{Yes.}
      \choice{No.}
    \end{multipleChoice}
    
    \begin{feedback}
      Yes, because
      \[
        \begin{bmatrix}
          1&1&0&6\\
          5&2&-2&3\\
          2&-1&3&4
        \end{bmatrix}
        \xrightarrow{\rowopadd{-2}{1}{2}}
        \begin{bmatrix}
          1&1&0&6\\
          3&0&-2&-9\\
          2&-1&3&4
        \end{bmatrix}.
      \]
      Consequently, 
      \[
        A=\begin{bmatrix}
          2&-1&3&4\\
          5&2&-2&3\\
          1&1&0&6
        \end{bmatrix}
      \]
      is also row-equivalent to
      \[
        C = \begin{bmatrix}
          1&1&0&6\\
          3&0&-2&-9\\
          2&-1&3&4
        \end{bmatrix}.
      \]
      We can also say that any pair of these three matrices $A$, $B$, and $C$ are row-equivalent.
    \end{feedback}
  \end{question}
\end{example}

\begin{exercise}
  Suppose $A$ is row-equivalent to $B$.  Does this mean that $B$ is row-equivalent to $A$?

  \begin{multipleChoice}
    \choice[correct]{Yes.}
    \choice{No.}
  \end{multipleChoice}  

  \begin{feedback}
    Notice that each of the three row operations is reversible, so we
    do not have to be careful about the distinction between ``$A$ is
    row-equivalent to $B$'' and ``$B$ is row-equivalent to $A$.''
  \end{feedback}
\end{exercise}

The preceding definitions are designed to make the following theorem possible.  It says that row-equivalent matrices represent systems of linear equations that have identical solution sets.

\begin{theorem}[Row-Equivalent Matrices represent Equivalent Systems]
  Suppose that $A$ and $B$ are row-equivalent augmented matrices.
  Then the systems of linear equations that they represent are
  equivalent systems.
\end{theorem}

\begin{proof}
  If we perform a single row operation on an augmented matrix, it will
  have the same effect as if we did the analogous equation operation
  on the system of equations the matrix represents.  We can see that
  each of these row operations will preserve the set of solutions for
  the system of equations the matrix represents.
\end{proof}

So at this point, our strategy is to begin with a system of equations,
represent the system by an augmented matrix, perform row operations
(which will preserve solutions for the system) to get a ``simpler''
augmented matrix, convert back to a ``simpler'' system of equations
and then solve that system, knowing that its solutions are those of
the original system.

\begin{example}
  We solve the following system using augmented matrices and row operations.
  \begin{align*}
    x_1+2x_2+2x_3&=4\\
    x_1+3x_2+3x_3&=5\\
    2x_1+6x_2+5x_3&=6
  \end{align*}
  Form the augmented matrix,
  \begin{align*}
    A=\begin{bmatrix}
      1&\answer{2}&2&4\\
      1&\answer{3}&3&5\\
      2&\answer{6}&5&6
    \end{bmatrix}
  \end{align*}
  and apply row operations,
  \begin{align*}
    \xrightarrow{\rowopadd{-1}{1}{2}}
    &
      \begin{bmatrix}
        1&\answer{2}&2&4\\
        0&\answer{1}&1&1\\
        2&6&5&6
      \end{bmatrix} \\
    \xrightarrow{\rowopadd{-2}{1}{3}} 
     &
      \begin{bmatrix}
        \answer{1}&2&2&4\\
        0&1&1&1\\
        \answer{0}&2&1&-2
      \end{bmatrix}\\
    \xrightarrow{\rowopadd{-2}{2}{3}}
    &
      \begin{bmatrix}
        1&2&2&4\\
        0&\answer{1}&1&1\\
        0&\answer{0}&-1&-4
      \end{bmatrix} \\
    \xrightarrow{\rowopmult{-1}{3}}
     &
       \begin{bmatrix}
         1&2&2&4\\
         0&1& 1&1\\
         0&0&\answer{1}&4
       \end{bmatrix}
  \end{align*}
  So the matrix
  \[
    B=\begin{bmatrix}
      1&2&2&4\\
      0&1& 1&1\\
      0&0&1&4
    \end{bmatrix}
  \]
  is row equivalent to $A$
  and by the theorem, the system of equations
  \begin{align*}
    x_1+2x_2+2x_3&=\answer{4}\\
    x_2+ x_3&=\answer{1}\\
    x_3&=\answer{4}
  \end{align*}
  has the same solution set as the original system of equations.
\end{example}


\end{document}
