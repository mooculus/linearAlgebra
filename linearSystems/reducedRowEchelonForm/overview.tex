\documentclass{ximera}
\author{Crichton Ogle}

\input{../../preamble.tex}

\title{Plan for Row Reduction}

\begin{document}
\begin{abstract}
  We row reduce a matrix by performing row operations, in order to
  find a simpler but equivalent system for which the solution set is
  easily read off.
\end{abstract}
\maketitle

Given a system of equations, how can we find its solution set? The
answer is by finding a simpler but equivalent system for which the
solution set is easily read off. The method used for finding this
simpler system is called \dfn{row reduction}, and the operations used
to perform row reduction are called \dfn{row operations}, of which
there are three types.  \vskip.1in
\begin{description}
\item[Type I] Switching two rows;
\item[Type II] Multiplying a row by a non-zero scalar $\alpha$;
\item[Type III] Adding a multiple of one row to another distinct row.
\end{description} 

It is quite easy to see the the first two types of row operations
won't change the solution set, and it can also be shown that the third
doesn't either. We'll say that two systems of the same dimensions are
\dfn{row equivalent} if one can be gotten from the other by a sequence
of row operations. The following theorem tells us about when this
happens.

\begin{theorem}
  Two systems of the same dimensions are row equivalent if and only if
  they are equivalent.
\end{theorem}

Now to proceed in an efficient manner, it will be convenient to
represent a system in terms of its essential information. This is
accomplished by means of the \dfn{augmented coefficient matrix} or
\dfn{ACM} corresponding to a given system. Starting with the
$m\times n$ system, the augmented coefficient matrix is given by
\[
A = 
\begin{amatrix}{4}
a_{11} &a_{12} &{}\ldots{} &a_{1n} &b_1 \\
a_{21} &a_{22} &{}\ldots{} &a_{2n} &b_2 \\
\vdots &\vdots &{}\ldots{} &\vdots &\vdots \\
a_{m1} &a_{m2} &{}\ldots{} &a_{mn} &b_m 
\end{amatrix}
\]
This is a matrix with $m$ rows and $(n+1)$ columns. Executing a number
of row operations and then computing the ACM gives the same result as
first forming the ACM and performing the row operations on that
matrix. Because the ACM contains all of the information relevant for
solving the system, and is less cumbersome to work with, we will
always form the ACM \textit{first}, and perform row operations on that
matrix.

The simplified form in which we would like to get our matrix is
referred to as \dfn{reduced row echelon form}. This is a term which
applies to matrices in general, not just augmented coefficient
matrices.

\begin{definition} A matrix $B$ of numbers is in \dfn{row-echelon form} if
\begin{itemize}
\item Every row of zeros lies below every non-zero row;
\item the left-most non-zero entry of a non-zero row is 1 (called a \dfn{leading} 1);
\item if both row $k$ and row $(k+1)$ are non-zero, the leading 1 in row $(k+1)$ appears to the right of the leading row in row $k$.
\end{itemize}
It is in \dfn{reduced row-echelon form} if in addition to being in row echelon form it satisfies the property
\begin{itemize}
\item In every column that contains a leading 1, that leading 1 is the only non-zero entry.
\end{itemize}
\end{definition}

\begin{example} Consider the three matrices
\[
A = \begin{bmatrix}
1 & 2 & -1 & 4\\
0 & 0 & 1 & 3\\
0 & 1 & 0 & 2
\end{bmatrix},\qquad
B = \begin{bmatrix}
1 & 2 & 3 & 4\\
0 & 1 & 5 & 7\\
0 & 0 & 0 & 1
\end{bmatrix},\qquad
C = \begin{bmatrix}
1 & 0 & 0 & 4\\
0 & 1 & 0 & 3\\
0 & 0 & 1 & 1
\end{bmatrix}
\]
\end{example}

The matrix $A$ is not in row echelon form, while $B$ is in row echelon but not reduced row echelon form, and $C$ is in reduced row echelon form (the reader should check this, and understand why for each example). The main fact we will need to know is

\begin{theorem} Every matrix of numbers is row-equivalent to one which is in reduced row echelon form.
\end{theorem}

The reduced row echelon form of a matrix is unique; for that reason we will refer to {\it the} reduced row echelon form of a matrix $A$, and write it as $rref(A)$. When $A$ is the ACM of a system of equations, $rref(A)$ tells us essentially everything we would like to know about the original system. The way it does this is summarized by the next result.

\begin{theorem} Let $A$ be the ACM of an $m\times n$ system of equations. Then the system
\begin{itemize}
\item is inconsistent precisely when $rref(A)$ contains a row of the form
\[
[0\ \ 0\ \ 0\ \ 0\ \dots\ \ 0\ \ |\ \ 1];
\]
\item has a unique solution precisely when each column of $rref(A)$ except the right-most column contains a leading 1;
\item has infinitely many solutions when it is i) consistent, and ii) at least one of the first n columns of $rref(A)$ does not contain a leading 1.
\end{itemize}
In the last case, the solution set is parametrized by the variables appearing in the original system which are indexed by the columns of $rref(A)$ to the left of the bar which do not contain a leading 1.
\end{theorem}

It is important to note that when $A$ is the ACM of a system, $rref(A)$ is the ACM of another system equivalent to the original one, which is the reduced row echelon form of the original system. In this reduced row echelon form, it {\it is} possible for an equation to consist of all zeros. If it does, we do not delete it from the system, because we want to maintain the original dimensions.

\begin{example} The rref of the ACM is given by
\[
rref(A) = 
\begin{amatrix}{4}
1 & 2 & 3 & 0 & 5\\
0 & 0 & 0 & 1 & 7 \\
0 & 0 & 0 & 0 & 1
\end{amatrix}
\]
In this case, we would conclude that the system is {\it inconsistent}, because of the last row (to understand why a row like this makes the system inconsistent, the reader should write down the equation to which it corresponds, and see what they can say about solutions to that single equation).
\end{example}
\vskip.2in
\begin{example} The rref of the ACM is given by
\[
rref(A) = 
\begin{amatrix}{4}
1 & 0 & 0 & 0 & 5\\
0 & 1 & 0 & 0 & -2 \\
0 & 0 & 1 & 0 & 4\\
0 & 0 & 0 & 1 & 10
\end{amatrix}
\]

In this case, every column to the left of the bar dividing the {\it coefficient matrix} from the augmented part contains a leading 1. Hence there is a unique solution (the reader should write down the equations corresponding to this ACM, and see how those equations actually give the exact solution).
\end{example} 
\vskip.2in

\begin{example} The rref of the ACM is given by
\[
rref(A) = 
\begin{amatrix}{4}
1 & 0 & 0 & 3 & 7\\
0 & 1 & 0 & 1 & 5 \\
0 & 0 & 1 & 2 & 2\\
0 & 0 & 0 & 0 & 0
\end{amatrix}
\]
Here we have a $4\times 4$ system which has infinitely many solutions; this is seen by noting i) it is consistent, and ii) the fourth column does not contains a leading 1 (the reader should write down the four equations corresponding to this matrix and then rewrite each equation so that they express the original four variables $x_1,x_2,x_3,x_4$ as linear functions in $x_4$, which is the independent parameter in this case. Since we only have one parameter, the result is a \dfn{one-parameter family of solutions}).
\end{example}

\end{document}
