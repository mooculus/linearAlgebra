\documentclass{ximera}
\author{Rob Beezer}
\input{../../preamble.tex}

\title{Free Variables}

\begin{document}
\begin{abstract}
  The number of free variables can be determined easily from a
  row-equivalent matrix in reduced row-echelon form.
\end{abstract}
\maketitle

In this section we develop a foolproof procedure for solving any system of linear equations, no matter how many equations or variables.  We also provide a handful of theorems that allow us to determine partial information about a solution set without actually constructing the whole set itself.

\begin{theorem}[Free Variables for Consistent Systems]
\label{theorem:FVCS}

Suppose $A$ is the augmented matrix of a \textit{consistent} system of
linear equations with $n$ variables.  Suppose also that $B$ is a
row-equivalent matrix in reduced row-echelon form with $r$ rows that
are not completely zeros.  Then the solution set can be described with
$n-r$ free variables.

\begin{proof}
See the proof of \ref{theorem:CSRN}.
\end{proof}
\end{theorem}

Quite a bit of information can be gleaned from a linaer system knowing
only the values of $n$ and $r$ and knowing whether or not the system
is consistent.

\begin{example}
Suppose we have a consistent system with $n=3$ and $r=2$.  Then its solution set then has $n-r=\answer{1}$ free variables, and therefore the system has
\begin{multipleChoice}
  \choice{no solutions.}
  \choice{exactly one solution.}
  \choice[correct]{infinitely many solutions.}
\end{multipleChoice}
\end{example}

\begin{example}
Suppose we have a consistent system with $n=3$ and $r=3$.  Then its solution set then has $n-r=\answer{0}$ free variables, and therefore the system has
\begin{multipleChoice}
  \choice{no solutions.}
  \choice[correct]{exactly one solution.}
  \choice{infinitely many solutions.}
\end{multipleChoice}
\end{example}

\begin{example}
  Suppose we have a system with $n=2$ and $r=3$.  Then this system
  \begin{multipleChoice}
    \choice[correct]{is inconsistent.}
    \choice{is consistent.}
    \choice{could be either consistent or inconsistent---it depends!}
  \end{multipleChoice}
  \begin{feedback}
    Indeed, in this case, column 3 must be a pivot column, so the
    system is inconsistent.

    We should not try to apply the above theorem to count free
    variables, since the theorem only applies to consistent
    systems.
  \end{feedback}
\end{example}

\begin{example}
  Suppose we have a consistent system with $n=4$ and $r=3$.  Then this system is
  \begin{multipleChoice}
    \choice{is inconsistent.}
    \choice{is consistent.}
    \choice[correct]{could be either consistent or inconsistent---it depends!}
  \end{multipleChoice}
\end{example}

We have accomplished a lot so far, but our main goal has been the
following theorem, which is now very simple to prove.  The proof is so
simple that we ought to call it a corollary, but the result is
important enough that it deserves to be called a theorem.

\begin{theorem}[Possible Solution Sets for Linear Systems]
\label{theorem:PSSLS}

A system of linear equations has no solutions, a unique solution or infinitely many solutions.

\begin{proof}
By its definition, a system is either inconsistent or consistent (see \ref{definition:CS}).  The first case describes systems with no solutions.  For consistent systems, we have the remaining two possibilities as guaranteed by, and described in, \ref{theorem:CSRN}.
\end{proof}
\end{theorem}

We have one more theorem to round out our set of tools for determining solution sets to systems of linear equations.

\begin{theorem}[Consistent, More Variables than Equations, Infinite solutions]
\label{theorem:CMVEI}

Suppose a consistent system of linear equations has $m$ equations in $n$ variables.  If $n>m$, then the system has infinitely many solutions.

\begin{proof}
Suppose that the augmented matrix of the system of equations is row-equivalent to $B$, a matrix in reduced row-echelon form with $r$ nonzero rows.
Because $B$ has $m$ rows in total, the number of nonzero rows is less than or equal to $m$.  In other words, $r\leq m$.
Follow this with the hypothesis that $n>m$ and we find that the system has a solution set described by at least one free variable because
\[
n-r\geq n-m>0.
\]
A consistent system with free variables will have an infinite number of solutions, as given by \ref{theorem:CSRN}.
\end{proof}
\end{theorem}

Notice that to use this theorem we need only know that the system is consistent, together with the values of $m$ and $n$.  We do not necessarily have to compute a row-equivalent reduced row-echelon form matrix, even though we discussed such a matrix in the proof.  This is the substance of the following example.

\begin{example}[One solution gives many]
Consider the system of $m=3$ equations in $n=4$ variables:
\begin{align*}
2x_1  + x_2 + 7x_3 - 7x_4 &= 8 \\
-3x_1 + 4x_2 -5x_3 - 6x_4 &=  -12 \\
x_1 +x_2 + 4x_3 - 5x_4 &=  4.
\end{align*}
This system has $x_1 = 0$, $x_2 = 1$, $x_3 = 2$, $x_4 = 1$ as a
solution; this can be checked easily by substitution.  Having been
\textit{handed} this solution, we know the system is
\wordChoice{\choice{inconsistent}\choice[correct]{consistent}}.  This,
together with the fact that $n>m$, allows us to conclude that the
system has
\begin{multipleChoice}
  \choice{no solutions.}
  \choice{exactly one solution.}
  \choice[correct]{infinitely many solutions.}
\end{multipleChoice}
\end{example}

These theorems give us the procedures and implications that allow us to completely solve any system of linear equations.  The main computational tool is using row operations to convert an augmented matrix into reduced row-echelon form.  Here is a broad outline of how we would instruct a computer to solve a system of linear equations.

\begin{enumerate}
\item Represent a system of linear equations in $n$ variables by an augmented matrix (an array is the appropriate data structure in most computer languages).
\item Convert the matrix to a row-equivalent matrix in reduced row-echelon form.
\item Identify the location of the pivot columns, and their number $r$.
\item If column $n+1$ is a pivot column, output the statement that the system is inconsistent and halt.
\item If column $n+1$ is not a pivot column, there are two possibilities:
\begin{itemize}\item $r=n$ and the solution is unique.  It can be read off directly from the entries in rows 1 through $n$ of column $n+1$.
\item
$r<n$ and there are infinitely many solutions.
If only a single solution is needed, set all the free variables to zero and read off the dependent variable values from column $n+1$.  If the entire solution set is required, figure out some nice compact way to describe it, since your finite computer is not big enough to hold all the solutions (we will have such a way soon).
\end{itemize}
\end{enumerate}

% BADBAD: It'd be nice to actually include an example of round-off errors

The above makes it all sound a bit simpler than it really is.  In practice, row operations employ division (usually to get a leading entry of a row to convert to a leading 1) and that will introduce round-off errors.  Entries that should be zero sometimes end up being very, very small nonzero entries, or small entries lead to overflow errors when used as divisors.  A variety of strategies can be employed to minimize these sorts of errors, and this is one of the main topics in the important subject known as numerical linear algebra.


\end{document}
%%% Local Variables:
%%% mode: latex
%%% TeX-master: t
%%% End:
