\documentclass{ximera}
\author{Rob Beezer}
\input{../../preamble.tex}

\title{Null Space}

\begin{document}
\begin{abstract}
  The set of solutions to a homogeneous system is of enough interest to warrant its own name.  
\end{abstract}
\maketitle

\begin{question}
  What is true of the solution set for a homogeneous system of
  equations?

  \begin{multipleChoice}
    \choice{The solution set is $\varnothing$.}
    \choice[correct]{The solution set contains $\vect{0}$.}
  \end{multipleChoice}

  \begin{feedback}[correct]
    The set of solutions to a homogeneous system is never empty; since
    the set of solutions to a homogeneous system consists of vectors
    which make the left hand sides of the equations equal to zero, and
    because ``null'' is another word for ``zero,'' we call this set of
    solutions the ``null space.''
  \end{feedback}
\end{question}

We define the ``null space'' as a property of the coefficient matrix,
not as a property of some system of equations.

\begin{definition}[Null Space of a Matrix]
  The \dfn{null space} of a matrix $A$, denoted $\nsp{A}$, is the set
  of all the vectors that are solutions to the homogeneous system
  $\homosystem{A}$.
\end{definition}

\begin{example}
  Consider the homogeneous system
  \begin{align*}
    x_1 +4x_2  - x_4  + 7x_6 - 9x_7 &= 0\\
    2x_1 + 8x_2 - x_3 + 3x_4 + 9x_5 - 13x_6 + 7x_7 &= 0\\
    2x_3 -3x_4 -4x_5 +12x_6 -8x_7 &= 0\\
    -x_1  - 4x_2 + 2x_3 +4x_4 + 8x_5 - 31x_6 + 37x_7 &= 0,
  \end{align*}
  Here are two solutions written as solution vectors.
  \begin{align*}
    \vect{x}=\colvector{3\\0\\-5\\-6\\0\\0\\1}&&
    \vect{y}=\colvector{-4\\1\\-3\\-2\\1\\1\\1}
  \end{align*}
  We can say that they are in the null space of the coefficient matrix
  for the system of equations.

  Is the vector
  \[
    \vect{z}=\colvector{1\\0\\0\\0\\0\\0\\2}
  \]
  in the null space?

  \begin{multipleChoice}
    \choice[correct]{No.}
    \choice{Yes.}
  \end{multipleChoice}

  \begin{feedback}
    The vector $\vect{z}$ is not a solution to the homogeneous system.  For example, it fails to make the first equation true.
  \end{feedback}
\end{example}

Here are two (prototypical) examples of the computation of the null space of a matrix.

\begin{example}[Computing a null space]
  Let us compute the null space of
  \[
    A=\begin{bmatrix}
      2 & -1 & 7 & -3 & -8 \\
      1 & 0 & 2 & 4 & 9 \\
      2 & 2 & -2 & -1 & 8
    \end{bmatrix}
  \]
  which we write as $\nsp{A}$.  We simply desire to solve the
  homogeneous system $\homosystem{A}$.  So we row-reduce the augmented
  matrix to obtain
  \[
    \begin{bmatrix}
      \leading{1} & 0 & 2 & 0 & 1 & 0 \\
      0 & \leading{1} & -3 & 0 & 4 & 0 \\
      0 & 0 & 0 & \leading{1} & 2 & 0
    \end{bmatrix}
  \]
  The variables (of the homogeneous system) $x_3$ and $x_5$ are free
  (since columns 1, 2 and \answer{4} are pivot columns), so we arrange
  the equations represented by the matrix in reduced row-echelon form
  to
  \begin{align*}
    x_1&=-2x_3-x_5\\
    x_2&=\answer{3}x_3-4x_5\\
    x_4&=-2x_5\\
  \end{align*}
  So we can write the infinite solution set as sets using column vectors,
  \[
    \nsp{A}=\setparts{
      \colvector{-2x_3-x_5\\3x_3-4x_5\\x_3\\-2x_5\\x_5}
    }{
      x_3,\,x_5\in\complexes
    }
  \]
\end{example}

\begin{example}[Computing a null space again]
  Let us compute the null space of
  \[
    C=\begin{bmatrix}
      -4 & 6 & 1 \\
      -1 & 4 & 1 \\
      5 & 6 & 7 \\
      4 & 7 & 1
    \end{bmatrix}
  \]
  which we write as $\nsp{C}$.   We simply desire to solve the homogeneous system $\homosystem{C}$.  So we row-reduce the augmented matrix to obtain
  \[
    \begin{bmatrix}
      \leading{1} & 0 & 0 & 0 \\
      0 & \leading{1} & 0 & 0 \\
      0 & 0 & \leading{1} & 0 \\
      0 & 0 & 0 & 0
    \end{bmatrix}
  \]
  How many free variables are there?
  \begin{multipleChoice}
    \choice[correct]{0}
    \choice{1}
    \choice{2}
    \choice{3}
  \end{multipleChoice}
  
  \begin{feedback}[correct]
    There are no free variables in the homogeneous system represented
    by the row-reduced matrix, so there is only the trivial solution,
    the zero vector, $\zerovector$.  So we can write the (trivial)
    solution set as
    \[
      \nsp{C}=\set{\zerovector}=\set{\colvector{0\\0\\0}}
    \]
  \end{feedback}
\end{example}

\begin{question}
  Suppose a homogeneous system of equations has 13 variables and 8
  equations.  How many solutions will it have?

  \begin{multipleChoice}
    \choice{No solutions.}
    \choice{Exactly one solution.}
    \choice[correct]{Infinitely many solutions.}
  \end{multipleChoice}

  \begin{hint}
    A homogeneous system of equations with 13 variables and 8
    equations is necessarily consistent, because $\vect{0}$ is a
    solution.
  \end{hint}

  \begin{hint}
    Moreover, there will be free variables, and since we can assign
    any (of infinitely many possible!) values to a free variable,
    there will be infinitely many solutions.
  \end{hint}
\end{question}

\end{document}
