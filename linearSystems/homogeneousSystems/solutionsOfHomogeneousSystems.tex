\documentclass{ximera}
\author{Rob Beezer}
\input{../../preamble.tex}

\title{Solutions of Homogeneous Systems}

\begin{document}
\begin{abstract}
  We specialize to systems of linear equations where every equation
  has a zero as its constant term.
\end{abstract}
\maketitle

\begin{definition}[Homogeneous System]
  A system of linear equations, $\linearsystem{A}{\vect{b}}$ is
  \dfn{homogeneous} if the vector of constants is the zero vector, in
  other words, if $\vect{b}=\zerovector$.
\end{definition}

\begin{example}
  Is the system
  \begin{align*}
    2x_1  - 3x_2 + x_3 - 6x_4 &= -7 \\
    4x_1 +x_2 +2x_3 + 9x_4 &=  -7 \\
    3x_1 +x_2 +x_3 + 8x_4 &=  -8
  \end{align*}
  a homogeneous system?

  \begin{multipleChoice}
    \choice[correct]{No, it is not homogeneous.}
    \choice{Yes, it is homogeneous.}
  \end{multipleChoice}

  \begin{feedback}[correct]
    That's correct: it is not a homogeneous system.  Bu we could convert it into a homogeneous system, i.e.,
    \begin{align*}
      2x_1  - 3x_2 + x_3 - 6x_4 &= 0 \\
      4x_1 +x_2 +2x_3 + 9x_4 &=  0 \\
      3x_1 +x_2 +x_3 + 8x_4 &=  0.
    \end{align*}
  \end{feedback}
\end{example}

\begin{question}
  Can you quickly find a solution to a homogeneous system without row-reducing the augmented matrix?

  \begin{multipleChoice}
    \choice[correct]{Yes.}
    \choice{No.}
  \end{multipleChoice}

  \begin{question}
    Setting each variable to zero will
    \wordChoice{\choice[correct]{always}\choice{never}} be a solution
    of a homogeneous system.  This is the substance of the following
    theorem.
  \end{question}
\end{question}

\begin{theorem}[Homogeneous Systems are Consistent]
\label{theorem:HSC}
Suppose that a system of linear equations is homogeneous.  Then the system is consistent and one solution is found by setting each variable to zero.

\begin{proof}
  Set each variable of the system to zero.  When substituting these
  values into each equation, the left-hand side evaluates to zero, no
  matter what the coefficients are.  Since a homogeneous system has
  zero on the right-hand side of each equation as the constant term,
  each equation is true.  With one demonstrated solution, we can call
  the system consistent.
\end{proof}
\end{theorem}

Since this solution is so obvious, we now define it as the trivial
solution.

\begin{definition}[Trivial Solution to Homogeneous Systems of Equations]
  Suppose a homogeneous system of linear equations has $n$ variables.
  The solution $x_1=0$, $x_2=0$, \ldots, $x_n=0$ (i.e.,
  $\vect{x}=\zerovector$) is called the \dfn{trivial solution}.
\end{definition}

Here are three typical examples.  Work through the row operations as
we bring each to reduced row-echelon form.  Also notice what is
similar in each example, and what differs.

\begin{example}
  Consider the homogeneous system
  \begin{align*}
    -7x_1 -6 x_2 - 12x_3 &=0\\
    5x_1  + 5x_2 + 7x_3 &=0\\
    x_1 +4x_3 &=0.
  \end{align*}
  The corresponding augmented matrix row-reduces to
  \[
    \begin{bmatrix}
      \leading{1} & 0 & 0 & 0\\
      0 & \leading{1} & 0 & 0\\
      0 & 0 & \leading{1} & 0
    \end{bmatrix}
  \]
  The system is consistent, and so the computation $n-r=3-3=0$ means 
  \begin{multipleChoice}
    \choice{the solution set is empty.}
    \choice[correct]{the solution set contains just a single solution.}
    \choice{the solution set is infinite.}
  \end{multipleChoice}
  
  \begin{feedback}[correct]
    Then, this lone solution must be the trivial solution.  
  \end{feedback}
\end{example}

\begin{example}
The augmented matrix corresponding to the homogeneous system
\begin{align*}
x_1 -x_2 +2x_3 & = 0\\
 2x_1+ x_2 + x_3 & = 0\\
 x_1 + x_2\quad\quad & = 0
\end{align*}
row-reduces to
\[
  \begin{bmatrix}
    \leading{1} & 0 & 1 & 0 \\
    0 & \leading{1} & -1 & 0\\
    0 & 0 & 0 & 0
  \end{bmatrix}
\]
The system is consistent, and so the computation $n-r=3-2=1$ means the solution set contains \wordChoice{\choice[correct]{one free variable}{one dependent variable}}, and hence has \wordChoice{\choice{exactly one solution}\choice[correct]{infinitely many solutions}}.

\begin{feedback}[correct]
  We can describe this solution set using the free variable $x_3$,
  \[
    S=\setparts{\colvector{x_1\\x_2\\x_3}}{x_1=-x_3,\,x_2=x_3}
    =\setparts{\colvector{-x_3\\x_3\\x_3}}{x_3\in\complexes}
  \]
Geometrically, these are points in three dimensions that lie on a line through the origin.
\end{feedback}

\end{example}

\begin{example}
Consider the system
\begin{align*}
2x_1  + x_2 + 7x_3 - 7x_4 &= 0 \\
-3x_1 + 4x_2 -5x_3 - 6x_4 &= 0 \\
x_1 +x_2 + 4x_3 - 5x_4 &=  0
\end{align*}
whose augmented matrix row-reduces to
\[
  \begin{bmatrix}
    \leading{1} & 0 & 3 & -2 & 0 \\
    0 & \leading{1} & 1 &  -3 & 0\\
    0 & 0 & 0 &  0 & 0
  \end{bmatrix}
\]
Being homogeneous, the system is consistent, and so the computation $n-r=4-2=2$ means the solution set contains two free variables by \ref{theorem:FVCS}, and hence has infinitely many solutions.  We can describe this solution set using the free variables $x_3$ and $x_4$,
\begin{align*}
  S&=\setparts{\colvector{x_1\\x_2\\x_3\\x_4}}{x_1=-3x_3+2x_4,\,x_2=-x_3+3x_4}\\
   &=\setparts{\colvector{-3x_3+2x_4\\-x_3+3x_4\\x_3\\x_4}}{ x_3,\,x_4\in\complexes}\\
\end{align*}
\end{example}

Notice that when we do row operations on the augmented matrix of a
homogeneous system of linear equations the last column of the matrix
is all zeros.  Any one of the three allowable row operations will
convert zeros to zeros and thus, the final column of the matrix in
reduced row-echelon form will also be all zeros.  So in this case, we
may be as likely to reference only the coefficient matrix and presume
that we remember that the final column begins with zeros, and after
any number of row operations is still zero.

\begin{theorem}[Homogeneous, More Variables than Equations, Infinite solutions]
  \label{theorem:HMVEI}
  Suppose that a homogeneous system of linear equations has $m$
  equations and $n$ variables with $n>m$.  Then the system has
  infinitely many solutions.
  
  \begin{proof}
    We are assuming the system is homogeneous, so it is consistent.  Then the hypothesis that $n>m$ gives infinitely many solutions.
  \end{proof}
\end{theorem}

\begin{question}
  For a homogeneous system, when $n>m$,
  \begin{multipleChoice}
    \choice{there are no solutions.}
    \choice{there is exactly one solution.}
    \choice[correct]{there are infinitely many solutions.}
    \choice{more information is needed to determine the number of solutions.}
  \end{multipleChoice}
  
  \begin{question}
    For a homogeneous system, when $n=m$,
    \begin{multipleChoice}
      \choice{there are no solutions.}
      \choice{there is exactly one solution.}
      \choice{there are infinitely many solutions.}
      \choice[correct]{more information is needed to determine the number of solutions.}
    \end{multipleChoice}
  \end{question}
  
\end{question}

\end{document}
