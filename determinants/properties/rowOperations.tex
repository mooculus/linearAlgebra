\documentclass{ximera}

\input{../../preamble.tex}

\title{Determinants and Row Operations}

\begin{document}
\begin{abstract}
  One can understand the effect of row operations on the determinant.
\end{abstract}
\maketitle

We have seen how to compute the determinant of a matrix, and the
incredible fact that we can perform expansion about \textit{any} row
\textit{or} column to make this computation.  In the nex few largely
theoretical activities, we will state and prove several more
intriguing properties about determinants.  Our main goal will be the
two results in \ref{theorem:SMZD} and \ref{theorem:DRMM}, but more
specifically, we will see how the value of a determinant will allow us
to gain insight into the various properties of a square matrix.

\begin{theorem}[Determinant with Zero Row or Column]
\label{theorem:DZRC}

Suppose that $A$ is a square matrix with a row where every entry is
zero, or a column where every entry is zero.  Then $\detname{A}=0$.

\begin{proof}
  Suppose that $A$ is a square matrix of size $n$ and row $i$ has every entry equal to zero.  We compute $\detname{A}$ via expansion about row $i$.
  \begin{align*}
    \detname{A}
    &=
      \sum_{j=1}^{n}(-1)^{i+j}\matrixentry{A}{ij}\detname{\submatrix{A}{i}{j}}
    &&\ref{theorem:DER}\\
    &=
      \sum_{j=1}^{n}(-1)^{i+j}\,0\,\detname{\submatrix{A}{i}{j}}
    &&\text{Row $i$ is zeros}\\
    &=
      \sum_{j=1}^{n}0=0
  \end{align*}
  
  The proof for the case of a zero column is entirely similar, or
  could be derived from an application of \ref{theorem:DT} employing
  the transpose of the matrix.
\end{proof}
\end{theorem}

\begin{theorem}[Determinant for Row or Column Swap]
\label{theorem:DRCS}

Suppose that $A$ is a square matrix.  Let $B$ be the square matrix
obtained from $A$ by interchanging the location of two rows, or
interchanging the location of two columns.  Then
$\detname{B}=-\detname{A}$.

  \begin{proof}
    Begin with the special case where $A$ is a square matrix of size
    $n$ and we form $B$ by swapping \textit{adjacent} rows $i$ and
    $i+1$ for some $1\leq i\leq n-1$.  Notice that the assumption
    about swapping adjacent rows means that
    $\submatrix{B}{i+1}{j}=\submatrix{A}{i}{j}$ for all
    $1\leq j\leq n$, and $\matrixentry{B}{i+1,j}=\matrixentry{A}{ij}$
    for all $1\leq j\leq n$.  We compute $\detname{B}$ via expansion
    about row $i+1$.
    \begin{align*}
      \detname{B}
      &=
        \sum_{j=1}^{n}(-1)^{(i+1)+j}\matrixentry{B}{i+1,j}\detname{\submatrix{B}{i+1}{j}}
      &&\ref{theorem:DER}\\
      &=
        \sum_{j=1}^{n}(-1)^{(i+1)+j}\matrixentry{A}{ij}\detname{\submatrix{A}{i}{j}}
      &&\text{Hypothesis}\\
      &=
        \sum_{j=1}^{n}(-1)^{1}(-1)^{i+j}\matrixentry{A}{ij}\detname{\submatrix{A}{i}{j}}\\
      &=
        (-1)\sum_{j=1}^{n}(-1)^{i+j}\matrixentry{A}{ij}\detname{\submatrix{A}{i}{j}}\\
      &=
        -\detname{A}&&\ref{theorem:DER}
    \end{align*}

    So the result holds for the special case where we swap adjacent
    rows of the matrix.  As any computer scientist knows, we can
    accomplish \textit{any} rearrangement of an ordered list by
    swapping adjacent elements.  This principle can be demonstrated by
    na\"ive sorting algorithms such as ``bubble sort.''  In any event,
    we do not need to discuss every possible reordering, we just need
    to consider a swap of two rows, say rows $s$ and $t$ with
    $1\leq s<t\leq n$.

    Begin with row $s$, and repeatedly swap it with each row just
    below it, including row $t$ and stopping there.  This will total
    $t-s$ swaps.  Now swap the former row $t$, which currently lives
    in row $t-1$, with each row above it, stopping when it becomes row
    $s$.  This will total another $t-s-1$ swaps.  In this way, we
    create $B$ through a sequence of $2(t-s)-1$ swaps of adjacent
    rows, each of which adjusts $\detname{A}$ by a multiplicative
    factor of $-1$.  So
    \[
      \detname{B}
      =
      (-1)^{2(t-s)-1}\detname{A}
      =
      \left((-1)^2\right)^{t-s}(-1)^{-1}\detname{A}
      =
      -\detname{A}
    \]
    as desired.

    The proof for the case of swapping two columns is entirely
    similar, or could be derived from an application of
    \ref{theorem:DT} employing the transpose of the matrix.
  \end{proof}
\end{theorem}

So \ref{theorem:DRCS} tells us the effect of the first row operation
(\ref{definition:RO}) on the determinant of a matrix.  Here is the
effect of the second row operation.



\begin{theorem}[Determinant for Row or Column Multiples]
  \label{theorem:DRCM}

  Suppose that $A$ is a square matrix.  Let $B$ be the square matrix
  obtained from $A$ by multiplying a single row by the scalar
  $\alpha$, or by multiplying a single column by the scalar $\alpha$.
  Then $\detname{B}=\alpha\detname{A}$.

  \begin{proof}
    Suppose that $A$ is a square matrix of size $n$ and we form the
    square matrix $B$ by multiplying each entry of row $i$ of $A$ by
    $\alpha$.  Notice that the other rows of $A$ and $B$ are equal, so
    $\submatrix{A}{i}{j}=\submatrix{B}{i}{j}$, for all
    $1\leq j\leq n$.  We compute $\detname{B}$ via expansion about row
    $i$.
    \begin{align*}
      \detname{B}
      &=
        \sum_{j=1}^{n}(-1)^{i+j}\matrixentry{B}{ij}\detname{\submatrix{B}{i}{j}}
      &&\ref{theorem:DER}\\
      &=
        \sum_{j=1}^{n}(-1)^{i+j}\matrixentry{B}{ij}\detname{\submatrix{A}{i}{j}}
      &&\text{Hypothesis}\\
      &=
        \sum_{j=1}^{n}(-1)^{i+j}\alpha\matrixentry{A}{ij}\detname{\submatrix{A}{i}{j}}
      &&\text{Hypothesis}\\
      &=
        \alpha\sum_{j=1}^{n}(-1)^{i+j}\matrixentry{A}{ij}\detname{\submatrix{A}{i}{j}}\\
      &=
        \alpha\detname{A}&&\ref{theorem:DER}\\
    \end{align*}

    The proof for the case of a multiple of a column is entirely
    similar, or could be derived from an application of
    \ref{theorem:DT} employing the transpose of the matrix.
  \end{proof}
\end{theorem}

Let us go for understanding the effect of all three row operations.
But first we need an intermediate result, but it is an easy one.

\begin{theorem}[Determinant with Equal Rows or Columns]
  \label{theorem:DERC}

  Suppose that $A$ is a square matrix with two equal rows, or two
  equal columns.  Then $\detname{A}=0$.

  \begin{proof}
    Suppose that $A$ is a square matrix of size $n$ where the two rows
    $s$ and $t$ are equal.  Form the matrix $B$ by swapping rows $s$
    and $t$.  Notice that as a consequence of our hypothesis, $A=B$.
    Then
    \begin{align*}
      \detname{A}
      &=
        \frac{1}{2}\left(\detname{A}+\detname{A}\right)\\
      &=
        \frac{1}{2}\left(\detname{A}-\detname{B}\right)
      &&\ref{theorem:DRCS}\\
      &=
        \frac{1}{2}\left(\detname{A}-\detname{A}\right)
      &&\text{Hypothesis, $A=B$}\\
      &=
        \frac{1}{2}\,(0)=0
    \end{align*}

    The proof for the case of two equal columns is entirely similar,
    or could be derived from an application of \ref{theorem:DT}
    employing the transpose of the matrix.

  \end{proof}
\end{theorem}

Now explain the third row operation.  Here we go.

\begin{theorem}[Determinant for Row or Column Multiples and Addition]
  \label{theorem:DRCMA}

  Suppose that $A$ is a square matrix.  Let $B$ be the square matrix
  obtained from $A$ by multiplying a row by the scalar $\alpha$ and
  then adding it to another row, or by multiplying a column by the
  scalar $\alpha$ and then adding it to another column.  Then
  $\detname{B}=\detname{A}$.

  \begin{proof}
    Suppose that $A$ is a square matrix of size $n$.  Form the matrix
    $B$ by multiplying row $s$ by $\alpha$ and adding it to row $t$.
    Let $C$ be the auxiliary matrix where we replace row $t$ of $A$ by
    row $s$ of $A$.  Notice that
    $\submatrix{A}{t}{j}=\submatrix{B}{t}{j}=\submatrix{C}{t}{j}$ for
    all $1\leq j\leq n$.  We compute the determinant of $B$ by
    expansion about row $t$.
    \begin{align*}
      \detname{B}
      &=
        \sum_{j=1}^{n}(-1)^{t+j}\matrixentry{B}{tj}\detname{\submatrix{B}{t}{j}}
      &&\ref{theorem:DER}\\
      &=
        \sum_{j=1}^{n}(-1)^{t+j}\left(\alpha\matrixentry{A}{sj}+\matrixentry{A}{tj}\right)\detname{\submatrix{B}{t}{j}}
      &&\text{Hypothesis}\\
      &=
        \sum_{j=1}^{n}(-1)^{t+j}\alpha\matrixentry{A}{sj}\detname{\submatrix{B}{t}{j}}\\
      &\quad\quad+
        \sum_{j=1}^{n}(-1)^{t+j}\matrixentry{A}{tj}\detname{\submatrix{B}{t}{j}}\\
      &=
        \alpha\sum_{j=1}^{n}(-1)^{t+j}\matrixentry{A}{sj}\detname{\submatrix{B}{t}{j}}\\
      &\quad\quad+
        \sum_{j=1}^{n}(-1)^{t+j}\matrixentry{A}{tj}\detname{\submatrix{B}{t}{j}}\\
      &=
        \alpha\sum_{j=1}^{n}(-1)^{t+j}\matrixentry{C}{tj}\detname{\submatrix{C}{t}{j}}\\
      &\quad\quad+
        \sum_{j=1}^{n}(-1)^{t+j}\matrixentry{A}{tj}\detname{\submatrix{A}{t}{j}}\\
      &=
        \alpha\detname{C}+\detname{A}&&\ref{theorem:DER}\\
      &=
        \alpha\,0+\detname{A} = \detname{A}&&\ref{theorem:DERC}
    \end{align*}

    The proof for the case of adding a multiple of a column is
    entirely similar, or could be derived from an application of
    \ref{theorem:DT} employing the transpose of the matrix.
  \end{proof}
\end{theorem}

Is this what you expected?  We could argue that the third row
operation is the most popular, and yet it has no effect whatsoever on
the determinant of a matrix!  We can exploit this, along with our
understanding of the other two row operations, to provide another
approach to computing a determinant.  We'll explain this in the
context of an example.

\begin{example}[Determinant by row operations]

  Suppose we desire the determinant of the $4\times 4$ matrix
  \[
    A=
    \begin{bmatrix}
      2 & 0 & 2 & 3 \\
      1 & 3 &-1 & 1 \\
      -1& 1 &-1 & 2 \\
      3 & 5 & 4 & 0
    \end{bmatrix}
  \]

  We will perform a sequence of row operations on this matrix,
  shooting for an upper triangular matrix, whose determinant will be
  simply the product of its diagonal entries.  For each row operation,
  we will track the effect on the determinant via \ref{theorem:DRCS},
  \ref{theorem:DRCM}, \ref{theorem:DRCMA}.
  \begin{align*}
    A&=
       \begin{bmatrix}
         2 & 0 & 2 & 3 \\
         1 & 3 &-1 & 1 \\
         -1& 1 &-1 & 2 \\
         3 & 5 & 4 & 0
       \end{bmatrix}
                   &&\detname{A}&&\\
    \xrightarrow{\rowopswap{1}{2}}
    A_1&=
         \begin{bmatrix}
           1 & 3 &-1 & 1 \\
           2 & 0 & 2 & 3 \\
           -1& 1 &-1 & 2 \\
           3 & 5 & 4 & 0
         \end{bmatrix}
                   &
           &=-\detname{A_1}&&\ref{theorem:DRCS}\\
    \xrightarrow{\rowopadd{-2}{1}{2}}
    A_2&=
         \begin{bmatrix}
           1 & 3 &-1 & 1 \\
           0 &-6 & 4 & 1 \\
           -1& 1 &-1 & 2 \\
           3 & 5 & 4 & 0
         \end{bmatrix}
                   &
           &=-\detname{A_2}&&\ref{theorem:DRCMA}\\
    \xrightarrow{\rowopadd{1}{1}{3}}
    A_3&=
         \begin{bmatrix}
           1 & 3 &-1 & 1 \\
           0 &-6 & 4 & 1 \\
           0 & 4 &-2 & 3 \\
           3 & 5 & 4 & 0
         \end{bmatrix}
                   &
           &=-\detname{A_3}&&\ref{theorem:DRCMA}\\
    \xrightarrow{\rowopadd{-3}{1}{4}}
    A_4&=
         \begin{bmatrix}
           1 & 3 &-1 & 1 \\
           0 &-6 & 4 & 1 \\
           0 & 4 &-2 & 3 \\
           0 &-4 & 7 &-3
         \end{bmatrix}
                   &
           &=-\detname{A_4}&&\ref{theorem:DRCMA}\\
    \xrightarrow{\rowopadd{1}{3}{2}}
    A_5&=
         \begin{bmatrix}
           1 & 3 &-1 & 1 \\
           0 &-2 & 2 & 4 \\
           0 & 4 &-2 & 3 \\
           0 &-4 & 7 &-3
         \end{bmatrix}
                   &
           &=-\detname{A_5}&&\ref{theorem:DRCMA}\\
    \xrightarrow{\rowopmult{-\frac{1}{2}}{2}}
    A_6&=
         \begin{bmatrix}
           1 & 3 &-1 & 1 \\
           0 & 1 &-1 &-2 \\
           0 & 4 &-2 & 3 \\
           0 &-4 & 7 &-3
         \end{bmatrix}
                   &
           &=\answer{2}\detname{A_6}&&\ref{theorem:DRCM}\\
    \xrightarrow{\rowopadd{-4}{2}{3}}
    A_7&=
         \begin{bmatrix}
           1 & 3 &-1 & 1 \\
           0 & 1 &-1 &-2 \\
           0 & 0 & 2 &11 \\
           0 &-4 & 7 &-3
         \end{bmatrix}
                   &
           &=2\detname{A_7}&&\ref{theorem:DRCMA}\\
    \xrightarrow{\rowopadd{4}{2}{4}}
    A_8&=
         \begin{bmatrix}
           1 & 3 &-1 & 1 \\
           0 & 1 &-1 &-2 \\
           0 & 0 & 2 &11 \\
           0 & 0 & 3 &-11
         \end{bmatrix}
                   &
           &=2\detname{A_8}&&\ref{theorem:DRCMA}\\
    \xrightarrow{\rowopadd{-1}{3}{4}}
    A_9&=
         \begin{bmatrix}
           1 & 3 &-1 & 1 \\
           0 & 1 &-1 &-2 \\
           0 & 0 & 2 &11 \\
           0 & 0 & 1 &-22
         \end{bmatrix}
                   &
           &=2\detname{A_9}&&\ref{theorem:DRCMA}\\
    \xrightarrow{\rowopadd{-2}{4}{3}}
    A_{10}&=
            \begin{bmatrix}
              1 & 3 &-1 & 1 \\
              0 & 1 &-1 &-2 \\
              0 & 0 & 0 &55 \\
              0 & 0 & 1 &-22
            \end{bmatrix}
                   &
           &=2\detname{A_{10}}&&\ref{theorem:DRCMA}\\
    \xrightarrow{\rowopswap{3}{4}}
    A_{11}&=
            \begin{bmatrix}
              1 & 3 &-1 & 1 \\
              0 & 1 &-1 &-2 \\
              0 & 0 & 1 &-22\\
              0 & 0 & 0 &55
            \end{bmatrix}
                   &
           &=\answer{-2}\detname{A_{11}}&&\ref{theorem:DRCS}\\
    \xrightarrow{\rowopmult{\frac{1}{55}}{4}}
    A_{12}&=
            \begin{bmatrix}
              1 & 3 &-1 & 1 \\
              0 & 1 &-1 &-2 \\
              0 & 0 & 1 &-22\\
              0 & 0 & 0 & 1
            \end{bmatrix}
                   &
           &=\answer{-110}\detname{A_{12}}&&\ref{theorem:DRCM}\\
  \end{align*}

  The matrix $A_{12}$ is upper triangular, so expansion about the
  first column (repeatedly) will result in
  $\detname{A_{12}}=(1)(1)(1)(1)=1$ (see \ref{example:DUTM}) and thus,
  $\detname{A}=\answer{-110}$.

  Notice that our sequence of row operations was somewhat \textit{ad
    hoc}, such as the transformation to $A_5$.  We could have been
  even more methodical, and strictly followed the process that
  converts a matrix to reduced row-echelon form (\ref{theorem:REMEF}),
  eventually achieving the same numerical result with a final matrix
  that equaled the $4\times 4$ identity matrix.  Notice too that we
  could have stopped with $A_8$, since at this point we could compute
  $\detname{A_8}$ by two expansions about first columns, followed by a
  simple determinant of a $2\times 2$ matrix (\ref{theorem:DMST}).

  The beauty of this approach is that computationally we should
  already have written a procedure to convert matrices to reduced-row
  echelon form, so all we need to do is track the multiplicative
  changes to the determinant as the algorithm proceeds.  Further, for
  a square matrix of size $n$ this approach requires on the order of
  $n^3$ multiplications, while a recursive application of expansion
  about a row or column (\ref{theorem:DER}, \ref{theorem:DEC}) will
  require in the vicinity of $(n-1)(n!)$ multiplications.  So even for
  very small matrices, a computational approach utilizing row
  operations will have superior run-time.  Tracking, and controlling,
  the effects of round-off errors is another story, best saved for a
  numerical linear algebra course.

\end{example}

\end{document}

