\documentclass{ximera}

\input{../../preamble.tex}

\title{Determinants, Row Operations, Elementary Matrices}

\begin{document}
\begin{abstract}
  As a final preparation for our two most important theorems about
  determinants, we prove a handful of facts about the interplay of row
  operations and matrix multiplication with elementary matrices with
  regard to the determinant.
\end{abstract}
\maketitle

First, a simple, but crucial, fact about the identity matrix.

\begin{theorem}[Determinant of the Identity Matrix]
  \label{theorem:DIM} 

  For every $n\geq 1$, $\detname{I_n}=1$.

  \begin{proof}
    It may be overkill, but this is a good situation to run through a
    proof by induction on $n$.  Is the result true when $n=1$? Yes,
    \begin{align*}
      \detname{I_1}
      &=\matrixentry{I_1}{11}&&\ref{definition:DM}\\
      &=1&&\ref{definition:IM}\\
    \end{align*}
    
    Now assume the theorem is true for the identity matrix of size
    $n-1$ and investigate the determinant of the identity matrix of
    size $n$ with expansion about row 1,
    \begin{align*}
      \detname{I_n}
      &=
        \sum_{j=1}^{n}(-1)^{1+j}\matrixentry{I_n}{1j}\detname{\submatrix{I_n}{1}{j}}
      &&\ref{definition:DM}\\
      &=
        (-1)^{1+1}\matrixentry{I_n}{11}\detname{\submatrix{I_n}{1}{1}}\\
      &\quad\quad+
        \sum_{j=2}^{n}(-1)^{1+j}\matrixentry{I_n}{1j}\detname{\submatrix{I_n}{1}{j}}\\
      &=
        1\detname{I_{n-1}}+
        \sum_{j=2}^{n}(-1)^{1+j}\,0\,\detname{\submatrix{I_n}{1}{j}}
      &&\ref{definition:IM}\\
      &=
        1(1)+\sum_{j=2}^{n}\,0=1
      &&\text{Induction Hypothesis}\\
    \end{align*}
  \end{proof}
\end{theorem}

\begin{theorem}[Determinants of Elementary Matrices]
  \label{theorem:DEM}
  For the three possible versions of an elementary matrix (\ref{definition:ELEM}) we have the determinants,
  \begin{enumerate}
  \item $\detname{\elemswap{i}{j}}=-1$
  \item $\detname{\elemmult{\alpha}{i}}=\alpha$
  \item $\detname{\elemadd{\alpha}{i}{j}}=1$
  \end{enumerate}

  \begin{proof}
    Swapping rows $i$ and $j$ of the identity matrix will create $\elemswap{i}{j}$  (\ref{definition:ELEM}), so
    \begin{align*}
      \detname{\elemswap{i}{j}}
      &=-\detname{I_n}&&\ref{theorem:DRCS}\\
      &=-1&&\ref{theorem:DIM}\\
    \end{align*}

    Multiplying row $i$ of the identity matrix by $\alpha$ will create $\elemmult{\alpha}{i}$ (\ref{definition:ELEM}), so
    \begin{align*}
      \detname{\elemmult{\alpha}{i}}
      &=\alpha\detname{I_n}&&\ref{theorem:DRCM}\\
      &=\alpha(1)=\alpha&&\ref{theorem:DIM}\\
    \end{align*}

    Multiplying row $i$ of the identity matrix by $\alpha$ and adding to row $j$ will create $\elemadd{\alpha}{i}{j}$ (\ref{definition:ELEM}), so
    \begin{align*}
      \detname{\elemadd{\alpha}{i}{j}}
      &=\detname{I_n}&&\ref{theorem:DRCMA}\\
      &=1&&\ref{theorem:DIM}\\
    \end{align*}
    
  \end{proof}
\end{theorem}

\begin{theorem}[Determinants, Elementary Matrices, Matrix Multiplication]
  \label{theorem:DEMMM}
  Suppose that $A$ is a square matrix of size $n$ and $E$ is any elementary matrix of size $n$.  Then
  \[
    \detname{EA}=\detname{E}\detname{A}
  \]

  \begin{proof}
    The proof procedes in three parts, one for each type of elementary matrix, with each part very similar to the other two.

    First, let $B$ be the matrix obtained from $A$ by swapping rows $i$ and $\answer{j}$,
    \begin{align*}
      \detname{\elemswap{i}{j}A}
      &=\detname{B}&&\ref{theorem:EMDRO}\\
      &=\answer{-1}\cdot\detname{A}&&\ref{theorem:DRCS}\\
      &=\detname{\elemswap{i}{j}}\detname{A}&&\ref{theorem:DEM}
    \end{align*}

    Second, let $B$ be the matrix obtained from $A$ by multiplying row $\answer{i}$ by $\alpha$,
    \begin{align*}
      \detname{\elemmult{\alpha}{i}A}
      &=\detname{B}&&\ref{theorem:EMDRO}\\
      &=\alpha\detname{A}&&\ref{theorem:DRCM}\\
      &=\detname{\elemmult{\alpha}{i}}\detname{A}&&\ref{theorem:DEM}
    \end{align*}

    Third, let $B$ be the matrix obtained from $A$ by multiplying row $i$ by $\alpha$ and adding to row $j$,
    \begin{align*}
      \detname{\elemadd{\alpha}{i}{j}A}
      &=\detname{B}&&\ref{theorem:EMDRO}\\
      &=\detname{A}&&\ref{theorem:DRCMA}\\
      &=\detname{\elemadd{\alpha}{i}{j}}\detname{A}&&\ref{theorem:DEM}
    \end{align*}

    Since the desired result holds for each variety of elementary matrix individually, we are done.
  \end{proof}
\end{theorem}

\end{document}


