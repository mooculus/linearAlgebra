\documentclass{ximera}

\input{../../preamble.tex}

\title{Definition of the Determinant}

\begin{document}
\begin{abstract}
  Our definition of the determinant function is recursive, that is,
  the determinant of a large matrix is defined in terms of the
  determinant of smaller matrices.
\end{abstract}
\maketitle

\begin{definition}[SubMatrix]
  Suppose that $A$ is an $m\times n$ matrix.  Then the \dfn{submatrix}
  $\submatrix{A}{i}{j}$ is the $(m-1)\times (n-1)$ matrix obtained
  from $A$ by removing row $i$ and column $j$.
\end{definition}

\begin{example}[Some submatrices]
  For the matrix
  \[
    A=
    \begin{bmatrix}
      1 & -2 & 3 & 9\\
      4 & -2 & 0 & 1\\
      3 & 5 & 2 & 1
    \end{bmatrix}
  \]
  we have the submatrices
  \[
    \submatrix{A}{\answer{2}}{\answer{3}}
    =
    \begin{bmatrix}
      1 & -2 & 9\\
      3 & 5 & 1
    \end{bmatrix}
  \]
  and
  \[
    \submatrix{A}{\answer{3}}{\answer{1}}
    =
    \begin{bmatrix}
      -2 & 3 & 9\\
      -2 & 0 & 1
    \end{bmatrix}.
  \]
\end{example}

\begin{definition}[Determinant of a Matrix]
  Suppose $A$ is a square matrix.  Then its \dfn{determinant},
  $\detname{A}=\detbars{A}$, is an element of $\complexes$ defined
  recursively by:
\begin{enumerate}
\item If $A$ is a $1\times 1$ matrix, then $\detname{A}=\matrixentry{A}{11}$.
\item If $A$ is a matrix of size $n$ with $n\geq 2$, then
  \begin{align*}
    \detname{A}&=
                 \matrixentry{A}{11}\detname{\submatrix{A}{1}{1}}
                 -\matrixentry{A}{12}\detname{\submatrix{A}{1}{2}}
                 +\matrixentry{A}{13}\detname{\submatrix{A}{1}{3}}-\\
               &\quad \matrixentry{A}{14}\detname{\submatrix{A}{1}{4}}
                 +\cdots
                 +(-1)^{n+1}\matrixentry{A}{1n}\detname{\submatrix{A}{1}{n}}
  \end{align*}
\end{enumerate}
\end{definition}

So to compute the determinant of a $5\times 5$ matrix we must build 5
submatrices, each of size $4$.  To compute the determinants of each
the $4\times 4$ matrices we need to create 4 submatrices each, these
now of size $3$ and so on.  To compute the determinant of a
$10\times 10$ matrix would require computing the determinant of
$10!=10\times 9\times 8\times 7\times 6\times 5\times 4\times 3\times
2=3,628,800$ $1\times 1$ matrices.  Fortunately there are better ways.
However this does suggest an excellent computer programming exercise
to write a recursive procedure to compute a determinant.

Let us compute the determinant of a reasonably sized matrix by hand.

\begin{example}[Determinant of a $3\times 3$ matrix]
  Suppose that we have the $3\times 3$ matrix
  \[
    A=
    \begin{bmatrix}
      3 & 2 & -1\\
      4 & 1 & 6\\
      -3 & -1 & 2
    \end{bmatrix}
  \]

  Then
  \begin{align*}
    \detname{A}=\detbars{A}
    &=\begin{vmatrix}
      3 & 2 & -1\\
      4 & 1 & 6\\
      -3 & -1 & 2
    \end{vmatrix}\\
    &=
      3
      \begin{vmatrix}
        1 & 6\\
        -1 & 2
      \end{vmatrix}
             -2
             \begin{vmatrix}
               4 & 6\\
               -3 & 2
             \end{vmatrix}
                    +(-1)
                    \begin{vmatrix}
                      4 & 1\\
                      -3 & -1
                    \end{vmatrix}\\
    &=
      3\left(
      1\begin{vmatrix}
        2\\
      \end{vmatrix}
    -6\begin{vmatrix}
      -1
    \end{vmatrix}\right)
    -2\left(
    4\begin{vmatrix}
      2
    \end{vmatrix}
    -6\begin{vmatrix}
      -3
    \end{vmatrix}\right)
    -\left(
    4\begin{vmatrix}
      -1
    \end{vmatrix}
    -1\begin{vmatrix}
      -3
    \end{vmatrix}\right)\\
    &=
      3\left(1(2)-6(-1)\right)
      -2\left(4(2)-6(-3)\right)
      -\left(4(-1)-1(-3)\right)\\
    &=24-52+1\\
    &=-27
  \end{align*}
\end{example}

In practice it is a bit silly to decompose a $2\times 2$ matrix down
into a couple of $1\times 1$ matrices and then compute the exceedingly
easy determinant of these puny matrices.  So here is a theorem that
most people who use linear algebra often employ unconsciously.

\begin{theorem}[Determinant of Matrices of Size Two]
\label{theorem:DMST}

Suppose that $A=\begin{bmatrix}a&b\\c&d\end{bmatrix}$.  Then
$\detname{A}=ad-bc$.

\begin{proof}
  Applying \ref{definition:DM},
  \[
    \begin{vmatrix}
      a&b\\c&d
    \end{vmatrix}=
    \answer{a}\begin{vmatrix}d\end{vmatrix}-b\begin{vmatrix}c\end{vmatrix}=ad-bc
  \]
\end{proof}
\end{theorem}

Do you recall seeing the expression $ad-bc$ before?  (Hint: \ref{theorem:TTMI}.)

\end{document}
