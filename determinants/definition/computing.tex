\documentclass{ximera}

\input{../../preamble.tex}

\title{Computing Determinants}

\begin{document}
\begin{abstract}
  There are a variety of ways to compute the determinant.  We will
  establish first that we can choose to mimic our definition of the
  determinant, but by using matrix entries and submatrices based on a
  row other than the first one.
\end{abstract}
\maketitle

\begin{theorem}[Determinant Expansion about Rows]
\label{theorem:DER}

Suppose that $A$ is a square matrix of size $n$.  Then for $1\leq i\leq n$
\begin{align*}
  \detname{A}&=
               (-1)^{i+1}\matrixentry{A}{i1}\detname{\submatrix{A}{i}{1}}+
               (-1)^{i+2}\matrixentry{A}{i2}\detname{\submatrix{A}{i}{2}}\\
             &\quad+(-1)^{i+3}\matrixentry{A}{i3}\detname{\submatrix{A}{i}{3}}+
               \cdots+
               (-1)^{i+n}\matrixentry{A}{in}\detname{\submatrix{A}{i}{n}}
\end{align*}
which is known as \dfn{expansion} about row $i$.

\begin{proof}
  First, the statement of the theorem coincides with
  \ref{definition:DM} when $i=1$, so throughout, we need only consider
  $i>1$.

  Given the recursive definition of the determinant, it should be no
  surprise that we will use induction for this proof
  (\ref{technique:I}).  When $n=1$, there is nothing to prove since
  there is but one row.  When $n=2$, we just examine expansion about
  the second row,
  \begin{align*}
(-1)^{2+1}\matrixentry{A}{21}&\detname{\submatrix{A}{2}{1}}+
                               (-1)^{2+2}\matrixentry{A}{22}\detname{\submatrix{A}{2}{2}}\\
                             &=-\matrixentry{A}{21}\matrixentry{A}{12}+\matrixentry{A}{22}\matrixentry{A}{11}
                             &&\ref{definition:DM}\\
&=\matrixentry{A}{11}\matrixentry{A}{22}-\matrixentry{A}{12}\matrixentry{A}{21}\\
                             &=
                               \detname{A}&&\ref{theorem:DMST}\\
  \end{align*}
  
  So the theorem is true for matrices of size $n=1$ and $n=2$.  Now
  assume the result is true for all matrices of size $n-1$ as we
  derive an expression for expansion about row $i$ for a matrix of
  size $n$.  We will abuse our notation for a submatrix slightly, so
  $\submatrix{A}{i_1,i_2}{j_1,j_2}$ will denote the matrix formed by
  removing rows $i_1$ and $i_2$, along with removing columns $j_1$ and
  $j_2$.  Also, as we take a determinant of a submatrix, we will need
  to ``jump up'' the index of summation partway through as we ``skip
  over'' a missing column.  To do this smoothly we will set
  \[
    \epsilon_{\ell j}=
    \begin{cases}
      0 & \ell<j\\
      1 & \ell>j
    \end{cases}
  \]

  Now,
  \begin{align*}
    &\detname{A}\\
&\quad=
  \sum_{j=1}^{n}(-1)^{1+j}\matrixentry{A}{1j}\detname{\submatrix{A}{1}{j}}
&&\ref{definition:DM}\\
    &\quad=
      \sum_{j=1}^{n}(-1)^{1+j}\matrixentry{A}{1j}
      \sum_{\substack{1\leq\ell\leq n\\\ell\neq j}}
    (-1)^{i-1+\ell-\epsilon_{\ell j}}\matrixentry{A}{i\ell}\detname{\submatrix{A}{1,i}{j,\ell}}
    &&\text{Induction}\\
    &\quad=
      \sum_{j=1}^{n}\sum_{\substack{1\leq\ell\leq n\\\ell\neq j}}
    (-1)^{j+i+\ell-\epsilon_{\ell j}}
\matrixentry{A}{1j}\matrixentry{A}{i\ell}\detname{\submatrix{A}{1,i}{j,\ell}}
    &&\ref{property:DCN}\\
    &\quad=
      \sum_{\ell=1}^{n}\sum_{\substack{1\leq j\leq n\\j\neq\ell}}
    (-1)^{j+i+\ell-\epsilon_{\ell j}}
    \matrixentry{A}{1j}\matrixentry{A}{i\ell}\detname{\submatrix{A}{1,i}{j,\ell}}
&&\ref{property:CACN}\\
    &\quad=
      \sum_{\ell=1}^{n}(-1)^{i+\ell}\matrixentry{A}{i\ell}
      \sum_{\substack{1\leq j\leq n\\j\neq\ell}}
    (-1)^{j-\epsilon_{\ell j}}
    \matrixentry{A}{1j}\detname{\submatrix{A}{1,i}{j,\ell}}
    &&\ref{property:DCN}\\
    &\quad=
      \sum_{\ell=1}^{n}(-1)^{i+\ell}\matrixentry{A}{i\ell}
      \sum_{\substack{1\leq j\leq n\\j\neq\ell}}
    (-1)^{\epsilon_{\ell j}+j}
    \matrixentry{A}{1j}\detname{\submatrix{A}{i,1}{\ell,j}}
    &&\text{$2\epsilon_{\ell j}$ is even}\\
    &\quad=
      \sum_{\ell=1}^{n}(-1)^{i+\ell}\matrixentry{A}{i\ell}\detname{\submatrix{A}{i}{\ell}}
&&\ref{definition:DM}
  \end{align*}

\end{proof}
\end{theorem}

We can also obtain a formula that computes a determinant by expansion
about a column, but this will be simpler if we first prove a result
about the interplay of determinants and transposes.  Notice how the
following proof makes use of the ability to compute a determinant by
expanding about \textit{any} row.

\begin{theorem}[Determinant of the Transpose]
  \label{theorem:DT}
  Suppose that $A$ is a square matrix.  Then $\detname{\transpose{A}}=\detname{A}$.

  \begin{proof}
    With our definition of the determinant (\ref{definition:DM}) and
    theorems like \ref{theorem:DER}, using induction
    (\ref{technique:I}) is a natural approach to proving properties of
    determinants.  And so it is here.  Let $n$ be the size of the
    matrix $A$, and we will use induction on $n$.

    For $n=1$, the transpose of a matrix is identical to the original
    matrix, so vacuously, the determinants are equal.

    Now assume the result is true for matrices of size $n-1$.  Then,
    \begin{align*}
      \detname{\transpose{A}}
      &=\frac{1}{n}\sum_{i=1}^{n}\detname{\transpose{A}}\\
      &=
        \frac{1}{n}\sum_{i=1}^{n}\sum_{j=1}^{n}(-1)^{i+j}
        \matrixentry{\transpose{A}}{ij}\detname{\submatrix{\transpose{A}}{i}{j}}
      &&\ref{theorem:DER}\\
      &=
        \frac{1}{n}\sum_{i=1}^{n}\sum_{j=1}^{n}(-1)^{i+j}
        \matrixentry{A}{ji}\detname{\submatrix{\transpose{A}}{i}{j}}
      &&\ref{definition:TM}\\
      &=
        \frac{1}{n}\sum_{i=1}^{n}\sum_{j=1}^{n}(-1)^{i+j}
        \matrixentry{A}{ji}\detname{\transpose{\left(\submatrix{A}{j}{i}\right)}}
      &&\ref{definition:TM}\\
      &=
        \frac{1}{n}\sum_{i=1}^{n}\sum_{j=1}^{n}(-1)^{i+j}
        \matrixentry{A}{ji}\detname{\submatrix{A}{j}{i}}
      &&\text{Induction Hypothesis}\\
      &=
        \frac{1}{n}\sum_{j=1}^{n}\sum_{i=1}^{n}(-1)^{j+i}
        \matrixentry{A}{ji}\detname{\submatrix{A}{j}{i}}
      &&\ref{property:CACN}\\
      &=
        \frac{1}{n}\sum_{j=1}^{n}\detname{A}
      &&\ref{theorem:DER}\\
      &=\detname{A}
    \end{align*}
  \end{proof}
\end{theorem}

Now we can easily get the result that a determinant can be computed by
expansion about any column as well.

\begin{theorem}[Determinant Expansion about Columns]
  \label{theorem:DEC}
  Suppose that $A$ is a square matrix of size $n$.  Then for
  $1\leq j\leq n$
  \begin{align*}
  \detname{A}&=
               (-1)^{1+j}\matrixentry{A}{1j}\detname{\submatrix{A}{1}{j}}+
               (-1)^{2+j}\matrixentry{A}{2j}\detname{\submatrix{A}{2}{j}}\\
             &\quad+(-1)^{3+j}\matrixentry{A}{3j}\detname{\submatrix{A}{3}{j}}+
               \cdots+
               (-1)^{n+j}\matrixentry{A}{nj}\detname{\submatrix{A}{n}{j}}
  \end{align*}
  which is known as \dfn{expansion} about column $j$.

  \begin{proof}

    \begin{align*}
      \detname{A}
      &=
        \detname{\transpose{A}}&&\ref{theorem:DT}\\
      &=
        \sum_{i=1}^{n}(-1)^{j+i}\matrixentry{\transpose{A}}{ji}\detname{\submatrix{\transpose{A}}{j}{i}}
                               &&\ref{theorem:DER}\\
      &=
        \sum_{i=1}^{n}(-1)^{j+i}\matrixentry{\transpose{A}}{ji}\detname{\transpose{\left(\submatrix{A}{i}{j}\right)}}
                               &&\ref{definition:TM}\\
      &=
        \sum_{i=1}^{n}(-1)^{j+i}\matrixentry{\transpose{A}}{ji}\detname{\submatrix{A}{i}{j}}
                               &&\ref{theorem:DT}\\
      &=
        \sum_{i=1}^{n}(-1)^{i+j}\matrixentry{A}{ij}\detname{\submatrix{A}{i}{j}}
                               &&\ref{definition:TM}
    \end{align*}
  \end{proof}
\end{theorem}

That the determinant of an $n\times n$ matrix can be computed in $2n$
different (albeit similar) ways is nothing short of remarkable.  For
the doubters among us, we will do an example, computing a $4\times 4$
matrix in two different ways.

\begin{example}[Two computations, same determinant]
  Let
  \[
    A=
    \begin{bmatrix}
      -2 & 3 & 0 & 1\\
      9 & -2 & 0 & 1\\
      1 & 3 & -2 & -1\\
      4 & 1 & 2 & 6
    \end{bmatrix}
  \]

  Then expanding about the fourth row (\ref{theorem:DER} with $i=4$)
  yields,
  \begin{align*}
    \detbars{A}
    &=
      (4)(-1)^{4+1}
      \begin{vmatrix}
        \answer{3} & 0 & 1\\
        \answer{-2} & 0 & 1\\
        \answer{3} & -2 & -1
      \end{vmatrix}
                 +(1)(-1)^{4+2}
                 \begin{vmatrix}
                   -2 &  0 & 1\\
                   9 &  0 & 1\\
                   1 &  -2 & -1
                 \end{vmatrix}\\
    &\quad\quad+(2)(-1)^{4+3}
      \begin{vmatrix}
        -2 & 3 &  1\\
        9 & -2 &  1\\
        1 & 3  & -1
      \end{vmatrix}
                 +(6)(-1)^{4+4}
                 \begin{vmatrix}
                   -2 & 3 & 0 \\
                   9 & -2 & 0 \\
                   1 & 3 & -2
                 \end{vmatrix}\\
    &=
      (-4)(10)+(1)(-22)+(-2)(61)+6(46)=92
  \end{align*}

  Expanding about column 3 (\ref{theorem:DEC} with $j=3$) gives
  \begin{align*}
    \detbars{A}
    &=
      (0)(-1)^{1+3}
      \begin{vmatrix}
        9 & -2 & 1\\
        1 & 3 & -1\\
        4 & 1 & 6
      \end{vmatrix}
                +
                (0)(-1)^{2+3}
                \begin{vmatrix}
                  -2 & 3 & 1\\
                  1 & 3 & -1\\
                  4 & 1 & 6
                \end{vmatrix}
                          +\\
    &\quad\quad(-2)(-1)^{3+3}
      \begin{vmatrix}
        -2 & 3 & 1\\
        9 & -2 & 1\\
        4 & 1 & 6
      \end{vmatrix}
                +
                (2)(-1)^{4+3}
                \begin{vmatrix}
                  -2 & 3 & 1\\
                  9 & -2 & 1\\
                  1 & 3 & -1
                \end{vmatrix}\\
    &=0+0+(-2)(-107)+(-2)(61)=\answer{92}
  \end{align*}

  Notice how much easier the second computation was.  By choosing to
  expand about the third column, we have two entries that are zero, so
  two $3\times 3$ determinants need not be computed at all!
\end{example}

When a matrix has all zeros above (or below) the diagonal, exploiting
the zeros by expanding about the proper row or column makes computing
a determinant insanely easy.

\begin{example}[Determinant of an upper triangular matrix]
  Suppose that
  \[
    T=
    \begin{bmatrix}
      2 & 3 & -1 & 3 & 3\\
      0 & -1 & 5 & 2 & -1\\
      0 & 0 & 3 & 9 & 2\\
      0 & 0 & 0 & -1 & 3\\
      0 & 0 & 0 & 0 & 5
    \end{bmatrix}
  \]

  We will compute the determinant of this $5\times 5$ matrix by
  consistently expanding about the first column for each submatrix
  that arises and does not have a zero entry multiplying it.
  \begin{align*}
    \detname{T}&=
                 \begin{vmatrix}
                   2 & 3 & -1 & 3 & 3\\
                   0 & -1 & 5 & 2 & -1\\
                   0 & 0 & 3 & 9 & 2\\
                   0 & 0 & 0 & -1 & 3\\
                   0 & 0 & 0 & 0 & 5
                 \end{vmatrix}\\
               &=2(-1)^{1+1}
                 \begin{vmatrix}
                   -1 & 5 & 2 & -1\\
                   0 & 3 & 9 & 2\\
                   0 & 0 & -1 & 3\\
                   0 & 0 & 0 & 5
                 \end{vmatrix}\\
               &=2(-1)(-1)^{1+1}
                 \begin{vmatrix}
                   3 & 9 & 2\\
                   0 & -1 & 3\\
                   0 & 0 & 5
                 \end{vmatrix}\\
               &=2(-1)(3)(-1)^{1+1}
                 \begin{vmatrix}
                   -1 & 3\\
                   0 & 5
                 \end{vmatrix}\\
               &=2(-1)(3)(-1)(-1)^{1+1}
                 \begin{vmatrix}
                   5
                 \end{vmatrix}\\
               &=2(-1)(3)(-1)(5)=30
  \end{align*}
\end{example}

When you consult other texts in your study of determinants, you may
run into the terms ``minor'' and ``cofactor,'' especially in a
discussion centered on expansion about rows and columns.  We have
chosen not to make these definitions formally since we have been able
to get along without them.  However, informally, a \dfn{minor} is a
determinant of a submatrix, specifically
$\detname{\submatrix{A}{i}{j}}$ and is usually referenced as the minor
of $\matrixentry{A}{ij}$.  A \dfn{cofactor} is a signed minor,
specifically the cofactor of $\matrixentry{A}{ij}$ is
$(-1)^{i+j}\detname{\submatrix{A}{i}{j}}$.

\end{document}
