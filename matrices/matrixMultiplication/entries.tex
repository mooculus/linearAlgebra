\documentclass{ximera}

\input{../../preamble.tex}

\title{Matrix Multiplication, Entry-by-Entry}

\begin{document}
\begin{abstract}
  While certain ``natural'' properties of multiplication do not hold, many more do.
\end{abstract}
\maketitle

We need a theorem that provides an alternate means of multiplying two
matrices.  In many texts, this would be given as the
\textit{definition} of matrix multiplication.  We prefer to turn it
around and have the following formula as a consequence of our
definition.  It will prove useful for proofs of matrix equality, where
we need to examine products of matrices, entry-by-entry.

\begin{theorem}[Entries of Matrix Products]
  \label{theorem:EMP}
  Suppose $A$ is an $m\times n$ matrix and $B$ is an $n\times p$ matrix.
  Then for $1\leq i\leq m$, $1\leq j\leq p$, the individual entries of
  $AB$ are given by
  \begin{align*}
    \matrixentry{AB}{ij}
    &=
      \matrixentry{A}{i1}\matrixentry{B}{1j}+
      \matrixentry{A}{i2}\matrixentry{B}{2j}+
      \matrixentry{A}{i3}\matrixentry{B}{3j}+
      \cdots+
      \matrixentry{A}{in}\matrixentry{B}{nj}\\
    &=
      \sum_{k=1}^{n}\matrixentry{A}{ik}\matrixentry{B}{kj}
  \end{align*}
  
  \begin{proof}
    Let the vectors $\vectorlist{A}{n}$ denote the columns of $A$ and let the vectors $\vectorlist{B}{p}$ denote the columns of $B$.  Then for $1\leq i\leq m$, $1\leq j\leq p$,
    \begin{align*}
      \matrixentry{AB}{ij}
      &=\vectorentry{A\vect{B}_j}{i}&&\ref{definition:MM}\\
      &=\vectorentry{
        \vectorentry{\vect{B}_j}{1}\vect{A}_1+
        \vectorentry{\vect{B}_j}{2}\vect{A}_2+
        \cdots+
        \vectorentry{\vect{B}_j}{n}\vect{A}_n
        }{i}&&\ref{definition:MVP}\\
      &=
        \vectorentry{\vectorentry{\vect{B}_j}{1}\vect{A}_1}{i}+
        \vectorentry{\vectorentry{\vect{B}_j}{2}\vect{A}_2}{i}+
        \cdots+
        \vectorentry{\vectorentry{\vect{B}_j}{n}\vect{A}_n}{i}
                                    &&\ref{definition:CVA}\\
      &=
        \vectorentry{\vect{B}_j}{1}\vectorentry{\vect{A}_1}{i}+
        \vectorentry{\vect{B}_j}{2}\vectorentry{\vect{A}_2}{i}+
        \cdots+
        \vectorentry{\vect{B}_j}{n}\vectorentry{\vect{A}_n}{i}
                                    &&\ref{definition:CVSM}\\
      &=
        \matrixentry{B}{1j}\matrixentry{A}{i1}+
        \matrixentry{B}{2j}\matrixentry{A}{i2}+
        \cdots+
        \matrixentry{B}{nj}\matrixentry{A}{in}
                                    &&\ref{definition:M}\\
      &=
        \matrixentry{A}{i1}\matrixentry{B}{1j}+
        \matrixentry{A}{i2}\matrixentry{B}{2j}+
        \cdots+
        \matrixentry{A}{in}\matrixentry{B}{nj}
                                    &&\ref{property:CMCN}\\
      &=\sum_{k=1}^{n}\matrixentry{A}{ik}\matrixentry{B}{kj}
    \end{align*}
  \end{proof}
\end{theorem}

\begin{example}[Product of two matrices, entry-by-entry]
  
  Consider again the two matrices from \ref{example:PTM}
  \begin{align*}
    A=
    \begin{bmatrix}
      1 & 2 & -1 & 4 & 6\\
      0 & -4 & 1 & 2 & 3\\
      -5 & 1 & 2 & -3 & 4
    \end{bmatrix}
        &&
           B=
           \begin{bmatrix}
             1 & 6 & 2 & 1\\
             -1 & 4 & 3 & 2\\
             1 & 1 & 2 & 3\\
             6 & 4 & -1 & 2\\
             1 & -2 & 3 & 0
           \end{bmatrix} &
  \end{align*}
  
  Then suppose we just wanted the entry of $AB$ in the second row, third column:
  \begin{align*}
    \matrixentry{AB}{23} = \answer{-3}
  \end{align*}
  \begin{hint}
    \begin{align*}
      \matrixentry{AB}{23} =&
                              \matrixentry{A}{21}\matrixentry{B}{13}+
                              \matrixentry{A}{22}\matrixentry{B}{23}+
                              \matrixentry{A}{23}\matrixentry{B}{33}+
                              \matrixentry{A}{24}\matrixentry{B}{43}+
                              \matrixentry{A}{25}\matrixentry{B}{53}\\
      =&(0)(2)+(-4)(3)+(1)(2)+(2)(-1)+(3)(3)=-3
    \end{align*}
  \end{hint}
  
  Notice how there are 5 terms in the sum, since 5 is the common
  dimension of the two matrices (column count for $A$, row count for
  $B$).  In the conclusion of \ref{theorem:EMP}, it would be the index
  $k$ that would run from 1 to 5 in this computation.  Here is a bit
  more practice.

  The entry of third row, first column:
  \[
    \matrixentry{AB}{31} = \answer{-18}
  \]
  
  \begin{hint}
    \begin{align*}
      \matrixentry{AB}{31} = 
      =&
         \matrixentry{A}{31}\matrixentry{B}{11}+
         \matrixentry{A}{32}\matrixentry{B}{21}+
         \matrixentry{A}{33}\matrixentry{B}{31}+
         \matrixentry{A}{34}\matrixentry{B}{41}+
         \matrixentry{A}{35}\matrixentry{B}{51}\\
      =&(-5)(1)+(1)(-1)+(2)(1)+(-3)(6)+(4)(1)=-18
    \end{align*}
  \end{hint}

  To get some more practice on your own, complete the computation of
  the other 10 entries of this product.  Construct some other pairs of
  matrices (of compatible sizes) and compute their product two ways.
  First use \ref{definition:MM}.  Since linear combinations are
  straightforward for you now, this should be easy to do and to do
  correctly.  Then do it again, using \ref{theorem:EMP}.  Since this
  process may take some practice, use your first computation to check
  your work.
\end{example}

\ref{theorem:EMP} is the way many people compute matrix products by
hand.  It will also be very useful for the theorems we are going to
prove shortly.  However, the definition (\ref{definition:MM}) is
frequently the most useful for its connections with deeper ideas like
the null space and the upcoming column space.

\end{document}
