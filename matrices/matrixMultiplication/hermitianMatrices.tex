\documentclass{ximera}

\input{../../preamble.tex}

\title{Hermitian Matrices}

\begin{document}
\begin{abstract}
  The adjoint of a matrix has a basic property when employed in a matrix-vector product as part of an inner product.
\end{abstract}
\maketitle

You could even use the following result as a motivation for the definition of an adjoint.

\begin{theorem}[Adjoint and Inner Product]
  \label{theorem:AIP}

  Suppose that $A$ is an $m\times n$ matrix and
  $\vect{x}\in\complex{n}$, $\vect{y}\in\complex{m}$.  Then
  $\innerproduct{A\vect{x}}{\vect{y}}=\innerproduct{\vect{x}}{\adjoint{A}\vect{y}}$.

  \begin{proof}
    
    \begin{align*}
      \innerproduct{A\vect{x}}{\vect{y}}
      &=\transpose{\left(\conjugate{A\vect{x}}\right)}\vect{y}
      &&\ref{theorem:MMIP}\\
      &=\transpose{\left(\conjugate{A}\,\conjugate{\vect{x}}\right)}\vect{y}
      &&\ref{theorem:MMCC}\\
      &=\transpose{\conjugate{\vect{x}}}\transpose{\conjugate{A}}\vect{y}
      &&\ref{theorem:MMT}\\
      &=\transpose{\conjugate{\vect{x}}}\left(\adjoint{A}\vect{y}\right)
      &&\ref{definition:A}\\
      &=\innerproduct{\vect{x}}{\adjoint{A}\vect{y}}
      &&\ref{theorem:MMIP}
    \end{align*}
  \end{proof}
\end{theorem}

Sometimes a matrix is equal to its adjoint (\ref{definition:A}), and
these matrices have interesting properties.  One of the most common
situations where this occurs is when a matrix has only real number
entries.  Then we are simply talking about symmetric matrices
(\ref{definition:SYM}), so you can view this as a generalization of a
symmetric matrix.

\begin{definition}[Hermitian Matrix]
  The square matrix $A$ is \dfn{Hermitian} (or \dfn{self-adjoint}) if $A=\adjoint{A}$.
\end{definition}

Again, the set of real matrices that are Hermitian is exactly the set
of symmetric matrices.  In \ref{section:PEE} we will uncover some
amazing properties of Hermitian matrices, so when you get there, run
back here to remind yourself of this definition.  Further properties
will also appear in \ref{section:OD}.  Right now we prove a
fundamental result about Hermitian matrices, matrix vector products
and inner products.  As a characterization, this could be employed as
a definition of a Hermitian matrix and some authors take this
approach.

\begin{theorem}[Hermitian Matrices and Inner Products]
  \label{theorem:HMIP}

  Suppose that $A$ is a square matrix of size $n$.  Then $A$ is
  Hermitian if and only if
  $\innerproduct{A\vect{x}}{\vect{y}}=\innerproduct{\vect{x}}{A\vect{y}}$
  for all $\vect{x},\,\vect{y}\in\complex{n}$.

\begin{proof}
  ($\Rightarrow$) This is the ``easy half'' of the proof, and makes
  the rationale for a definition of Hermitian matrices most obvious.
  Assume $A$ is Hermitian,
  \begin{align*}
    \innerproduct{A\vect{x}}{\vect{y}}
    &=\innerproduct{\vect{x}}{\adjoint{A}\vect{y}}&&\ref{theorem:AIP}\\
    &=\innerproduct{\vect{x}}{A\vect{y}}&&\ref{definition:HM}\\
  \end{align*}

  ($\Leftarrow$) This ``half'' will take a bit more work.  Assume that
  $\innerproduct{A\vect{x}}{\vect{y}}=\innerproduct{\vect{x}}{A\vect{y}}$
  for all $\vect{x},\,\vect{y}\in\complex{n}$.  We show that
  $A=\adjoint{A}$ by establishing that $A\vect{x}=\adjoint{A}\vect{x}$
  for all $\vect{x}$, so we can then apply \ref{theorem:EMMVP}.  With
  only this much motivation, consider the inner product for any
  $\vect{x}\in\complex{n}$.
  \begin{align*}
    \innerproduct{A\vect{x}-\adjoint{A}\vect{x}}{A\vect{x}-\adjoint{A}\vect{x}}
    &=
      \innerproduct{A\vect{x}-\adjoint{A}\vect{x}}{A\vect{x}}-
      \innerproduct{A\vect{x}-\adjoint{A}\vect{x}}{\adjoint{A}\vect{x}}
    &&\ref{theorem:IPVA}\\
    &=
      \innerproduct{A\vect{x}-\adjoint{A}\vect{x}}{A\vect{x}}-
      \innerproduct{A\left(A\vect{x}-\adjoint{A}\vect{x}\right)}{\vect{x}}
    &&\ref{theorem:AIP}\\
    &=
      \innerproduct{A\vect{x}-\adjoint{A}\vect{x}}{A\vect{x}}-
      \innerproduct{A\vect{x}-\adjoint{A}\vect{x}}{A\vect{x}}
    &&\text{Hypothesis}\\
    &
      =0
    &&\ref{property:AICN}
  \end{align*}

  Because this first inner product equals zero, and has the same
  vector in each argument ($A\vect{x}-\adjoint{A}\vect{x}$),
  \ref{theorem:PIP} gives the conclusion that
  $A\vect{x}-\adjoint{A}\vect{x}=\zerovector$.  With
  $A\vect{x}=\adjoint{A}\vect{x}$ for all $\vect{x}\in\complex{n}$,
  \ref{theorem:EMMVP} says $A=\adjoint{A}$, which is the defining
  property of a Hermitian matrix (\ref{definition:HM}).
\end{proof}
\end{theorem}

So, informally, Hermitian matrices are those that can be tossed around from one side of an inner product to the other with reckless abandon.  We will see later what this buys us.

\begin{question}
  Suppose that $A$ is an $n\times n$ matrix.  Is $\adjoint{A}A$ a Hermitian matrix?
  
  \begin{multipleChoice}
    \choice[correct]{Yes}
    \choice{No}
  \end{multipleChoice}
\end{question}

\begin{question}
  Suppose that $A$ is an $n\times n$ matrix.  Is $A\adjoint{A}$ a Hermitian matrix?
  
  \begin{multipleChoice}
    \choice[correct]{Yes}
    \choice{No}
  \end{multipleChoice}
\end{question}

\begin{question}
  Suppose that $A$ is an $n\times n$ matrix.

  For all 
  $\vect{x}\in\complex{n}$, $\vect{y}\in\complex{n}$, the expression
  \[
    \innerproduct{A\adjoint{A}\vect{x}}{\vect{y}}
  \] is equal to
  \begin{multipleChoice}
    \choice{$\innerproduct{\vect{x}}{\adjoint{A}A\vect{y}}$.}
    \choice[correct]{$\innerproduct{\vect{x}}{A\adjoint{A}\vect{y}}$.}
    \choice{$\innerproduct{\vect{y}}{\adjoint{A}A\vect{x}}$.}
    \choice{$\innerproduct{\vect{y}}{A\adjoint{A}\vect{x}}$.}
  \end{multipleChoice}
\end{question}

\end{document}
