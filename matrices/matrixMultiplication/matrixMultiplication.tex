\documentclass{ximera}

\input{../../preamble.tex}

\title{Matrix Multiplication}

\begin{document}
\begin{abstract}
  We now define how to multiply two matrices together.  
\end{abstract}
\maketitle

Stop for a minute and think about how you might define the product of
two matrices.

Many books would present this definition much earlier in the course.
However, we have taken great care to delay it as long as possible and
to present as many ideas as practical based mostly on the notion of
linear combinations.  Towards the conclusion of the course, or when
you perhaps take a second course in linear algebra, you may be in a
position to appreciate the reasons for this.  For now, understand that
matrix multiplication is a central definition and perhaps you will
appreciate its importance more by having saved it for later.

\begin{definition}[Matrix Multiplication]
  Suppose $A$ is an $m\times n$ matrix and $\vectorlist{B}{p}$ are the
  columns of an $n\times p$ matrix $B$.  Then the \dfn{matrix product}
  of $A$ with $B$ is the $m\times p$ matrix where column $i$ is the
  matrix-vector product $A\vect{B}_i$.  Symbolically,
  \[
    AB=A\matrixcolumns{B}{p}=\left[A\vect{B}_1|A\vect{B}_2|A\vect{B}_3|\ldots|A\vect{B}_p\right].
  \]
\end{definition}

\begin{example}[Product of two matrices]
Set
\begin{align*}
A=
\begin{bmatrix}
1 & 2 & -1 & 4 & 6\\
0 & -4 & 1 & 2 & 3\\
-5 & 1 & 2 & -3 & 4
\end{bmatrix}
&&
B=
\begin{bmatrix}
1 & 6 & 2 & 1\\
-1 & 4 & 3 & 2\\
1 & 1 & 2 & 3\\
6 & 4 & -1 & 2\\
1 & -2 & 3 & 0
\end{bmatrix} &
\end{align*}

Then
\[
AB=\begin{bmatrix}
\answer{28} & 17 & 20 & 10\\
\answer{20} & -13 & -3 & -1\\
\answer{-18} & -44 & 12 & -3
\end{bmatrix}.
\]

\begin{hint}
  \[
    AB=\left[
      A\colvector{1\\-1\\1\\6\\1}
      \left\lvert A\colvector{6\\4\\1\\4\\-2}\right.
      \left\lvert A\colvector{2\\3\\2\\-1\\3}\right.
      \left\lvert A\colvector{1\\2\\3\\2\\0}\right.
    \right]
    =
    \begin{bmatrix}
      28 & 17 & 20 & 10\\
      20 & -13 & -3 & -1\\
      -18 & -44 & 12 & -3
    \end{bmatrix}.
  \]
\end{hint}

\end{example}

Is this the definition of matrix multiplication you expected?  Perhaps
our previous operations for matrices caused you to think that we might
multiply two matrices of the \textit{same} size,
\textit{entry-by-entry}?  Notice that our current definition uses
matrices of different sizes (though the number of columns in the first
must equal the number of rows in the second), and the result is of a
third size.  Notice too in the previous example that we cannot even
consider the product $BA$, since the sizes of the two matrices in this
order are not right.

\begin{example}%[Matrix multiplication is not commutative]

  Set
  \begin{align*}
    A=
    \begin{bmatrix}
      1 & 3\\
      -1 & 2
    \end{bmatrix}
         &&
            B=
            \begin{bmatrix}
              4&0\\
              5&1
            \end{bmatrix}.
  \end{align*}
  
  Then we have two square, $2\times 2$ matrices, so \ref{definition:MM} allows us to multiply them in either order.  We find
  \[
    AB=
    \begin{bmatrix}
      \answer{19} & 3\\
      6 & 2
    \end{bmatrix}
  \]
  and
  \[
    BA=
    \begin{bmatrix}
      \answer{4} & 12\\
      4 & 17
    \end{bmatrix}.
  \]
  and therefore $AB\neq BA$.  Not even close.  It should not be hard
  for you to construct other pairs of matrices that do not commute
  (try a couple of $3\times 3$'s).

  Can you find a pair of non-identical matrices that \textit{do}
  commute?

\end{example}

And it gets weirder than $AB \neq BA$.  Many of your old ideas about
``multiplication'' will not apply to matrix multiplication, but some
still will.  So make no assumptions, and do not do anything until you
have a theorem that says you can.  Even if the sizes are right, matrix
multiplication is not commutative---order matters.

\end{document}
