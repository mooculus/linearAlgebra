\documentclass{ximera}

\input{../../preamble.tex}

\title{Matrix Equality, Addition, Scalar Multiplication}

\begin{document}
\begin{abstract}
  We begin with a definition of a totally general set of matrices, and see where that takes us.
\end{abstract}
\maketitle

\begin{definition}[Vector Space of $m\times n$ Matrices]
  The vector space $M_{mn}$ is the set of all $m\times n$ matrices with entries from the set of complex numbers.
\end{definition}

Just as we made, and used, a careful definition of equality for column
vectors, so too, we have precise definitions for matrices.

\begin{definition}[Matrix Equality]
  The $m\times n$ matrices $A$ and $B$ are \dfn{equal}, written $A=B$
  provided $\matrixentry{A}{ij}=\matrixentry{B}{ij}$ for all
  $1\leq i\leq m$, $1\leq j\leq n$.
\end{definition}

So equality of matrices translates to the equality of complex numbers,
on an entry-by-entry basis.  Notice that we now have yet another
definition that uses the symbol ``='' for shorthand.  Whenever a
theorem has a conclusion saying two matrices are equal (think about
your objects), we will consider appealing to this definition as a way
of formulating the top-level structure of the proof.

We will now define two operations on the set $M_{mn}$.  Again, we will
overload a symbol (`+') and a convention (juxtaposition for scalar
multiplication).

\begin{definition}[Matrix Addition]
  Given the $m\times n$ matrices $A$ and $B$, define the \dfn{sum} of
  $A$ and $B$ as an $m\times n$ matrix, written $A+B$, according to
  \begin{align*}
    \matrixentry{A+B}{ij}&=\matrixentry{A}{ij}+\matrixentry{B}{ij}
    &&1\leq i\leq m,\,1\leq j\leq n
  \end{align*}
\end{definition}

So matrix addition takes two matrices of the same size and combines
them (in a natural way!) to create a new matrix of the same size.
Perhaps this is the ``obvious'' thing to do, but it does not relieve
us from the obligation to state it carefully.

\begin{example}[Addition of two matrices in $M_{23}$]
  If
  \begin{align*}
    A=
    \begin{bmatrix}
      2&-3&4\\
      1&0&-7
    \end{bmatrix}
       &&
          B=
          \begin{bmatrix}
            6&2&-4\\
            3&5&2
          \end{bmatrix}
  \end{align*}
  then
  \begin{align*}
    A+B&=\begin{bmatrix}
      \answer{8}&\answer{-1}&0\\
      \answer{4}&5&-5
    \end{bmatrix}
  \end{align*}

     \begin{hint}
     This can be computed as
     \begin{align*}
       A+B&=
            \begin{bmatrix}
              2&-3&4\\
              1&0&-7
            \end{bmatrix}
                   +
                   \begin{bmatrix}
                     6&2&-4\\
                     3&5&2
                   \end{bmatrix}\\
          &=
            \begin{bmatrix}
              2+6&-3+2&4+(-4)\\
              1+3&0+5&-7+2
            \end{bmatrix}
                       =\begin{bmatrix}
                         8&-1&0\\
                         4&5&-5
                       \end{bmatrix}
     \end{align*}
   \end{hint}
\end{example}

Our second operation takes two objects of different types,
specifically a number and a matrix, and combines them to create
another matrix.  As with vectors, in this context we call a number a
\dfn{scalar} in order to emphasize that it is not a matrix.

\begin{definition}[Matrix Scalar Multiplication]
Given the $m\times n$ matrix $A$
and the scalar $\alpha\in\complexes$, the \dfn{scalar multiple} of $A$ is an $m\times n$ matrix, written $\alpha A$ and defined according to
\begin{align*}
\matrixentry{\alpha A}{ij}&=\alpha\matrixentry{A}{ij}&&
\quad 1\leq i\leq m,\,1\leq j\leq n
\end{align*}
\end{definition}

Notice again that we have yet another kind of multiplication, and it is again written putting two symbols side-by-side.  Computationally, scalar matrix multiplication is very easy.

\begin{example}[Scalar multiplication in $M_{32}$]
  If
  \[
    A=
    \begin{bmatrix}
      2&8\\
      -3&5\\0&1
    \end{bmatrix}
  \]
  and $\alpha=7$, then
    \[
      \alpha A=
      \begin{bmatrix}\answer{14}&56\\-21&\answer{35}\\0&7\end{bmatrix}
    \]
    \begin{hint}
      This computation could be done via
      \[
        \alpha A=
        7\begin{bmatrix}2&8\\-3&5\\0&1\end{bmatrix}=
        \begin{bmatrix}7(2)&7(8)\\7(-3)&7(5)\\7(0)&7(1)\end{bmatrix}=
        \begin{bmatrix}14&56\\-21&35\\0&7\end{bmatrix}
      \]
    \end{hint}
\end{example}

\end{document}
