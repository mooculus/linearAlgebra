\documentclass{ximera}

\input{../../preamble.tex}

\title{Adjoint of a Matrix}

\begin{document}
\begin{abstract}
  The combination of transposing and conjugating a matrix is an especially important operation.
\end{abstract}
\maketitle

\begin{definition}[Adjoint]
If $A$ is a matrix, then its \dfn{adjoint} is
$\adjoint{A}=\transpose{\left(\conjugate{A}\right)}$.
\end{definition}

You will see the adjoint written elsewhere variously as $A^H$,
$A^\ast$ or $A^\dagger$.

\begin{question}
  Does it matter if we conjugate and then transposed, or transposed
  and then conjugated?

  \begin{multipleChoice}
    \choice[correct]{No}
    \choice{Yes}
  \end{multipleChoice}

  \begin{feedback}[correct]
    By \ref{theorem:MCT}, it does not really matter if we conjugate
    and then transpose, or transpose and then conjugate.
  \end{feedback}

\end{question}

\begin{theorem}[Adjoint and Matrix Addition]
\label{theorem:AMA}

Suppose $A$ and $B$ are matrices of the same size.  Then $\adjoint{\left(A+B\right)}=\adjoint{A}+\adjoint{B}$.

\begin{proof}

  \begin{align*}
    \adjoint{\left(A+B\right)}
    &=\transpose{\left(\conjugate{A+B}\right)}
    &&\ref{definition:A}\\
    &=\transpose{\left(\conjugate{A}+\conjugate{B}\right)}
    &&\ref{theorem:CRMA}\\
    &=\transpose{\left(\conjugate{A}\right)}+\transpose{\left(\conjugate{B}\right)}
    &&\ref{theorem:TMA}\\
    &=\adjoint{A}+\adjoint{B}
    &&\ref{definition:A}
  \end{align*}
\end{proof}
\end{theorem}

\begin{theorem}[Adjoint and Matrix Scalar Multiplication]
\label{theorem:AMSM}

Suppose $\alpha\in\complexes$ is a scalar and $A$ is a matrix.  Then $\adjoint{\left(\alpha A\right)}=\conjugate{\alpha}\adjoint{A}$.

\begin{proof}

  \begin{align*}
    \adjoint{\left(\alpha A\right)}
    &=\transpose{\left(\conjugate{\alpha A}\right)}
    &&\ref{definition:A}\\
    &=\transpose{\left(\conjugate{\alpha}\conjugate{A}\right)}
    &&\ref{theorem:CRMSM}\\
    &=\conjugate{\alpha}\transpose{\left(\conjugate{A}\right)}
    &&\ref{theorem:TMSM}\\
    &=\conjugate{\alpha}\adjoint{A}
    &&\ref{definition:A}
\end{align*}

\end{proof}
\end{theorem}

\begin{theorem}[Adjoint of an Adjoint]
\label{theorem:AA}

Suppose that $A$ is a matrix.  Then $\adjoint{\left(\adjoint{A}\right)}=A$.

\begin{proof}
  
  \begin{align*}
    \adjoint{\left(\adjoint{A}\right)}
    &=\transpose{\left(\conjugate{\left(\adjoint{A}\right)}\right)}
    &&\ref{definition:A}\\
    &=\conjugate{\left(\transpose{\left(\adjoint{A}\right)}\right)}
    &&\ref{theorem:MCT}\\
    &=\conjugate{\left(\transpose{\left(\transpose{\left(\conjugate{A}\right)}\right)}\right)}
    &&\ref{definition:A}\\
    &=\conjugate{\left(\conjugate{A}\right)}
    &&\ref{theorem:TT}\\
    &=A
    &&\ref{theorem:CCM}
  \end{align*}
\end{proof}
\end{theorem}

Take note of how the theorems in this section, while simple, build on
earlier theorems and definitions and never descend to the level of
entry-by-entry proofs based on \ref{definition:ME}.  In other words,
the equal signs that appear in the previous proofs are equalities of
matrices, not scalars (which is the opposite of a proof like that of
\ref{theorem:TMA}).

\end{document}