\documentclass{ximera}

\input{../../preamble.tex}

\title{Transposes and Symmetric Matrices}

\begin{document}
\begin{abstract}
  Informally, to transpose a matrix is to build a new matrix by swapping its rows and columns.
\end{abstract}
\maketitle


\begin{definition}[Transpose of a Matrix]

  Given an $m\times n$ matrix $A$, its \dfn{transpose} is the $n\times m$ matrix $\transpose{A}$ given by
  \[
    \matrixentry{\transpose{A}}{ij}=\matrixentry{A}{ji},\quad 1\leq i\leq n,\,1\leq j\leq m.
  \]
\end{definition}

\begin{example}[Transpose of a $3\times 4$ matrix]

Suppose
\[
D=
\begin{bmatrix}
3&7&2&-3\\
-1&4&2&8\\
0&3&-2&5
\end{bmatrix}.
\]

We could formulate the transpose, entry-by-entry, using the
definition.  But it is easier to just systematically rewrite rows as
columns (or vice-versa).  The form of the definition given will be
more useful in proofs.  So we have
\[
  \transpose{D}=
  \begin{bmatrix}
    \answer{3}&\answer{-1}&0\\
    \answer{7}&\answer{4}&3\\
    2&2&-2\\
    -3&8&5
  \end{bmatrix}
\]

\end{example}

It will sometimes happen that a matrix is equal to its transpose.  In
this case, we will call a matrix \dfn{symmetric}.  These matrices
occur naturally in certain situations, and also have some nice
properties, so it is worth stating the definition carefully.
Informally a matrix is symmetric if we can ``flip'' it about the main
diagonal (upper-left corner, running down to the lower-right corner)
and have it look unchanged.

\begin{definition}[Symmetric Matrix]
The matrix $A$ is \dfn{symmetric} if $A=\transpose{A}$.
\end{definition}

\begin{example}[A $5\times 5$ matrix]

  The matrix
  \[
    E=
    \begin{bmatrix}
      2&3&-9&5&7\\
      3&1&6&-2&-3\\
      -9&6&0&-1&9\\
      5&-2&-1&4&-8\\
      7&-3&9&-8&-3
    \end{bmatrix}
  \]
  is\ldots
  \begin{multipleChoice}
    \choice[correct]{symmetric.}
    \choice{not symmetric.}
  \end{multipleChoice}
\end{example}

\begin{example}[Another $5\times 5$ matrix]

  The matrix
  \[
    F=
    \begin{bmatrix}
      2&4&8&4&6\\
      3&2&5&1&3\\
      1&2&0&1&5\\
      1&2&1&3&3\\
      6&3&5&3&3
    \end{bmatrix}
  \]
  is\ldots
  \begin{multipleChoice}
    \choice{symmetric.}
    \choice[correct]{not symmetric.}
  \end{multipleChoice}
\end{example}

You might have noticed that \ref{definition:SYM} did not specify the
size of the matrix $A$, as has been our custom.  That is because it
was not necessary.  An alternative would have been to state the
definition just for square matrices, but this is the substance of the
next proof.

\begin{theorem}[Symmetric Matrices are Square]
  \label{theorem:SMS}
  Suppose that $A$ is a symmetric matrix.  Then $A$ is square.
  
  \begin{proof}
    We start by specifying $A$'s size, without assuming it is square,
    since we are trying to \textit{prove} that, so we cannot also
    assume it.  Suppose $A$ is an $m\times n$ matrix.  Because $A$ is
    symmetric, we know by \ref{definition:SYM} that $A=\transpose{A}$.
    So, in particular, \ref{definition:ME} requires that $A$ and
    $\transpose{A}$ must have the same size.  The size of
    $\transpose{A}$ is $n\times m$.  Because $A$ has $m$ rows and
    $\transpose{A}$ has $n$ rows, we conclude that $m=n$, and hence
    $A$ must be square by \ref{definition:SQM}.
  \end{proof}
\end{theorem}

We finish this section with three easy theorems, but they illustrate the interplay of our three new operations, our new notation, and the techniques used to prove matrix equalities.

\begin{theorem}[Transpose and Matrix Addition]
  \label{theorem:TMA}

  Suppose that $A$ and $B$ are $m\times n$ matrices.  Then  $\transpose{(A+B)}=\transpose{A}+\transpose{B}$.

  \begin{proof}
    The statement to be proved is an equality of matrices, so we work
    entry-by-entry and use \ref{definition:ME}.  Think carefully about
    the objects involved here, and the many uses of the plus sign.
    Realize too that while $A$ and $B$ are $m\times n$ matrices, the
    conclusion is a statement about the equality of two $n\times m$
    matrices.  So we begin with: for $1\leq i\leq n$, $1\leq j\leq m$,
    \begin{align*}
      \matrixentry{\transpose{(A+B)}}{ij}
      &=\matrixentry{A+B}{ji}\\ %&&\ref{definition:TM}\\
      &=\matrixentry{A}{ji}+\matrixentry{B}{ji}\\ %&&\ref{definition:MA}\\
      &=\matrixentry{\transpose{A}}{ij}+\matrixentry{\transpose{B}}{ij}\\ %&&\ref{definition:TM}\\
      &=\matrixentry{\transpose{A}+\transpose{B}}{ij}\\ %&&\ref{definition:MA}
    \end{align*}
    
    Since the matrices $\transpose{(A+B)}$ and
    $\transpose{A}+\transpose{B}$ agree at each entry,
    the two matrices are equal.
\end{proof}
\end{theorem}

\begin{theorem}[Transpose and Matrix Scalar Multiplication]
\label{theorem:TMSM}

Suppose that $\alpha\in\complexes$ and $A$ is an $m\times n$ matrix.  Then $\transpose{(\alpha A)}=\alpha\transpose{A}$.

\begin{proof}
  The statement  to be proved is an equality of matrices, so we work entry-by-entry and use \ref{definition:ME}.  Notice that the desired equality is of $n\times m$ matrices, and think carefully about the objects involved here, plus the many uses of juxtaposition.  For $1\leq i\leq m$, $1\leq j\leq n$,
  \begin{align*}
    \matrixentry{\transpose{(\alpha A)}}{ji}&=
                                              \matrixentry{\alpha A}{ij}\\ %&&\ref{definition:TM}\\
                                            &=\alpha\matrixentry{A}{ij}\\ %&&\ref{definition:MSM}\\
                                            &=\alpha\matrixentry{\transpose{A}}{ji}\\ %&&\ref{definition:TM}\\
                                            &=\matrixentry{\alpha\transpose{A}}{ji}\\ %&&\ref{definition:MSM}
  \end{align*}

  Since the matrices $\transpose{(\alpha A)}$ and $\alpha\transpose{A}$ agree at each entry, \ref{definition:ME} tells us the two matrices are equal.
  
\end{proof}
\end{theorem}

\begin{theorem}[Transpose of a Transpose]
  \label{theorem:TT}
  
  Suppose that $A$ is an $m\times n$ matrix.  Then $\transpose{\left(\transpose{A}\right)}=A$.

  \begin{proof}
    We again want to prove an equality of matrices, so we work entry-by-entry and use \ref{definition:ME}.  For $1\leq i\leq m$, $1\leq j\leq n$,
    \begin{align*}
      \matrixentry{\transpose{\left(\transpose{A}\right)}}{ij}
      &=\matrixentry{\transpose{A}}{ji}\\ %&&\ref{definition:TM}\\
      &=\matrixentry{A}{ij}\\ %&&\ref{definition:TM}
    \end{align*}
    
    Since the matrices $\transpose{\left(\transpose{A}\right)}$ and $A$ agree at each entry, \ref{definition:ME} tells us the two matrices are equal.
    
  \end{proof}
\end{theorem}

\end{document}
