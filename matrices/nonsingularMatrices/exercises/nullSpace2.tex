\documentclass{ximera}
\author{Rob Beezer}
\input{../../../preamble.tex}

\begin{document}

\begin{exercise}
  Let $A$ be the coefficient matrix of the system of equations below.
  \begin{align*}
    -x_1+5x_2&=-8\\
    -2x_1+5x_2+5x_3+2x_4&=9\\
    -3x_1-x_2+3x_3+x_4&=3\\
    7x_1+6x_2+5x_3+x_4&=30
  \end{align*}
  Is $A$ nonsingular or singular?
  The coefficient matrix row reduces to
  \begin{align*}
    \begin{bmatrix}
      -1 & 5 & 0 & 0  \\
      -2 & 5 & 5 & 2  \\
      -3 & -1 & 3 & 1  \\
      7 & 6 & 5 & 1
    \end{bmatrix}
         &\rref
           \begin{bmatrix}
             \leading{1} & 0 & 0 & 0  \\
             0 & \leading{1} & 0 & 0  \\
             0 & 0 & \leading{1} & 0  \\
             0 & 0 & 0 & \leading{1}
           \end{bmatrix}
  \end{align*}
  What can we infer about the solution set for the system based only on what you have learned about $A$ being singular or nonsingular?  The null space of $A$ is
  \begin{multipleChoice}
    \choice{the empty set}
    \choice[correct]{the set containing the zero vector}
    \choice{a set containing infinitely many vectors}
  \end{multipleChoice}

  \begin{feedback}[correct]
    Since the row-reduced version of the coefficient matrix is the
    $4\times 4$ identity matrix, $I_4$ (\ref{definition:IM}
    by\ref{theorem:NMRRI}, we know the coefficient matrix is
    nonsingular.  According to \ref{theorem:NMUS} we know that the
    system is guaranteed to have a unique solution, based only on the
    extra information that the coefficient matrix is nonsingular.
  \end{feedback}
\end{exercise}

\end{document}



%%% Local Variables:
%%% mode: latex
%%% TeX-master: t
%%% End:
