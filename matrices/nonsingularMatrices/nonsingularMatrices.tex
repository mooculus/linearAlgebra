\documentclass{ximera}
\author{Rob Beezer}
\input{../../preamble.tex}

\title{Nonsingular Matrices}

\begin{document}
\begin{abstract}
  We specialize further and consider matrices with equal numbers of
  rows and columns, which when considered as coefficient matrices lead
  to systems with equal numbers of equations and variables.
\end{abstract}
\maketitle

Our theorems will now establish connections between systems of
equations (homogeneous or otherwise), augmented matrices representing
those systems, coefficient matrices, constant vectors, the reduced
row-echelon form of matrices (augmented and coefficient) and solution
sets.  Be very careful in your reading, writing and speaking about
systems of equations, matrices and sets of vectors.  A system of
equations is not a matrix, a matrix is not a solution set, and a
solution set is not a system of equations.

\begin{definition}[Square Matrix]
  A matrix with $m$ rows and $n$ columns is \dfn{square} if $m=n$.  In
  this case, we say the matrix has \dfn{size} $n$.
\end{definition}

\begin{warning}
  Despite the fact tha squares are rectangles, sometimes people will
  use \dfn{``rectangular matrix''} to refer to a non-square matrix.
\end{warning}

We can now present one of the central definitions of linear algebra.

\begin{definition}[Nonsingular Matrix]
  Suppose $A$ is a square matrix.  Suppose further that the solution
  set to the homogeneous linear system of equations
  $\linearsystem{A}{\zerovector}$ is $\set{\zerovector}$, in other
  words, the system has \textit{only} the trivial solution.  Then we
  say that $A$ is a \dfn{nonsingular} matrix.  Otherwise we say $A$ is
  a \dfn{singular} matrix.
\end{definition}

We can investigate whether any square matrix is nonsingular or not, no
matter if the matrix is derived somehow from a system of equations or
if it is simply a matrix.  The definition says that to perform this
investigation we must construct a very specific system of equations
(homogeneous, with the matrix as the coefficient matrix) and look at
its solution set.  We will have theorems in this section that connect
nonsingular matrices with systems of equations, creating more
opportunities for confusion.  Convince yourself now of two
observations, (1) we can decide nonsingularity for any square matrix,
and (2) the determination of nonsingularity involves the solution set
for a certain homogeneous system of equations.

Notice that it makes no sense to call a system of equations
nonsingular (the term does not apply to a system of equations), nor
does it make any sense to call a $5\times 7$ matrix singular (the
matrix is not square).

\begin{example}
The matrix
\[
  A = \begin{bmatrix}
    1 & -1 & 2 \\
    2 & 1 & 1 \\
    1 & 1 & 0  
  \end{bmatrix}
\]
row reduces to
\[
  \begin{bmatrix}
    \leading{1} & 0 & 1 \\
    0 & \leading{1} & -1 \\
    0 & 0 & 0 
  \end{bmatrix}.
\]
Consequently, the $3\times 3$ matrix $A$ is
\begin{multipleChoice}
  \choice[correct]{singular.}
  \choice{nonsingular.}
  \choice{possibly singular or nonsingular.}
\end{multipleChoice}

\begin{feedback}[correct]
  The matrix $A$ is a singular matrix since there are nontrivial
  solutions to the homogeneous system
  \begin{align*}
    x_1 -x_2 +2x_3 & = 0\\
    2x_1+ x_2 + x_3 & =0 \\
    x_1 + x_2 & =0.
  \end{align*}
\end{feedback}
\end{example}

\begin{example}
  Consider the matrix
  \[
    B = \begin{bmatrix}
      -7&-6&- 12\\
      5&5&7 \\
      1&0&4
    \end{bmatrix},
  \]
  which row-reduces to
  \[
    \begin{bmatrix}
      \leading{1}&0&0 \\
      0&\leading{1}&0 \\
      0&0&\leading{1}
    \end{bmatrix},
  \]
  and so the matrix $B$ is 
  \begin{multipleChoice}
    \choice{singular.}
    \choice[correct]{nonsingular.}
    \choice{possibly singular or nonsingular.}
  \end{multipleChoice}
  \begin{feedback}[correct]
    Exactly!  The matrix $B$ is nonsingular because the homogeneous system
    \begin{align*}
      -7x_1 -6 x_2 - 12x_3 &=0\\
      5x_1  + 5x_2 + 7x_3 &=0\\
      x_1 +4x_3 &=0
    \end{align*}
    has exactly one solution.
  \end{feedback}
\end{example}

\begin{example}
  Notice that we will not discuss whether
  \[
    \begin{bmatrix}
      -7&-6&- 12&-33\\
      5&5&7&24\\
      1&0&4&5
    \end{bmatrix}
  \]
  singular or nonsingular matrix since the matrix is
  \wordChoice{\choice[correct]{not square}\choice{square}}.
\end{example}


\end{document}

