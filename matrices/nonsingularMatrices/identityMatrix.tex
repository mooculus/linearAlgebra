\documentclass{ximera}
\author{Rob Beezer}
\input{../../preamble.tex}

\title{Identity matrix}

\begin{document}
\begin{abstract}
  Row reduction makes it easy to recognize a nonsingular matrix.
\end{abstract}
\maketitle

\begin{definition}[Identity Matrix]
The $m\times m$ \dfn{identity matrix}, $I_m$, is defined by
\begin{align*}
  \matrixentry{I_m}{ij}&=
                         \begin{cases}
                           1 & i=j\\
                           0 & i\neq j
                         \end{cases}
                               \quad\quad
                               1\leq i,\,j\leq m
\end{align*}
\end{definition}

\begin{example}[An identity matrix]
  The $4\times 4$ identity matrix is
  \[
    I_4=
    \begin{bmatrix}
      1&\answer{0}&0&0\\
      0&\answer{1}&0&0\\
      0&\answer{0}&1&0\\
      0&\answer{0}&0&1
    \end{bmatrix}.
  \]
\end{example}

\begin{question}
  Notice that an identity matrix is square, and
  \begin{multipleChoice}
    \choice{is not in reduced row-echelon form.}
    \choice[correct]{is in reduced row-echelon form.}
  \end{multipleChoice}
  
  \begin{feedback}[correct]
    Also, every column is a pivot column, and every possible pivot column
    appears once.

    We mentioned above that an identity matrix is in reduced row-echelon
    form.  What happens if we try to row-reduce a matrix that is already
    in reduced row-echelon form?  By the uniqueness of the result, there
    should be no change.
  \end{feedback}
\end{question}

\begin{theorem}[Nonsingular Matrices Row Reduce to the Identity matrix]
  \label{theorem:NMRRI}
  Suppose that $A$ is a square matrix and $B$ is a row-equivalent
  matrix in reduced row-echelon form.  Then $A$ is nonsingular if and
  only if $B$ is the identity matrix.

\begin{proof}
  $(\Leftarrow)$ Suppose $B$ is the identity matrix.  When the
  augmented matrix $\augmented{A}{\zerovector}$ is row-reduced, the
  result is $\augmented{B}{\zerovector}=\augmented{I_n}{\zerovector}$.
  The number of nonzero rows is equal to the number of variables in
  the linear system of equations $\linearsystem{A}{\zerovector}$, so
  $n=r$ and \ref{theorem:FVCS} gives $n-r=0$ free variables.  Thus,
  the homogeneous system $\homosystem{A}$ has just $\answer{1}$
  solution, which must be the trivial solution.  This is exactly the
  definition of a nonsingular matrix.

  $(\Rightarrow)$ If $A$ is nonsingular, then the homogeneous system
  $\linearsystem{A}{\zerovector}$ has a unique solution, and has no
  free variables in the description of the solution set.  The
  homogeneous system is consistent (\ref{theorem:HSC}) so
  \ref{theorem:FVCS} applies and tells us there are $n-r$ free
  variables.  Thus, $n-r=0$, and so $n=r$.  So $B$ has $\answer{n}$
  pivot columns among its total of $n$ columns.  This is enough to
  force $B$ to be the $n\times n$ identity matrix $I_n$.
\end{proof}
\end{theorem}

Notice that since this theorem is an equivalence it will always allow
us to determine if a matrix is either nonsingular or singular.

\begin{example}
  \[
    A = \begin{bmatrix}
      1 & -1 & 2 \\
      2 & 1 & 1 \\
      1 & 1 & 0  
    \end{bmatrix}
  \]
  row-reduces to
  \[
    \begin{bmatrix}
      \leading{1} & 0 & 1 \\
      0 & \leading{1} & -1 \\
      0 & 0 & 0 
    \end{bmatrix}.
  \]
  Since this row-reduced matrix
  \wordChoice{\choice{is}\choice[correct]{is not}} the $3\times 3$
  identity matrix, the above theorem tells us that $A$ is a
  \wordChoice{\choice{nonsingular}\choice[correct]{singular}} matrix.
\end{example}

\begin{example}
  The matrix
  \[
    B = \begin{bmatrix}
      -7&-6&- 12\\
      5&5&7 \\
      1&0&4
    \end{bmatrix},
  \]
  row-reduces to
  \[
    \begin{bmatrix}
      \leading{1}&0&0 \\
      0&\leading{1}&0 \\
      0&0&\leading{1}
    \end{bmatrix},
  \]
  which \wordChoice{\choice[correct]{is}\choice{is not}} the $3\times 3$
  identity matrix, so the above theorem tells us that $A$ is a
  \wordChoice{\choice[correct]{nonsingular}\choice{singular}} matrix.
\end{example}

\end{document}

%%% Local Variables:
%%% mode: latex
%%% TeX-master: t
%%% End:
