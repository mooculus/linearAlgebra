\documentclass{ximera}
\input{../preamble.tex}
\title{The Superposition Principle}
\author{Crichton Ogle}
\pgfplotsset{compat=1.15}
\begin{document}
\begin{abstract}
  Sums of solution to homogeneous systems are also solutions.
\end{abstract}
\maketitle

 Given a matrix equation $A*{\bf x} = {\bf b}$, its {\it associated homogeneous equation} is the equation $A*{\bf x} = {\bf 0}$ (that results from replacing $\bf b$ by $\bf 0$). Note that consistency status may change in passing from an arbitrary non-homogeneous equation to its associated counterpart. However, if the original system is consistent, there is an important relation between the two solution sets, which is a manifestation of the {\it superposition principle}.
\vskip.2in

To see this, note that if ${\bf x}', {\bf x}''$ are two solutions to the equation $A*{\bf x} = {\bf b}$, then 
\[
A*({\bf x}' - {\bf x}'') = A*{\bf x}' - A*{\bf x}'' = {\bf b} - {\bf b} = {\bf 0}
\]
so $({\bf x}' - {\bf x}'')$ is a solution to the associated homogeneous equation. On the other hand, given a solution ${\bf x}_h$ to the associated homogeneous equation, and a solution ${\bf x}$ to the original equation, we see
\[
A*({\bf x}_h + {\bf x}) = A*{\bf x}_h + A*{\bf x} = {\bf 0} + {\bf b} = {\bf b}
\]
so ${\bf x}_h + {\bf x}$ is again a solution to the original equation. Thus

\begin{theorem}[Superposition Principle] Suppose $A*{\bf x} = {\bf b}$ is a consistent matrix equation, with ${\bf x}_p$ a particular solution to the equation. Then the set of solutions to the original equation can be expressed as
\[
\{{\bf x}_h + {\bf x}_p\ |\ {\bf x}_h\in S_0\}
\]
where $S_0$ denotes the set of solutions to the associated homogeneous equation.
\end{theorem}

\begin{example} Suppose we are given a $3\times 5$ matrix $A$ and a $3\times 1$ vector $\bf b$ with
\[
A = \begin{bmatrix}
-8 & 8 & 15 & 1 & 11\\
10 & 11 & 2 & -2 & 13\\
-19 & -7 & -14 & 8& -4
\end{bmatrix},\quad
{\bf b} = 
\begin{bmatrix}
35\\
5\\
21
\end{bmatrix}
\]
and we want to describe the set of solutions to the equation $A*{\bf x} = {\bf b}$. Using Octave or MATLAB, we compute the rref(ACM):
\[
rref([A\ \  {\bf b}]) = 
\begin{bmatrix}
1 & 0 & 0 & -475/1419 & -853/4021 & -666/335\\
0 & 1 & 0 & 636/4021 & 5619/4021 & 9041/4021\\
0 & 0 & 1 & -789/4021 & -503/4021 & 297/4021
\end{bmatrix}
\]
The columns (except the right-most) that do not contain leading ones are the fourth and fifth. We also see that the system is consistent. It follows that there is a 2-parameter family of solutions parametrized by the free, or independent variables $x_4$ and $x_5$, with the dependent variables $x_1, x_2, x_3$ expressable as linear functions in $x_4$ and $x_5$. Written in standard parametrized form, the full solution set is then given by
\begin{align*}
x_1 &= (475/1419)x_4 + (853/4021)x_5 - 666/335 \\
x_2 &= (-636/4021)x_4 - (5619/4021)x_5 + 9041/4021\\
x_3 &= (789/4021)x_4 + (503/4021)x_5 + 297/4021\\
x_4 &= x_4\\
x_5 &= x_5
\end{align*}
\vskip.2in
To rewrite in superpositional format, we first choose a particular solution. The easiest one to compute is that corresponding to $x_4 = x_5 = 0$. In other words the vector determined by the constant terms appearing on the right-hand sides of the above set of equations:
\[
{\bf x}_p = 
\begin{bmatrix}
-666/335\\ 9041/4021\\ 297/4021\\ 0\\ 0
\end{bmatrix}
\]
The homogeneous part is then given as the 2-parameter set of vectors corresponding to the linear part of the same set of equations:
\[
{\bf x}_h = 
\begin{bmatrix}
(475/1419)x_4 + (853/4021)x_5\\
(-636/4021)x_4 - (5619/4021)x_5\\
(789/4021)x_4 + (503/4021)x_5\\
x_4\\
x_5
\end{bmatrix} = x_4 \begin{bmatrix}
475/1419\\ -636/4021\\ 789/4021\\ 1\\ 0
\end{bmatrix} + x_5 \begin{bmatrix}
853/4021\\ 5619/4021\\ 503/4021\\ 0\\ 1
\end{bmatrix}
\]
The solution set given above can then be alternately expressed as
\[
{\bf x} = {\bf x}_p + {\bf x}_h
\]
where ${\bf x}_p$ and ${\bf x}_h$ are as computed above.
\end{example}



\end{document}
