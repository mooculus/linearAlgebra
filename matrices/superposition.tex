\documentclass{ximera}
\input{../preamble.tex}
\title{The Superposition Principle}
\author{Crichton Ogle}

\begin{document}
\begin{abstract}
  Sums of solution to homogeneous systems are also solutions.
\end{abstract}
\maketitle

 Given a matrix equation $A*{\bf x} = {\bf b}$, its {\it associated homogeneous equation} is $A*{\bf x} = {\bf 0}$ (resulting from replacing $\bf b$ by $\bf 0$). Note that consistency status may change in passing from an arbitrary non-homogeneous equation to its associated counterpart. However, if the original system is consistent, there is an important relation between the two solution sets, which is a manifestation of the {\it superposition principle}.
\vskip.2in

To see this, note that if ${\bf x}', {\bf x}''$ are two solutions to the equation $A*{\bf x} = {\bf b}$, then 
\[
A*({\bf x}' - {\bf x}'') = A*{\bf x}' - A*{\bf x}'' = {\bf b} - {\bf b} = {\bf 0}
\]
so $({\bf x}' - {\bf x}'')$ is a solution to the associated homogeneous equation. On the other hand, given a solution ${\bf x}_h$ to the associated homogeneous equation, and a solution ${\bf x}$ to the original equation, we see
\[
A*({\bf x}_h + {\bf x}) = A*{\bf x}_h + A*{\bf x} = {\bf 0} + {\bf b} = {\bf b}
\]
so ${\bf x}_h + {\bf x}$ is again a solution to the original equation. Thus

\begin{theorem}[Superposition Principle] Suppose $A*{\bf x} = {\bf b}$ is a consistent matrix equation, with ${\bf x}_p$ a particular solution to the equation. Then the set of solutions to the original equation can be expressed as
\[
\{{\bf x}_h + {\bf x}_p\ |\ {\bf x}_h\in S_0\}
\]
where $S_0$ denotes the set of solutions to the associated homogeneous equation.
\end{theorem}


\end{document}
