\documentclass{ximera}
\input{../preamble.tex}
\title{Elementary Matrices}
\author{Crichton Ogle}

\begin{document}
\begin{abstract}
  Sums of solution to homogeneous systems are also solutions.
\end{abstract}
\maketitle

As we have seen, systems of equations---or equivalently matrix
equations---are solved by i) forming the ACM associated with the set
of equations and ii) applying row operations to the ACM until it is in
reduced row echelon form.

It turns out these row operations can be realized by left
multiplication by a certain type of matrix, and these matrices have
uses beyond that of performing row operations. To explain how matrix
multiplication comes into play, let us write $\cal R(_-)$ for a
particular row operation on $m\times n$ matrices, so that the given
operation is represented by $A\mapsto {\cal R}(A)$. It turns out that
for any of the three types of row operations we have considered above,
one has the identity
\[
{\cal R}(A) = {\cal R}(I*A) = {\cal R}(I)*A
\]
In other words, {\it the row operation ${\cal R}(_-)$, applied to $A$, can be realized in terms of left multiplication by the $m\times m$ matrix ${\cal R}(I)$ gotten by applying $\cal R$ to the $m\times m$ identity matrix}.

As one would then expect, one has - for each row operation - a
corresponding {\it elementary matrix} derived from the identity matrix
of the appropriate dimension by application of that given operation.

These matrices, and the notation used to define them, can be recorded
in an expanded version of the table above in which we indicated the
types of operations and their representation:

\begin{center}
    \begin{tabular}{|c|c|c|c|}\hline\hline
    \phantom{x} & \phantom{x} & \phantom{x} & \phantom{x}\\
    \Large{Type} &\Large{What it does} &\Large{Indicated by} & \Large{Elementary matrix}\\ \hline
    \phantom{x} & {\large Switches} & \phantom{\Huge X} & \phantom{X}\\
    \large{Type I} &{\large $i^{th}$ and $j^{th}$} & {\large $R_i\leftrightarrow R_j$} & {\large ${\cal R}(I) = P_{ij}$}\\ 
    \phantom{x} & {\large rows} & \phantom{x} & \phantom{x}\\ \hline
    \phantom{x} &{\large Multiplies} & \phantom{\Huge X} & \phantom{x}\\
    \large{Type II} &{\large $i^{th}$ row} & {\large $r\cdot R_i$} & {\large ${\cal R}(I) = D_i(r)$}\\
    \phantom{x} & {\large by $r\ne 0$} & \phantom{x} & \phantom{x}\\ \hline
    \phantom{x} &{\large Adds} & \phantom{\Huge X} & \phantom{x}\\
    \large{Type III} &{\large $a$ times the $i^{th}$ row} & {\large $a\cdot R_i$ added to $R_j$} & {\large ${\cal R}(I) = E_{ji}(a)$}\\
    \phantom{x} & {\large to the $j^{th}$ row} & \phantom{x} & \phantom{x}\\\hline\hline
    \end{tabular}
    \end{center}
\vskip.2in

%BADBAD: Examples to be included

\end{document}
