\documentclass{ximera}

\input{../../preamble.tex}

\title{Inverse of a Matrix}

\begin{document}
\begin{abstract}
  When a matrix is multiplied by its ``inverse,'' the result is, by
  definition, the identity matrix.
\end{abstract}
\maketitle

\begin{definition}[Matrix Inverse]
  Suppose $A$ and $B$ are square matrices of size $n$ such that
  $AB=I_n$ and $BA=I_n$.  Then $A$ is \dfn{invertible} and $B$ is the
  \dfn{inverse} of $A$.  In this situation, we write $B=\inverse{A}$.
\end{definition}

Notice that if $B$ is the inverse of $A$, then we can just as easily
say $A$ is the inverse of $B$, or $A$ and $B$ are inverses of each
other.

Not every square matrix has an inverse.  

\begin{example}[A matrix without an inverse]

Consider the coefficient matrix 
\[
  A = \begin{bmatrix}
    1 & -1 & 2\\
    2 & 1 & 1\\
    1 & 1 & 0
  \end{bmatrix}
\]
Suppose that $A$ is invertible and does have an inverse, say $B$.
Choose the vector of constants
\[
\vect{b}=\colvector{1\\3\\2}
\]
and consider the system of equations $\linearsystem{A}{\vect{b}}$.
This vector equation would then have a unique solution, namely
\begin{multipleChoice}
  \choice{$\vect{x}=A\vect{b}$.}
  \choice[correct]{$\vect{x}=B\vect{b}$.}
  \choice[correct]{$\vect{x}=B^{-1}\vect{b}$.}
\end{multipleChoice}
However, the system $\linearsystem{A}{\vect{b}}$ is inconsistent.
Form the augmented matrix $\augmented{A}{\vect{b}}$ and row-reduce to
\[
\begin{bmatrix}
\leading{1} & 0 & 1 & 0\\
0 & \leading{1} & -1 & 0\\
0 & 0 & 0 & \leading{1}
\end{bmatrix}
\]
which allows us to recognize the inconsistency.

So the assumption of $A$'s inverse leads to a logical inconsistency
(the system cannot be both consistent and inconsistent), so our
assumption is false.  $A$ is \wordChoice{\choice[correct]{not
    invertible}\choice{invertible}}.

It is possible this example is less than satisfying.  Just where did
that particular choice of the vector $\vect{b}$ come from anyway?
\end{example}

Let us look at one more matrix inverse before we embark on a more systematic study.

\begin{example}[Matrix inverse]
Consider the matrix
\begin{align*}
  A&=
     \begin{bmatrix}
       1 & 2 & 1 & 2 & 1 \\
       -2 & -3 & 0 & -5 & -1 \\
       1 & 1 & 0 & 2 & 1 \\
       -2 & -3 & -1 & -3 & -2 \\
       -1 & -3 & -1 & -3 & 1
     \end{bmatrix}
\end{align*}
and the matrix
\begin{align*}
  B&=
     \begin{bmatrix}
       -3 & 3 & 6 & -1 & -2 \\
       0 & -2 & -5 & -1 & 1 \\
       1 & 2 & 4 & 1 & -1 \\
       1 & 0 & 1 & 1 & 0 \\
       1 & -1 & -2 & 0 & 1
     \end{bmatrix}
\end{align*}
Then
\begin{align*}
  AB
  &=
    \begin{bmatrix}
      1 & 2 & 1 & 2 & 1 \\
      -2 & -3 & 0 & -5 & -1 \\
      1 & 1 & 0 & 2 & 1 \\
      -2 & -3 & -1 & -3 & -2 \\
      -1 & -3 & -1 & -3 & 1
    \end{bmatrix}
                          \begin{bmatrix}
                            -3 & 3 & 6 & -1 & -2 \\
                            0 & -2 & -5 & -1 & 1 \\
                            1 & 2 & 4 & 1 & -1 \\
                            1 & 0 & 1 & 1 & 0 \\
                            1 & -1 & -2 & 0 & 1
                          \end{bmatrix} \\
  &=
    \begin{bmatrix}
      1 & 0 & 0 & 0 & 0\\
      0 & 1 & 0 & 0 & 0\\
      0 & 0 & 1 & 0 & 0\\
      0 & 0 & 0 & 1 & 0\\
      0 & 0 & 0 & 0 & 1
    \end{bmatrix}
\end{align*}
and
\begin{align*}
  BA
  &=
    \begin{bmatrix}
      -3 & 3 & 6 & -1 & -2 \\
      0 & -2 & -5 & -1 & 1 \\
      1 & 2 & 4 & 1 & -1 \\
      1 & 0 & 1 & 1 & 0 \\
      1 & -1 & -2 & 0 & 1
    \end{bmatrix}
                        \begin{bmatrix}
                          1 & 2 & 1 & 2 & 1 \\
                          -2 & -3 & 0 & -5 & -1 \\
                          1 & 1 & 0 & 2 & 1 \\
                          -2 & -3 & -1 & -3 & -2 \\
                          -1 & -3 & -1 & -3 & 1
                        \end{bmatrix} \\
  &=
    \begin{bmatrix}
      1 & 0 & 0 & 0 & 0\\
      0 & 1 & 0 & 0 & 0\\
      0 & 0 & 1 & 0 & 0\\
      0 & 0 & 0 & 1 & 0\\
      0 & 0 & 0 & 0 & 1
    \end{bmatrix}
\end{align*}
so we can say that $A$ is invertible and write $B=\inverse{A}$.
\end{example}

We will now concern ourselves less with whether or not an inverse of a
matrix exists, but instead with how you can find one when it does
exist.  We will have some theorems that allow us to more quickly and
easily determine just when a matrix is invertible.

\end{document}
