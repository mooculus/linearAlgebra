\documentclass{ximera}

\input{../../preamble.tex}

\title{Properties of Matrix Inverses}

\begin{document}
\begin{abstract}
  The inverse of a matrix enjoys many wonderful properties.
\end{abstract}
\maketitle

A matrix can have but one inverse.

\begin{theorem}[Matrix Inverse is Unique]
  \label{theorem:MIU}
  
  Suppose the square matrix $A$ has an inverse.  Then $\inverse{A}$ is
  unique.

  \begin{proof}
    We will assume that $A$ has two inverses.  The hypothesis tells
    there is at least one.  Suppose then that $B$ and $C$ are both
    inverses for $A$, so $AB=BA=I_n$ and $AC=CA=I_n$.  Then we have,
    \begin{align*}
      B
      &=BI_n\\
      &=B(AC)\\
      &=(BA)C\\
      &=I_nC \\
      &=C
    \end{align*}
    So we conclude that $B$ and $C$ are the same, and cannot be different.  So any matrix that acts like \textit{an} inverse, must be \textit{the} inverse.
  \end{proof}
\end{theorem}

When most of us dress in the morning, we put on our socks first,
followed by our shoes.  In the evening we must then first remove our
shoes, followed by our socks.  Try to connect the conclusion of the
following theorem with this everyday example.

\begin{theorem}[Socks and Shoes]
\label{theorem:SS}

Suppose $A$ and $B$ are invertible matrices of size $n$.  Then $AB$ is
an invertible matrix and $\inverse{(AB)}=\inverse{B}\inverse{A}$.

\begin{proof}
  At the risk of carrying our everyday analogies too far, the proof of
  this theorem is quite easy when we compare it to the workings of a
  dating service.  We have a statement about the inverse of the matrix
  $AB$, which for all we know right now might not even exist.  Suppose
  $AB$ was to sign up for a dating service with two requirements for a
  compatible date.  Upon multiplication on the left, and on the right,
  the result should be the identity matrix.  In other words, $AB$'s
  ideal date would be its inverse.

  Now along comes the matrix $\inverse{B}\inverse{A}$ (which we know
  exists because our hypothesis says both $A$ and $B$ are invertible
  and we can form the product of these two matrices), also looking for
  a date.  Let us see if $\inverse{B}\inverse{A}$ is a good match for
  $AB$.  First they meet at a noncommittal neutral location, say a
  coffee shop, for quiet conversation:
  \begin{align*}
    (\inverse{B}\inverse{A})(AB)
    &=\inverse{B}(\inverse{A}A)B\\ %\ref{theorem:MMA}\\
    &=\inverse{B}I_nB\\ %\ref{definition:MI}\\
    &=\inverse{B}B\\ %\ref{theorem:MMIM}\\
    &=I_n\\ %\ref{definition:MI}
  \end{align*}
  The first date having gone smoothly, a second, more serious, date is arranged, say dinner and a show:
  \begin{align*}
    (AB)(\inverse{B}\inverse{A})
    &=A(B\inverse{B})\inverse{A}\\ %\ref{theorem:MMA}\\
    &=AI_n\inverse{A}\\ %\ref{definition:MI}\\
    &=A\inverse{A}\\ %\ref{theorem:MMIM}\\
    &=I_n\\ %\ref{definition:MI}
  \end{align*}

  So the matrix $\inverse{B}\inverse{A}$ has met all of the
  requirements to be $AB$'s inverse (date) and with the ensuing
  marriage proposal we can announce that
  $\inverse{(AB)}=\inverse{B}\inverse{A}$.
\end{proof}
\end{theorem}

\begin{theorem}[Matrix Inverse of a Matrix Inverse]
  \label{theorem:MIMI}

  Suppose $A$ is an invertible matrix.  Then $\inverse{A}$ is invertible and $\inverse{(\inverse{A})}=A$.

  \begin{proof}
    As with the proof of \ref{theorem:SS}, we examine if $A$ is a suitable inverse for $\inverse{A}$ (by definition, the opposite is true).
    \begin{align*}
      A\inverse{A}&=I_n\\ %\ref{definition:MI}
      \end{align*}
      and
      \begin{align*}
        \inverse{A}A&=I_n\\ %\ref{definition:MI}
      \end{align*}

    The matrix $A$ has met all the requirements to be the inverse of $\inverse{A}$, and so is invertible and we can write $A=\inverse{(\inverse{A})}$.

  \end{proof}
\end{theorem}

\begin{theorem}[Matrix Inverse of a Transpose]
  \label{theorem:MIT}

  Suppose $A$ is an invertible matrix.  Then $\transpose{A}$ is invertible and $\inverse{(\transpose{A})}=\transpose{(\inverse{A})}$.

  \begin{proof}
    As with the proof of \ref{theorem:SS}, we see if $\transpose{(\inverse{A})}$ is a suitable inverse for $\transpose{A}$. Apply \ref{theorem:MMT} to see that
    \begin{align*}
      \transpose{(\inverse{A})}\transpose{A}
      &=\transpose{(A\inverse{A})}\\ %\ref{theorem:MMT}\\
      &=\transpose{I_n}\\ %\ref{definition:MI}\\
      &=I_n\\ %\ref{definition:SYM}
      \end{align*}
      and
      \begin{align*}
        \transpose{A}\transpose{(\inverse{A})}
      &=\transpose{(\inverse{A}A)}\\ %\ref{theorem:MMT}\\
      &=\transpose{I_n}\\ %\ref{definition:MI}\\
      &=I_n\\ %\ref{definition:SYM}
    \end{align*}

    The matrix $\transpose{(\inverse{A})}$ has met all the
    requirements to be the inverse of $\transpose{A}$, and so is
    invertible and we can write
    $\inverse{(\transpose{A})}=\transpose{(\inverse{A})}$.

\end{proof}
\end{theorem}

\begin{theorem}
  \label{theorem:MISM}
  [Matrix Inverse of a Scalar Multiple]

  Suppose $A$ is an invertible matrix and $\alpha$ is a nonzero
  scalar.  Then
  $\inverse{\left(\alpha A\right)}=\frac{1}{\alpha}\inverse{A}$ and
  $\alpha A$ is invertible.

  \begin{proof}
    As with the proof of \ref{theorem:SS}, we see if $\frac{1}{\alpha}\inverse{A}$ is a suitable inverse for $\alpha A$.
    \begin{align*}
      \left(\frac{1}{\alpha}\inverse{A}\right)\left(\alpha A\right)
      &=\left(\frac{1}{\alpha}\alpha\right)\left(\inverse{A}A\right)\\ %\ref{theorem:MMSMM}\\
      &=1I_n&&\text{Scalar multiplicative inverses}\\
      &=I_n\\ %\ref{property:OM}
      \end{align*}
      and
      \begin{align*}
              \left(\alpha A\right)\left(\frac{1}{\alpha}\inverse{A}\right)&=
                                                                             \left(\alpha\frac{1}{\alpha}\right)\left(A\inverse{A}\right)\\ %\ref{theorem:MMSMM}\\
      &=1I_n&&\text{Scalar multiplicative inverses}\\
      &=I_n\\ %\ref{property:OM}
    \end{align*}

    The matrix $\frac{1}{\alpha}\inverse{A}$ has met all the
    requirements to be the inverse of $\alpha A$, so we can write
    $\inverse{\left(\alpha A\right)}=\frac{1}{\alpha}\inverse{A}$.

  \end{proof}
\end{theorem}

\begin{question}
  Notice that there are some likely theorems that are missing here.  For
  example, is it the case that
  $\inverse{(A+B)}=\inverse{A}+\inverse{B}$?
  \begin{multipleChoice}
    \choice[correct]{No.}
    \choice{Yes.}
  \end{multipleChoice}

  \begin{feedback}[correct]
    Indeed, it is \textit{not} the case that, for all invertible $A$
    and $B$, that $\inverse{(A+B)}=\inverse{A}+\inverse{B}$.  Can you
    find a counterexample?
  \end{feedback}

\end{question}

\end{document}



