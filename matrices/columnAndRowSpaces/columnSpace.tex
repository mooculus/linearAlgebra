\documentclass{ximera}

\input{../../preamble.tex}

\title{Column Spaces and Systems of Equations}

\begin{document}
\begin{abstract}
  There is a natural correspondence between solutions to linear systems and linear combinations of the columns of the coefficient matrix.
\end{abstract}
\maketitle

A matrix-vector product is a linear combination of the columns of the matrix and this allows us to connect matrix multiplication with systems of equations.

% Row operations are linear combinations of the rows of a matrix, and
% of course, reduced row-echelon form (\ref{definition:RREF}) is also
% intimately related to solving systems of equations.  In this section
% we will formalize these ideas with two key definitions of sets of
% vectors derived from a matrix.

\begin{definition}[Column Space of a Matrix]
  Suppose that $A$ is an $m\times n$ matrix with columns
  $\vectorlist{A}{n}$.  Then the \dfn{column space} of $A$, written
  $\csp{A}$, is the subset of $\complex{m}$ containing all linear
  combinations of the columns of $A$,
  \[
    \csp{A}=\spn{\set{\vectorlist{A}{n}}}
  \]
\end{definition}

Some authors refer to the column space of a matrix as the \dfn{range}
or \dfn{image}, but we will reserve such terms for use with linear
transformations (\ref{definition:RLT}).

Upon encountering any new set, the first question we ask is what
objects are in the set, and which objects are not?  Here is an example
of one way to answer this question, and it will motivate a theorem
that will then answer the question precisely.

%BADBAD
\begin{example}[Column space of a matrix and consistent systems]
  Consider the linear system
  \begin{align*}
    2x_1  + x_2 + 7x_3 - 7x_4 &= 8 \\
    -3x_1 + 4x_2 -5x_3 - 6x_4 &=  -12 \\
    x_1 +x_2 + 4x_3 - 5x_4 &=  4
  \end{align*}
  and the linear system
  \begin{align*}
    2x_1  + x_2 + 7x_3 - 7x_4 &= 2 \\
    -3x_1 + 4x_2 -5x_3 - 6x_4 &=  3 \\
    x_1 +x_2 + 4x_3 - 5x_4 &=  2.
  \end{align*}
  These two systems are built from the same $3\times 4$ coefficient matrix, namely
  \begin{align*}
    A&=\begin{bmatrix}
      2 & 1 & 7 & -7 \\
      -3 & 4 &  -5 & -6 \\
      1 & 1 & 4 &  -5
    \end{bmatrix}
  \end{align*}
  However, only one of the above systems are consistent!  We can
  explain this difference by employing the column space of the matrix
  $A$.

  The column vector of constants, $\vect{b}$, from the first linear
  system is given below, and one solution listed for
  $\linearsystem{A}{\vect{b}}$ is $\vect{x}$,
  \begin{align*}
    \vect{b}&=\colvector{8\\-12\\4}
            &
              \vect{x}&=\colvector{7\\8\\1\\3}
  \end{align*}

  By \ref{theorem:SLSLC}, we can summarize this solution as a linear
  combination of the columns of $A$ that equals $\vect{b}$,
  \[
    7\colvector{2\\-3\\1}+
    8\colvector{1\\4\\1}+
    1\colvector{7\\-5\\4}+
    3\colvector{-7\\-6\\-5}=
    \colvector{8\\-12\\4}=\vect{b}.
  \]

  This equation says that $\vect{b}$ is a linear combination of the
  columns of $A$, and then by \ref{definition:CSM}, we can say that
  $\vect{b}\in\csp{A}$.

  On the other hand, the second linear system is
  $\linearsystem{A}{\vect{c}}$, where the vector of constants is
  \[
    \vect{c}=\colvector{2\\3\\2}
  \]
  and this system of equations is inconsistent.  This means
  $\vect{c}\not\in\csp{A}$, for if it were, then it would equal a
  linear combination of the columns of $A$ and \ref{theorem:SLSLC}
  would lead us to a solution of the system
  $\linearsystem{A}{\vect{c}}$.
\end{example}

So if we fix the coefficient matrix, and vary the vector of constants,
we can sometimes find consistent systems, and sometimes inconsistent
systems.  The vectors of constants that lead to consistent systems are
exactly the elements of the column space.  This is the content of the
next theorem, and since it is an equivalence, it provides an alternate
view of the column space.

\begin{theorem}[Column Spaces and Consistent Systems]
  \label{theorem:CSCS}
  Suppose $A$ is an $m\times n$ matrix and $\vect{b}$ is a vector of
  size $m$.  Then $\vect{b}\in\csp{A}$ \wordChoice{\choice[correct]{if
      and only if}\choice{if}\choice{only if}}
  $\linearsystem{A}{\vect{b}}$ is consistent.

\begin{proof}
  ($\Rightarrow$) Suppose $\vect{b}\in\csp{A}$.  Then we can write
  $\vect{b}$ as some linear combination of the
  \wordChoice{\choice{rows}\choice[correct]{columns}} of $A$.  By
  \ref{theorem:SLSLC} we can use the scalars from this linear
  combination to form a solution to $\linearsystem{A}{\vect{b}}$, so
  this system is consistent.

  ($\Leftarrow$) If $\linearsystem{A}{\vect{b}}$ is
  \wordChoice{\choice[correct]{consistent}\choice{incorrect}}, there
  is a solution that may be used with \ref{theorem:SLSLC} to write
  $\vect{b}$ as a linear combination of the columns of $A$.  This
  qualifies $\vect{b}$ for membership in $\csp{A}$.
\end{proof}
\end{theorem}

This theorem tells us that asking if the system
$\linearsystem{A}{\vect{b}}$ is consistent is exactly the same
question as asking if $\vect{b}$ is in the column space of $A$.  Or
equivalently, it tells us that the column space of the matrix $A$ is
precisely those vectors of constants, $\vect{b}$, that can be paired
with $A$ to create a system of linear equations
$\linearsystem{A}{\vect{b}}$ that is consistent.

Employing \ref{theorem:SLEMM} we can form the chain of equivalences
\begin{align*}
  \vect{b}\in\csp{A}
  \iff
  \linearsystem{A}{\vect{b}}\text{ is consistent}
  \iff
  A\vect{x}=\vect{b}\text{ for some }\vect{x}
\end{align*}

Thus, an alternative (and popular) definition of the column space of an $m\times n$ matrix $A$ is
\begin{align*}
  \csp{A}&=
           \setparts{
           \vect{y}\in\complex{m}
           }{
           \vect{y}=A\vect{x}\text{ for some }\vect{x}\in\complex{n}
           }
           =
           \setparts{A\vect{x}}{\vect{x}\in\complex{n}}
           \subseteq\complex{m}
\end{align*}

We recognize this as saying create \textit{all} the matrix vector
products possible with the matrix $A$ by letting $\vect{x}$ range over
all of the possibilities.  By \ref{definition:MVP} we see that this
means take all possible linear combinations of the columns of $A$
<mdash /> precisely the definition of the column space
(\ref{definition:CSM}) we have chosen.

Notice how this formulation of the column space looks very much like
the definition of the null space of a matrix (\ref{definition:NSM}),
but for a rectangular matrix the column vectors of $\csp{A}$ and
$\nsp{A}$ have different sizes, so the sets are very different.

Given a vector $\vect{b}$ and a matrix $A$ it is now very mechanical
to test if $\vect{b}\in\csp{A}$.  Form the linear system
$\linearsystem{A}{\vect{b}}$, row-reduce the augmented matrix,
$\augmented{A}{\vect{b}}$, and test for consistency with
\ref{theorem:RCLS}.  Here is an example of this procedure.

\begin{example}[Membership in the column space of a matrix]
  Consider the column space of the $3\times 4$ matrix $A$,
  \[
    A=
    \begin{bmatrix}
      3 & 2 & 1 & -4\\
      -1 & 1 & -2 & 3\\
      2 & -4 & 6 & -8
    \end{bmatrix}
  \]
  We first show that $\vect{v}=\colvector{18\\-6\\12}$ is in the
  column space of $A$, $\vect{v}\in\csp{A}$.  \ref{theorem:CSCS} says
  we need only check the consistency of $\linearsystem{A}{\vect{v}}$.
  Form the augmented matrix and row-reduce,
  \[
    \begin{bmatrix}
      3 & 2 & 1 & -4 & 18\\
      -1 & 1 & -2 & 3 & -6\\
      2 & -4 & 6 & -8 & 12
    \end{bmatrix}
    \rref
    \begin{bmatrix}
      \leading{1} & 0 & 1 & -2 & 6\\
      0 & \leading{1} & -1 & 1 & 0\\
      0 & 0 & 0 & 0 & 0
    \end{bmatrix}
  \]
  
  Since the final column \wordChoice{\choice[correct]{is
      not}\choice{is}} a pivot column, the system is consistent and
  therefore by \ref{theorem:CSCS}, $\vect{v}\in\csp{A}$.

  If we wished to demonstrate explicitly that $\vect{v}$ is a linear
  combination of the columns of $A$, we can find a solution (any
  solution) of $\linearsystem{A}{\vect{v}}$ and use
  \ref{theorem:SLSLC} to construct the desired linear combination.
  For example, set the free variables to $x_3=2$ and $x_4=1$.  Then a
  solution has $x_2=1$ and $x_1=6$.  Then by \ref{theorem:SLSLC},
  \[
    \vect{v}=\colvector{18\\-6\\12}=
    6\colvector{3\\-1\\2}+
    1\colvector{2\\1\\-4}+
    2\colvector{1\\-2\\6}+
    1\colvector{-4\\3\\-8}
  \]

  Now we show that $\vect{w}=\colvector{2\\1\\-3}$ is not in the
  column space of $A$, $\vect{w}\not\in\csp{A}$.  \ref{theorem:CSCS}
  says we need only check the consistency of
  $\linearsystem{A}{\vect{w}}$.  Form the augmented matrix and
  row-reduce,
  \[
    \begin{bmatrix}
      3 & 2 & 1 & -4 & 2\\
      -1 & 1 & -2 & 3 & 1\\
      2 & -4 & 6 & -8 & -3
    \end{bmatrix}
    \rref
    \begin{bmatrix}
      \leading{1} & 0 & 1 & -2 & 0\\
      0 & \leading{1} & -1 & 1 & 0\\
      0 & 0 & 0 & 0 & \leading{1}
    \end{bmatrix}
  \]
  and since the final column is a pivot column, \ref{theorem:RCLS}
  tells us the system is inconsistent and therefore by
  \ref{theorem:CSCS}, $\vect{w}\not\in\csp{A}$.

\end{example}

\ref{theorem:CSCS} completes a collection of three theorems, and one
definition, that deserve comment.  Many questions about spans, linear
independence, null space, column spaces and similar objects can be
converted to questions about systems of equations (homogeneous or
not), which we understand well from our previous results, especially
those in \ref{chapter:SLE}.  These previous results include theorems
like \ref{theorem:RCLS} which allows us to quickly decide consistency
of a system, and \ref{theorem:BNS} which allows us to describe
solution sets for homogeneous systems compactly as the span of a
linearly independent set of column vectors.

\begin{tabular}{|r|l|}\hline\hline
  \multicolumn{1}{|l}{\ref{definition:NSM}}&\\\hline
  Synopsis&Null space is solution set of homogeneous system\\\hline
  Example&General solution sets described by \ref{theorem:PSPHS}\\\hline\hline
  \multicolumn{1}{|l}{\ref{theorem:SLSLC}}&\\\hline
  Synopsis&Solutions for linear combinations with unknown scalars\\\hline
  Example&Deciding membership in spans\\\hline\hline
  \multicolumn{1}{|l}{\ref{theorem:SLEMM}}&\\\hline
  Synopsis&System of equations represented by matrix-vector product\\\hline
  Example&Solution to $\linearsystem{A}{\vect{b}}$ is $\inverse{A}\vect{b}$ when $A$ is nonsingular\\\hline\hline
  \multicolumn{1}{|l}{\ref{theorem:CSCS}}&\\\hline
  Synopsis&Column space vectors create consistent systems\\\hline
  Example&Deciding membership in column spaces\\\hline\hline
\end{tabular}

\end{document}
