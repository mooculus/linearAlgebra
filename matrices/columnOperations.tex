\documentclass{ximera}
\input{../preamble.tex}
\title{Column Operations}
\author{Crichton Ogle}

\begin{document}
\begin{abstract}
  Analogous to row operations, one can perform column operations.
\end{abstract}
\maketitle

In an analogous fashion, one can also perform column operations on a
matrix. As with row operations there are three types, and each type
can be achieved via {\it right} multiplication with the corresponding
matrix. In other words, if ${\cal C}(_-)$ indicates the given column
operation, then for each type one has
\[
{\cal C}(A) = {\cal C}(A*I) = A*{\cal C}(I)
\]
Denoting the $k^{th}$ column by $C_k$, we have

\begin{center}
    \begin{tabular}{|c|c|c|c|}\hline\hline
    \phantom{x} & \phantom{x} & \phantom{x} & \phantom{x}\\
    \Large{Type} &\Large{What it does} &\Large{Indicated by} & \Large{Elementary matrix}\\ \hline
    \phantom{x} & {\large Switches} & \phantom{\Huge X} & \phantom{X}\\
    \large{Type I} &{\large $i^{th}$ and $j^{th}$} & {\large $C_i\leftrightarrow C_j$} & {\large ${\cal C}(I) = P_{ij}$}\\ 
    \phantom{x} & {\large colmns} & \phantom{x} & \phantom{x}\\ \hline
    \phantom{x} &{\large Multiplies} & \phantom{\Huge X} & \phantom{x}\\
    \large{Type II} &{\large $i^{th}$ column} & {\large $r\cdot C_i$} & {\large ${\cal C}(I) = D_i(r)$}\\
    \phantom{x} & {\large by $r\ne 0$} & \phantom{x} & \phantom{x}\\ \hline
    \phantom{x} &{\large Adds} & \phantom{\Huge X} & \phantom{x}\\
    \large{Type III} &{\large $a$ times the $i^{th}$ column} & {\large $a\cdot C_i$ added to $C_j$} & {\large ${\cal C}(I) = E_{ij}(a)$}\\
    \phantom{x} & {\large to the $j^{th}$ column} & \phantom{x} & \phantom{x}\\\hline\hline
    \end{tabular}
    \end{center}
\vskip.2in

% BADBAD: [Examples to be included]

Each elementary matrix is invertible, and of the same type. The
following indicates how each elementary matrix behaves under i)
inversion and ii) transposition:
\begin{align}
P_{ij}^{-1} = P_{ij},&\qquad P_{ij}^T = P_{ij}\\
D_i(r)^{-1} = D_i(r^{-1}),&\qquad D_i(r)^T = D_i(r)\\
E_{ij}(a)^{-1} = E_{ij}(-a),&\qquad E_{ij}(a)^T = E_{ji}(a)
\end{align}

This is useful in problems where one wants to express the inverse of a
matrix explicitly as a product of elementary matrices.




\end{document}
