\documentclass{ximera}
\input{../preamble.tex}
\title{Definition}
\author{Crichton Ogle}

\begin{document}
\begin{abstract}
  A nonzero vector which is scaled by a linear transformation is an eigenvector for that transformation.
\end{abstract}
\maketitle

If $A$ is an $m\times n$ matrix, $\bf v$ an $n\times 1$ non-zero vector, we say that $\bf v$ is {\it an eigenvector of A with eigenvalue $\lambda$} if one has the equality
\[
A*{\bf v} = \lambda{\bf v}
\]
In other words, if multiplying $\bf v$ on the left by the matrix $A$ has the same effect as multiplying it by the scalar $\lambda$. We note first that, in order for this to be at all  possible, $A*{\bf v}$ must also be an $n\times 1$ vector; in other words, $A$ must be a {\it square} matrix.
\vskip.2in

Given a square matrix, then, the {\it eigenvalue problem} is to find a complete description of the eigenvalues and associated eigenvectors for that matrix. Our goal in this section is to determine a systematic way of doing this.
\vskip.2in

Before proceeding with examples, we note that

\begin{proposition} If $\bf v$ is an eigenvector of a matrix $A$, the eigenvalue associated with it is unique.
\end{proposition}

\begin{proof} Suppose $\lambda_1{\bf v} = A*{\bf v} = \lambda_2{\bf v}$. Then $\lambda_1{\bf v} - \lambda_2{\bf v} = (\lambda_1 - \lambda_2){\bf v} = {\bf 0}$. But since ${\bf v}\ne {\bf 0}$, the only way this could happen is if the coefficient $(\lambda_1 - \lambda_2)$ is equal to zero, or equivalently, if $\lambda_1 = \lambda_2$.
\end{proof}
\vskip.2in

\begin{example} Let $A = I^{n\times n}$ be the $n\times n$ identity matrix. Then for any ${\bf v}\in\mathbb R^n$, one has $A*{\bf v} = I*{\bf v} = {\bf v} = 1{\bf v}$. So in this case, we see that {\it every non-zero vector in $\mathbb R^n$ is an eigenvector of $A$ with corresponding eigenvalue $1$}.
\end{example}
\vskip.2in

\begin{example} Let $B = \begin{bmatrix}2 & 0\\0 & 3\end{bmatrix}$. Then $B*{\bf e}_1 = 2{\bf e}_1$, and $B*{\bf e}_2 = 3{\bf e}_2$. Here the two standard basis vectors of $\mathbb R^2$ are eigenvectors of $B$, but with different eigenvalues. Note that, unlike the matrix $A$ in the previous example, not every vector in $\mathbb R^2$ is an eigenvector of $B$. In particular, if we let ${\bf v} = {\bf e}_1 + {\bf e}_2 = \begin{bmatrix} 1\\ 1\end{bmatrix}$, then $B*{\bf v} = \begin{bmatrix}2\\ 3\end{bmatrix}$, which cannot be written as a scalar multiple of $\bf v$.
\end{example}
\vskip.2in

So how does one go about finding the eigenvalues of a matrix? In order to understand more clearly what it is we are looking for, we consider a reformulation of the defining equation above. First, we note that the scalar product $\lambda{\bf v}$ can be rewritten as a matrix product
\[
\lambda{\bf v} = (\lambda I)*{\bf v}
\]
From this we have the following equivalent statements:
\[
A*{\bf v} = \lambda{\bf v}\quad\Leftrightarrow\quad A*{\bf v} = (\lambda I)*{\bf v}\quad\Leftrightarrow\quad (A - \lambda I)*{\bf v} = {\bf 0}\quad\Leftrightarrow\quad {\bf v}\in N(A - \lambda I)
\]

Thus

\begin{observation}\label{obs:eigen} A non-zero vector $\bf v$ is an eigenvector of $A$ with eigenvalue $\lambda$ if and only if it lies in the nullspace of the matrix $A-\lambda I$. In particular, in order for such an vector to exist, the matrix $A-\lambda I$ must be singular.
\end{observation}
\vskip.2in

This suggests that in order to solve the eigenvalue problem, we should first determine the values $\lambda$ for which $A - \lambda I$ is singular; then, for each such $\lambda$, determine a basis for the nullspace of $A - \lambda I$. In other words, {\it eigenvalues first, then eigenvectors}. And to determine the values for which $A - \lambda I$ is singular, we will use the determinant.

\end{document}
