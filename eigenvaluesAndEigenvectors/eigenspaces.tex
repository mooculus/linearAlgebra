\documentclass{ximera}
\input{../preamble.tex}
\title{Eigenspaces}
\author{Crichton Ogle}

\begin{document}
\begin{abstract}
  Establish algebraic criteria for determining exactly when a real number can occur as an eigenvalue of $A$.
\end{abstract}
\maketitle

%%%%%%%%%%%%%%%%%%%%%%%%%%%%%%%%%%%%%%

As we saw above, $\lambda$ is an eigenvalue of $A$ iff $N(A-\lambda I)\ne 0$, with the non-zero vectors in this nullspace comprising the set of eigenvectors of $A$ with eigenvalue $\lambda$.

\begin{definition} The eigenspace of $A$ corresponding to an eigenvalue $\lambda$ is $E_\lambda (A) := N(A-\lambda I)$.
\end{definition}

\begin{exercise} Suppose $A = \begin{bmatrix} 2 & 0\\0 & 3\end{bmatrix}$. Show that the two possible eigenvalues of $A$ are $\lambda = 2$ and $\lambda = 3$. Then show the two corresponding eigenspaces for these eigenvalues are $E_2(A) = Span\{{\bf e}_1\}$, $E_3(A) = Span\{{\bf e}_2\}$.
\end{exercise}

Note that the dimension of the eigenspace corresponding to a given eigenvalue must be at least 1, since eigenspaces must contain non-zero vectors by definition.

\end{document}
