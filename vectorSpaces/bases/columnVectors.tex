\documentclass{ximera}

\input{../../preamble.tex}

\title{Bases for Spans of Column Vectors}

\begin{document}
\begin{abstract}
  We build bases for vector spaces built from column vectors.
\end{abstract}
\maketitle

Suppose we have a subspace of $\real{m}$ that is expressed as the span
of a set of vectors, $S$, and $S$ is not necessarily linearly
independent, or perhaps not very attractive.  \ref{theorem:REMRS} says
that row-equivalent matrices have identical row spaces, while
\ref{theorem:BRS} says the nonzero rows of a matrix in reduced
row-echelon form are a basis for the row space.  These theorems
together give us a great computational tool for quickly finding a
basis for a subspace that is expressed originally as a span.

\begin{example}[Row space basis]
  When we first defined the span of a set of column vectors, in \ref{example:SCAD} we looked at the set
  \[
    W=\spn{\set{
        \colvector{2\\-3\\1},\,
        \colvector{1\\4\\1},\,
        \colvector{7\\-5\\4},\,
        \colvector{-7\\-6\\-5}
      }}
  \]
  with an eye towards realizing $W$ as the span of a smaller set.  By
  building relations of linear dependence (though we did not know them
  by that name then) we were able to remove two vectors and write $W$
  as the span of the other two vectors.  These two remaining vectors
  formed a linearly independent set, even though we did not know that
  at the time.

  Now we know that $W$ is a subspace and must have a basis.  Consider
  the matrix, $C$, whose rows are the vectors in the spanning set for
  $W$,
  \[
    C=\begin{bmatrix}
      2 & -3 & 1\\
      1 & 4 & 1\\
      7 & -5 & 4\\
      -7 & -6 & -5
    \end{bmatrix}
  \]
  Then
  \begin{multipleChoice}
    \choice[correct]{the row space of $C$ will be $W$, $\rsp{C}=W$.}
    \choice{the column space of $C$ will be $W$, $\csp{C}=W$.}
  \end{multipleChoice}

  So if we row-reduce $C$, the nonzero rows of the row-equivalent
  matrix in reduced row-echelon form will be a basis for $\rsp{C}$,
  and hence a basis for $W$.  Let us do it: $C$ row-reduces to
  \[
    \begin{bmatrix}
      \leading{1} & 0 & \frac{7}{11}\\
      0 & \leading{1} & \frac{1}{11}\\
      0 & 0 & 0\\
      0 & 0 & 0
    \end{bmatrix}
  \]
  If we convert the two nonzero rows to column vectors then we have a basis,
  \[
    B=\set{\colvector{1\\0\\\frac{7}{11}},\,\colvector{0\\1\\\frac{1}{11}}}
  \]
  and
  \[
    W=\spn{\set{\colvector{1\\0\\\frac{7}{11}},\,\colvector{0\\1\\\frac{1}{11}}}}
  \]
  
  For aesthetic reasons, we might wish to multiply each vector in $B$
  by $11$, which will not change the spanning or linear independence
  properties of $B$ as a basis.  Then we can also write
  \[
    W=\spn{\set{\colvector{11\\0\\7},\,\colvector{0\\11\\1}}}
  \]

\end{example}

\ref{example:IAS} provides another example of this flavor, though now
we can notice that $X$ is a subspace, and that the resulting set of
three vectors is a basis.  This is such a powerful technique that we
should do one more example.


\begin{example}[Reducing a span]
  We began with a set of $n=4$ vectors from $\real{5}$,
  \[
    R=\set{\vect{v}_1,\,\vect{v}_2,\,\vect{v}_3,\,\vect{v}_4}
    =
    \set{
      \colvector{1\\2\\-1\\3\\2},\,
      \colvector{2\\1\\3\\1\\2},\,
      \colvector{0\\-7\\6\\-11\\-2},\,
      \colvector{4\\1\\2\\1\\6}
    }\\
  \]
  and defined $V=\spn{R}$.  Our goal in that problem was to find a
  relation of linear dependence on the vectors in $R$, solve the
  resulting equation for one of the vectors, and re-express $V$ as the
  span of a set of three vectors.

  Here is another way to accomplish something similar.  The row space
  of the matrix
  \[
    A=\begin{bmatrix}
      1 & 2 & -1 & 3 & 2\\
      2 & 1 & 3 & 1 & 2\\
      0 & -7 & 6 & -11 & -2\\
      4 & 1 & 2 & 1 & 6
    \end{bmatrix}
  \]
  is equal to $\spn{R}$.  By \ref{theorem:BRS} we can row-reduce this
  matrix, ignore any zero rows, and use the nonzero rows as column
  vectors that are a basis for the row space of $A$.  Row-reducing $A$
  creates the matrix
  \[
    \begin{bmatrix}
      1 & 0 & 0 & -\frac{1}{17} & \frac{30}{17}\\
      0 & 1 & 0 & \frac{25}{17} & -\frac{2}{17}\\
      0 & 0 & 1 & -\frac{2}{17} & -\frac{8}{17}\\
      0 & 0 & 0 & 0 & 0
    \end{bmatrix}
  \]

  So
  \[
    \set{
      \colvector{\answer{1}\\0\\0\\-\frac{1}{17}\\\frac{30}{17}},\,
      \colvector{0\\1\\0\\\frac{25}{17}\\-\frac{2}{17}},\,
      \colvector{0\\0\\1\\-\frac{2}{17}\\-\frac{8}{17}}
    }
  \]
  is a basis for $V$.  Our theorem tells us this is a basis, there is
  no need to verify that the subspace spanned by three vectors (rather
  than four) is the identical subspace, and there is no need to verify
  that we have reached the limit in reducing the set, since the set of
  three vectors is guaranteed to be linearly independent.

\end{example}


\end{document}
