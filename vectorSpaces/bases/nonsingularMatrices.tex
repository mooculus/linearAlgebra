\documentclass{ximera}

\input{../../preamble.tex}

\title{Bases and Nonsingular Matrices}

\begin{document}
\begin{abstract}
  The columns of a nonsingular matrix form a basis.
\end{abstract}
\maketitle

A quick source of diverse bases for $\real{m}$ is the set of columns of a nonsingular matrix.

\begin{theorem}[Columns of Nonsingular Matrix are a Basis]
  \label{theorem:CNMB}

  Suppose that $A$ is a square matrix of size $m$.  Then the columns
  of $A$ are a basis of $\real{m}$ if and only if $A$ is nonsingular.
  
  \begin{proof}
    ($\Rightarrow$) Suppose that the columns of $A$ are a basis for
    $\real{m}$.  Then \ref{definition:B} says the set of columns is
    linearly independent.  \ref{theorem:NMLIC} then says that $A$ is
    \wordChoice{\choice[correct]{nonsingular}\choice{singular}}.
    
    ($\Leftarrow$) Suppose that $A$ is nonsingular.  Then by
    \ref{theorem:NMLIC} this set of columns is linearly
    \wordChoice{\choice[correct]{independent}\choice{dependent}}.
    \ref{theorem:CSNM} says that for a nonsingular matrix,
    $\csp{A}=\real{m}$.  This is equivalent to saying that the columns
    of $A$ are a spanning set for the vector space $\real{m}$.  As a
    linearly independent spanning set, the columns of $A$ qualify as a
    basis for $\real{m}$ (\ref{definition:B}).
  \end{proof}
\end{theorem}

\begin{example}[Columns as Basis]
  Consider the $5\times 5$ matrix
  \[
    K=\begin{bmatrix}
      10 & 18 & 24 & 24 & -12 \\
      12 & -2 &  -6 & 0 & -18 \\
      -30 &  -21 & -23 &  -30 & 39 \\
      27 & 30 &  36 & 37 &  -30 \\
      18 & 24 & 30 &  30 & -20
    \end{bmatrix}
  \]
  which is row-equivalent to the $5\times 5$ identity matrix $I_5$.
  So by \ref{theorem:NMRRI}, $K$ is nonsingular.  Then
  \ref{theorem:CNMB} says the set
  \[
    \set{\colvector{10\\12\\-30\\27\\18},\,\colvector{18\\-2\\-21\\30\\24},\,\colvector{24\\-6\\-23\\36\\30},\,\colvector{24\\0\\-30\\37\\30},\,\colvector{-12\\-18\\39\\-30\\-20}}
  \]
  is a (novel) basis of $\real{5}$.
\end{example}

Perhaps we should view the fact that the standard unit vectors are a
basis (\ref{theorem:SUVB}) as just a simple corollary of
\ref{theorem:CNMB}?  (See \ref{technique:LC}.)

With a new equivalence for a nonsingular matrix, we can update our list of equivalences.

\begin{theorem}[Nonsingular Matrix Equivalences, Round 5]

  Suppose that $A$ is a square matrix of size $n$.  The following are equivalent.
  \begin{enumerate}\item $A$ is nonsingular.
  \item $A$ row-reduces to the identity matrix.
  \item The null space of $A$ contains only the zero vector, $\nsp{A}=\set{\zerovector}$.
  \item The linear system $\linearsystem{A}{\vect{b}}$ has a unique solution for every possible choice of $\vect{b}$.
  \item The columns of $A$ are a linearly independent set.
  \item $A$ is invertible.
  \item The column space of $A$ is $\real{n}$, $\csp{A}=\real{n}$.
  \item The columns of $A$ are a basis for $\real{n}$.
  \end{enumerate}

  \begin{proof}
    With a new equivalence for a nonsingular matrix in \ref{theorem:CNMB} we can expand our previous theorem.
  \end{proof}
\end{theorem}

\end{document}
