\documentclass{ximera}

\input{../../preamble.tex}

\title{Rank and Nullity of a Nonsingular Matrix}

\begin{document}
\begin{abstract}
  The dimensions of the row space, column space, and null space are
  important quantities associated to a matrix.
\end{abstract}
\maketitle

Let us take a look at the rank and nullity of a square matrix.

\begin{example}[Rank and nullity of a square matrix]
  The matrix
  \[
    E=\begin{bmatrix}
      0 & 4 & -1 & 2 & 2 & 3 & 1\\
      2 & -2 & 1 & -1 & 0 & -4 & -3\\
      -2 & -3 & 9 & -3 & 9 & -1 & 9\\
      -3 & -4 & 9 & 4 & -1 & 6 & -2\\
      -3 & -4 & 6 & -2 & 5 & 9 & -4\\
      9 & -3 & 8 & -2 & -4 & 2 & 4\\
      8 & 2 & 2 & 9 & 3 & 0 & 9
    \end{bmatrix}
  \]
  is row-equivalent to the matrix in reduced row-echelon form,
  \[
    \begin{bmatrix}
      \leading{1} & 0 & 0 & 0 & 0 & 0 & 0\\
      0 & \leading{1} & 0 & 0 & 0 & 0 & 0\\
      0 & 0 & \leading{1} & 0 & 0 & 0 & 0\\
      0 & 0 & 0 & \leading{1} & 0 & 0 & 0\\
      0 & 0 & 0 & 0 & \leading{1} & 0 & 0\\
      0 & 0 & 0 & 0 & 0 & \leading{1} & 0\\
      0 & 0 & 0 & 0 & 0 & 0 & \leading{1}
    \end{bmatrix}
  \]
  
  With $n=7$ columns and $r=7$ nonzero rows \ref{theorem:CRN} tells us
  the rank is $\rank{E}=\answer{7}$ and the nullity is $\nullity{E}=\answer{0}$.
\end{example}

The value of either the nullity or the rank are enough to characterize a nonsingular matrix.

\begin{theorem}[Rank and Nullity of a Nonsingular Matrix]
  \label{theorem:RNNM}
  
  Suppose that $A$ is a square matrix of size $n$.  The following are equivalent.
  \begin{enumerate}
  \item A is nonsingular.
  \item The rank of $A$ is $n$, $\rank{A}=n$.
  \item The nullity of $A$ is zero, $\nullity{A}=0$.
  \end{enumerate}

  \begin{proof}
    (1 $\Rightarrow$ 2) \ref{theorem:CSNM} says that if $A$ is
    nonsingular then $\csp{A}=\complex{n}$.  If $\csp{A}=\complex{n}$,
    then the column space has dimension $n$ by \ref{theorem:DCM}, so
    the rank of $A$ is $n$.

    (2 $\Rightarrow$ 3)  Suppose $\rank{A}=n$.  Then \ref{theorem:RPNC} gives
    \begin{align*}
      \nullity{A}&=n-\rank{A}&&\ref{theorem:RPNC}\\
                 &=n-n&&\text{Hypothesis}\\
                 &=0
    \end{align*}

    (3 $\Rightarrow$ 1) Suppose $\nullity{A}=0$, so a basis for the
    null space of $A$ is 
    \begin{multipleChoice}
      \choice{the rows of $A$}
      \choice{the columns of $A$}
      \choice[correct]{the empty set.}
    \end{multipleChoice}
    
    \begin{feedback}[correct]
      Because the basis is $\varnothing$, this implies that
      $\nsp{A}=\set{\zerovector}$ and \ref{theorem:NMTNS} says $A$ is
      nonsingular.
    \end{feedback}
  \end{proof}
\end{theorem}

With a new equivalence for a nonsingular matrix, we can update our
list of equivalences, which now becomes a list requiring double digits
to number.

\begin{theorem}[Nonsingular Matrix Equivalences, Round 6]

  Suppose that $A$ is a square matrix of size $n$.  The following are equivalent.
  \begin{enumerate}\item $A$ is nonsingular.
  \item $A$ row-reduces to the identity matrix.
  \item The null space of $A$ contains only the zero vector, $\nsp{A}=\set{\zerovector}$.
  \item The linear system $\linearsystem{A}{\vect{b}}$ has a unique solution for every possible choice of $\vect{b}$.
  \item The columns of $A$ are a linearly independent set.
  \item $A$ is invertible.
  \item The column space of $A$ is $\complex{n}$, $\csp{A}=\complex{n}$.
  \item The columns of $A$ are a basis for $\complex{n}$.
  \item The rank of $A$ is $n$, $\rank{A}=n$.
  \item The nullity of $A$ is zero, $\nullity{A}=0$.
  \end{enumerate}

  \begin{proof}
    We can add two of the statements from \ref{theorem:RNNM}.
  \end{proof}
\end{theorem}

\end{document}
