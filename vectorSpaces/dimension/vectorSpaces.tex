\documentclass{ximera}

\input{../../preamble.tex}

\title{Dimension of Vector Spaces}

\begin{document}
\begin{abstract}
  We collect the dimension of some common, and not so common, vector spaces.
\end{abstract}
\maketitle

\begin{theorem}[Dimension of $\complex{m}$]
  \label{theorem:DCM}

  The dimension of $\complex{m}$ (\ref{example:VSCV}) is $m$.

  \begin{proof}
    \ref{theorem:SUVB} provides a basis with $m$ vectors.
  \end{proof}
\end{theorem}

\begin{theorem}[Dimension of $P_n$]
  \label{theorem:DP}

  The dimension of $P_{n}$  (\ref{example:VSP}) is $\answer{n+1}$.

  \begin{feedback}[correct]
    \begin{proof}
      \ref{example:BP} provides \textit{two} bases with $n+1$ vectors.  Take your pick.
    \end{proof}
  \end{feedback}
\end{theorem}

\begin{theorem}[Dimension of $M_{mn}$]
  \label{theorem:DM}

  The dimension of $M_{mn}$  (\ref{example:VSM}) is $mn$.
  
  \begin{proof}
    \ref{example:BM} provides a basis with $mn$ vectors.
  \end{proof}
\end{theorem}

\begin{example}[Dimension of a subspace of $M_{22}$]
  It should now be plausible that
  \[
    Z=\setparts{\begin{bmatrix}a&b\\c&d\end{bmatrix}}{2a+b+3c+4d=0,\,-a+3b-5c-2d=0}
  \]
  is a subspace of the vector space $M_{22}$ (\ref{example:VSM}).  

  To find the dimension of $Z$ we must first find a basis, though any old basis will do.

  First concentrate on the conditions relating $a,\,b,\,c$ and $d$.  They form a homogeneous system of two equations in four variables with coefficient matrix
  \[
    \begin{bmatrix}
      2 & 1 & 3 & 4\\
      -1 & 3 & -5 & -2
    \end{bmatrix}
  \]

  We can row-reduce this matrix to obtain
  \[
    \begin{bmatrix}
      \leading{1} & 0 & 2 & 2\\
      0 & \leading{1} & -1 & 0
    \end{bmatrix}
  \]
  
  Rewrite the two equations represented by each row of this matrix, expressing the dependent variables ($a$ and $b$) in terms of the free variables ($c$ and $d$), and we obtain,
  \begin{align*}
    a&=-2c-2d\\
    b&=c
  \end{align*}

  We can now write a typical entry of $Z$ strictly in terms of $c$ and $d$, and we can decompose the result,
  \[
    \begin{bmatrix}a&b\\c&d\end{bmatrix}=
    \begin{bmatrix}-2c-2d&c\\c&d\end{bmatrix}=
    \begin{bmatrix}-2c&c\\c&0\end{bmatrix}+
    \begin{bmatrix}-2d&0\\0&d\end{bmatrix}=
    c\begin{bmatrix}-2&1\\1&0\end{bmatrix}+
    d\begin{bmatrix}-2&0\\0&1\end{bmatrix}
  \]

  This equation says that an arbitrary matrix in $Z$ can be written as a linear combination of the two vectors in
  \[
    S=\set{\begin{bmatrix}-2&1\\1&0\end{bmatrix},\,\begin{bmatrix}-2&0\\0&1\end{bmatrix}}
  \]
  so we know that
  \[
    Z=\spn{S}=
    \spn{\set{
        \begin{bmatrix}-2&1\\1&0\end{bmatrix},\,
        \begin{bmatrix}-2&0\\0&1\end{bmatrix}
      }}
  \]

  Are these two matrices (vectors) also linearly independent?  Begin with a relation of linear dependence on $S$,
  \begin{align*}
    a_1\begin{bmatrix}-2&1\\1&0\end{bmatrix}+
                               a_2\begin{bmatrix}-2&0\\0&1\end{bmatrix}&=\zeromatrix\\
    \begin{bmatrix}-2a_1-2a_2&a_1\\a_1&a_2\end{bmatrix}&=
                                                         \begin{bmatrix}0&0\\0&0\end{bmatrix}
  \end{align*}
  
  From the equality of the two entries in the last row, we conclude
  that $a_1=0$, $a_2=0$.  Thus the only possible relation of linear
  dependence is the trivial one, and therefore $S$ is linearly
  independent (\ref{definition:LI}).  So $S$ is a basis for $Z$
  (\ref{definition:B}).  Finally, we can conclude that
  $\dimension{Z}=2$ (\ref{definition:D}) since $S$ has two elements.
\end{example}

\begin{example}[Dimension of a subspace of $P_4$]
  We showed that
  \[
    S=\set{x-2,\,x^2-4x+4,\,x^3-6x^2+12x-8,\,x^4-8x^3+24x^2-32x+16}
  \]
  is a basis for $W=\setparts{p(x)}{p\in P_4,\ p(2)=0}$.  Thus, the
  dimension of $W$ is four, $\dimension{W}=4$.

  Note that $\dimension{P_4}=5$ by \ref{theorem:DP}, so $W$ is a
  subspace of dimension 4 within the vector space $P_4$ of dimension
  5, illustrating the upcoming \ref{theorem:PSSD}.
\end{example}

\begin{example}[Dimension of the crazy vector space]
  In \ref{example:BC} we determined that the set
  $R=\set{(1,\,0),\,(6,\,3)}$ from the crazy vector space, $C$
  (\ref{example:CVS}), is a basis for $C$.  By \ref{definition:D} we
  see that $C$ has dimension 2, $\dimension{C}=2$.
\end{example}

It is possible for a vector space to have no finite bases, in which
case we say it has infinite dimension.  Many of the best examples of
this are vector spaces of functions, which lead to constructions like
Hilbert spaces.  We will focus exclusively on finite-dimensional
vector spaces.  OK, one infinite-dimensional example, and
\textit{then} we will focus exclusively on finite-dimensional vector
spaces.

\begin{example}[Vector space of polynomials with unbounded degree]
  Define the set $P$ by
  \[
    P=\setparts{p}{p(x)\text{ is a polynomial in }x}
  \]

  Our operations will be the same as those defined for $P_n$
  (\ref{example:VSP}).

  With no restrictions on the possible degrees of our polynomials, any
  finite set that is a candidate for spanning $P$ will come up short.
  We will give a proof by contradiction (\ref{technique:CD}).  To this
  end, suppose that the dimension of $P$ is finite, say
  $\dimension{P}=n$.

  The set $T=\set{1,\,x,\,x^2,\,\ldots,\,x^n}$ is a linearly
  independent set (check this!) containing $n+1$ polynomials from $P$.
  However, a basis of $P$ will be a spanning set of $P$ containing $n$
  vectors.  This situation is a contradiction of \ref{theorem:SSLD},
  so our assumption that $P$ has finite dimension is false.  Thus, we
  say $\dimension{P}=\infty$.
\end{example}

\end{document}
