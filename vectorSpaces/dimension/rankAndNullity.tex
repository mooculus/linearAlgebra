\documentclass{ximera}

\input{../../preamble.tex}

\title{Rank and Nullity of a Matrix}

\begin{document}
\begin{abstract}
  The dimensions of the row space, column space, and null space are
  important quantities associated to a matrix.
\end{abstract}
\maketitle

For any matrix, we have seen that we can associate several
subspaces---the null space (\ref{theorem:NSMS}), the column space
(\ref{theorem:CSMS}), row space (\ref{theorem:RSMS}) and the left null
space (\ref{theorem:LNSMS}).  As vector spaces, each of these has a
dimension, and for the null space and column space, they are important
enough to warrant names.

\begin{definition}[Nullity Of a Matrix]
  Suppose that $A$ is an $m\times n$ matrix.  Then the \dfn{nullity}
  of $A$ is the dimension of the null space of $A$,
  $\nullity{A}=\dimension{\nsp{A}}$.
\end{definition}

\begin{definition}[Rank Of a Matrix]
  Suppose that $A$ is an $m\times n$ matrix.  Then the \dfn{rank} of
  $A$ is the dimension of the column space of $A$,
  $\rank{A}=\dimension{\csp{A}}$.
\end{definition}

\begin{example}[Rank and nullity of a matrix]
  Let us compute the rank and nullity of
  \[
    A=\begin{bmatrix}
      2 & -4 & -1 & 3 & 2 & 1 & -4\\
      1 & -2 & 0 & 0 & 4 & 0 & 1\\
      -2 & 4 & 1 & 0 & -5 & -4 & -8\\
      1 & -2 & 1 & 1 & 6 & 1 & -3\\
      2 & -4 & -1 & 1 & 4 & -2 & -1\\
      -1 & 2 & 3 & -1 & 6 & 3 & -1
    \end{bmatrix}
  \]

  To do this, we will first row-reduce the matrix since that will help us determine bases for the null space and column space.
  \[
    \begin{bmatrix}
      \leading{1} & -2 & 0 & 0 & 4 & 0 & 1\\
      0 & 0 & \leading{1} & 0 & 3 & 0 & -2\\
      0 & 0 & 0 & \leading{1} & -1 & 0 & -3\\
      0 & 0 & 0 & 0 & 0 & \leading{1} & 1\\
      0 & 0 & 0 & 0 & 0 & 0 & 0\\
      0 & 0 & 0 & 0 & 0 & 0 & 0
    \end{bmatrix}
  \]

  From this row-equivalent matrix in reduced row-echelon form we
  record $D=\set{1,\,3,\,4,\,6}$ and $F=\set{2,\,5,\,7}$.

  For each index in $D$, \ref{theorem:BCS} creates a single basis
  vector.  In total the basis will have $4$ vectors, so the column
  space of $A$ will have dimension $4$ and we write
  $\rank{A}=\answer{4}$.

  For each index in $F$, \ref{theorem:BNS} creates a single basis
  vector.  In total the basis will have $3$ vectors, so the null space
  of $A$ will have dimension $3$ and we write
  $\nullity{A}=\answer{3}$.
\end{example}

There were no accidents or coincidences in the previous example---with
the row-reduced version of a matrix in hand, the rank and nullity are
easy to compute.

\begin{theorem}[Computing Rank and Nullity]
  \label{theorem:CRN}
  Suppose that $A$ is an $m\times n$ matrix and $B$ is a
  row-equivalent matrix in reduced row-echelon form.  Let $r$ denote
  the number of pivot columns (or the number of nonzero rows).  Then
  $\rank{A}=r$ and $\nullity{A}=n-r$.

  \begin{proof}
    \ref{theorem:BCS} provides a basis for the column space by
    choosing columns of $A$ that that have the same indices as the
    pivot columns of $B$.  In the analysis of $B$, each leading 1
    provides one nonzero row and one pivot column.  So there are $r$
    column vectors in a basis for $\csp{A}$.

    \ref{theorem:BNS} provides a basis for the null space by creating
    basis vectors of the null space of $A$ from entries of $B$, one
    basis vector for each column that is \textit{not} a pivot column.
    So there are $n-r$ column vectors in a basis for $\nullity{A}$.
  \end{proof}
\end{theorem}

\begin{question}
  You may have noticed as that the larger the column space is the
  \wordChoice{\choice[correct]{smaller}\choice{larger}} the null space
  is.  A simple corollary states this trade-off succinctly.
\end{question}

\begin{theorem}[Rank Plus Nullity is Columns]
  \label{theorem:RPNC}

  Suppose that $A$ is an $m\times n$ matrix.  Then
  $\rank{A}+\nullity{A}=n$.

  \begin{proof}
    Let $r$ be the number of nonzero rows in a row-equivalent matrix
    in reduced row-echelon form.  By \ref{theorem:CRN},
    \[
      \rank{A}+\nullity{A}= r+(n-r)=n
    \]
  \end{proof}
\end{theorem}

When we first introduced $r$ as our standard notation for the number
of nonzero rows in a matrix in reduced row-echelon form you might have
thought $r$ stood for ``rows.''  Not really---it stands for ``rank''!

\end{document}
