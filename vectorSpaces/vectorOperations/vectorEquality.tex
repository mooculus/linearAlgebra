\documentclass{ximera}

\input{../../preamble.tex}

\title{Vector Equality}

\begin{document}
\begin{abstract}
  We continue our study of vectors by first defining what it means for
  two vectors to be the same.
\end{abstract}
\maketitle

\begin{definition}[Column Vector Equality]
  Suppose that $\vect{u},\,\vect{v}\in\complex{m}$.  Then $\vect{u}$
  and $\vect{v}$ are \dfn{equal}, written $\vect{u}=\vect{v}$ if
  \begin{align*}
    \vectorentry{\vect{u}}{i}&=\vectorentry{\vect{v}}{i}
    &&1\leq i\leq m
  \end{align*}
\end{definition}

Now this may seem like a silly thing to say so carefully.  Of course
two vectors are equal if they are equal for each corresponding entry!
Well, this is not as silly as it appears.  We will see a few occasions
later where the obvious definition is \textit{not} the right one.  And
besides, in doing mathematics we need to be very careful about making
all the necessary definitions and making them unambiguous.

Let us do an example of vector equality that begins to hint at the
utility of this definition.

\begin{question}
  Consider two vectors $\vect{a}, \vect{b} \in \complex{3}$ with the
  property that $\vectorentry{\vect{a}}{n} = n^2$ and
  $\vect{b} = \left\langle 1, 4, 9 \right\rangle$.

  Is it the case that $\vect{a} = \vect{b}$?

  \begin{multipleChoice}
    \choice{No.}
    \choice[correct]{Yes.}
    \choice{It depends.}
  \end{multipleChoice}

  \begin{feedback}[correct]
    Notice now that the symbol ``='' is now doing triple-duty.  We
    know from our earlier education what it means for two numbers
    (real or complex) to be equal, and we take this for granted.  We
    already defined what it meant for two sets to be equal.  Now we
    have defined what it means for two vectors to be equal, and that
    definition builds on our definition for when two numbers are equal
    when we use the condition $u_i=v_i$ for all $1\leq i\leq m$.  So
    think carefully about your objects when you see an equal sign and
    think about just which notion of equality you have encountered.
    This will be especially important when you are asked to construct
    proofs whose conclusion states that two objects are equal.
  \end{feedback}

\end{question}

\begin{example}
  Consider the system of linear equations
  \begin{align*}
    -7x_1 -6 x_2 - 12x_3 &=-33\\
    5x_1  + 5x_2 + 7x_3 &=24\\
    x_1 +4x_3 &=5
  \end{align*}
  Note the use of three equals signs---each indicates an equality of
  \wordChoice{\choice[correct]{numbers}\choice{vectors}} (the linear
  expressions are numbers when we evaluate them with fixed values of
  the variable quantities).  Now write the vector equality,
  \[
    \colvector{-7x_1 -6 x_2 - 12x_3\\ 5x_1  + 5x_2 + 7x_3\\ x_1 +4x_3}
    =
    \colvector{-33\\24\\5}.
  \]
  This \textit{single} equality (of two
  \wordChoice{\choice{numbers}\choice[correct]{column vectors}})
  translates into \textit{three} simultaneous equalities of
  \wordChoice{\choice[correct]{numbers}\choice{vectors}} that form the
  system of equations.  So with this new notion of vector equality we
  can become less reliant on referring to \textit{systems} of
  \textit{simultaneous} equations.  There is more to vector equality
  than just this, but this is a good example for starters and we will
  develop it further.
\end{example}

\end{document}
