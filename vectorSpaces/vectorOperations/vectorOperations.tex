\documentclass{ximera}

\input{../../preamble.tex}

\title{Vector Operations}

\begin{document}
\begin{abstract}
  We define two operations involving vectors, and collect some basic
  properties of these operations.
\end{abstract}
\maketitle

We will now define two operations on the set $\complex{m}$.  By this
we mean well-defined procedures that somehow convert vectors into
other vectors.  Here are two of the most basic definitions of the
entire course.

\begin{definition}[Column Vector Addition]
  Suppose that $\vect{u},\,\vect{v}\in\complex{m}$. The \dfn{sum} of
  $\vect{u}$ and $\vect{v}$ is the vector $\vect{u}+\vect{v}$ defined
  by
  \begin{align*}
    \vectorentry{\vect{u}+\vect{v}}{i}
    &=\vectorentry{\vect{u}}{i}+\vectorentry{\vect{v}}{i}
    &&1\leq i\leq m
  \end{align*}
\end{definition}

So vector addition takes two vectors of the same size and combines
them to create a new vector of the same size.

Notice that this definition is required, even if we agree that this is
the obvious, right, natural or correct way to do it.  Notice too that
the symbol $+$ is being recycled.  We all know how to add
\textit{numbers}, but now we have the same symbol extended to
double-duty and we use it to indicate how to add two new objects,
\textit{vectors.}  And this definition of our new meaning is built on
our previous meaning of addition via the expressions $u_i+v_i$.

\begin{example}[Addition of two vectors in $\complex{4}$]
  If
  \begin{align*}
    \vect{u}=\colvector{2\\-3\\4\\2}&&\vect{v}=\colvector{-1\\5\\2\\-7}
  \end{align*}
  then
  \[\vect{u}+\vect{v}=
    \colvector{\answer{2}\\-3\\4\\2}+\colvector{-1\\5\\2\\-7}=
    \colvector{2+(-1)\\-3+5\\4+2\\2+(-7)}=
    \colvector{\answer{1}\\\answer{2}\\6\\-5}\]
\end{example}

Our second operation takes two objects of different types,
specifically a number and a vector, and combines them to create
another vector.  In this context we call a number a \dfn{scalar} in
order to emphasize that it is not a vector.

\begin{definition}[Column Vector Scalar Multiplication]
  Suppose $\vect{u}\in\complex{m}$ and $\alpha\in\complexes$, then the
  \dfn{scalar multiple} of $\vect{u}$ by $\alpha$ is the vector
  $\alpha\vect{u}$ defined by
  \begin{align*}
    \vectorentry{\alpha\vect{u}}{i}
    &=\alpha\vectorentry{\vect{u}}{i}
    &&1\leq i\leq m
  \end{align*}
\end{definition}

Notice that we are doing a kind of multiplication here, but we are
\textit{defining} a new type, perhaps in what appears to be a natural
way.  We use juxtaposition to denote this operation rather than using
a symbol like we did with vector addition.  So this can be another
source of confusion.  When two symbols are next to each other, are we
doing regular old multiplication or are we doing scalar vector
multiplication, the operation we just defined?  Think about your
objects---if the first object is a scalar, and the second is a vector,
then it \textit{must} be that we are doing our new operation, and the
\textit{result} of this operation will be another vector.

\begin{warning}
  Notice how consistency in notation can be an aid here.  If we write
  scalars as lower case Greek letters from the start of the alphabet
  (such as $\alpha$, $\beta$, \ldots ) and write vectors in Latin
  letters from the end of the alphabet ($\vect{u}$, $\vect{v}$, \ldots
  ), then we have some hints about what type of objects we are working
  with.  This can be a blessing \textit{and} a curse, since when we go
  read another book about linear algebra, or read an application in
  another discipline (physics, economics, ldots ) the types of
  notation employed may be very different and hence unfamiliar.
\end{warning}

Computationally, vector scalar multiplication is easy.

\begin{example}[Scalar multiplication in $\complex{5}$]
  If
  \[
    \vect{u}=\colvector{3\\1\\-2\\4\\-1}
  \]
  and $\alpha=6$, then
  \[
    \alpha\vect{u}=
    6\colvector{3\\1\\\answer{-2}\\4\\-1}=
%    \colvector{6(3)\\6(1)\\6(-2)\\6(4)\\6(-1)}=
    \colvector{18\\6\\\answer{-12}\\24\\-6}.
  \]
\end{example}

\end{document}
