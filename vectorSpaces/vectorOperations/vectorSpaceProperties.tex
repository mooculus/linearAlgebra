\documentclass{ximera}

\input{../../preamble.tex}

\title{Vector Space Properties}

\begin{document}
\begin{abstract}
  There are key properties satisfied by vector addition and scalar multiplication.
\end{abstract}
\maketitle

With definitions of vector addition and scalar multiplication we can
state, and prove, several properties of each operation, and some
properties that involve their interplay.  We collect ten of them here.

\begin{theorem}[Vector Space Properties of Column Vectors]\label{theorem:VSPCV}
  Suppose that $\complex{m}$ is the set of column vectors of size $m$
  with addition and scalar multiplication as defined before.  Then
  \begin{description}
  \item[Additive Closure, Column Vectors]
    If $\vect{u},\,\vect{v}\in\complex{m}$, then $\vect{u}+\vect{v}\in\complex{m}$.
  \item[Scalar Closure, Column Vectors]
    If $\alpha\in\complexes$ and $\vect{u}\in\complex{m}$, then $\alpha\vect{u}\in\complex{m}$.
  \item[Commutativity, Column Vectors]
    If $\vect{u},\,\vect{v}\in\complex{m}$, then $\vect{u}+\vect{v}=\vect{v}+\vect{u}$.
  \item[Additive Associativity, Column Vectors]
    If $\vect{u},\,\vect{v},\,\vect{w}\in\complex{m}$, then $\vect{u}+\left(\vect{v}+\vect{w}\right)=\left(\vect{u}+\vect{v}\right)+\vect{w}$.
  \item[Zero Vector, Column Vectors]
    There is a vector, $\zerovector$, called the \dfn{zero vector}, such that  $\vect{u}+\zerovector=\vect{u}$  for all $\vect{u}\in\complex{m}$.
  \item[Additive Inverses, Column Vectors]
    If  $\vect{u}\in\complex{m}$, then there exists a vector $\vect{-u}\in\complex{m}$ so that $\vect{u}+ (\vect{-u})=\zerovector$.
  \item[Scalar Multiplication Associativity, Column Vectors]
    If $\alpha,\,\beta\in\complexes$ and $\vect{u}\in\complex{m}$, then $\alpha(\beta\vect{u})=(\alpha\beta)\vect{u}$.
  \item[Distributivity across Vector Addition, Column Vectors]
    If $\alpha\in\complexes$ and $\vect{u},\,\vect{v}\in\complex{m}$, then $\alpha(\vect{u}+\vect{v})=\alpha\vect{u}+\alpha\vect{v}$.
  \item[Distributivity across Scalar Addition, Column Vectors]
    If $\alpha,\,\beta\in\complexes$ and $\vect{u}\in\complex{m}$, then
    $(\alpha+\beta)\vect{u}=\alpha\vect{u}+\beta\vect{u}$.
  \item[One, Column Vectors]
    If $\vect{u}\in\complex{m}$, then $1\vect{u}=\vect{u}$.
  \end{description}

\begin{proof}
  While some of these properties seem very obvious, they all require
  proof.  However, the proofs are not very interesting, and border on
  tedious. We will prove one version of distributivity very carefully,
  and you can test your proof-building skills on some of the others.
  We need to establish an equality, so we will do so by beginning with
  one side of the equality, apply various definitions and theorems
  (listed to the right of each step) to massage the expression from
  the left into the expression on the right.  Here we go with a proof
  of one of these properties.

  For $1\leq i\leq m$,
  \begin{align*}
    \vectorentry{(\alpha+\beta)\vect{u}}{i}
    &=(\alpha+\beta)\vectorentry{\vect{u}}{i} \\
%    &&\ref{definition:CVSM}\\
    &=\alpha\vectorentry{\vect{u}}{i}+\beta\vectorentry{\vect{u}}{i} \\
%    &&\ref{property:DCN}\\
    &=\vectorentry{\alpha\vect{u}}{i}+\vectorentry{\beta\vect{u}}{i} \\
%    &&\ref{definition:CVSM}\\
    &=\vectorentry{\alpha\vect{u}+\beta\vect{u}}{i}
%    &&\ref{definition:CVA}\\
  \end{align*}

  Since the individual components of the vectors
  $(\alpha+\beta)\vect{u}$ and $\alpha\vect{u}+\beta\vect{u}$ are
  equal for \textit{all} $i$, $1\leq i\leq m$, we conclude the vectors
  are equal.
\end{proof}
\end{theorem}

Many of the conclusions of our theorems can be characterized as
``identities,'' especially when we are establishing basic properties
of operations such as those in this section.  Most of the properties
listed above are examples.

Be careful with the notion of the vector $\vect{-u}$.  This is a
vector that we add to $\vect{u}$ so that the result is the particular
vector $\zerovector$.  This is basically a property of vector
addition.  It happens that we can compute $\vect{-u}$ using the
\textit{other} operation, scalar multiplication.  We can prove this
directly by writing that
\[
  \vectorentry{\vect{-u}}{i}
  =-\vectorentry{\vect{u}}{i}
  =(\answer{-1})\vectorentry{\vect{u}}{i}
  =\vectorentry{(-1)\vect{u}}{i}
\]
We will see later how to derive this property as a
\textit{consequence} of several of the ten properties listed above.

\begin{question}
  Similarly, we will often write something you would immediately
  recognize as ``vector subtraction.''  This could be placed on a firm
  theoretical foundation, by defining ``$\vect{u}-\vect{v}$'' by
  \begin{align*}
    \vectorentry{\vect{u}-\vect{v}}{i}=\vectorentry{\vect{u}}{i}-\vectorentry{\vect{v}}{i}&&1\leq
    i\leq m
  \end{align*}

  Is vector subtraction commutative?
  \begin{multipleChoice}
    \choice[correct]{No.}
    \choice{Yes.}
  \end{multipleChoice}

  Is vector subtraction associative?
  \begin{multipleChoice}
    \choice[correct]{No.}
    \choice{Yes.}
  \end{multipleChoice}

\end{question}

\begin{question}
  For all $\vect{a}, \vect{b}, \vect{c} \in \complex{5}$, which two
  expressions below are equal?
  \begin{selectAll}
    \choice[correct]{$\left(\vect{a} + \vect{b} \right) + \vect{c}$}
    \choice{$\vect{a} - \left( \vect{b} - \vect{c} \right)$}
    \choice{$\left(\vect{a} - \vect{b} \right) - \vect{c}$}
    \choice[correct]{$\vect{a} + \left( \vect{b} + \vect{c} \right)$}
  \end{selectAll}
  
  \begin{feedback}
    Associativity implies that we do not have to be careful about how
    we ``parenthesize'' the addition of vectors.  In other words,
    there is nothing to be gained by writing
    $\left(\vect{u}+\vect{v}\right)+\left(\vect{w}+\left(\vect{x}+\vect{y}\right)\right)$
    rather than $\vect{u}+\vect{v}+\vect{w}+\vect{x}+\vect{y}$, since
    we get the same result no matter which order we choose to perform
    the four additions.  So we will not be careful about using
    parentheses this way.
\end{feedback}
\end{question}

\end{document}