\documentclass{ximera}

\input{../../preamble.tex}

\title{Vector Spaces}

\begin{document}
\begin{abstract}
  An example of a vector space is the result of collecting together
  all possible vectors of a fixed size.
\end{abstract}
\maketitle

Begin by recalling our definition of a column vector as an ordered
list of complex numbers, written vertically.  The collection of all
possible vectors of a fixed size is a commonly used set, so we start
with its definition.

\begin{definition}[Vector Space of Column Vectors]
  The vector space $\complex{m}$ is the set of all column vectors of
  size $m$ with entries from the set of complex numbers, $\complexes$.

  When a set similar to this is defined using only column vectors
  where all the entries are from the real numbers, it is written as
  ${\mathbb R}^m$ and is known as \dfn{Euclidean $m$-space}.
\end{definition}

\begin{warning}
  Often people conflate the notion of vector with that of a point.  To
  relate these ideas, we can construct an arrow from the origin to the
  point which is consistent with the notion that a vector has
  direction and magnitude.
\end{warning}

\begin{question}
  The term \dfn{vector} is used in a variety of different ways.  We
  have defined it as an ordered list written vertically.  A vector
  could simply be
  \begin{multipleChoice}
    \choice[correct]{an ordered list of numbers}
    \choice{a set of numbers}
  \end{multipleChoice}

  \begin{feedback}[correct]
    A vector could be written as
    $\left\langle 2,\,3,\,-1,\,6\right\rangle$.  Or it could be
    interpreted as a point in $m$ dimensions, such as
    $\left(3,\,4,\,-2\right)$ representing a point in three dimensions
    relative to $x$, $y$ and $z$ axes.

    All of these ideas---points, vectors with direction and magnitude,
    ordered lists numbers of numbers---can be shown to be related and
    equivalent, so keep that in mind as you connect the ideas of this
    course with ideas from other disciplines.

    For now, we will stick with the idea that a vector is just a list
    of numbers, in some particular order.
  \end{feedback}
\end{question}

\end{document}
