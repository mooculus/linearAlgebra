\documentclass{ximera}

\input{../../preamble.tex}

\title{Particular Solutions, Homogeneous Solutions}

\begin{document}
\begin{abstract}
  To find all of the solutions to a linear system of equations, it is
  sufficient to find just one solution, and then find all of the
  solutions to the corresponding homogeneous system.
\end{abstract}
\maketitle

The next theorem tells us that in order to find all of the solutions
to a linear system of equations, it is sufficient to find just one
solution, and then find all of the solutions to the corresponding
homogeneous system.  This explains part of our interest in the null
space, the set of all solutions to a homogeneous system.

\begin{theorem}[Particular Solution Plus Homogeneous Solutions]
  \label{theorem:PSPHS}
  Suppose that $\vect{w}$ is one solution to the linear system of
  equations $\linearsystem{A}{\vect{b}}$.  Then $\vect{y}$ is a
  solution to $\linearsystem{A}{\vect{b}}$ if and only if
  $\vect{y}=\vect{w}+\vect{z}$ for some vector $\vect{z}\in\nsp{A}$.

  \begin{proof}
    Let $\vectorlist{A}{n}$ be the columns of the coefficient matrix $A$.

    ($\Leftarrow$) Suppose $\vect{y}=\vect{w}+\vect{z}$ and
    $\vect{z}\in\nsp{A}$. Then
    \begin{align*}
      \vect{b}
      &=
        \vectorentry{\vect{w}}{1}\vect{A}_1+
        \vectorentry{\vect{w}}{2}\vect{A}_2+
        \vectorentry{\vect{w}}{3}\vect{A}_3+
        \cdots+
        \vectorentry{\vect{w}}{n}\vect{A}_n
      &&\ref{theorem:SLSLC}\\
      &=
        \vectorentry{\vect{w}}{1}\vect{A}_1+
        \vectorentry{\vect{w}}{2}\vect{A}_2+
        \vectorentry{\vect{w}}{3}\vect{A}_3+
        \cdots+
        \vectorentry{\vect{w}}{n}\vect{A}_n
        +\zerovector
      &&\ref{property:ZC}\\
      &=
        \vectorentry{\vect{w}}{1}\vect{A}_1+
        \vectorentry{\vect{w}}{2}\vect{A}_2+
        \vectorentry{\vect{w}}{3}\vect{A}_3+
        \cdots+
        \vectorentry{\vect{w}}{n}\vect{A}_n
      &&\ref{theorem:SLSLC}\\
      &\quad\quad
        +
        \vectorentry{\vect{z}}{1}\vect{A}_1+
        \vectorentry{\vect{z}}{2}\vect{A}_2+
        \vectorentry{\vect{z}}{3}\vect{A}_3+
        \cdots+
        \vectorentry{\vect{z}}{n}\vect{A}_n\\
      &=
        \left(\vectorentry{\vect{w}}{1}+\vectorentry{\vect{z}}{1}\right)\vect{A}_1+
        \left(\vectorentry{\vect{w}}{2}+\vectorentry{\vect{z}}{2}\right)\vect{A}_2
      &&\ref{theorem:VSPCV}\\
      &\quad\quad
        \left(\vectorentry{\vect{w}}{3}+\vectorentry{\vect{z}}{3}\right)\vect{A}_3+
        \cdots+
        \left(\vectorentry{\vect{w}}{n}+\vectorentry{\vect{z}}{n}\right)\vect{A}_n\\
      &=
        \vectorentry{\vect{w}+\vect{z}}{1}\vect{A}_1+
        \vectorentry{\vect{w}+\vect{z}}{2}\vect{A}_2+
        \cdots+
        \vectorentry{\vect{w}+\vect{z}}{n}\vect{A}_n
      &&\ref{definition:CVA}\\
      &=
        \vectorentry{\vect{y}}{1}\vect{A}_1+
        \vectorentry{\vect{y}}{2}\vect{A}_2+
        \vectorentry{\vect{y}}{3}\vect{A}_3+
        \cdots+
        \vectorentry{\vect{y}}{n}\vect{A}_n
      &&\text{Definition of $\vect{y}$}
    \end{align*}
    Applying \ref{theorem:SLSLC} we see that the vector $\vect{y}$ is
    a solution to $\linearsystem{\answer{A}}{\vect{b}}$.

    ($\Rightarrow$) Suppose $\vect{y}$ is a solution to
    $\linearsystem{A}{\vect{b}}$.  Then
    \begin{align*}
      \zerovector
      &=
        \vect{b}-\vect{b}&&\text{}\\
      &=
        \vectorentry{\vect{y}}{1}\vect{A}_1+
        \vectorentry{\vect{y}}{2}\vect{A}_2+
        \vectorentry{\vect{y}}{3}\vect{A}_3+
        \cdots+
        \vectorentry{\vect{y}}{n}\vect{A}_n
                         &&\ref{theorem:SLSLC}\\
      &\quad\quad
        -
        \left(
        \vectorentry{\vect{w}}{1}\vect{A}_1+
        \vectorentry{\vect{w}}{2}\vect{A}_2+
        \vectorentry{\vect{w}}{3}\vect{A}_3+
        \cdots+
        \vectorentry{\vect{w}}{n}\vect{A}_n
        \right)\\
      &=
        \left(\vectorentry{\vect{y}}{1}-\vectorentry{\vect{w}}{1}\right)\vect{A}_1+
        \left(\vectorentry{\vect{y}}{2}-\vectorentry{\vect{w}}{2}\right)\vect{A}_2
                         &&\ref{theorem:VSPCV}\\
      &\quad\quad
        +
        \left(\vectorentry{\vect{y}}{3}-\vectorentry{\vect{w}}{3}\right)\vect{A}_3+
        \cdots+
        \left(\vectorentry{\vect{y}}{n}-\vectorentry{\vect{w}}{n}\right)\vect{A}_n\\
      &=
        \vectorentry{\vect{y}-\vect{w}}{1}\vect{A}_1+
        \vectorentry{\vect{y}-\vect{w}}{2}\vect{A}_2
                         &&\ref{definition:CVA}\\
      &\quad\quad
        +\vectorentry{\vect{y}-\vect{w}}{3}\vect{A}_3+
        \cdots+
        \vectorentry{\vect{y}-\vect{w}}{n}\vect{A}_n
    \end{align*}
    By \ref{theorem:SLSLC} we see that the vector $\vect{y}-\vect{w}$
    is a solution to the homogeneous system $\homosystem{A}$ and by
    \ref{definition:NSM}, $\vect{y}-\vect{w}\in\nsp{A}$.  In other
    words, $\vect{y}-\vect{w}=\vect{z}$ for some vector
    $\vect{z}\in\nsp{A}$.  Rewritten, this is
    %$\vect{y}=\vect{w}+\vect{z}$
    $\vect{y}=\answer{w+z}$, as desired.
  \end{proof}
\end{theorem}

Nonsingular coefficient matrices lead to unique solutions for every
choice of the vector of constants.  What does this say about singular
matrices?  A singular matrix $A$ has a nontrivial null space
(\ref{theorem:NMTNS}).  For a given vector of constants, $\vect{b}$,
the system $\linearsystem{A}{\vect{b}}$ could be inconsistent, meaning
there are no solutions.  But if there is at least one solution
($\vect{w}$), then \ref{theorem:PSPHS} tells us there will be
infinitely many solutions because of the role of the infinite null
space for a singular matrix.  So a system of equations with a singular
coefficient matrix \textit{never} has a unique solution.  With a
singular coefficient matrix, either there are no solutions, or
infinitely many solutions, depending on the choice of the vector of
constants ($\vect{b}$).

\begin{example}[Particular solutions, homogeneous solutions]
  The system
  \begin{align*}
    2x_1  + x_2 + 7x_3 - 7x_4 &= 8 \\
    -3x_1 + 4x_2 -5x_3 - 6x_4 &=  -12 \\
    x_1 +x_2 + 4x_3 - 5x_4 &=  4
  \end{align*}
  is a consistent system of equations with a nontrivial null space.
  Let $A$ denote the coefficient matrix of this system.  The write-up
  for this system begins with three solutions,
  \begin{align*}
    \vect{y}_1=\colvector{0\\1\\2\\1}&&
    \vect{y}_2=\colvector{4\\0\\0\\0}&&
    \vect{y}_3=\colvector{7\\8\\1\\3}
  \end{align*}

  We will choose to have $\vect{y}_1$ play the role of $\vect{w}$ in
  the statement of \ref{theorem:PSPHS}, any one of the three vectors
  listed here (or others) could have been chosen.  To illustrate the
  theorem, we should be able to write each of these three solutions as
  the vector $\vect{w}$ plus a solution to the corresponding
  homogeneous system of equations.  Since $\zerovector$ is always a
  solution to a homogeneous system we can easily write
  \[
    \vect{y}_1=\vect{w}=\vect{w}+\zerovector.
  \]

  The vectors $\vect{y}_2$ and $\vect{y}_3$ will require a bit more
  effort.  Solutions to the homogeneous system $\homosystem{A}$ are
  exactly the elements of the null space of the coefficient matrix,
  which by an application of \ref{theorem:VFSLS} is
  \[
    \nsp{A}=\setparts{
      x_3\colvector{-3\\-1\\1\\0}+x_4\colvector{2\\3\\0\\1}
    }{
      x_3,\,x_4\in\complexes
    }
  \]
  Then
  \[
    \vect{y}_2=\colvector{4\\0\\0\\0}
    =\colvector{0\\1\\2\\1}+\colvector{4\\-1\\-2\\-1}
    =\colvector{0\\1\\2\\1}+\left((-2)\colvector{-3\\-1\\1\\0}+(-1)\colvector{2\\3\\0\\1}\right)
    =\vect{w}+\vect{z}_2
  \]
  where
  \[
    \vect{z}_2
    =\colvector{4\\-1\\-2\\-1}
    =(-2)\colvector{-3\\-1\\1\\0}+(-1)\colvector{2\\3\\0\\1}
  \]
  is obviously a solution of the homogeneous system since it is
  written as a linear combination of the vectors describing the null
  space of the coefficient matrix (or as a check, you could just
  evaluate the equations in the homogeneous system with $\vect{z}_2$).

  Again
  \[
    \vect{y}_3=\colvector{7\\8\\1\\3}
    =\colvector{0\\1\\2\\1}+\colvector{7\\7\\-1\\2}
    =\colvector{0\\1\\2\\1}+\left((-1)\colvector{-3\\-1\\1\\0}+2\colvector{2\\3\\0\\1}\right)
    =\vect{w}+\vect{z}_3
  \]
  where
  \[
    \vect{z}_3=\colvector{7\\7\\-1\\2}=
    (-1)\colvector{-3\\-1\\1\\0}+2\colvector{2\\3\\0\\1}
  \]
  is obviously a solution of the homogeneous system since it is
  written as a linear combination of the vectors describing the null
  space of the coefficient matrix (or as a check, you could just
  evaluate the equations in the homogeneous system with $\vect{z}_2$).

  Here is another view of this theorem, in the context of this
  example.  Grab two new solutions of the original system of
  equations, say
  \begin{align*}
    \vect{y}_4=\colvector{11\\0\\-3\\-1}&&
    \vect{y}_5=\colvector{-4\\2\\4\\2}
  \end{align*}
  and form their difference,
  \[
    \vect{u}=\colvector{11\\0\\-3\\-1}-\colvector{-4\\2\\4\\2}=\colvector{15\\-2\\-7\\-3}.
  \]
  Is $\vect{u}$ a solution to the homogeneous system?
  \begin{multipleChoice}
    \choice{No.}
    \choice[correct]{Yes.}
  \end{multipleChoice}

  \begin{feedback}[correct]
    It is no accident that $\vect{u}$ is a solution to the homogeneous
    system (check this!).  In other words, the difference between any
    two solutions to a linear system of equations is an element of the
    null space of the coefficient matrix.  This is an equivalent way
    to state \ref{theorem:PSPHS}.
  \end{feedback}
\end{example}

The ideas of this subsection will appear again when we discuss
preimages of linear transformations..

\end{document}
