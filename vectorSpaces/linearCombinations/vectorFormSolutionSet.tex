\documentclass{ximera}

\input{../../preamble.tex}

\title{Vector Form of Solution Sets}

\begin{document}
\begin{abstract}
  We use column vectors and linear combinations to express all of the
  solutions to a linear system of equations.
\end{abstract}
\maketitle

We have written solutions to systems of equations as column vectors.
For example, we might write the solution
$x_1 = -3,\,x_2 = 5,\,x_3 = 2$ as
\[
  \vect{x}=\colvector{x_1\\x_2\\x_3}=\colvector{-3\\5\\2}
\]
Now, we will use column vectors and linear combinations to express
\textit{all} of the solutions to a linear system of equations in a
compact and understandable way.  First, here are two examples that
will motivate our next theorem.  This is a valuable technique, almost
the equal of row-reducing a matrix, so be sure you get comfortable
with it over the course of this section.

\begin{example}
  Consider the linear system of 3 equations in 4 variables:
  \begin{align*}
    2x_1  + x_2 + 7x_3 - 7x_4 &= 8, \\
    -3x_1 + 4x_2 -5x_3 - 6x_4 &=  -12, \\
    x_1 +x_2 + 4x_3 - 5x_4 &=  4.
  \end{align*}
  Row-reducing the corresponding augmented matrix yields
  \[
    \begin{bmatrix}
      \leading{1} & 0 & 3 & -2 & 4 \\
      0 & \leading{1} & 1 &  -3 & 0\\
      0 & 0 & 0 &  0 & 0
    \end{bmatrix}
  \]
  and we see $r=2$ pivot columns. Also, $D=\set{1,\,2}$ so the
  dependent variables are then $x_1$ and $x_2$.  Since
  $F=\set{3,\,4,\,5}$, the two free variables are $x_3$ and $x_4$.  We
  will express a generic solution for the system by two slightly
  different methods, though both arrive at the same conclusion.

  First, we decompose a solution vector.  Rearranging each equation
  represented in the row-reduced form of the augmented matrix by
  solving for the dependent variable in each row yields the vector
  equality,
  \begin{align*}
    \vect{x} &= \colvector{x_1\\x_2\\x_3\\x_4}=
    \colvector{4-3x_3+2x_4\\ -x_3+3x_4\\x_3\\x_4} \\
  \end{align*}
  Now we will use the definitions of column vector addition and scalar multiplication to express this vector as a linear combination,
\begin{align*}
  \vect{x}
                                  &=\colvector{4\\0\\0\\0}+
    \colvector{-3x_3\\-x_3\\x_3\\0}+
    \colvector{2x_4\\3x_4\\0\\x_4}&&\ref{definition:CVA}\\
                                  &=\colvector{4\\0\\0\\0}+
    x_3\colvector{-3\\-1\\1\\0}+
    x_4\colvector{2\\3\\0\\1}&&\ref{definition:CVSM}\\
  \end{align*}
  We will develop the same linear combination a bit quicker, using
  three steps.  While the method above is instructive, the method
  below will be our preferred approach.

  \textbf{Step 1: Building the scaffold.}  Write the vector of variables as a fixed vector,
  plus a linear combination of $n-r$ vectors, using the free variables
  as the scalars.
  \[
    \vect{x}=\colvector{x_1\\x_2\\x_3\\x_4}=
    \colvector{\ \\\ \\\ \\\ }+x_3\colvector{\ \\\ \\\ \\\ }+x_4\colvector{\ \\\ \\\ \\\ }
  \]

  \textbf{Step 2: Address the free variables.}  Use 0's and 1's to ensure equality for the entries
  of the vectors with indices in $F$ (corresponding to the free
  variables).
  \[
    \vect{x}=\colvector{x_1\\x_2\\x_3\\x_4}=
    \colvector{\ \\\ \\0\\0}+x_3\colvector{\ \\\ \\1\\0}+x_4\colvector{\ \\\ \\\answer{0}\\\answer{1}}
  \]

  \textbf{Step 3: Finish the dependent variables.}  For each dependent
  variable, use the augmented matrix to formulate an equation
  expressing the dependent variable as a constant plus multiples of
  the free variables.  Convert this equation into entries of the
  vectors that ensure equality for each dependent variable, one at a
  time.
  \begin{align*}
    x_1=4-3x_3+2x_4&&\Rightarrow&&
                                   \vect{x}=\colvector{x_1\\x_2\\x_3\\x_4}=
    \colvector{4\\\ \\0\\0}+x_3\colvector{-3\\\ \\1\\0}+x_4\colvector{\answer{2}\\\ \\0\\1}\\
    x_2=0-1x_3+3x_4&&\Rightarrow&&
                                   \vect{x}=\colvector{x_1\\x_2\\x_3\\x_4}=
    \colvector{4\\0\\0\\0}+x_3\colvector{-3\\-1\\1\\0}+x_4\colvector{2\\3\\0\\1}
  \end{align*}

  This final \textit{form} of a typical solution is especially
  pleasing and useful.  For example, we can build solutions quickly by
  choosing values for our free variables, and then compute a linear
  combination.  Such as
  \begin{align*}
    x_3=2,\,x_4=-5&&\Rightarrow&&
                                  \vect{x}=\colvector{x_1\\x_2\\x_3\\x_4}=
    \colvector{4\\0\\0\\0}+(2)\colvector{-3\\-1\\1\\0}+(-5)\colvector{2\\3\\0\\1}
    =\colvector{-12\\-17\\2\\-5}
    \end{align*}
    or,
    \begin{align*}
      x_3=1,\,x_4=3&&\Rightarrow&&
                                   \vect{x}=\colvector{x_1\\x_2\\x_3\\x_4}=
      \colvector{4\\0\\0\\0}+(1)\colvector{-3\\-1\\1\\0}+(3)\colvector{2\\3\\0\\1}
      =\colvector{7\\8\\1\\3}
  \end{align*}
  
  \begin{question}
    How many solutions are there?

    While the above form is useful for quickly creating solutions, it
    is even better because it tells us \textit{exactly} what every
    solution looks like.  We know the solution set is 
    \begin{multipleChoice}
      \choice{empty}
      \choice{finite}
      \choice[correct]{infinite}
      \end{multipleChoice}

      \begin{feedback}[correct]
        The solution set is infinite, and we can say more: we can say
        that a solution is some multiple of $\colvector{-3\\-1\\1\\0}$
        plus a multiple of $\colvector{2\\3\\0\\1}$ plus the fixed
        vector $\colvector{4\\0\\0\\0}$.  Period.  So it only takes us
        \textit{three} vectors to describe the entire infinite
        solution set, provided we also agree on how to combine the
        three vectors into a linear combination.
      \end{feedback}
    \end{question}
  \end{example}

That was an important and fundamental technique, so we will do another
example.

\begin{example}[Vector form of solutions]
  Consider a linear system of $m=5$ equations in $n=7$ variables,
  having the augmented matrix $A$.
  \[
    A=
    \begin{bmatrix}
      2 & 1 & -1 & -2 & 2 & 1 & 5 & 21 \\
      1 & 1 & -3 & 1 & 1 & 1 & 2 & -5 \\
      1 & 2 & -8 & 5 & 1 & 1 & -6 & -15 \\
      3 & 3 & -9 & 3 & 6 & 5 & 2 & -24 \\
      -2 & -1 & 1 & 2 & 1 & 1 & -9 & -30
    \end{bmatrix}
  \]
  Row-reducing we obtain the matrix
  \[
    B=
    \begin{bmatrix}
      \leading{1} & 0 & 2 & -3 & 0 & 0 & 9 &  15 \\
      0 & \leading{1} & -5 & 4 & 0 & 0 & -8 &  -10 \\
      0 & 0 & 0 & 0 & \leading{1} & 0 & -6 &  11 \\
      0 & 0 & 0 & 0 & 0 & \leading{1} & 7 &  -21 \\
      0 & 0 & 0 & 0 & 0 & 0 & 0 & 0
    \end{bmatrix}
  \]
  and we see $r=\answer{4}$ pivot columns. Also,
  $D=\set{1,\,2,\,5,\,\answer{6}}$ so the dependent variables are then
  $x_1,\,x_2,\,x_5,$ and $x_6$.  $F=\set{3,\,4,\,7,\,8}$ so the
  $n-r=3$ free variables are $x_3,\,x_4$ and $x_7$.  We will express a
  generic solution for the system by two different methods: both a
  decomposition and a construction.

  First, we will decompose a solution vector.  Rearranging each
  equation represented in the row-reduced form of the augmented matrix
  by solving for the dependent variable in each row yields the vector
  equality,
  \begin{align*}
    \vect{v}
    &= \colvector{x_1\\x_2\\x_3\\x_4\\x_5\\x_6\\x_7} \\
    &=\colvector{
      15-2x_3+3x_4-9x_7\\
    -10+5x_3-4x_4+8x_7\\
    x_3\\
    x_4\\
    11+6x_7\\
    -21-7x_7\\
    x_7
    }
  \end{align*}
  Now we will use the definitions of column vector addition and scalar multiplication to decompose this generic solution vector as a linear combination,
  \begin{align*}
    \vect{v}
                                                 &=
                                                   \colvector{15\\ -10\\ 0\\ 0\\ 11\\ -21\\ 0 }
    +
    \colvector{ -2x_3\\ 5x_3\\ x_3\\ 0\\ 0\\ 0\\ 0 }
    +
    \colvector{ 3x_4\\ -4x_4\\ 0\\ x_4\\ 0\\ 0\\ 0 }
    +
    \colvector{ -9x_7\\ 8x_7\\ 0\\ 0\\ 6x_7\\ -7x_7\\ x_7 }
                                                 &&\ref{definition:CVA}\\
                                                 &=
                                                   \colvector{15\\ -10\\ 0\\ 0\\ 11\\ -21\\ 0 }
    +
    x_3\colvector{ -2\\ 5\\ 1\\ 0\\ 0\\ 0\\ 0 }
    +
    x_4\colvector{ 3\\ -4\\ 0\\ 1\\ 0\\ 0\\ 0 }
    +
    x_7\colvector{ -9\\ 8\\ 0\\ 0\\ 6\\ -7\\ 1 }
                                                 &&\ref{definition:CVSM}
  \end{align*}
  We will now develop the same linear combination a bit quicker, using three steps.  While the method above is instructive, the method below will be our preferred approach.

  \textbf{Step 1.}  Write the vector of variables as a fixed vector,
  plus a linear combination of $n-r$ vectors, using the free variables
  as the scalars.
  \[
    \vect{x}=
    \colvector{x_1\\x_2\\x_3\\x_4\\x_5\\x_6\\x_7}=
    \colvector{\ \\\ \\\ \\\ \\\ \\\ \\\ }+x_3\colvector{\ \\\ \\\ \\\ \\\ \\\ \\\ }+x_4\colvector{\ \\\ \\\ \\\ \\\ \\\ \\\ }+x_7\colvector{\ \\\ \\\ \\\ \\\ \\\ \\\ }
  \]

  \textbf{Step 2.}  Use 0's and 1's to ensure equality for the entries
  of the vectors with indices in $F$ (corresponding to the free
  variables).
  \[
    \vect{x}=
    \colvector{x_1\\x_2\\x_3\\x_4\\x_5\\x_6\\x_7}=
    \colvector{\\ \\ 0\\ 0\\ \\ \\ 0}+x_3\colvector{\\ \\ 1\\ 0\\ \\ \\ 0}+x_4\colvector{\\ \\ 0\\ 1\\ \\ \\ 0}+x_7\colvector{\\ \\ 0\\ 0\\ \\ \\ 1}
  \]
  
  \textbf{Step 3.}  For each dependent variable, use the augmented
  matrix to formulate an equation expressing the dependent variable as
  a constant plus multiples of the free variables.  Convert this
  equation into entries of the vectors that ensure equality for each
  dependent variable, one at a time.
  \begin{align*}
    x_1&=15-2x_3+3x_4-9x_7\ \Rightarrow\\&\vect{x}=
                                           \colvector{x_1\\x_2\\x_3\\x_4\\x_5\\x_6\\x_7}=
    \colvector{15\\ \\ 0\\ 0\\ \\ \\ 0}+
    x_3\colvector{-2\\ \\ 1\\ 0\\ \\ \\ 0}+
    x_4\colvector{3\\ \\ 0\\ 1\\ \\ \\ 0}+
    x_7\colvector{-9\\ \\ 0\\ 0\\ \\ \\ 1}\\
    x_2&=-10+5x_3-4x_4+8x_7\ \Rightarrow\\&\vect{x}=
                                            \colvector{x_1\\x_2\\x_3\\x_4\\x_5\\x_6\\x_7}=
    \colvector{15\\ -10\\ 0\\ 0\\ \\ \\ 0}+
    x_3\colvector{-2\\ 5\\ 1\\ 0\\ \\ \\ 0}+
    x_4\colvector{3\\ -4\\ 0\\ 1\\ \\ \\ 0}+
    x_7\colvector{-9\\ 8\\ 0\\ 0\\ \\ \\ 1}\\
    x_5&=11+6x_7\ \Rightarrow\\&\vect{x}=
                                 \colvector{x_1\\x_2\\x_3\\x_4\\x_5\\x_6\\x_7}=
    \colvector{15\\ -10\\ 0\\ 0\\ 11\\ \\ 0}+
    x_3\colvector{-2\\ 5\\ 1\\ 0\\ 0\\ \\ 0}+
    x_4\colvector{3\\ -4\\ 0\\ 1\\ 0\\ \\ 0}+
    x_7\colvector{-9\\ 8\\ 0\\ 0\\ 6\\ \\ 1}\\
    x_6&=-21-7x_7\ \Rightarrow\\&\vect{x}=
                                  \colvector{x_1\\x_2\\x_3\\x_4\\x_5\\x_6\\x_7}=
    \colvector{15\\ -10\\ 0\\ 0\\ 11\\ -21\\ 0}+
    x_3\colvector{-2\\ 5\\ 1\\ 0\\ 0\\ 0\\ 0}+
    x_4\colvector{3\\ -4\\ 0\\ 1\\ 0\\ 0\\ 0}+
    x_7\colvector{-9\\ 8\\ 0\\ 0\\ 6\\ -7\\ 1}
  \end{align*}

  This final \textit{form} of a typical solution is especially
  pleasing and useful.  For example, we can build solutions quickly by
  choosing values for our free variables, and then compute a linear
  combination.  For example,
  \begin{align*}
    x_3&=2,\,
         x_4=-4,\,
         x_7=3
         \quad\quad\Rightarrow\\
    \vect{x}&=
              \colvector{x_1\\x_2\\x_3\\x_4\\x_5\\x_6\\x_7}=
    \colvector{15\\ -10\\ 0\\ 0\\ 11\\ -21\\ 0}+
    (\answer{2})\colvector{-2\\ 5\\ 1\\ 0\\ 0\\ 0\\ 0}+
    (-4)\colvector{3\\ -4\\ 0\\ 1\\ 0\\ 0\\ 0}+
    (3)\colvector{-9\\ 8\\ 0\\ 0\\ 6\\ -7\\ 1}
    =
    \colvector{-28\\40\\2\\-4\\29\\-42\\3}
  \end{align*}
  or perhaps,
  \begin{align*}
    x_3&=5,\,
         x_4=2,\,
         x_7=1
         \quad\quad\Rightarrow\\
    \vect{x}&=
              \colvector{x_1\\x_2\\x_3\\x_4\\x_5\\x_6\\x_7}=
    \colvector{15\\ -10\\ 0\\ 0\\ 11\\ -21\\ 0}+
    (5)\colvector{-2\\ 5\\ 1\\ 0\\ 0\\ 0\\ 0}+
    (2)\colvector{3\\ -4\\ 0\\ 1\\ 0\\ 0\\ 0}+
    (1)\colvector{-9\\ 8\\ 0\\ 0\\ 6\\ -7\\ 1}
    =
    \colvector{2\\15\\5\\2\\17\\-28\\1}
  \end{align*}
  or even,
  \begin{align*}
    x_3&=0,\,
         x_4=0,\,
         x_7=0
         \quad\quad\Rightarrow\\
    \vect{x}&=
              \colvector{x_1\\x_2\\x_3\\x_4\\x_5\\x_6\\x_7}=
    \colvector{15\\ -10\\ 0\\ 0\\ 11\\ -21\\ 0}+
    (0)\colvector{-2\\ 5\\ 1\\ 0\\ 0\\ 0\\ 0}+
    (0)\colvector{3\\ -4\\ 0\\ 1\\ 0\\ 0\\ 0}+
    (0)\colvector{-9\\ 8\\ 0\\ 0\\ 6\\ -7\\ 1}
    =
    \colvector{15\\ -10\\ 0\\ 0\\ 11\\ -21\\ 0}
  \end{align*}
  So we can compactly express \textit{all} of the solutions to this
  linear system with just \answer{4} fixed vectors, provided we agree
  how to combine them in a linear combinations to create solution
  vectors.

  \begin{question}
    Is the vector 
    \[
      \vect{w}=\colvector{100\\-75\\7\\9\\-37\\35\\-8}
    \]
    a solution to this system of equations?

    \begin{multipleChoice}
      \choice{No.}
      \choice[correct]{Yes.}
    \end{multipleChoice}

    \begin{feedback}[correct]
      You could turn the problem around and write $\vect{w}$ as a
      linear combination of the four vectors $\vect{c}$, $\vect{u}_1$,
      $\vect{u}_2$, $\vect{u}_3$.
      \begin{align*}
        \vect{c}&=\colvector{15\\ -10\\ 0\\ 0\\ 11\\ -21\\ 0}
                &
                  \vect{u}_1&=\colvector{-2\\ 5\\ 1\\ 0\\ 0\\ 0\\ 0}
                &
                  \vect{u}_2&=\colvector{3\\ -4\\ 0\\ 1\\ 0\\ 0\\ 0}
                &
                  \vect{u}_3&=\colvector{-9\\ 8\\ 0\\ 0\\ 6\\ -7\\ 1}
      \end{align*}
      In that case, the coefficient of $\vect{c}$ is 1.  The
      coefficients of $\vect{u}_1$, $\vect{u}_2$, $\vect{u}_3$ lie in
      the third, fourth and seventh entries of $\vect{w}$.  Can you
      see why?  (Hint: $F=\set{3,\,4,\,7,\,8}$, so the free variables
      are $x_3,\,x_4$ and $x_7$.)
    \end{feedback}
  \end{question}
\end{example}

Did you think a few weeks ago that you could so quickly and easily
list \textit{all} the solutions to a linear system of 5 equations in 7
variables?

In the next activity, We will now formalize the last two (important)
examples as a theorem.

\end{document}
