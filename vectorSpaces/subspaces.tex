\documentclass{ximera}
\input{../preamble.tex}
\title{Subspaces}
\author{Crichton Ogle}

\begin{document}
\begin{abstract}
  A subset is a subspace if it is non-empty and, using the restriction to the subset of the sum and scalar product operations, the subset satisfies the
  axioms of a vector space.
\end{abstract}
\maketitle

If $V = \{V,+,\cdot\}$ is a vector space, then $W\subseteq V$ is called a {\it subspace} of $V$ if the restriction to $W$ of the sum and scalar product operations of $V$ make $W$ a vector space. It might seem that, in order to show some collection of vectors in $V$ form a subspace, one would have to go through the entire list of axioms checking each one. In fact, one only needs to check the closure axioms.

\begin{theorem} Let $W$ be a non-empty subset of vectors in $V$, which satisfy the two closure axioms C1 and C2 with respect to the operations on $V$. Then $W$ is a subspace of $V$.
\end{theorem}

\begin{proof} As we have already observed, the vector space axioms A1 - A8 fall into two types: {\it existential} (claiming the existence of certain vectors), and {\it universal} (indicating some property is universally true). Of these eight axioms, all but A3 and A4 are of the second type. These are automatically satisfied for vectors in $W$, because they hold for the larger space $V$ in which $W$ lies. In other words, for these six axioms, there is nothing to prove. 
\vskip.2in

The issue is with A3 and A4. To this end, we first show

\begin{claim} For any vector ${\bf v}\in V$, ${\bf z} = 0{\bf v}$. In addition, $-{\bf v} = (-1){\bf v}$.
\end{claim}
In other words, the zero vector of $V$ can be realized by taking any vector in $V$ and multiplying it by the scalar $0$, while the additive inverse of any vector can be gotten by multiplying it by the scalar $-1$.

\begin{proof} Fix ${\bf v}\in V$. Then
\[
{\bf v} + 0{\bf v} = 1{\bf v} + 0{\bf v} = (1+0){\bf v} = 1{\bf v} = {\bf v}
\]
so adding $-{\bf v}$ to both sides gives
\[
0{\bf v} = z + 0{\bf v} = (-{\bf v} + {\bf v}) + 0{\bf v} = -{\bf v} + ({\bf v} + 0{\bf v}) = -{\bf v} + {\bf v} = z
\]
verifying the first claim. Knowing this, we then have
\[
(-1){\bf v} + {\bf v} = (-1){\bf v} + 1{\bf v} = (-1+1){\bf v} = 0{\bf v} = z
\]
implying $(-1){\bf v} = -{\bf v}$ by the uniqueness of the additive inverse indicated by A4. 
\end{proof}

Axioms A3 and A4 then follow immediately for $W$, by virtue of the fact that $W$ is closed under scalar multiplication.
\end{proof}

\end{document}
