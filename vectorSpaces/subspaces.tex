\documentclass{ximera}
\input{../preamble.tex}
\title{Subspaces}
\author{Crichton Ogle}

\begin{document}
\begin{abstract}
  A subset of a vector space is a subspace if it is non-empty and, using the restriction to the subset of the sum and scalar product operations, the subset satisfies the
  axioms of a vector space.
\end{abstract}
\maketitle

If $V = \{V,+,\cdot\}$ is a vector space, then $W\subseteq V$ is called a {\it subspace} of $V$ if the restriction to $W$ of the sum and scalar product operations of $V$ make $W$ a vector space. It might seem that, in order to show some collection of vectors in $V$ form a subspace, one would have to go through the entire list of axioms checking each one. In fact, one only needs to check the closure axioms.

\begin{theorem} Let $W$ be a non-empty subset of vectors in $V$, which satisfy the two closure axioms C1 and C2 with respect to the operations on $V$. Then $W$ is a subspace of $V$.
\end{theorem}

\begin{proof} As we have already observed, the vector space axioms A1 - A8 fall into two types: {\it existential} (claiming the existence of certain vectors), and {\it universal} (indicating some property is universally true). Of these eight axioms, all but A3 and A4 are of the second type. These are automatically satisfied for vectors in $W$, because they hold for the larger space $V$ in which $W$ lies. In other words, for these six axioms, there is nothing to prove. 
\vskip.2in

The issue is with A3 and A4. To this end, we first show

\begin{claim} For any vector ${\bf v}\in V$, ${\bf z} = 0{\bf v}$. In addition, $-{\bf v} = (-1){\bf v}$.
\end{claim}
In other words, the zero vector of $V$ can be realized by taking any vector in $V$ and multiplying it by the scalar $0$, while the additive inverse of any vector can be gotten by multiplying it by the scalar $-1$.

\begin{proof} Fix ${\bf v}\in V$. Then
\[
{\bf v} + 0{\bf v} = 1{\bf v} + 0{\bf v} = (1+0){\bf v} = 1{\bf v} = {\bf v}
\]
so adding $-{\bf v}$ to both sides gives
\[
0{\bf v} = z + 0{\bf v} = (-{\bf v} + {\bf v}) + 0{\bf v} = -{\bf v} + ({\bf v} + 0{\bf v}) = -{\bf v} + {\bf v} = z
\]
verifying the first claim. Knowing this, we then have
\[
(-1){\bf v} + {\bf v} = (-1){\bf v} + 1{\bf v} = (-1+1){\bf v} = 0{\bf v} = z
\]
implying $(-1){\bf v} = -{\bf v}$ by the uniqueness of the additive inverse indicated by A4. 
\end{proof}

Axioms A3 and A4 then follow immediately for $W$, by virtue of the fact that $W$ is closed under scalar multiplication.
\end{proof}

\begin{proposition} For a closed interval $[a,b]\subset\mathbb R$, let $C[a,b]\subset F[a,b]$ denote the subset of continuous (real-valued) functions on $[a,b]$. Then $C[a,b]$ is a subspace of $F[a,b]$.
\end{proposition}

\begin{proof} $C[a,b]$ contains the zero function, so it is non-empty. Now the sum of two continuous functions on $[a,b]$ is again continuous on $[a,b]$. Also, any scalar multiple of a continuous function is continuous (these results were verified in first-term Calculus). By the above theorem $C[a,b]$ is a subspace of $F[a,b]$.
\end{proof}

\begin{exercise} Let $P_\infty$ be the set of polynomials in $x$ with real coefficients, and let $P_n\subset P_\infty$ be the subset of $P_\infty$ consisting of polynomials in $x$ of degree less than $n$ ($n\ge 1$). Show that $P_\infty$ is a subspace of $F(-\infty,\infty)$, and that $P_n$ is a subspace of $P_\infty$ for all $n\ge 1$.
\end{exercise}

In the following sections we will explore how to construct subspaces using various different methods. However we first need to revisit the operation of forming linear combinations in the more general setting of vector spaces. 

\end{document}
