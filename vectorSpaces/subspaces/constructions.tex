\documentclass{ximera}

\input{../../preamble.tex}

\title{Subspace Constructions}

\begin{document}
\begin{abstract}
  Several of the special subsets, like the column space, of vector spaces are also subspaces.
\end{abstract}
\maketitle

Several of the subsets of vectors spaces that we worked with in
\ref{chapter:M} are also subspaces---they are closed under vector
addition and scalar multiplication in $\complex{m}$.

\begin{theorem}[Column Space of a Matrix is a Subspace]
  \label{theorem:CSMS}

  Suppose that $A$ is an $m\times n$ matrix.  Then $\csp{A}$ is a subspace of $\complex{m}$.

  \begin{proof}
    \ref{definition:CSM} shows us that $\csp{A}$ is a subset of
    $\complex{m}$, and that it is defined as the
    \wordChoice{\choice[correct]{span of a set}\choice{finite set}} of
    vectors from $\complex{m}$ (the columns of the matrix).  Since
    $\csp{A}$ is a span, \ref{theorem:SSS} says it is a subspace.
  \end{proof}
\end{theorem}

That was easy!  Notice that we could have used this same approach to
prove that the null space is a subspace, since \ref{theorem:SSNS}
provided a description of the null space of a matrix as the span of a
set of vectors.  However, I much prefer the current proof of
\ref{theorem:NSMS}.  Speaking of easy, here is a very easy theorem
that exposes another of our constructions as creating subspaces.

\begin{theorem}[Row Space of a Matrix is a Subspace]
  \label{theorem:RSMS}

  Suppose that $A$ is an $m\times n$ matrix.  Then $\rsp{A}$ is a
  subspace of $\complex{n}$.

  \begin{proof}
    \ref{definition:RSM} says $\rsp{A}=\csp{\transpose{A}}$, so the
    row space of a matrix is a column space, and every column space is
    a subspace by \ref{theorem:CSMS}.  That's enough.
  \end{proof}
\end{theorem}

One more.

\begin{theorem}
  \label{theorem:LNSMS}
  [Left Null Space of a Matrix is a Subspace]
  
  Suppose that $A$ is an $m\times n$ matrix.  Then $\lns{A}$ is a subspace of $\complex{m}$.
  
  \begin{proof}
    \ref{definition:LNS} says $\lns{A}=\nsp{\transpose{A}}$, so the
    left null space is a null space, and every null space is a
    subspace by \ref{theorem:NSMS}.  Done.
  \end{proof}
\end{theorem}

So the span of a set of vectors, and the null space, column space, row
space and left null space of a matrix are all subspaces, and hence are
all vector spaces, meaning they have all the properties detailed in
\ref{definition:VS} and in the basic theorems presented in
\ref{section:VS}.  We have worked with these objects as just sets in
\ref{chapter:V} and \ref{chapter:M}, but now we understand that they
have much more structure.  In particular, being closed under vector
addition and scalar multiplication means a subspace is also closed
under linear combinations.

\begin{question}
  Consider the set of vectors
  \begin{align*}
    W&=\setparts{\colvector{a\\b\\c}}{3a-2b+c=0}
  \end{align*}

  Is the set $W$ a subspace of $\real{3}$?
  \begin{multipleChoice}
    \choice[correct]{Yes.}
    \choice{No.}
  \end{multipleChoice}

  \begin{question}
    The set $W$ is 
    \begin{multipleChoice}
      \choice{$\csp{\begin{bmatrix} 3 & -2 & 1 \end{bmatrix}}$}
      \choice{$\rsp{\begin{bmatrix} 3 & -2 & 1 \end{bmatrix}}$}
      \choice[correct]{$\nsp{\begin{bmatrix} 3 & -2 & 1 \end{bmatrix}}$}
    \end{multipleChoice}
  \end{question}
  
\end{question}

\end{document}
