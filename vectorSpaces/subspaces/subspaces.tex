\documentclass{ximera}

\input{../../preamble.tex}

\title{Subspaces}

\begin{document}
\begin{abstract}
  A subspace is a vector space that is contained within another vector
  space.  Every subspace is a vector space in its own right, but also
  defined relative to some larger vector space.
\end{abstract}
\maketitle

Here is the principal definition for this section.  We will discover
shortly that we are already familiar with a wide variety of subspaces
from previous activities.

\begin{definition}[Subspace]
  Suppose that $V$ and $W$ are two vector spaces that have identical
  definitions of vector addition and scalar multiplication, and that
  $W$ is a subset of $V$, $W\subseteq V$.  Then $W$ is a
  \dfn{subspace} of $V$.
\end{definition}

Let us look at an example of a vector space inside another vector space.

\begin{example}[A subspace of $\real{3}$]
  We know that $\real{3}$ is a vector space (\ref{example:VSCV}).
  Consider the subset,
  \[
    W=\setparts{\colvector{x_1\\x_2\\x_3}}{2x_1-5x_2+7x_3=0}
  \]
  Then
  \begin{multipleChoice}
    \choice[correct]{$W\subseteq\real{3}$}
    \choice{$\real{3} \subseteq W$}
  \end{multipleChoice}

  \begin{feedback}[correct]
    The objects in $W$ are column vectors of size 3, so
    $W\subseteq\real{3}$.
  \end{feedback}

  \begin{question}
    But is $W$ a vector space?  Does it satisfy the ten properties of
    \ref{definition:VS} when we use the same operations?

    \begin{multipleChoice}
      \choice[correct]{Yes}
      \choice{No}
    \end{multipleChoice}

    Suppose $\vect{x}=\colvector{x_1\\x_2\\x_3}$ and
    $\vect{y}=\colvector{y_1\\y_2\\y_3}$ are vectors from $W$.  Then
    we know that these vectors cannot be totally arbitrary, they must
    have gained membership in $W$ by virtue of meeting the membership
    test.  For example, we know that $\vect{x}$ must satisfy
    $2x_1-5x_2+7x_3=0$ while $\vect{y}$ must satisfy
    $2y_1-5y_2+7y_3=0$.  Our first property (\ref{property:AC}) asks
    the question, is $\vect{x}+\vect{y}\in W$?  When our set of
    vectors was $\real{3}$, this was an easy question to answer.  Now
    it is not so obvious.  Notice first that
    \[
      \vect{x}+\vect{y}=\colvector{x_1\\x_2\\x_3}+\colvector{y_1\\y_2\\y_3}=
      \colvector{x_1+y_1\\x_2+y_2\\x_3+y_3}
    \]
    and we can test this vector for membership in $W$ as follows.
    Because $\vect{x}\in W$ we know $2x_1-5x_2+7x_3=0$ and because
    $\vect{y}\in W$ we know $2y_1-5y_2+7y_3=0$.  Therefore,
    \begin{align*}
      2(x_1+y_1)-5(x_2+y_2)+7(x_3+y_3)
      &=2x_1+2y_1-5x_2-5y_2+7x_3+7y_3\\
      &=(2x_1-5x_2+7x_3)+(2y_1-5y_2+7y_3)\\
      &=0 + 0\\
      &=0
    \end{align*}
    and by this computation we see that $\vect{x}+\vect{y}\in W$.  One
    property down, nine to go.
    
    If $\alpha$ is a scalar and $\vect{x}\in W$, is it always true that
    $\alpha\vect{x}\in W$?  This is what we need to establish
    \ref{property:SC}.  Again, the answer is not as obvious as it was
    when our set of vectors was all of $\real{3}$.  Let us see.  First,
    note that because $\vect{x}\in W$ we know $2x_1-5x_2+7x_3=0$.
    Therefore,
    \[
      \alpha\vect{x}=\alpha\colvector{x_1\\x_2\\x_3}=\colvector{\alpha x_1\\\alpha x_2\\\alpha x_3}
    \]
    and we can test this vector for membership in $W$.  First, note that
    because $\vect{x}\in W$ we know $2x_1-5x_2+7x_3=0$.  Therefore,
    \begin{align*}
      2(\alpha x_1)-5(\alpha x_2)+7(\alpha x_3)
      &=\alpha(2x_1-5x_2+7x_3)\\
      &=\alpha 0\\
      &=0
    \end{align*}
    and we see that indeed $\alpha\vect{x}\in W$.  Always.
    
    If $W$ has a zero vector, it will be unique (\ref{theorem:ZVU}).
    The zero vector for $\real{3}$ should also perform the required
    duties when added to elements of $W$.  So the likely candidate for a
    zero vector in $W$ is the same zero vector that we know $\real{3}$
    has.  You can check that $\zerovector=\colvector{0\\0\\0}$ is a zero
    vector in $W$ too (\ref{property:Z}).
    
    With a zero vector, we can now ask about additive inverses
    (\ref{property:AI}).  As you might suspect, the natural candidate
    for an additive inverse in $W$ is the same as the additive inverse
    from $\real{3}$.  However, we must insure that these additive
    inverses actually are elements of $W$.  Given $\vect{x}\in W$, is
    $\vect{-x}\in W$?
    \[
      \vect{-x}=\colvector{-x_1\\-x_2\\-x_3}
    \]
    and we can test this vector for membership in $W$.  As before, because $\vect{x}\in W$ we know $2x_1-5x_2+7x_3=0$.
    \begin{align*}
      2(-x_1)-5(-x_2)+7(-x_3)
      &=-(2x_1-5x_2+7x_3)\\
      &=-0\\
      &=0
    \end{align*}
    and we now believe that $\vect{-x}\in W$.
    
    Is the vector addition in $W$ commutative (\ref{property:C})?  Is
    $\vect{x}+\vect{y}=\vect{y}+\vect{x}$?  Of course!  Nothing about
    restricting the scope of our set of vectors will prevent the
    operation from still being commutative.  Indeed, the remaining five
    properties are unaffected by the transition to a smaller set of
    vectors, and so remain true.  That was convenient.
    
    So $W$ satisfies all ten properties, is therefore a vector space,
    and thus earns the title of being a subspace of $\real{3}$.
  \end{question}
\end{example}

\end{document}
