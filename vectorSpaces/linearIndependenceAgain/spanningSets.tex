\documentclass{ximera}

\input{../../preamble.tex}

\title{Spanning Sets}

\begin{document}
\begin{abstract}
  Given a vector space, a ``spanning set'' is a set the span of which is the vector space.
\end{abstract}
\maketitle

In a vector space $V$, suppose we are given a set of vectors
$S\subseteq V$.  Then we can immediately construct a subspace,
$\spn{S}$, using \ref{definition:SS} and then be assured by
\ref{theorem:SSS} that the construction does provide a subspace.  We
now turn the situation upside-down.  Suppose we are first given a
subspace $W\subseteq V$.  Can we find a set $S$ so that $\spn{S}=W$?
Typically $W$ is infinite and we are searching for a finite set of
vectors $S$ that we can combine in linear combinations and ``build''
all of $W$.

I like to think of $S$ as the raw materials that are sufficient for
the construction of $W$.  If you have nails, lumber, wire, copper
pipe, drywall, plywood, carpet, shingles, paint (and a few other
things), then you can combine them in many different ways to create a
house (or infinitely many different houses for that matter).  A
fast-food restaurant may have beef, chicken, beans, cheese, tortillas,
taco shells and hot sauce and from this small list of ingredients
build a wide variety of items for sale.  Or maybe a better analogy
comes from Ben Cordes---the additive primary colors (red, green and
blue) can be combined to create many different colors by varying the
intensity of each.  The intensity is like a scalar multiple, and the
combination of the three intensities is like vector addition.  The
three individual colors, red, green and blue, are the elements of the
spanning set.

Because we will use terms like ``spanned by'' and ``spanning set,''
there is the potential for confusion with ``the span.''  Come back and
reread the first paragraph of this subsection whenever you are
uncertain about the difference.  Here is the working definition.

\begin{definition}[Spanning Set of a Vector Space]
  Suppose $V$ is a vector space.  A subset $S$ of $V$ is a
  \dfn{spanning set} of $V$ if $\spn{S}=V$.  In this case, we also
  frequently say $S$ \dfn{spans} $V$.
\end{definition}

The definition of a spanning set requires that two sets (subspaces
actually) be equal.  If $S$ is a subset of $V$, then
$\spn{S}\subseteq V$, always.  Thus it is usually only necessary to
prove that $V\subseteq\spn{S}$.

\begin{example}[Spanning set in $P_4$]
  We showed that
  \[
    W=\setparts{p(x)}{p\in P_4,\ p(2)=0}
  \]
  is a subspace of $P_4$, the vector space of polynomials with degree
  at most $4$ (\ref{example:VSP}).  In this example, we will show that
  the set
  \[
    S=\set{x-2,\,x^2-4x+4,\,x^3-6x^2+12x-8,\,x^4-8x^3+24x^2-32x+16}
  \]
  is a spanning set for $W$.  To do this, we require that 
  \begin{multipleChoice}
    \choice{$W=S$.}
    \choice[correct]{$W=\spn{S}$.}
  \end{multipleChoice}
  This is an equality of sets.  We can check that every polynomial in
  $S$ has $x=2$ as a root and therefore $S\subseteq W$.  Since $W$ is
  closed under addition and scalar multiplication,
  $\spn{S}\subseteq W$ also.

  So it remains to show that $W\subseteq \spn{S}$
  (\ref{definition:SE}).  To do this, begin by choosing an arbitrary
  polynomial in $W$, say $r(x)=ax^4+bx^3+cx^2+dx+e\in W$.  This
  polynomial is not as arbitrary as it would appear, since we also
  know it must have $x=2$ as a root.  This translates to
  \[
    0=a(2)^4+b(2)^3+c(2)^2+d(2)+e=16a+8b+4c+2d+e
  \]
  as a condition on $r$.

  We wish to show that $r$ is a polynomial in $\spn{S}$, that is, we
  want to show that $r$ can be written as a linear combination of the
  vectors (polynomials) in $S$.  So let us try.
  \begin{align*}
    r(x)&=ax^4+bx^3+cx^2+dx+e\\
        &=\alpha_1\left(x-2\right)+\alpha_2\left(x^2-4x+4\right)+\alpha_3\left(x^3-6x^2+12x-8\right)\\
        &\quad +\alpha_4\left(x^4-8x^3+24x^2-32x+16\right)\\
        &=\alpha_4x^4+
          \left(\alpha_3-8\alpha_4\right)x^3+
          \left(\alpha_2-6\alpha_3+24\alpha_4\right)x^2\\
        &\quad +
          \left(\alpha_1-4\alpha_2+12\alpha_3-32\alpha_4\right)x+
          \left(-2\alpha_1+4\alpha_2-8\alpha_3+16\alpha_4\right)
  \end{align*}

  Equating coefficients (vector equality in $P_4$) gives the system of
  five equations in four variables,
  \begin{align*}
    \alpha_4&=a\\
    \alpha_3-8\alpha_4&=b\\
    \alpha_2-6\alpha_3+24\alpha_4&=c\\
    \alpha_1-4\alpha_2+12\alpha_3-32\alpha_4&=d\\
    -2\alpha_1+4\alpha_2-8\alpha_3+16\alpha_4&=e\\
  \end{align*}

  Any solution to this system of equations will provide the linear
  combination we need to determine if $r\in\spn{S}$, but we need to be
  convinced there is a solution for any values of $a,\,b,\,c,\,d,\,e$
  that qualify $r$ to be a member of $W$.  So the question is: is this
  system of equations consistent?  We will form the augmented matrix,
  and row-reduce. (We probably need to do this by hand, since the
  matrix is symbolic---reversing the order of the first four rows is
  the best way to start).  We obtain a matrix in reduced row-echelon
  form
  \begin{align*}
    &\begin{bmatrix}
      \leading{1}&0&0&0&32a+12b+4c+d\\
      0&\leading{1}&0&0&24a+6b+c\\
      0&0&\leading{1}&0&8a+b\\
      0&0&0&\leading{1}&a\\
      0&0&0&0&16a+8b+4c+2d+e
    \end{bmatrix}\\
    =&
       \begin{bmatrix}
         \leading{1}&0&0&0&32a+12b+4c+d\\
         0&\leading{1}&0&0&24a+6b+c\\
         0&0&\leading{1}&0&8a+b\\
         0&0&0&\leading{1}&a\\
         0&0&0&0&0
       \end{bmatrix}
  \end{align*}

  For your results to match our first matrix, you may find it
  necessary to multiply the final row of your row-reduced matrix by
  the appropriate scalar, and/or add multiples of this row to some of
  the other rows.  To obtain the second version of the matrix, the
  last entry of the last column has been simplified to zero according
  to the one condition we were able to impose on an arbitrary
  polynomial from $W$.  Since the last column is not a pivot column,
  \ref{theorem:RCLS} tells us this system is consistent.  Therefore,
  \textit{any} polynomial from $W$ can be written as a linear
  combination of the polynomials in $S$, so
  $W\subseteq\spn{S}$. Therefore, $W=\spn{S}$ and $S$ is a spanning
  set for $W$ by \ref{definition:SSVS}.

  Notice that an alternative to row-reducing the augmented matrix by
  hand would be to appeal to \ref{theorem:FS} by expressing the column
  space of the coefficient matrix as a null space, and then verifying
  that the condition on $r$ guarantees that $r$ is in the column
  space, thus implying that the system is always consistent.  Give it
  a try, we will wait.  This has been a complicated example, but worth
  studying carefully.
\end{example}

Given a subspace and a set of vectors, it can take some work to
determine that the set actually is a spanning set.  An even harder
problem is to be confronted with a subspace and required to construct
a spanning set with no guidance.  We will now work an example of this
flavor, but some of the steps will be unmotivated.  Fortunately, we
will have some better tools for this type of problem later on.

\begin{example}[Spanning set in $M_{22}$]
  In the space of all $2\times 2$ matrices, $M_{22}$ consider the subspace
  \[
    Z=\setparts{\begin{bmatrix}a&b\\c&d\end{bmatrix}}{a+3b-c-5d=0,\ -2a-6b+3c+14d=0}
  \]
  and find a spanning set for $Z$.

  We need to construct a limited number of matrices in $Z$ so that
  every matrix in $Z$ can be expressed as a linear combination of this
  limited number of matrices.  Suppose that
  $B=\begin{bmatrix}a&b\\c&d\end{bmatrix}$ is a matrix in $Z$.  Then
  we can form a column vector with the entries of $B$ and write
  \[
    \colvector{a\\b\\c\\d}\in
    \nsp{\begin{bmatrix}1 & 3 & -1 & -5\\-2 & -6 & 3 & 14\end{bmatrix}}
  \]

  Row-reducing this matrix and applying \ref{theorem:REMES} we obtain
  the equivalent statement,
  \[
    \colvector{a\\b\\c\\d}\in
    \nsp{\begin{bmatrix}\leading{1} & 3 & 0 & -1\\0 & 0 & \leading{1} & 4\end{bmatrix}}
  \]

  We can then express the subspace $Z$ in the following equal forms,
  \begin{align*}
    Z&=\setparts{\begin{bmatrix}a&b\\c&d\end{bmatrix}}{a+3b-c-5d=0,\ -2a-6b+3c+14d=0}\\
     &=\setparts{\begin{bmatrix}a&b\\c&d\end{bmatrix}}{a+3b-d=0,\ c+4d=0}\\
     &=\setparts{\begin{bmatrix}a&b\\c&d\end{bmatrix}}{a=-3b+d,\ c=-4d}\\
     &=\setparts{\begin{bmatrix}-3b+d&b\\-4d&d\end{bmatrix}}{b,\,d\in\complexes}\\
     &=\setparts{
       \begin{bmatrix}-3b&b\\0&0\end{bmatrix}+
                                \begin{bmatrix}d&0\\-4d&d\end{bmatrix}
                                                         }{b,\,d\in\complexes}\\
     &=\setparts{
       b\begin{bmatrix}-3&1\\0&0\end{bmatrix}+
                                d\begin{bmatrix}1&0\\-4&1\end{bmatrix}
                                                         }{b,\,d\in\complexes}\\
     &=\spn{\set{
       \begin{bmatrix}-3&1\\0&0\end{bmatrix},\,
                               \begin{bmatrix}1&0\\-4&1\end{bmatrix}
                                                       }}
  \end{align*}

  So the set
  \[
    Q=\set{
      \begin{bmatrix}-3&1\\0&0\end{bmatrix},\,
      \begin{bmatrix}1&0\\-4&1\end{bmatrix}
    }
  \]
  spans $Z$ by \ref{definition:SSVS}.
\end{example}


\end{document}
