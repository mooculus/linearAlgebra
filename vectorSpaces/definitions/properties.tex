\documentclass{ximera}

\input{../../preamble.tex}

\title{Vector Space Properties}

\begin{document}
\begin{abstract}
  We prove some general properties of vector spaces.  Some of these results will again seem obvious, but it is important to understand why it is necessary to state and prove them.
\end{abstract}
\maketitle

As we prove general properties, a typical hypothesis will be ``Let $V$
be a vector space.''  From this we may assume the ten properties of
\ref{definition:VS}, \textit{and nothing more}.  It is like starting
over, as we learn about what can happen in this new algebra we are
learning.  But the power of this careful approach is that we can apply
these theorems to any vector space we encounter---those in the
previous examples, or new ones we have not yet contemplated.  Or
perhaps new ones that nobody has ever contemplated.  We will
illustrate some of these results with examples from the crazy vector
space (\ref{example:CVS}), but mostly we are stating theorems and
doing proofs.  These proofs do not get too involved, but are not
trivial either, so these are good theorems to try proving yourself
before you study the proof given here.

First we show that there is just one zero vector.  Notice that the
properties only require there to be <em>at least</em> one, and say
nothing about there possibly being more.  That is because we can use
the ten properties of a vector space (\ref{definition:VS}) to learn
that there can \textit{never} be more than one.  To require that this
extra condition be stated as an eleventh property would make the
definition of a vector space more complicated than it needs to be.

\begin{theorem}[Zero Vector is Unique]
  \label{theorem:ZVU}
  
  Suppose that $V$ is a vector space.  The zero vector, $\zerovector$,
  is unique.
  
  \begin{proof}
    To prove uniqueness, a standard technique is to suppose the
    existence of two objects.  So let $\zerovector_1$ and
    $\zerovector_2$ be two zero vectors in $V$.  Then
    \begin{align*}
      \zerovector_1
      &=\zerovector_1+\zerovector_2
      &&\ref{property:Z}\text{ for }\zerovector_2\\
      &=\zerovector_2+\zerovector_1
      &&\ref{property:C}\\
      &=\zerovector_2
      &&\ref{property:Z}\text{ for }\zerovector_1
    \end{align*}
    
    This proves the uniqueness since the two zero vectors are really the same.
  \end{proof}
\end{theorem}

\begin{theorem}[Additive Inverses are Unique]
  \label{theorem:AIU}
  
  Suppose that $V$ is a vector space.   For each $\vect{u}\in V$, the additive inverse, $\vect{-u}$, is unique.
  
  \begin{proof}
    To prove uniqueness, a standard technique is to suppose the existence of two objects (\ref{technique:U}).  So let $\vect{-u}_1$ and $\vect{-u}_2$ be two additive inverses for $\vect{u}$.  Then
    \begin{align*}
      \vect{-u}_1&=\vect{-u}_1+\zerovector&&\ref{property:Z}\\
                 &=\vect{-u}_1+(\vect{u}+\vect{-u}_2)&&\ref{property:AI}\\
                 &=(\vect{-u}_1+\vect{u})+\vect{-u}_2&&\ref{property:AA}\\
                 &=\zerovector+\vect{-u}_2&&\ref{property:AI}\\
                 &=\vect{-u}_2&&\ref{property:Z}
    \end{align*}
    
    So the two additive inverses are really the same.
    
  \end{proof}
\end{theorem}

As obvious as the next three theorems appear, nowhere have we
guaranteed that the zero scalar, scalar multiplication and the zero
vector all interact this way.  Until we have proved it, anyway.

\begin{theorem}[Zero Scalar in Scalar Multiplication]
  \label{theorem:ZSSM}

  Suppose that $V$ is a vector space and $\vect{u}\in V$.  Then $0\vect{u}=\zerovector$.

  \begin{proof}
    Notice that $0$ is a scalar, $\vect{u}$ is a vector, so \ref{property:SC} says $0\vect{u}$ is again a vector.  As such, $0\vect{u}$ has an additive inverse, $-(0\vect{u})$ by \ref{property:AI}.
    \begin{align*}
      0\vect{u}
      &=\zerovector+0\vect{u}&&\ref{property:Z}\\
      &=\left(-(0\vect{u}) + 0\vect{u}\right)+0\vect{u}&&\ref{property:AI}\\
      &=-(0\vect{u}) + \left(0\vect{u}+0\vect{u}\right)&&\ref{property:AA}\\
      &=-(0\vect{u}) + (0+0)\vect{u}&&\ref{property:DSA}\\
      &=-(0\vect{u}) + 0\vect{u}&&\ref{property:ZCN}\\
      &=\zerovector&&\ref{property:AI}
    \end{align*}

  \end{proof}
\end{theorem}

Here is another theorem that \textit{looks} like it should be obvious, but is still in need of a proof.

\begin{theorem}
  \label{theorem:ZVSM}
  [Zero Vector in Scalar Multiplication]
  
  Suppose that $V$ is a vector space and $\alpha\in\complexes$.   Then $\alpha\zerovector=\zerovector$.

  \begin{proof}
    Notice that $\alpha$ is a scalar, $\zerovector$ is a vector, so \ref{property:SC} means $\alpha\zerovector$ is again a vector.  As such, $\alpha\zerovector$ has an additive inverse, $-(\alpha\zerovector)$ by \ref{property:AI}.
    \begin{align*}
      \alpha\zerovector
      &=\zerovector+\alpha\zerovector&&\ref{property:Z}\\
      &=\left(-(\alpha\zerovector)+\alpha\zerovector\right)+\alpha\zerovector&&\ref{property:AI}\\
      &=-(\alpha\zerovector)+\left(\alpha\zerovector+\alpha\zerovector\right)&&\ref{property:AA}\\
      &=-(\alpha\zerovector)+\alpha\left(\zerovector+\zerovector\right)&&\ref{property:DVA}\\
      &=-(\alpha\zerovector)+\alpha\zerovector&&\ref{property:Z}\\
      &=\zerovector&&\ref{property:AI}
    \end{align*}
  \end{proof}
\end{theorem}

Here is another one that sure looks obvious.  But understand that we
have chosen to use certain notation because it makes the theorem's
conclusion look so nice.  The theorem is not true because the notation
looks so good; it still needs a proof.  If we had really wanted to
make this point, we might have used notation like $\vect{u}^\sharp$
for the additive inverse of $\vect{u}$.  Then we would have written
the defining property, \ref{property:AI}, as
$\vect{u}+\vect{u}^\sharp=\zerovector$.  This theorem would become
$\vect{u}^\sharp=(-1)\vect{u}$.  Not really quite as pretty, is it?

\begin{theorem}[Additive Inverses from Scalar Multiplication]
  \label{theorem:AISM}

  Suppose that $V$ is a vector space and $\vect{u}\in V$.  Then $\vect{-u}=(-1)\vect{u}$.

  \begin{proof}
    \begin{align*}
      \vect{-u}
      &=\vect{-u}+\zerovector&&\ref{property:Z}\\
      &=\vect{-u}+0\vect{u}&&\ref{theorem:ZSSM}\\
      &=\vect{-u}+\left(1+(-1)\right)\vect{u}&&\ref{property:AICN}\\
&=\vect{-u}+\left(1\vect{u}+(-1)\vect{u}\right)&&\ref{property:DSA}\\
      &=\vect{-u}+\left(\vect{u}+(-1)\vect{u}\right)&&\ref{property:O}\\
      &=\left(\vect{-u}+\vect{u}\right)+(-1)\vect{u}&&\ref{property:AA}\\
      &=\zerovector+(-1)\vect{u}&&\ref{property:AI}\\
      &=(-1)\vect{u}&&\ref{property:Z}
    \end{align*}
  \end{proof}
\end{theorem}

Because of this theorem, we can now write linear combinations like
$6\vect{u}_1+(-4)\vect{u}_2$ as $6\vect{u}_1-4\vect{u}_2$, even though
we have not formally defined an operation called \dfn{vector
  subtraction}.

Our next theorem is a bit different from several of the others in the
list.  Rather than making a declaration (``the zero vector is
unique'') it is an implication (``\ldots, then\ldots'') and so can be
used in proofs to convert a vector equality into two possibilities,
one a scalar equality and the other a vector equality.  It should
remind you of the situation for complex numbers.  If
$\alpha,\,\beta\in\complexes$ and $\alpha\beta=0$, then $\alpha=0$ or
$\beta=0$.  This critical property is the driving force behind using a
factorization to solve a polynomial equation.

\begin{theorem}[Scalar Multiplication Equals the Zero Vector]
  \label{theorem:SMEZV}

  Suppose that $V$ is a vector space and $\alpha\in\complexes$.  If
  $\alpha\vect{u}=\zerovector$, then either $\alpha=0$ or
  $\vect{u}=\zerovector$.

  \begin{proof}
    We prove this theorem by breaking up the analysis into two cases.  The first seems too trivial, and it is, but the logic of the argument is still legitimate.

    \textbf{Case 1.}  Suppose $\alpha=0$.  In this case our conclusion
    is true (the first part of the either/or is true) and we are done.
    That was easy.
    
    \textbf{Case 2.}  Suppose $\alpha\neq 0$.
    \begin{align*}
      \vect{u}
      &=1\vect{u}
      &&\ref{property:O}\\
      &=\left(\frac{1}{\alpha}\alpha\right)\vect{u}
      &&\alpha\neq 0, \ref{property:MICN}\\
      &=\frac{1}{\alpha}\left(\alpha\vect{u}\right)
      &&\ref{property:SMA}\\
      &=\frac{1}{\alpha}\left(\zerovector\right)
      &&\text{Hypothesis}\\
      &=\zerovector&&\ref{theorem:ZVSM}
    \end{align*}

    So in this case, the conclusion is true (the second part of the either/or is true) and we are done since the conclusion was true in each of the two cases.
  \end{proof}
\end{theorem}

\begin{exercise}
  Suppose that $V$ is a vector space, and $\vect{u},\,\vect{v},\,\vect{w}\in V$.

  If $\vect{w}+\vect{u}=\vect{w}+\vect{v}$, then it is necessarily the case that
  \begin{multipleChoice}
    \choice[correct]{$\vect{u}=\vect{v}$}
    \choice{$\vect{w}=\zerovector$}
  \end{multipleChoice}
  
  \begin{feedback}[correct]
    We can prove the statement ``if
    $\vect{w}+\vect{u}=\vect{w}+\vect{v}$ then $\vect{u}=\vect{v}$''
    with an argument such as this:
    \begin{align*}
      \vect{u}
      &=\zerovector+\vect{u}&&\ref{property:Z}\\
      &=\left(\vect{-w}+\vect{w}\right)+\vect{u}&&\ref{property:AI}\\
      &=\vect{-w}+\left(\vect{w}+\vect{u}\right)&&\ref{property:AA}\\
      &=\vect{-w}+\left(\vect{w}+\vect{v}\right)&&\text{Hypothesis}\\
      &=\left(\vect{-w}+\vect{w}\right)+\vect{v}&&\ref{property:AA}\\
      &=\zerovector+\vect{v}&&\ref{property:AI}\\
      &=\vect{v}&&\ref{property:Z}
    \end{align*}
  \end{feedback}
\end{exercise}

\begin{example}[Properties for the Crazy Vector Space]
  Several of the above theorems have interesting demonstrations when
  applied to the crazy vector space, $C$ (\ref{example:CVS}).  We are
  not proving anything new here, or learning anything we did not know
  already about $C$.  It is just plain fun to see how these general
  theorems apply in a specific instance.  For most of our examples,
  the applications are obvious or trivial, but not with $C$.

  Suppose $\vect{u}\in C$.  Then, as given by \ref{theorem:ZSSM},
  \[
    0\vect{u}=0(x_1,\,x_2)=(0x_1+0-1,\,0x_2+0-1)=(-1,-1)=\zerovector
  \]
  And as given by \ref{theorem:ZVSM},
  \begin{align*}
    \alpha\zerovector
    &=\alpha(-1,\,-1)=(\alpha(-1)+\alpha-1,\,\alpha(-1)+\alpha-1)\\
    &=(-\alpha+\alpha-1,\,-\alpha+\alpha-1)=(-1,\,-1)=\zerovector
  \end{align*}
  Finally, as given by \ref{theorem:AISM},
  \begin{align*}
    (-1)\vect{u}
    &=(-1)(x_1,\,x_2)=((-1)x_1+(-1)-1,\,(-1)x_2+(-1)-1)\\
    &=(-x_1-2,\,-x_2-2)=-\vect{u}
  \end{align*}
\end{example}

\end{document}
