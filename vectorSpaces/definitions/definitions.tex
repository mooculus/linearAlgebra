\documentclass{ximera}

\input{../../preamble.tex}

\title{Definition of a vector space}

\begin{document}
\begin{abstract}
The formal definition of a vector space leads to an extra increment of abstraction.
\end{abstract}
\maketitle

Here is one of the two most important definitions in the entire course.

\begin{definition}[Vector Space]
  \label{definition:VS}
  Suppose that $V$ is a set upon which we have defined two operations:
  (1) \dfn{vector addition}, which combines two elements of $V$ and is
  denoted by ``+'', and (2) \dfn{scalar multiplication}, which
  combines a complex number with an element of $V$ and is denoted by
  juxtaposition.  Then $V$, along with the two operations, is a
  \dfn{vector space} over $\complexes$ if the following ten properties
  hold.

%<propertylist>
%<property acro="AC" index="additive closure!vectors">
\begin{description}
\item[Additive Closure]
If $\vect{u},\,\vect{v}\in V$, then $\vect{u}+\vect{v}\in V$.
%<property acro="SC" index="scalar closure!vectors">
\item[Scalar Closure]
If $\alpha\in\complexes$ and $\vect{u}\in V$, then $\alpha\vect{u}\in V$.
%<property acro="C" index="commutativity!vectors">
\item[Commutativity]
If $\vect{u},\,\vect{v}\in V$, then $\vect{u}+\vect{v}=\vect{v}+\vect{u}$.
%<property acro="AA" index="additive associativity!vectors">
\item[Additive Associativity]
If $\vect{u},\,\vect{v},\,\vect{w}\in V$, then $\vect{u}+\left(\vect{v}+\vect{w}\right)=\left(\vect{u}+\vect{v}\right)+\vect{w}$.
%<property acro="Z" index="zero vector!vectors">
\item[Zero Vector]
There is a vector, $\zerovector$, called the \dfn{zero vector}, such that  $\vect{u}+\zerovector=\vect{u}$  for all $\vect{u}\in V$.
%<property acro="AI" index="additive inverses!vectors">
\item[Additive Inverses]
If $\vect{u}\in V$, then there exists a vector $\vect{-u}\in V$ so that $\vect{u}+ (\vect{-u})=\zerovector$.
%<property acro="SMA" index="scalar multiplication associativity!vectors">
\item[Scalar Multiplication Associativity]
If $\alpha,\,\beta\in\complexes$ and $\vect{u}\in V$, then $\alpha(\beta\vect{u})=(\alpha\beta)\vect{u}$.
%<property acro="DVA" index="distributivity, vector addition!vectors">
\item[Distributivity across Vector Addition]
If $\alpha\in\complexes$ and $\vect{u},\,\vect{v}\in V$, then $\alpha(\vect{u}+\vect{v})=\alpha\vect{u}+\alpha\vect{v}$.
%<property acro="DSA" index="distributivity, scalar addition!vectors">
\item[Distributivity across Scalar Addition]
If $\alpha,\,\beta\in\complexes$ and $\vect{u}\in V$, then
$(\alpha+\beta)\vect{u}=\alpha\vect{u}+\beta\vect{u}$.
%<property acro="O" index="one!vectors">
\item[One]
If $\vect{u}\in V$, then $1\vect{u}=\vect{u}$.
\end{description}

The objects in $V$ are called \dfn{vectors}, no matter what else they might really be, simply by virtue of being elements of a vector space.

\end{definition}

Now, there are several important observations to make.  Many of these
will be easier to understand on a second or third reading, and
especially after carefully studying the examples in future activities.

An \dfn{axiom} is often a ``self-evident'' truth.  Something so
fundamental that we all agree it is true and accept it without proof.
Typically, it would be the logical underpinning that we would begin to
build theorems upon.  Some might refer to the ten properties of
\ref{definition:VS} as axioms, implying that a vector space is a very
natural object and the ten properties are the essence of a vector
space.  We will instead emphasize that we will begin with a definition
of a vector space.  After studying the remainder of this chapter, you
might return here and remind yourself how all our forthcoming theorems
and definitions rest on this foundation.

As we will see shortly, the objects in $V$ can be \textit{anything},
even though we will call them vectors.  We have been working with
vectors frequently, but we should stress here that these have so far
just been \textit{column} vectors---scalars arranged in a columnar
list of fixed length.  In a similar vein, you have used the symbol
``+'' for many years to represent the addition of numbers (scalars).
We have extended its use to the addition of column vectors and to the
addition of matrices, and now we are going to recycle it even further
and let it denote vector addition in \textit{any} possible vector
space.  So when describing a new vector space, we will have to
\textit{define} exactly what ``+'' is.  Similar comments apply to
scalar multiplication.  Conversely, we can \textit{define} our
operations any way we like, so long as the ten properties are
fulfilled (see \ref{example:CVS}).

In \ref{definition:VS}, the scalars do not have to be complex numbers.
They can come from what are called in more advanced mathematics,
``fields''.  Examples of fields are the set of complex numbers, the
set of real numbers, the set of rational numbers, and even the finite
set of ``binary numbers'', $\set{0,\,1}$.  There are many, many
others.  In this case we would call $V$ a \dfn{vector space} over (the
field) $F$.

A vector space is composed of three objects, a set and two operations.
Some would explicitly state in the definition that $V$ must be a
nonempty set, but we can infer this from \ref{property:Z}, since the
set cannot be empty and contain a vector that behaves as the zero
vector.  Also, we usually use the same symbol for both the set and the
vector space itself.  Do not let this convenience fool you into
thinking the operations are secondary!

This discussion has either convinced you that we are really embarking
on a new level of abstraction, or it has seemed cryptic, mysterious or
nonsensical.  You might want to return to this section in a few days
and give it another read then.  In any case, let us look at some
concrete examples now.

\begin{exercise}
%<exercise type="M" number="11" rough="Vector space or not?">
%<problem contributor="chrisblack">
  Let $V$ be the set $\real{2}$ with the usual vector addition, but
  with scalar multiplication defined by
  \[
    \alpha \colvector{x\\y} = \colvector{\alpha y \\ \alpha x}
  \]
  Then $V$, with these operations\ldots
  \begin{multipleChoice}
    \choice[correct]{is not a vector space}
    \choice{is a vector space}
  \end{multipleChoice}

  \begin{feedback}[correct]
    The set $\real{2}$ with the proposed operations is not a vector
    space since \ref{property:O} is not valid.  A counterexample is
    $1\colvector{3\\2 } = \colvector{2\\3}\ne\colvector{3\\2 }$, so in
    general, $1\vect{u}\ne\vect{u}$.
  \end{feedback}
\end{exercise}

\subsection{Recycling definitions}

When we say that $V$ is a vector space, we then know we have a set of
objects (the ``vectors''), but we also know we have been provided with
two operations (``vector addition'' and ``scalar multiplication'') and
these operations behave with these objects according to the ten
properties of \ref{definition:VS}.  One combines two vectors and
produces a vector, the other takes a scalar and a vector, producing a
vector as the result.  So if
$\vect{u}_1,\,\vect{u}_2,\,\vect{u}_3\in V$ then an expression like
\[
  5\vect{u}_1+7\vect{u}_2-13\vect{u}_3
\]
would be unambiguous in \textit{any} of the vector spaces we have
discussed in this section.  And the resulting object would be another
vector in the vector space.  If you were tempted to call the above
expression a linear combination, you would be right.  Four of the
definitions that were central to our discussions in \ref{chapter:V}
were stated in the context of vectors being \textit{column vectors},
but were purposely kept broad enough that they could be applied in the
context of any vector space.  They only rely on the presence of
scalars, vectors, vector addition and scalar multiplication to make
sense.  We will restate them shortly, unchanged, except that their
titles and acronyms no longer refer to column vectors, and the
hypothesis of being in a vector space has been added.  Take the time
now to look forward and review each one, and begin to form some
connections to what we have done earlier and what we will be doing in
subsequent activities.

\end{document}
