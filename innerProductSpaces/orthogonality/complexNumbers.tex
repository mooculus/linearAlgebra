\documentclass{ximera}

\input{../../preamble.tex}

\title{Complex Numbers}

\begin{document}
\begin{abstract}
  Because we have chosen to use complex numbers as our set of scalars,
  this section is more complex than it would be for the real
  numbers.
\end{abstract}
\maketitle

If you have not already, now would be a good time to review some of
the basic properties of arithmetic with complex numbers.  With that
done, we can extend the basics of complex number arithmetic to our
study of vectors in $\complex{m}$.

We know how the addition and multiplication of complex numbers is
employed in defining the operations for vectors in $\complex{m}$..  We
can also extend the idea of the conjugate to vectors.

\begin{definition}[Complex Conjugate of a Column Vector]
  Suppose that $\vect{u}$ is a vector from $\complex{m}$.  Then the
  conjugate of the vector, $\conjugate{\vect{u}}$, is defined by
  \begin{align*}
    \vectorentry{\conjugate{\vect{u}}}{i}
    &=\conjugate{\vectorentry{\vect{u}}{i}}
    &&\text{$1\leq i\leq m$}
  \end{align*}
\end{definition}

\begin{example}
  Suppose $\vect{x} = \begin{bmatrix} 2 + 3i \\ 4 - 2i \end{bmatrix}$.

  Then
  \[
    \conjugate{\vect{x}} = \begin{bmatrix} \answer{2 - 3i} \\ \answer{4 + 2i} \end{bmatrix}.
  \]
\end{example}

With this definition we can show that the conjugate of a column vector behaves as we would expect with regard to vector addition and scalar multiplication.

\begin{theorem}[Conjugation Respects Vector Addition]
  \label{theorem:CRVA}
  
  Suppose $\vect{x}$ and $\vect{y}$ are two vectors from $\complex{m}$.  Then
  \[
    \conjugate{\vect{x}+\vect{y}}=\conjugate{\vect{x}}+\conjugate{\vect{y}}
  \]
  
  \begin{proof}
    For each $1\leq i\leq m$,
    \begin{align*}
      \vectorentry{\conjugate{\vect{x}+\vect{y}}}{i}
      &=\conjugate{\vectorentry{\vect{x}+\vect{y}}{i}}\\ %\ref{definition:CCCV}\\
      &=\conjugate{\vectorentry{\vect{x}}{i}+\vectorentry{\vect{y}}{i}}\\ %\ref{definition:CVA}\\
      &=\conjugate{\vectorentry{\vect{x}}{i}}+\conjugate{\vectorentry{\vect{y}}{i}}\\ %\ref{theorem:CCRA}\\
      &=\vectorentry{\conjugate{\vect{x}}}{i}+\vectorentry{\conjugate{\vect{y}}}{i}\\ %\ref{definition:CCCV}\\
      &=\vectorentry{\conjugate{\vect{x}}+\conjugate{\vect{y}}}{i}\\ %\ref{definition:CVA}
    \end{align*}
    
    
    Then by \ref{definition:CVE} we have $\conjugate{\vect{x}+\vect{y}}=\conjugate{\vect{x}}+\conjugate{\vect{y}}$.
    
  \end{proof}
\end{theorem}

\begin{theorem}[Conjugation Respects Vector Scalar Multiplication]
  \label{theorem:CRSM}
  
  Suppose $\vect{x}$ is a vector from $\complex{m}$, and $\alpha\in\complexes$ is a scalar.  Then
  \[
    \conjugate{\alpha\vect{x}}=\conjugate{\alpha}\,\conjugate{\vect{x}}
  \]
  
  \begin{proof}
    For $1\leq i\leq m$,
    \begin{align*}
      \vectorentry{\conjugate{\alpha\vect{x}}}{i}
      &=\conjugate{\vectorentry{\alpha\vect{x}}{i}}\\ %\ref{definition:CCCV}\\
      &=\conjugate{\alpha\vectorentry{\vect{x}}{i}}\\ %\ref{definition:CVSM}\\
      &=\conjugate{\alpha}\,\conjugate{\vectorentry{\vect{x}}{i}}\\ %\ref{theorem:CCRM}\\
      &=\conjugate{\alpha}\,\vectorentry{\conjugate{\vect{x}}}{i}\\ %\ref{definition:CCCV}\\
      &=\vectorentry{\conjugate{\alpha}\,\conjugate{\vect{x}}}{i}\\ %\ref{definition:CVSM}\\
    \end{align*}

    Then by \ref{definition:CVE} we have $\conjugate{\alpha\vect{x}}=\conjugate{\alpha}\,\conjugate{\vect{x}}$.
\end{proof}
\end{theorem}

These two theorems together tell us how we can ``push'' complex conjugation through linear combinations.

\begin{example}
  Suppose $\vect{x}$ and $\vect{y}$ are two vectors in $\complex{m}$.
  
  Set $\vect{z} = \alpha \vect{x} + \beta \vect{y}$.

  Then $\conjugate{\vect{z}}$ is
  \begin{multipleChoice}
    \choice{$\alpha \conjugate{\vect{x}} + \beta \conjugate{\vect{y}}$}
    \choice{$\conjugate{\alpha}\, \vect{x} + \conjugate{\beta}\, \vect{y}$}
    \choice[correct]{$\conjugate{\alpha}\, \conjugate{\vect{x}} + \conjugate{\beta}\, \conjugate{\vect{y}}$}
    \choice{$\conjugate{\alpha}\, \conjugate{\vect{x}} - \conjugate{\beta}\, \conjugate{\vect{y}}$}
  \end{multipleChoice}
\end{example}


\end{document}

