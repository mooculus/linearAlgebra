\documentclass{ximera}

\input{../../preamble.tex}

\title{Norm}

\begin{document}
\begin{abstract}
  Geometrically, the length of a vector is a key concept.
\end{abstract}
\maketitle

If treating linear algebra in a more geometric fashion, the length of
a vector occurs naturally, and is what you would expect from its name.
With complex numbers, we will define a similar function.  Recall that
if $c$ is a complex number, then $\modulus{c}$ denotes its modulus.

\begin{definition}[Norm of a Vector]
  The \dfn{norm} of the vector $\vect{u}$ is the scalar quantity in $\complexes$
  \[
    \norm{\vect{u}}=
    \sqrt{
      \modulus{\vectorentry{\vect{u}}{1}}^2+
      \modulus{\vectorentry{\vect{u}}{2}}^2+
      \modulus{\vectorentry{\vect{u}}{3}}^2+
      \cdots+
      \modulus{\vectorentry{\vect{u}}{m}}^2
    }
    =
    \sqrt{\sum_{i=1}^{m}\modulus{\vectorentry{\vect{u}}{i}}^2}
  \]
\end{definition}

Computing a norm is also easy to do.

\begin{example}[Computing the norm of some vectors]
The norm of
\[
\vect{u}=\colvector{3+2i\\1-6i\\2+4i\\2+i}
\]
is $\answer{5\sqrt{3}}$.
\begin{hint}
  \begin{align*}
    \norm{\vect{u}}&=
                     \sqrt{\modulus{3+2i}^2+\modulus{1-6i}^2+\modulus{2+4i}^2+\modulus{2+i}^2}\\
                   &=\sqrt{13+37+20+5}=\sqrt{75}=5\sqrt{3}.
  \end{align*}
\end{hint}
\end{example}

\begin{example}[Computing the norm of more vectors]
  The norm of
  \[
    \vect{v}=\colvector{3\\-1\\2\\4\\-3}
  \]
  is $\answer{\sqrt{39}}$.
  \begin{hint}
    \[
      \norm{\vect{v}}=
      \sqrt{\modulus{3}^2+\modulus{-1}^2+\modulus{2}^2+\modulus{4}^2+\modulus{-3}^2}
      =\sqrt{3^2+1^2+2^2+4^2+3^2}=\sqrt{39}.
    \]
  \end{hint}
\end{example}

Notice how the norm of a vector with real number entries is just the
length of the vector.  Inner products and norms are related by the
following theorem.

\begin{theorem}[Inner Products and Norms]
\label{theorem:IPN}

Suppose that $\vect{u}$ is a vector in $\complex{m}$.  Then
$\norm{\vect{u}}^2=\innerproduct{\vect{u}}{\vect{u}}$.

\begin{proof}

  \begin{align*}
    \norm{\vect{u}}^2 &= \left(\sqrt{\sum_{i=1}^{m}\modulus{\vectorentry{\vect{u}}{i}}^2}\right)^2 \\ 
    &=\sum_{i=1}^{m}\modulus{\vectorentry{\vect{u}}{i}}^2 \\ 
    &=\sum_{i=1}^{m}\conjugate{\vectorentry{\vect{u}}{i}}\vectorentry{\vect{u}}{i} \\ 
    &=\innerproduct{\vect{u}}{\vect{u}} \\ 
  \end{align*}

\end{proof}
\end{theorem}

When our vectors have entries only from the real, thend the dot
product of a vector with itself is equal to the length of the vector
squared.

\begin{theorem}[Positive Inner Products]
\label{theorem:PIP}

Suppose that $\vect{u}$ is a vector in $\complex{m}$.  Then
$\innerproduct{\vect{u}}{\vect{u}}\geq 0$ with equality if and only if
$\vect{u}=\zerovector$.

\begin{proof}
  From the proof of \ref{theorem:IPN}  we see that
  \[
    \innerproduct{\vect{u}}{\vect{u}}
    =
    \modulus{\vectorentry{\vect{u}}{1}}^2+
    \modulus{\vectorentry{\vect{u}}{2}}^2+
    \modulus{\vectorentry{\vect{u}}{3}}^2+
    \cdots+
    \modulus{\vectorentry{\vect{u}}{m}}^2
  \]
  
  Since each modulus is squared, every term is positive, and the sum
  must also be positive.  (Notice that in general the inner product is
  a complex number and cannot be compared with zero, but in the
  special case of $\innerproduct{\vect{u}}{\vect{u}}$ the result is a
  real number.)

  The phrase, ``with equality if and only if'' means that we want to
  show that the statement $\innerproduct{\vect{u}}{\vect{u}}= 0$
  (i.e., with equality) is equivalent (``if and only if'') to the
  statement $\vect{u}=\zerovector$.

  If $\vect{u}=\zerovector$, then it is a straightforward computation
  to see that $\innerproduct{\vect{u}}{\vect{u}}= 0$.  In the other
  direction, assume that $\innerproduct{\vect{u}}{\vect{u}}= 0$.  As
  before, $\innerproduct{\vect{u}}{\vect{u}}$ is a sum of moduli.  So
  we have
  \[
    0=\innerproduct{\vect{u}}{\vect{u}}=
    \modulus{\vectorentry{\vect{u}}{1}}^2+
    \modulus{\vectorentry{\vect{u}}{2}}^2+
    \modulus{\vectorentry{\vect{u}}{3}}^2+
    \cdots+
    \modulus{\vectorentry{\vect{u}}{m}}^2
  \]

  Now we have a sum of squares equaling zero, so each term must be
  zero.  Then by similar logic,
  $\modulus{\vectorentry{\vect{u}}{i}}=0$ will imply that
  $\vectorentry{\vect{u}}{i}=0$, since $0+0i$ is the only complex
  number with zero modulus.  Thus every entry of $\vect{u}$ is zero
  and so $\vect{u}=\zerovector$, as desired.

\end{proof}
\end{theorem}

Notice that \ref{theorem:PIP} contains \textit{three} implications:
\begin{align*}
  \vect{u}\in\complex{m}&\Rightarrow\innerproduct{\vect{u}}{\vect{u}}\geq 0\\
  \vect{u}=\zerovector&\Rightarrow\innerproduct{\vect{u}}{\vect{u}}=0\\
  \innerproduct{\vect{u}}{\vect{u}}=0&\Rightarrow\vect{u}=\zerovector
\end{align*}

The results contained in \ref{theorem:PIP} are summarized by saying
``the inner product is \dfn{positive definite}.''


\end{document}

