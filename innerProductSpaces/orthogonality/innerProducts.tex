\documentclass{ximera}

\input{../../preamble.tex}

\title{Inner products}

\begin{document}
\begin{abstract}
  The inner product takes two vectors and produces a scalar.
\end{abstract}
\maketitle

\begin{definition}[Inner Product]
Given the vectors $\vect{u},\,\vect{v}\in\complex{m}$ the \dfn{inner product} of $\vect{u}$ and $\vect{v}$ is the scalar quantity in $\complexes$,
\[
\innerproduct{\vect{u}}{\vect{v}}=
\conjugate{\vectorentry{\vect{u}}{1}}\vectorentry{\vect{v}}{1}+
\conjugate{\vectorentry{\vect{u}}{2}}\vectorentry{\vect{v}}{2}+
\conjugate{\vectorentry{\vect{u}}{3}}\vectorentry{\vect{v}}{3}+
\cdots+
\conjugate{\vectorentry{\vect{u}}{m}}\vectorentry{\vect{v}}{m}
=
\sum_{i=1}^{m}\conjugate{\vectorentry{\vect{u}}{i}}\vectorentry{\vect{v}}{i}
\]
\end{definition}

This operation is a bit different in that we begin with two vectors but produce a scalar.  Computing one is straightforward.

\begin{example}[Computing some inner products]
  The inner product of
  \begin{align*}
    \vect{u}=\colvector{2+3i\\5+2i\\-3+i}&&\text{and}&&
                                                        \vect{v}=\colvector{1+2i\\-4+5i\\0+5i}
  \end{align*}
  is $\answer{3+19i}$.
  
  \begin{hint}
    \begin{align*}
      \innerproduct{\vect{u}}{\vect{v}}
      &=(\conjugate{2+3i})(1+2i)+(\conjugate{5+2i})(-4+5i)+(\conjugate{-3+i})(0+5i)\\
      &=(2-3i)(1+2i)+(5-2i)(-4+5i)+(-3-i)(0+5i)\\
      &=(8+i)+(-10+33i)+(5-15i)\\
      &=3+19i
    \end{align*}
  \end{hint}
\end{example}

\begin{example}[Computing some more inner products]
  The inner product of
  \begin{align*}
    \vect{w}=\colvector{2\\4\\-3\\2\\8}&&\text{and}&&
                                                      \vect{x}=\colvector{3\\1\\0\\-1\\-2}
  \end{align*}
  is $\answer{-8}$.
  
  \begin{hint}
    \begin{align*}
      \innerproduct{\vect{w}}{\vect{x}}&=
                                         (\conjugate{2})3+(\conjugate{4})1+(\conjugate{-3})0+(\conjugate{2})(-1)+(\conjugate{8})(-2)\\
                                       &=2(3)+4(1)+(-3)0+2(-1)+8(-2)=-8.
    \end{align*}
  \end{hint}
\end{example}

In the case where the entries of our vectors are all real numbers (as
in the second part of \ref{example:CSIP}), the computation of the
inner product may look familiar and be known to you as a \dfn{dot
  product} or \dfn{scalar product}.  So you can view the inner product
as a generalization of the scalar product to vectors from
$\complex{m}$ (rather than ${\mathbb R}^m$).

\begin{warning}
  Note that we have chosen to conjugate the entries of the
  \textit{first} vector listed in the inner product, while it is
  almost equally feasible to conjugate entries from the
  \textit{second} vector instead.  Conjugating the first vector leads
  to much nicer formulas for certain matrix decompositions and also
  shortens some proofs.
\end{warning}

There are several quick theorems we can now prove, and they will each be useful later.

\begin{theorem}[Inner Product and Vector Addition]
  \label{theorem:IPVA}
  
  Suppose $\vect{u},\,\vect{v},\,\vect{w}\in\complex{m}$.  Then
  \begin{itemize}
  \item $\innerproduct{\vect{u}+\vect{v}}{\vect{w}}=\innerproduct{\vect{u}}{\vect{w}}+\innerproduct{\vect{v}}{\vect{w}}$
  \item $\innerproduct{\vect{u}}{\vect{v}+\vect{w}}=\innerproduct{\vect{u}}{\vect{v}}+\innerproduct{\vect{u}}{\vect{w}}$
  \end{itemize}

  \begin{proof}
    The proofs of the two parts are very similar, with the second one requiring just a bit more effort due to the conjugation that occurs.  We will prove part 1 and you can prove part 2 (<acroref type="exercise" acro="O.T10" />).
    \begin{align*}
      \innerproduct{\vect{u}+\vect{v}}{\vect{w}}
      &=\sum_{i=1}^{m}\conjugate{\vectorentry{\vect{u}+\vect{v}}{i}}\vectorentry{\vect{w}}{i}
      \\ %&&\ref{definition:IP}\\
      &=\sum_{i=1}^{m}\left(\conjugate{\vectorentry{\vect{u}}{i}+
        \vectorentry{\vect{v}}{i}}\right)\vectorentry{\vect{w}}{i}
      \\ %&&\ref{definition:CVA}\\
      &=\sum_{i=1}^{m}\left(\conjugate{\vectorentry{\vect{u}}{i}}+
        \conjugate{\vectorentry{\vect{v}}{i}}\right)\vectorentry{\vect{w}}{i}
      \\ %&&\ref{theorem:CCRA}\\
      &=\sum_{i=1}^{m}\conjugate{\vectorentry{\vect{u}}{i}}\vectorentry{\vect{w}}{i}
        + \conjugate{\vectorentry{\vect{v}}{i}}\vectorentry{\vect{w}}{i}
      \\ %&&\ref{property:DCN}\\
      &=\sum_{i=1}^{m}\conjugate{\vectorentry{\vect{u}}{i}}\vectorentry{\vect{w}}{i}
        +\sum_{i=1}^{m}\conjugate{\vectorentry{\vect{v}}{i}}\vectorentry{\vect{w}}{i}
      \\ %&&\ref{property:CACN}\\
      &=\innerproduct{\vect{u}}{\vect{w}}+\innerproduct{\vect{v}}{\vect{w}} %&&\ref{definition:IP}
    \end{align*}
\end{proof}
\end{theorem}

\begin{theorem}[Inner Product and Scalar Multiplication]
\label{theorem:IPSM}

Suppose $\vect{u},\,\vect{v}\in\complex{m}$ and $\alpha\in\complexes$.  Then
\begin{itemize}
\item $\innerproduct{\alpha\vect{u}}{\vect{v}}=\conjugate{\alpha}\innerproduct{\vect{u}}{\vect{v}}$
\item $\innerproduct{\vect{u}}{\alpha\vect{v}}=\alpha\innerproduct{\vect{u}}{\vect{v}}$
\end{itemize}

\begin{proof}
  The proofs of the two parts are very similar, with the second one requiring just a bit more effort due to the conjugation that occurs.  We will prove part 1.
  \begin{align*}
    \innerproduct{\alpha\vect{u}}{\vect{v}}
    &=\sum_{i=1}^{m}\conjugate{\vectorentry{\alpha\vect{u}}{i}}\vectorentry{\vect{v}}{i}
    \\ %&&\ref{definition:IP}\\
    &=\sum_{i=1}^{m}\conjugate{\alpha\vectorentry{\vect{u}}{i}}\vectorentry{\vect{v}}{i}
    \\ %&&\ref{definition:CVSM}\\
    &=\sum_{i=1}^{m}\conjugate{\alpha}\,\conjugate{\vectorentry{\vect{u}}{i}}\vectorentry{\vect{v}}{i}
    \\ %&&\ref{theorem:CCRM}\\
    &=\conjugate{\alpha}\sum_{i=1}^{m}\conjugate{\vectorentry{\vect{u}}{i}}\vectorentry{\vect{v}}{i}
    \\ %&&\ref{property:DCN}\\
    &=\conjugate{\alpha}\innerproduct{\vect{u}}{\vect{v}}
    \\ %&&\ref{definition:IP}
  \end{align*}
\end{proof}
\end{theorem}

\begin{theorem}
\label{theorem:IPAC}
[Inner Product is Anti-Commutative]

Suppose that $\vect{u}$ and $\vect{v}$ are vectors in $\complex{m}$.  Then
$\innerproduct{\vect{u}}{\vect{v}}=\conjugate{\innerproduct{\vect{v}}{\vect{u}}}$.





\begin{proof}

\begin{align*}
\innerproduct{\vect{u}}{\vect{v}}
&=\sum_{i=1}^{m}\conjugate{\vectorentry{\vect{u}}{i}}\vectorentry{\vect{v}}{i}
\\ %&&\ref{definition:IP}\\
&=\sum_{i=1}^{m}\conjugate{\vectorentry{\vect{u}}{i}}\,\conjugate{\conjugate{\vectorentry{\vect{v}}{i}}}
\\ %&&\ref{theorem:CCT}\\
&=\sum_{i=1}^{m}\conjugate{\vectorentry{\vect{u}}{i}\conjugate{\vectorentry{\vect{v}}{i}}}
\\ %&&\ref{theorem:CCRM}\\
&=\conjugate{\left(\sum_{i=1}^{m}\vectorentry{\vect{u}}{i}\conjugate{\vectorentry{\vect{v}}{i}}\right)}
\\ %&&\ref{theorem:CCRA}\\
&=\conjugate{\left(\sum_{i=1}^{m}\conjugate{\vectorentry{\vect{v}}{i}}\vectorentry{\vect{u}}{i}\right)}
\\ %&&\ref{property:CMCN}\\
&=\conjugate{\innerproduct{\vect{v}}{\vect{u}}}
\\ %&&\ref{definition:IP}\\
\end{align*}




\end{proof}
\end{theorem}

\end{document}
