\documentclass{ximera}
\input{../preamble.tex}
\title{The dot product in $\mathbb R^n$}
\author{Crichton Ogle}

\begin{document}
\begin{abstract}
\end{abstract}
\maketitle

Consider $\mathbb R^n$ equipped with its standard basis (so that vectors in $\mathbb R^n$ are canonically identified with their $n\times 1$ coordinate representations). Then given ${\bf v} = [v_1\ v_2\ \dots\ v_n], {\bf w} = [w_1\ w_2\ \dots\ w_n]^T\in\mathbb R^n$, their {\it dot product} (also referred to as {\it scalar product}) is given by
\[
{\bf v}\cdot{\bf w} := {\bf v}^T*{\bf w} = \sum_{i=1}^n v_iw_i
\]

This operation on pairs of vectors satisfies three basic properties

\begin{description}
\item[(IP1)] It is symmetric:
\[
{\bf v}\cdot {\bf w} = {\bf w}\cdot {\bf v}
\]
\item[(IP2)] It is bilinear:
\begin{gather*}
(\alpha_1{\bf v}_1 + \alpha_2{\bf v}_2)\cdot {\bf w} = \alpha_1{\bf v}_1\cdot{\bf w} + \alpha_2{\bf v}_2\cdot{\bf w}\\
{\bf v}\cdot (\beta_1{\bf w}_1 + \beta_2{\bf w}_2) = \beta_1{\bf v}\cdot{\bf w}_1 + \beta_2{\bf v}\cdot{\bf w}_2
\end{gather*}
\item[(IP3)] It is positive non-degenerate:
\[
{\bf v}\cdot {\bf v}\ge 0;\quad {\bf v}\cdot {\bf v} = 0\,\text{ iff } {\bf v} = {\bf 0}
\]
\end{description}
\vskip.2in

Note that the standard {\it Euclidean norm} of $\bf v$  (also referred to as the $\ell^2$-norm) is closely related to the dot product; precisely
\begin{equation}
\|{\bf v}\| = \|{\bf v}\|_2 = \sqrt{v_1^2 + v_2^2 +\dots v_n^2} = ({\bf v}\cdot {\bf v})^{\frac12}
\end{equation}

Its essential features are

\begin{enumerate}
\item[(N1)] It is positive definite:
\[
\|{\bf v}\|\ge 0;\quad \|{\bf v}\| = 0 \text{ iff } {\bf v} = 0
\]
\item[(N2)] It satisfies the triangle inequality (for norms):
\[
\|{\bf v} + {\bf w}\|\le \|{\bf v}\| + \|{\bf w}\|
\]
\end{enumerate}
\vskip.2in

This is the norm used in $\mathbb R^n$ to define the standard Euclidean {\it metric}, which is the conventional way to measure the distance between two vectors:
\begin{equation}
d({\bf v},{\bf w}) := \|{\bf v} - {\bf w}\|_2
\end{equation}

Again, this distance function - or metric - satisfies three basic properties, which are direct consequences of the ones above.

\begin{enumerate}
\item[(M1)] It is symmetric:
\[
d({\bf v},{\bf w}) = d({\bf w},{\bf v})
\]
\item[(M2)] It is positive non-degenerate:
\[
d({\bf v},{\bf w})\ge 0\,\,\forall {\bf v}, {\bf w}\in\mathbb R^n;\text{ moreover } d({\bf v},{\bf w}) = 0\,\text{ iff } {\bf v} = {\bf w}
\]
\item[(M3]) It satisfies the triangle inequality (for metrics):
\[
d({\bf u},{\bf w})\le d({\bf u},{\bf v}) + d({\bf v},{\bf w})\quad\forall {\bf u}, {\bf v}, {\bf w}\in\mathbb R^n
\]
\end{enumerate}
\vskip.2in

So the i) dot product, ii) Euclidean norm, and iii) Euclidean distance are all closely related. In fact, any one of them determines the other two. Obviously, the dot product determines the norm, and the norm determines the distance. But also one has

\begin{itemize}
\item the equality
\[
{\bf v}\cdot {\bf w} = \frac12(\|{\bf v} + {\bf w}\|^2 - \|{\bf v}\|^2 - \|{\bf w}\|^2)
\]
so that one can also recover the dot product from the norm;
\item $\|{\bf v}\| = d({\bf v}, {\bf 0})$, so $d(_-,_-)$ and $\|_-\|$ determine each other.
\end{itemize}
\vskip.3in

If $\bf v$, $\bf w$ are two non-zero vectors in $\mathbb R^n$, then there span is at most two-dimensional, and so we may choose a plane in $\mathbb R^n$ containing them. Within that plane the vectors determine two angles. We define $\theta_{{\bf v}, {\bf w}}$ to be the smaller of the two angles; this angle will always lie between $0$ and $\pi$ radians\vskip.2in

[Note: Except in the case the two vectors are colinear and pointing in opposite directions, the smaller angle will be unique and less than $\pi$ radians. In that one remaining case the choice of angle representative will not be unique, but both angles will be equal to $\pi$, so the angle itself is well-defined.]
\vskip.2in

The angle formula is given by the following lemma.

\begin{lemma} For $\mathbb R^n\ni {\bf v}, {\bf w}\ne 0$, one has
\[
{\bf v}\cdot{\bf w} = \|{\bf v}\| \|{\bf w}\|\cos(\theta_{{\bf v},{\bf w}})
\]

\end{lemma}

As $\cos$ is 1-1 on the interval $[0,\pi]$, this equality implies

\begin{corollary} For ${\bf v}, {\bf w}$ as above, one has
\[
\theta_{{\bf v},{\bf w}} = \arccos\left({\bf v}\cdot{\bf w}/(\|{\bf v}\| \|{\bf w}\|)\right)
\]

\end{corollary}


\begin{example} Suppose ${\bf v} = \begin{bmatrix} 3\\ -1\\ 4  \end{bmatrix}, {\bf w} = \begin{bmatrix} -2\\ 1\\ 5\end{bmatrix}$, and we want to compute the angle $\theta_{{\bf v}, {\bf w}}$ determined by $\bf v$ and $bf w$.
\vskip.1in
Using Octave or MATLAB, we see that (using "short" format)
\begin{itemize}
\item ${\bf v}\cdot{\bf w} = 13$;
\item $\|{\bf v}\| = 5.0990$;
\item $\|{\bf w}\| = 5.4772$;
\item Therefore
\begin{gather*}
\theta_{{\bf v},{\bf w}} = \arccos\left({\bf v}\cdot{\bf w}/(\|{\bf v}\| \|{\bf w}\|)\right)
= \arccos(13/(5.0990*5.4772))
= \arccos(0.4655) = 1.0866\mathtext{\rm\ radians}
\end{gather*}
\end{itemize}


\end{example}

%%%%%%%%%%%%%%%%%%%%%%%%%%%%%%%%%%%%%%

\end{document}
