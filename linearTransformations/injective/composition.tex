\documentclass{ximera}

\input{../../preamble.tex}

\title{Composition of Injective Linear Transformations}

\begin{document}
\begin{abstract}
  The composition of injective linear transformations is again injective.
\end{abstract}
\maketitle

We saw how to combine linear transformations to build new linear
transformations, specifically, how to build the composition of two
linear transformations (\ref{definition:LTC}).  It will be useful
later to know that the composition of injective linear transformations
is again injective, so we prove that here.


\begin{theorem}[Composition of Injective Linear Transformations is Injective]
\label{theorem:CILTI}


Suppose that $\ltdefn{T}{U}{V}$ and $\ltdefn{S}{V}{W}$ are injective linear transformations.  Then $\ltdefn{(\compose{S}{T})}{U}{W}$ is \wordChoice{\choice[correct]{an injective}\choice{a surjective}} linear transformation.


\begin{proof}
That the composition is a linear transformation was established in \ref{theorem:CLTLT}, so we need only establish that the composition is injective.  Applying \ref{definition:ILT}, choose $\vect{x}$, $\vect{y}$ from $U$.  Then if $\lteval{\left(\compose{S}{T}\right)}{\vect{x}}=\lteval{\left(\compose{S}{T}\right)}{\vect{y}}$,
\begin{align*}
&\Rightarrow&\lteval{S}{\lteval{T}{\vect{x}}}&=\lteval{S}{\lteval{T}{\vect{y}}}
&&\ref{definition:LTC}\\
&\Rightarrow&\lteval{T}{\vect{x}}&=\lteval{T}{\vect{y}}
&&\ref{definition:ILT}\text{ for }S\\
&\Rightarrow&\vect{x}&=\vect{y}
&&\ref{definition:ILT}\text{ for }T
\end{align*}




\end{proof}
\end{theorem}


\end{document}
