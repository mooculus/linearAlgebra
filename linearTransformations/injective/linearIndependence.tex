\documentclass{ximera}

\input{../../preamble.tex}

\title{Injective Linear Transformations and Linear Independence}

\begin{document}
\begin{abstract}
There is a connection between injective linear transformations and linearly independent sets that we will make precise.
\end{abstract}
\maketitle


Informally, we can get a feel for the connection between injective
linear transformations and linearly independent sets when we think
about how each property is defined.  A set of vectors is linearly
independent if the \textbf{only} relation of linear dependence is the
trivial one.  A linear transformation is injective if the
\textbf{only} way two input vectors can produce the same output is in
the trivial way, when both input vectors are equal.


\begin{theorem}[Injective Linear Transformations and Linear Independence]
\label{theorem:ILTLI}

Suppose that $\ltdefn{T}{U}{V}$ is an injective linear transformation and
\begin{align*}
S&=\set{\vectorlist{u}{t}}
\end{align*}
is a linearly independent subset of $U$.  Then
\begin{align*}
R&=\set{\lteval{T}{\vect{u}_1},\,\lteval{T}{\vect{u}_2},\,\lteval{T}{\vect{u}_3},\,\ldots,\,\lteval{T}{\vect{u}_t}}
\end{align*}
is a linearly independent subset of $V$.

\begin{proof}
Begin with a relation of linear dependence on $R$ (\ref{definition:RLD}, \ref{definition:LI}),
\begin{align*}
a_1\lteval{T}{\vect{u}_1}+a_2\lteval{T}{\vect{u}_2}+a_3\lteval{T}{\vect{u}_3}+\ldots+a_t\lteval{T}{\vect{u}_t}&=\zerovector\\
\lteval{T}{\lincombo{a}{u}{t}}&=\zerovector&&\ref{theorem:LTLC}\\
\lincombo{a}{u}{t}&\in\krn{T}&&\ref{definition:KLT}\\
\lincombo{a}{u}{t}&\in\set{\zerovector}&&\ref{theorem:KILT}\\
\lincombo{a}{u}{t}&=\zerovector&&\ref{definition:SET}\\
\end{align*}

Since this is a relation of linear dependence on the linearly independent set $S$, we can conclude that
\begin{align*}
a_1&=0&a_2&=0&a_3&=0&\ldots&&a_t&=0
\end{align*}
and this establishes that $R$ is a linearly independent set.

\end{proof}
\end{theorem}

\begin{theorem}[Injective Linear Transformations and Bases]
\label{theorem:ILTB}

Suppose that $\ltdefn{T}{U}{V}$ is a linear transformation and
\begin{align*}
B&=\set{\vectorlist{u}{m}}
\end{align*}
is a basis of $U$.  Then $T$ is injective if and only if
\begin{align*}
C&=\set{\lteval{T}{\vect{u}_1},\,\lteval{T}{\vect{u}_2},\,\lteval{T}{\vect{u}_3},\,\ldots,\,\lteval{T}{\vect{u}_m}}
\end{align*}
is a linearly independent subset of $V$.


\begin{proof}
($\Rightarrow$)  Assume $T$ is injective.  Since $B$ is a basis, we know $B$ is linearly independent (\ref{definition:B}).  Then \ref{theorem:ILTLI} says that $C$ is a linearly independent subset of $V$.



($\Leftarrow$)  Assume that $C$ is linearly independent.  To establish that $T$ is injective, we will show that the kernel of $T$ is trivial (\ref{theorem:KILT}).  Suppose that $\vect{u}\in\krn{T}$.  As an element of $U$, we can write $\vect{u}$ as a linear combination of the basis vectors in $B$ (uniquely).  So there are are scalars, $\scalarlist{a}{m}$, such that
\[
\vect{u}=\lincombo{a}{u}{m}
\]




Then,
\begin{align*}
\zerovector
&=\lteval{T}{\vect{u}}
&&\ref{definition:KLT}\\
&=\lteval{T}{\lincombo{a}{u}{m}}
&&\ref{definition:SSVS}\\
&=a_1\lteval{T}{\vect{u}_1}+a_2\lteval{T}{\vect{u}_2}+a_3\lteval{T}{\vect{u}_3}+\cdots+a_m\lteval{T}{\vect{u}_m}
&&\ref{theorem:LTLC}
\end{align*}




This is a relation of linear dependence (\ref{definition:RLD}) on the linearly independent set $C$, so 
\begin{multipleChoice}
\choice{one of the scalars iszero:  $a_i \neq 0$.}
\choice[correct]{the scalars are all zero:  $a_1=a_2=a_3=\cdots=a_m=0$.}
\end{multipleChoice}

Then
\begin{align*}
\vect{u}&=\lincombo{a}{u}{m}\\
&=0\vect{u}_1+0\vect{u}_2+0\vect{u}_3+\cdots+0\vect{u}_m&&\ref{theorem:ZSSM}\\
&=\zerovector+\zerovector+\zerovector+\cdots+\zerovector&&\ref{theorem:ZSSM}\\
&=\zerovector&&\ref{property:Z}
\end{align*}


Since $\vect{u}$ was chosen as an arbitrary vector from $\krn{T}$, we have $\krn{T}=\set{\zerovector}$ and \ref{theorem:KILT} tells us that $T$ is injective.

\end{proof}
\end{theorem}

\end{document}
