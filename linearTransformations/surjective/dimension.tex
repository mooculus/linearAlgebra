\documentclass{ximera}

\input{../../preamble.tex}

\title{Surjective Linear Transformations and Dimension}

\begin{document}
\begin{abstract}
  Surjective functions cannot increase dimension.
\end{abstract}
\maketitle

\begin{theorem}[Surjective Linear Transformations and Dimension]
\label{theorem:SLTD}

Suppose that $\ltdefn{T}{U}{V}$ is a surjective linear transformation.  Then $\dimension{U}\geq\dimension{V}$.


\begin{proof}
Suppose to the contrary that
$m=\dimension{U}<\dimension{V}=t$.
Let $B$ be  a basis of $U$, which will then contain $m$ vectors.  Apply $T$ to each element of $B$ to form a set $C$ that is a subset of $V$.  By \ref{theorem:SLTB}, $C$ is a spanning set of $V$ with $m$ or fewer vectors.  So we have a set of $m$ or fewer vectors that span $V$, a vector space of dimension $t$, with
$m<t$.
However, this contradicts \ref{theorem:G}, so our assumption is false and $\dimension{U}\geq\dimension{V}$.



\end{proof}
\end{theorem}

\begin{example}

Consider the linear transformation 
\[\ltdefn{T}{P_4}{P_5},\quad
\lteval{T}{p(x)}=(x-2)p(x)
\]

Since
$\dimension{P_4}=5<6=\dimension{P_5}$, we may conclude
\begin{multipleChoice}
\choice[correct]{$T$ cannot be surjective}
\choice{$T$ must be surjective}
\end{multipleChoice}
\begin{feedback}[correct]
Exactly!  If it were, then it would violate \ref{theorem:SLTD}.
\end{feedback}

\end{example}

Notice that the previous example made no use of the actual formula defining the function.  Merely a comparison of the dimensions of the domain and codomain are enough to conclude that the linear transformation is not surjective.  There are examples of linear transformations that have ``small'' domains and ``big'' codomains, resulting in an inability to create all possible outputs and thus they are non-surjective linear transformations.

\end{document}
