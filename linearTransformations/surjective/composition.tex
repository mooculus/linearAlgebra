\documentclass{ximera}

\input{../../preamble.tex}

\title{Composition of Surjective Linear Transformations}

\begin{document}
\begin{abstract}
  Surjective functions cannot increase dimension.
\end{abstract}
\maketitle

We have already seen how to combine linear transformations to build
new linear transformations, specifically, how to build the composition
of two linear transformations (\ref{definition:LTC}).  It will be
useful later to know that the composition of surjective linear
transformations is again surjective, so we prove that here.

\begin{theorem}[Composition of Surjective Linear Transformations is Surjective]
\label{theorem:CSLTS}

Suppose that $\ltdefn{T}{U}{V}$ and $\ltdefn{S}{V}{W}$ are surjective linear transformations.  Then $\ltdefn{(\compose{S}{T})}{U}{W}$ is \wordChoice{\choice{an injective}\choice[correct]{a surjective}} linear transformation.


\begin{proof}
That the composition is a linear transformation was established in \ref{theorem:CLTLT}, so we need only establish that the composition is surjective.  Applying \ref{definition:SLT}, choose $\vect{w}\in W$.



Because $S$ is surjective, there must be a vector $\vect{v}\in V$, such that $\lteval{S}{\vect{v}}=\vect{w}$.  With the existence of $\vect{v}$ established, that $T$ is surjective guarantees a vector $\vect{u}\in U$ such that $\lteval{T}{\vect{u}}=\vect{v}$.  Now,
\begin{align*}
\lteval{\left(\compose{S}{T}\right)}{\vect{u}}&=\lteval{S}{\lteval{T}{\vect{u}}}&&\ref{definition:LTC}\\
&=\lteval{S}{\vect{v}}&&\text{Definition of $\vect{u}$}\\
&=\vect{w}&&\text{Definition of $\vect{v}$}
\end{align*}




This establishes that any element of the codomain ($\vect{w}$) can be created by evaluating $\compose{S}{T}$ with the right input ($\vect{u}$).  Thus, by \ref{definition:SLT}, $\compose{S}{T}$ is surjective.



\end{proof}
\end{theorem}

\end{document}
