\documentclass{ximera}
\input{../preamble.tex}
\title{Definition}
\author{Crichton Ogle}

\begin{document}
\begin{abstract}
  A linear transformation is a function between vector spaces preserving the structure of the vector spaces.
\end{abstract}
\maketitle

Suppose $V$ and $W$ are vector spaces (of arbitrary dimension). A {\it function} $f:V\to W$ is (as we recall) a rule which associates to each ${\bf v}\in V$ a unique vector $f({\bf v})\in W$. We will call such a function a {\it linear transformation} if it commutes with the linear structure in the domain and range:

\begin{eqnarray*}
f({\bf v} + {\bf w}) = f({\bf v}) + f({\bf w})\qquad\forall {\bf v,w}\in V\\
f(\alpha{\bf v}) = \alpha f({\bf v})\qquad\qquad\forall \alpha\in\mathbb R, {\bf v}\in V
\end{eqnarray*}

In other words, $f$ takes sums to sums and scalar products to scalar products. These two properties can be combined into one, using linear combinations. Precisely, 

\begin{definition} Let $V,W$ be vector spaces. A function $f:V\to W$ is a linear transformation if
\begin{equation}
f(\alpha{\bf v} + \beta{\bf w}) = \alpha f({\bf v}) + \beta f({\bf w})\qquad\forall \alpha,\beta\in\mathbb R, {\bf v}, {\bf w}\in V
\end{equation}
\end{definition}


\begin{exercise} Show that $f$ is a linear transformation iff
\[
f(\alpha_1{\bf v}_1 +\dots \alpha_n{\bf v}_n) = \alpha_1 f({\bf v}_1) +\dots \alpha_n f({\bf v}_n)\qquad\forall \alpha_i\in\mathbb R, {\bf v}_i\in V
\]
\end{exercise}

As a consequence of this exercise, we have

\begin{corollary} If $f: V\to W$ is a linear transformation, then it is uniquely determined by its values on a basis for $V$. Conversely, if $S := \{{\bf v}_1,\dots {\bf v}_n\}$ is a basis for $V$ and $g:S\to W$ is a function from the set $S$ to $W$, then $g$ extends uniquely to a linear transformation $f:V\to W$ with $f({\bf v}_i) = g({\bf v}_i), 1\le i\le n$. Thus, if $f_1, f_2:V\to W$ are two linear transformations which agree on a basis for $V$, then $f_1 = f_2$.
\end{corollary}
\vskip.2in

Linear transformations may be used to define subspaces. Let $L:V\to W$ be a linear transformation. The {\it kernel} of $L$ is then
\[
ker(L) := \{{\bf v}\in V\ |\ L({\bf v}) = {\bf 0}\}\subset V
\]
The {\it image} of $L$ is defined as
\[
im(L) := \{ {\bf w}\in W\ |\ \exists {\bf v}\in V \text{ with } L({\bf v}) = {\bf w}\} \subset W
\]
The image of $L$ is sometimes denoted $L(V)$. It is also referred to as the {\it range} of $L$. These subspaces are useful in defining specific types of linear transformations. We say $L:V\to W$ is

\begin{itemize}
\item {\it surjective} or {\it an epimorphism} iff $im(L) = W$;
\item {\it injective} or {\it a monomorphism} iff $ker(L) = \{\bf 0\}$;
\item {\it bijective} or {\it an isomorphism} iff $L$ is both surjective and injective.
\end{itemize}

We say that two vector spaces $V$ and $W$ are {\it isomorphic} if there exists a linear transformation $L:V\to W$ which is an isomorphism.

\begin{theorem}\label{thm:rank-null} For any linear transformation $L:V\to W$, $ker(L)$ is a subspace of $V$ and $im(L)$ is a subspace of $W$. Moreover, 
\[
Dim(V) = Dim(ker(L)) + Dim(im(L))
\]

\end{theorem}
\vskip.2in

It is natural to ask: what does it mean for a map to be a linear transformation? For example, if $f:\mathbb R^n\to \mathbb R^m$, or more generally if $f:V\to W$ satisfies the above property, does it admit some simple description? We investigate this question next.
\vskip.3in

\end{document}
