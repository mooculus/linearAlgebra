\documentclass{ximera}
\input{../preamble.tex}
\title{Complex eigenvalues and eigenvectors} 
\author{Crichton Ogle}

\begin{document}
\begin{abstract}
  There are advantages to working with complex numbers.
\end{abstract}
\maketitle

All of the constructions we have done so far over $\mathbb R$ extend naturally to $\mathbb C$, with some slight adjustment for the case of inner products (we will discuss this in more detail below). For now, the main reason for considering complex numbers has to do with the factorization of polynomials. The key result one want to know (whose proof involves techniques well beyond the scope of linear algebra) is

\begin{theorem} {\rm ( The Fundamental Theorem of Algebra)} Any non-constant polynomial $p(z)$ with complex coefficients has a complex root. Consequently, any non-constant polynomial with real or complex coefficients can be factored over $\mathbb C$ into a product of linear terms
\[
p(z) = c(z - r_1)(z - r_2)\dots (z - r_n),\qquad c,r_1,r_2,\dots,r_n\in\mathbb C
\]
\end{theorem}

\end{document}
