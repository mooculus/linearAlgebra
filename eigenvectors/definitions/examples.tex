\documentclass{ximera}

\input{../../preamble.tex}

\title{Examples of Computing Eigenvalues and Eigenvectors}

\begin{document}
\begin{abstract}
  A selection of examples illustrates the range of possibilities for the eigenvalues and eigenvectors of a matrix. 
\end{abstract}
\maketitle

These examples can all be done by hand, though the computation of the
characteristic polynomial would be very time-consuming and
error-prone.  It can also be difficult to find roots of an arbitrary
polynomial, though if we were to suggest that most of our eigenvalues
are going to be integers, then it can be easier to hunt for roots.
First, we will sneak in a pair of definitions so we can illustrate
them throughout this sequence of examples.

\begin{definition}[Algebraic Multiplicity of an Eigenvalue]
  Suppose that $A$ is a square matrix and $\lambda$ is an eigenvalue
  of $A$.  Then the \dfn{algebraic multiplicity} of $\lambda$,
  $\algmult{A}{\lambda}$, is the highest power of $(x-\lambda)$ that
  divides the characteristic polynomial, $\charpoly{A}{x}$.
\end{definition}

Since an eigenvalue $\lambda$ is a root of the characteristic
polynomial, there is always a factor of $(x-\lambda)$, and the
algebraic multiplicity is just the power of this factor in a
factorization of $\charpoly{A}{x}$.  So in particular,
$\algmult{A}{\lambda}\geq 1$.  Compare the definition of algebraic
multiplicity with the next definition.

\begin{definition}[Geometric Multiplicity of an Eigenvalue]
  Suppose that $A$ is a square matrix and $\lambda$ is an eigenvalue
  of $A$.  Then the \dfn{geometric multiplicity} of $\lambda$,
  $\geomult{A}{\lambda}$, is the dimension of the eigenspace
  $\eigenspace{A}{\lambda}$.
\end{definition}

Every eigenvalue must have at least one eigenvector, so the associated
eigenspace cannot be trivial, and so $\geomult{A}{\lambda}\geq 1$.

\begin{example}[Eigenvalue multiplicities, matrix of size 4]
  Consider the matrix
  \[
    B=
    \begin{bmatrix}
      -2 & 1 & -2 & -4\\
      12 & 1 & 4 & 9\\
      6 & 5 & -2 & -4\\
      3 & -4 & 5 & 10
    \end{bmatrix}
  \]
  then
  \[
    \charpoly{B}{x}=8-20x+18x^2-7x^3+x^4=(x-1)(x-\answer{2})^3
  \]
  So the eigenvalues are $\lambda=1,\,2$ with algebraic multiplicities
  $\algmult{B}{1}=1$ and $\algmult{B}{2}=\answer{3}$.

  Computing eigenvectors,
  \begin{align*}
    \lambda&=1&B- 1I_4&=
                        \begin{bmatrix}
                          -3 & 1 & -2 & -4\\
                          12 & 0 & 4 & 9\\
                          6 & 5 & -3 & -4\\
                          3 & -4 & 5 & 9
                        \end{bmatrix}
                                       \rref
                                       \begin{bmatrix}
                                         \leading{1} & 0 & \frac{1}{3} & 0\\
                                         0 & \leading{1} & -1 & 0\\
                                         0 & 0 & 0 & \leading{1}\\
                                         0 & 0 & 0 & 0
                                       \end{bmatrix}\\
           &&\eigenspace{B}{1}&=\nsp{B-1I_4}
                                =\spn{\set{\colvector{-\frac{1}{3}\\1\\1\\0}}}
    =\spn{\set{\colvector{-1\\3\\3\\0}}}\\
    \lambda&=2&B-2I_4&=
                       \begin{bmatrix}
                         -4 & 1 & -2 & -4\\
                         12 & -1 & 4 & 9\\
                         6 & 5 & -4 & -4\\
                         3 & -4 & 5 & 8
                       \end{bmatrix}
                                      \rref
                                      \begin{bmatrix}
                                        \leading{1} & 0 & 0 & 1/2\\
                                        0 & \leading{1} & 0 & -1\\
                                        0 & 0 & \leading{1} & 1/2\\
                                        0 & 0 & 0 & 0
                                      \end{bmatrix}\\
           &&\eigenspace{B}{2}&=\nsp{B-2I_4}
                                =\spn{\set{\colvector{-\frac{1}{2}\\1\\-\frac{1}{2}\\1}}}
    =\spn{\set{\colvector{-1\\2\\-1\\2}}}\\
  \end{align*}
  
  So each eigenspace has dimension 1 and so
  $\geomult{B}{1}=\answer{1}$ and $\geomult{B}{2}=\answer{1}$.  This
  example is of interest because of the discrepancy between the two
  multiplicities for $\lambda=\answer{2}$.  In many of our examples
  the algebraic and geometric multiplicities will be equal for all of
  the eigenvalues (as it was for $\lambda=1$ in this example), so keep
  this example in mind.  We will have some explanations for this
  phenomenon later.
\end{example}

\begin{example}[Eigenvalues, symmetric matrix of size 4]
  Consider the matrix
  \[
    C=
    \begin{bmatrix}
      1 &  0 &  1 &  1\\
      0 &  1 &  1 &  1\\
      1 &  1 &  1 &  0\\
      1 &  1 &  0 &  1
    \end{bmatrix}
  \]
  then
  \[
    \charpoly{C}{x}=-3+4x+2x^2-4x^3+x^4=(x-3)(x-1)^2(x+1)
  \]
  So the eigenvalues are $\lambda=3,\,1,\,-1$ with algebraic
  multiplicities $\algmult{C}{3}=1$, $\algmult{C}{1}=\answer{2}$ and
  $\algmult{C}{-1}=1$.

  Computing eigenvectors,
  \begin{align*}
    \lambda&=3&C- 3I_4&=
                        \begin{bmatrix}
                          -2 & 0 & 1 & 1\\
                          0 & -2 & 1 & 1\\
                          1 & 1 & -2 & 0\\
                          1 & 1 & 0 & -2
                        \end{bmatrix}
                                      \rref
                                      \begin{bmatrix}
                                        \leading{1} & 0 & 0 & -1\\
                                        0 & \leading{1} & 0 & -1\\
                                        0 & 0 & \leading{1} & -1\\
                                        0 & 0 & 0 & 0
                                      \end{bmatrix}\\
           &&\eigenspace{C}{3}&=\nsp{C-3I_4}
                                =\spn{\set{\colvector{1\\1\\1\\1}}}\\
    \lambda&=1&C-1I_4&=
                       \begin{bmatrix}
                         0 & 0 & 1 & 1\\
                         0 & 0 & 1 & 1\\
                         1 & 1 & 0 & 0\\
                         1 & 1 & 0 & 0
                       \end{bmatrix}
                                     \rref
                                     \begin{bmatrix}
                                       \leading{1} & 1 & 0 & 0\\
                                       0 & 0 & \leading{1} & 1\\
                                       0 & 0 & 0 & 0\\
                                       0 & 0 & 0 & 0
                                     \end{bmatrix}\\
           &&\eigenspace{C}{1}&=\nsp{C-1I_4}
                                =\spn{\set{\colvector{-1\\1\\0\\0},\,\colvector{0\\0\\-1\\1}}}\\
    \lambda&=-1&C+1I_4&=
                        \begin{bmatrix}
                          2 & 0 & 1 & 1\\
                          0 & 2 & 1 & 1\\
                          1 & 1 & 2 & 0\\
                          1 & 1 & 0 & 2
                        \end{bmatrix}
                                      \rref
                                      \begin{bmatrix}
                                        \leading{1} & 0 & 0 & 1\\
                                        0 & \leading{1} & 0 & 1\\
                                        0 & 0 & \leading{1} & -1\\
                                        0 & 0 & 0 & 0
                                      \end{bmatrix}\\
           &&\eigenspace{C}{-1}&=\nsp{C+1I_4}
                                 =\spn{\set{\colvector{-1\\-1\\1\\1}}}\\
  \end{align*}

  So the eigenspace dimensions yield geometric multiplicities
  $\geomult{C}{3}=1$, $\geomult{C}{1}=\answer{2}$ and
  $\geomult{C}{-1}=1$, the same as for the algebraic multiplicities.
  This example is of interest because $A$ is a symmetric matrix, and
  will be the subject of \ref{theorem:HMRE}.

\end{example}

\begin{example}[High multiplicity eigenvalues, matrix of size 5]
  
  Consider the matrix
  \[
    E=
    \begin{bmatrix}
      29 & 14 & 2 & 6 & -9\\
      -47 & -22 & -1 & -11 & 13\\
      19 & 10 & 5 & 4 & -8\\
      -19 & -10 & -3 & -2 & 8\\
      7 & 4 & 3 & 1 & -3
    \end{bmatrix}
  \]
  then
  \[
    \charpoly{E}{x}=-16+16x+8x^2-16x^3+7x^4-x^5=-(x-2)^4(x+1)
  \]
  So the eigenvalues are $\lambda=2,\,-1$ with algebraic multiplicities $\algmult{E}{2}=\answer{4}$  and $\algmult{E}{-1}=1$.
  
  Computing eigenvectors,
  \begin{align*}
    \lambda&=2\\
    E-2I_5&=
            \begin{bmatrix}
              27 & 14 & 2 & 6 & -9\\
              -47 & -24 & -1 & -11 & 13\\
              19 & 10 & 3 & 4 & -8\\
              -19 & -10 & -3 & -4 & 8\\
              7 & 4 & 3 & 1 & -5
            \end{bmatrix}
                              \rref
                              \begin{bmatrix}
                                \leading{1} & 0 & 0 & 1 & 0\\
                                0 & \leading{1} & 0 & -\frac{3}{2} & -\frac{1}{2}\\
                                0 & 0 & \leading{1} & 0 & -1\\
                                0 & 0 & 0 & 0 & 0\\
                                0 & 0 & 0 & 0 & 0
                              \end{bmatrix}\\
    \eigenspace{E}{2}&=\nsp{E-2I_5}
                       =\spn{\set{\colvector{-1\\\frac{3}{2}\\0\\1\\0},\,\colvector{0\\\frac{1}{2}\\1\\0\\1}}}
    =\spn{\set{\colvector{-2\\3\\0\\2\\0},\,\colvector{0\\1\\2\\0\\2}}}\\
    \lambda&=-1\\
    E+1I_5&=
            \begin{bmatrix}
              30 & 14 & 2 & 6 & -9\\
              -47 & -21 & -1 & -11 & 13\\
              19 & 10 & 6 & 4 & -8\\
              -19 & -10 & -3 & -1 & 8\\
              7 & 4 & 3 & 1 & -2
            \end{bmatrix}
                              \rref
                              \begin{bmatrix}
                                \leading{1} & 0 & 0 & 2 & 0\\
                                0 & \leading{1} & 0 & -4 & 0\\
                                0 & 0 & \leading{1} & 1 & 0\\
                                0 & 0 & 0 & 0 & \leading{1}\\
                                0 & 0 & 0 & 0 & 0
                              \end{bmatrix}\\
    \eigenspace{E}{-1}&=\nsp{E+1I_5}=\spn{\set{\colvector{-2\\4\\-1\\1\\0}}}\\
  \end{align*}
  
  So the eigenspace dimensions yield geometric multiplicities
  $\geomult{E}{2}=\answer{2}$ and $\geomult{E}{-1}=1$.  This example
  is of interest because $\lambda=2$ has such a large algebraic
  multiplicity, which is also not equal to its geometric multiplicity.

\end{example}

\begin{example}[Complex eigenvalues, matrix of size 6]

  Consider the matrix
  \[
    F=
    \begin{bmatrix}
      -59 & -34 & 41 & 12 & 25 & 30\\
      1 & 7 & -46 & -36 & -11 & -29\\
      -233 & -119 & 58 & -35 & 75 & 54\\
      157 & 81 & -43 & 21 & -51 & -39\\
      -91 & -48 & 32 & -5 & 32 & 26\\
      209 & 107 & -55 & 28 & -69 & -50
    \end{bmatrix}
  \]
  then
  \begin{align*}
    \charpoly{F}{x}&=-50+55x+13x^2-50x^3+32x^4-9x^5+x^6\\
                   &=(x-2)(x+1)(x^2-4x+5)^2\\
                   &=(x-2)(x+1)((x-(2+i))(x-(2-i)))^2\\
                   &=(x-2)(x+1)(x-(2+i))^2(x-(2-i))^2\\
  \end{align*}
  So the eigenvalues are $\lambda=2,\,-1,2+i,\,2-i$ with algebraic
  multiplicities $\algmult{F}{2}=1$, $\algmult{F}{-1}=1$,
  $\algmult{F}{2+i}=2$ and $\algmult{F}{2-i}=2$.

  We compute eigenvectors, noting that the last two basis vectors are
  each a scalar multiple of what \ref{theorem:BNS} will provide,
  \begin{align*}
    \lambda&=2\quad\quad F-2I_6=\\
           &
             \begin{bmatrix}
               -61 & -34 & 41 & 12 & 25 & 30\\
               1 & 5 & -46 & -36 & -11 & -29\\
               -233 & -119 & 56 & -35 & 75 & 54\\
               157 & 81 & -43 & 19 & -51 & -39\\
               -91 & -48 & 32 & -5 & 30 & 26\\
               209 & 107 & -55 & 28 & -69 & -52
             \end{bmatrix}
                                            \rref
                                            \begin{bmatrix}
                                              \leading{1} & 0 & 0 & 0 & 0 & \frac{1}{5}\\
                                              0 & \leading{1} & 0 & 0 & 0 & 0\\
                                              0 & 0 & \leading{1} & 0 & 0 & \frac{3}{5}\\
                                              0 & 0 & 0 & \leading{1} & 0 & -\frac{1}{5}\\
                                              0 & 0 & 0 & 0 & \leading{1} & \frac{4}{5}\\
                                              0 & 0 & 0 & 0 & 0 & 0
                                            \end{bmatrix}\\
           &\eigenspace{F}{2}=\nsp{F-2I_6}
             =\spn{\set{\colvector{-\frac{1}{5}\\0\\-\frac{3}{5}\\\frac{1}{5}\\-\frac{4}{5}\\1}}}
    =\spn{\set{\colvector{-1\\0\\-3\\1\\-4\\5}}}\\
  \end{align*}
  \begin{align*}
    \lambda&=-1\quad\quad F+1I_6=\\
           &
             \begin{bmatrix}
               -58 & -34 & 41 & 12 & 25 & 30\\
               1 & 8 & -46 & -36 & -11 & -29\\
               -233 & -119 & 59 & -35 & 75 & 54\\
               157 & 81 & -43 & 22 & -51 & -39\\
               -91 & -48 & 32 & -5 & 33 & 26\\
               209 & 107 & -55 & 28 & -69 & -49
             \end{bmatrix}
                                            \rref
                                            \begin{bmatrix}
                                              \leading{1} & 0 & 0 & 0 & 0 & \frac{1}{2}\\
                                              0 & \leading{1} & 0 & 0 & 0 & -\frac{3}{2}\\
                                              0 & 0 & \leading{1} & 0 & 0 & \frac{1}{2}\\
                                              0 & 0 & 0 & \leading{1} & 0 & 0\\
                                              0 & 0 & 0 & 0 & \leading{1} & -\frac{1}{2}\\
                                              0 & 0 & 0 & 0 & 0 & 0
                                            \end{bmatrix}\\
           &\eigenspace{F}{-1}=\nsp{F+I_6}
             =\spn{\set{\colvector{-\frac{1}{2}\\\frac{3}{2}\\-\frac{1}{2}\\0\\\frac{1}{2}\\1}}}
    =\spn{\set{\colvector{-1\\3\\-1\\0\\1\\2}}}\\
  \end{align*}
  \begin{align*}
    \lambda&=2+i\\
           &F-(2+i)I_6=
             \begin{bmatrix}
               -61-i & -34 & 41 & 12 & 25 & 30\\
               1 & 5-i & -46 & -36 & -11 & -29\\
               -233 & -119 & 56-i & -35 & 75 & 54\\
               157 & 81 & -43 & 19-i & -51 & -39\\
               -91 & -48 & 32 & -5 & 30-i & 26\\
               209 & 107 & -55 & 28 & -69 & -52-i
             \end{bmatrix}\\
           &
             \rref
             \begin{bmatrix}
               \leading{1} & 0 & 0 & 0 & 0 & \frac{1}{5}(7+ i)\\
               0 & \leading{1} & 0 & 0 & 0 & \frac{1}{5}(-9-2i)\\
               0 & 0 & \leading{1} & 0 & 0 & 1\\
               0 & 0 & 0 & \leading{1} & 0 & -1\\
               0 & 0 & 0 & 0 & \leading{1} & 1\\
               0 & 0 & 0 & 0 & 0 & 0
             \end{bmatrix}\\
           &\eigenspace{F}{2+i}=\nsp{F-(2+i)I_6}
             =\spn{\set{\colvector{-7-i\\9+2i\\-5\\5\\-5\\5}}}\\
  \end{align*}
  \begin{align*}
    \lambda&=2-i\\
           &F-(2-i)I_6=
             \begin{bmatrix}
               -61+i & -34 & 41 & 12 & 25 & 30\\
               1 & 5+i & -46 & -36 & -11 & -29\\
               -233 & -119 & 56+i & -35 & 75 & 54\\
               157 & 81 & -43 & 19+i & -51 & -39\\
               -91 & -48 & 32 & -5 & 30+i & 26\\
               209 & 107 & -55 & 28 & -69 & -52+i
             \end{bmatrix}\\
           &\rref
             \begin{bmatrix}
               \leading{1} & 0 & 0 & 0 & 0 & \frac{1}{5}(7-i)\\
               0 & \leading{1} & 0 & 0 & 0 & \frac{1}{5}(-9+2i)\\
               0 & 0 & \leading{1} & 0 & 0 & 1\\
               0 & 0 & 0 & \leading{1} & 0 & -1\\
               0 & 0 & 0 & 0 & \leading{1} & 1\\
               0 & 0 & 0 & 0 & 0 & 0
             \end{bmatrix}\\
           &\eigenspace{F}{2-i}=\nsp{F-(2-i)I_6}
             =\spn{\set{\colvector{-7+i\\9-2i\\-5\\5\\-5\\5}}}\\
  \end{align*}
  
  Eigenspace dimensions yield geometric multiplicities of
  $\geomult{F}{2}=1$, $\geomult{F}{-1}=1$, $\geomult{F}{2+i}=1$ and
  $\geomult{F}{2-i}=1$.  This example demonstrates some of the
  possibilities for the appearance of complex eigenvalues, even when
  all the entries of the matrix are real.  Notice how all the numbers
  in the analysis of $\lambda=2-i$ are conjugates of the corresponding
  number in the analysis of $\lambda=2+i$.  This is the content of the
  upcoming \ref{theorem:ERMCP}.
\end{example}

\begin{example}[Distinct eigenvalues, matrix of size 5]
  Consider the matrix
  \[
    H=
    \begin{bmatrix}
      15 & 18 & -8 & 6 & -5\\
      5 & 3 & 1 & -1 & -3\\
      0 & -4 & 5 & -4 & -2\\
      -43 & -46 & 17 & -14 & 15\\
      26 & 30 & -12 & 8 & -10
    \end{bmatrix}
  \]
  then
  \[
    \charpoly{H}{x}=-6x+x^2+7x^3-x^4-x^5=x(x-2)(x-1)(x+1)(x+3)
  \]
  So the eigenvalues are $\lambda=2,\,1,\,0,\,-1,\,-3$ with algebraic
  multiplicities $\algmult{H}{2}=1$, $\algmult{H}{1}=1$,
  $\algmult{H}{0}=1$, $\algmult{H}{-1}=1$ and $\algmult{H}{-3}=1$.
  
  Computing eigenvectors,
  \begin{align*}
    \lambda&=2\\&H-2I_5=
                  \begin{bmatrix}
                    13 & 18 & -8 & 6 & -5\\
                    5 & 1 & 1 & -1 & -3\\
                    0 & -4 & 3 & -4 & -2\\
                    -43 & -46 & 17 & -16 & 15\\
                    26 & 30 & -12 & 8 & -12
                  \end{bmatrix}
                                        \rref
                                        \begin{bmatrix}
                                          \leading{1} & 0 & 0 & 0 & -1\\
                                          0 & \leading{1} & 0 & 0 & 1\\
                                          0 & 0 & \leading{1} & 0 & 2\\
                                          0 & 0 & 0 & \leading{1} & 1\\
                                          0 & 0 & 0 & 0 & 0
                                        \end{bmatrix}\\
           &\eigenspace{H}{2}=\nsp{H-2I_5}
             =\spn{\set{\colvector{1\\-1\\-2\\-1\\1}}}
  \end{align*}
  \begin{align*}
    \lambda&=1\\&H-1I_5=
                  \begin{bmatrix}
                    14 & 18 & -8 & 6 & -5\\
                    5 & 2 & 1 & -1 & -3\\
                    0 & -4 & 4 & -4 & -2\\
                    -43 & -46 & 17 & -15 & 15\\
                    26 & 30 & -12 & 8 & -11
                  \end{bmatrix}
                                        \rref
                                        \begin{bmatrix}
                                          \leading{1} & 0 & 0 & 0 & -\frac{1}{2}\\
                                          0 & \leading{1} & 0 & 0 & 0\\
                                          0 & 0 & \leading{1} & 0 & \frac{1}{2}\\
                                          0 & 0 & 0 & \leading{1} & 1\\
                                          0 & 0 & 0 & 0 & 0
                                        \end{bmatrix}\\
           &\eigenspace{H}{1}=\nsp{H-1I_5}
             =\spn{\set{\colvector{\frac{1}{2}\\0\\-\frac{1}{2}\\-1\\1}}}
    =\spn{\set{\colvector{1\\0\\-1\\-2\\2}}}
  \end{align*}
  \begin{align*}
    \lambda&=0\\&H-0I_5=
                  \begin{bmatrix}
                    15 & 18 & -8 & 6 & -5\\
                    5 & 3 & 1 & -1 & -3\\
                    0 & -4 & 5 & -4 & -2\\
                    -43 & -46 & 17 & -14 & 15\\
                    26 & 30 & -12 & 8 & -10
                  \end{bmatrix}
                                        \rref
                                        \begin{bmatrix}
                                          \leading{1} & 0 & 0 & 0 & 1\\
                                          0 & \leading{1} & 0 & 0 & -2\\
                                          0 & 0 & \leading{1} & 0 & -2\\
                                          0 & 0 & 0 & \leading{1} & 0\\
                                          0 & 0 & 0 & 0 & 0
                                        \end{bmatrix}\\
           &\eigenspace{H}{0}=\nsp{H-0I_5}
             =\spn{\set{\colvector{-1\\2\\2\\0\\1}}}
  \end{align*}
  \begin{align*}
    \lambda&=-1\\&H+1I_5=
                   \begin{bmatrix}
                     16 & 18 & -8 & 6 & -5\\
                     5 & 4 & 1 & -1 & -3\\
                     0 & -4 & 6 & -4 & -2\\
                     -43 & -46 & 17 & -13 & 15\\
                     26 & 30 & -12 & 8 & -9
                   \end{bmatrix}
                                         \rref
                                         \begin{bmatrix}
                                           \leading{1} & 0 & 0 & 0 & -1/2\\
                                           0 & \leading{1} & 0 & 0 & 0\\
                                           0 & 0 & \leading{1} & 0 & 0\\
                                           0 & 0 & 0 & \leading{1} & 1/2\\
                                           0 & 0 & 0 & 0 & 0
                                         \end{bmatrix}\\
           &\eigenspace{H}{-1}=\nsp{H+1I_5}
             =\spn{\set{\colvector{\frac{1}{2}\\0\\0\\-\frac{1}{2}\\1}}}
    =\spn{\set{\colvector{1\\0\\0\\-1\\2}}}
  \end{align*}
  \begin{align*}
    \lambda&=-3\\&H+3I_5=
                   \begin{bmatrix}
                     18 & 18 & -8 & 6 & -5\\
                     5 & 6 & 1 & -1 & -3\\
                     0 & -4 & 8 & -4 & -2\\
                     -43 & -46 & 17 & -11 & 15\\
                     26 & 30 & -12 & 8 & -7
                   \end{bmatrix}
                                         \rref
                                         \begin{bmatrix}
                                           \leading{1} & 0 & 0 & 0 & -1\\
                                           0 & \leading{1} & 0 & 0 & \frac{1}{2}\\
                                           0 & 0 & \leading{1} & 0 & 1\\
                                           0 & 0 & 0 & \leading{1} & 2\\
                                           0 & 0 & 0 & 0 & 0
                                         \end{bmatrix}\\
           &\eigenspace{H}{-3}=\nsp{H+3I_5}
=\spn{\set{\colvector{1\\-\frac{1}{2}\\-1\\-2\\1}}}
    =\spn{\set{\colvector{-2\\1\\2\\4\\-2}}}
  \end{align*}


  So the eigenspace dimensions yield geometric multiplicities
  $\geomult{H}{2}=\answer{1}$, $\geomult{H}{1}=\answer{1}$,
  $\geomult{H}{0}=\answer{1}$, $\geomult{H}{-1}=\answer{1}$ and
  $\geomult{H}{-3}=\answer{1}$, identical to the algebraic
  multiplicities.  This example is of interest for two reasons.
  First, $\lambda=0$ is an eigenvalue, illustrating the upcoming
  \ref{theorem:SMZE}.  Second, all the eigenvalues are distinct,
  yielding algebraic and geometric multiplicities of 1 for each
  eigenvalue, illustrating \ref{theorem:DED}.
\end{example}

\end{document}
