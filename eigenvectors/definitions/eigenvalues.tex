\documentclass{ximera}

\input{../../preamble.tex}

\title{Eigenvalues and Eigenvectors of a Matrix}

\begin{document}
\begin{abstract}
  We will define the eigenvalues and eigenvectors of a matrix, and see how to compute them.  
\end{abstract}
\maketitle

We start with the principal definition for this chapter.

\begin{definition}[Eigenvalues and Eigenvectors of a Matrix]
  Suppose that $A$ is a square matrix of size $n$,
  $\vect{x}\neq\zerovector$ is a vector in $\complex{n}$, and
  $\lambda$ is a scalar in $\complexes$.  Then we say $\vect{x}$ is an
  \dfn{eigenvector} of $A$ with \dfn{eigenvalue} $\lambda$ if
  \[
    A\vect{x}=\lambda\vect{x}
  \]
\end{definition}

Before going any further, perhaps we should convince you that such
things ever happen at all.  Understand the next example, but do not
concern yourself with where the pieces come from.  We will have
methods soon enough to be able to discover these eigenvectors
ourselves.

\begin{example}[Some eigenvalues and eigenvectors]
  Consider the matrix
  \[
    A=
    \begin{bmatrix}
      204 & 98 & -26 & -10\\
      -280 & -134 & 36 & 14\\
      716 & 348 & -90 & -36\\
      -472 & -232 & 60 & 28
    \end{bmatrix}
  \]
  and the vectors
  \begin{align*}
\vect{x}=\colvector{1\\-1\\2\\5}&&     % ev=4
\vect{y}=\colvector{-3\\4\\-10\\4}&&  % ev = 0
\vect{z}=\colvector{-3\\7\\0\\8}&&     % ev=2
\vect{w}=\colvector{1\\-1\\4\\0}        % ev=2
\end{align*}

Then
\[
A\vect{x}=
\begin{bmatrix}
204 & 98 & -26 & -10\\
-280 & -134 & 36 & 14\\
716 & 348 & -90 & -36\\
-472 & -232 & 60 & 28
\end{bmatrix}
\colvector{1\\-1\\2\\5}=
\colvector{4\\-4\\8\\20}=
4\colvector{1\\-1\\2\\5}=4\vect{x}
\]
so $\vect{x}$ is an eigenvector of $A$ with eigenvalue $\lambda=\answer{4}$.

Also,
\[
A\vect{y}=
\begin{bmatrix}
204 & 98 & -26 & -10\\
-280 & -134 & 36 & 14\\
716 & 348 & -90 & -36\\
-472 & -232 & 60 & 28
\end{bmatrix}
\colvector{-3\\4\\-10\\4}=
\colvector{0\\0\\0\\0}=
0\colvector{-3\\4\\-10\\4}=0\vect{y}
\]
so $\vect{y}$ is an eigenvector of $A$ with eigenvalue $\lambda=\answer{0}$.

Also,
\[
A\vect{z}=
\begin{bmatrix}
204 & 98 & -26 & -10\\
-280 & -134 & 36 & 14\\
716 & 348 & -90 & -36\\
-472 & -232 & 60 & 28
\end{bmatrix}
\colvector{-3\\7\\0\\8}=
\colvector{-6\\14\\0\\16}=
2\colvector{-3\\7\\0\\8}=2\vect{z}
\]
so $\vect{z}$ is an eigenvector of $A$ with eigenvalue $\lambda=\answer{2}$.

Also,
\[
A\vect{w}=
\begin{bmatrix}
204 & 98 & -26 & -10\\
-280 & -134 & 36 & 14\\
716 & 348 & -90 & -36\\
-472 & -232 & 60 & 28
\end{bmatrix}
\colvector{1\\-1\\4\\0}=
\colvector{2\\-2\\8\\0}=
2\colvector{1\\-1\\4\\0}=2\vect{w}
\]
so $\vect{w}$ is an eigenvector of $A$ with eigenvalue $\lambda=2$.

So we have demonstrated four eigenvectors of $A$.  Are there more?
Yes, any nonzero scalar multiple of an eigenvector is again an
eigenvector.  In this example, set $\vect{u}=30\vect{x}$.  Then
\begin{align*}
  A\vect{u}
  &=A(30\vect{x})\\
  &=30A\vect{x}&&\ref{theorem:MMSMM}\\
  &=30(4\vect{x})&&\ref{definition:EEM}\\
  &=4(30\vect{x})&&\ref{property:SMAM}\\
  &=4\vect{u}
\end{align*}
so that $\vect{u}$ is also an eigenvector of $A$ for the same eigenvalue, $\lambda=4$.

The vectors $\vect{z}$ and $\vect{w}$ are both eigenvectors of $A$ for
the same eigenvalue $\lambda=2$, yet this is not as simple as the two
vectors just being scalar multiples of each other (they are not).
Look what happens when we add them together, to form
$\vect{v}=\vect{z}+\vect{w}$, and multiply by $A$,
\begin{align*}
  A\vect{v}
  &=A(\vect{z}+\vect{w})\\
  &=A\vect{z}+A\vect{w}&&\ref{theorem:MMDAA}\\
  &=2\vect{z}+2\vect{w}&&\ref{definition:EEM}\\
  &=2(\vect{z}+\vect{w})&&\ref{property:DVAC}\\
  &=2\vect{v}
\end{align*}
so that $\vect{v}$ is also an eigenvector of $A$ for the eigenvalue
$\lambda=2$.  So it would appear that the set of eigenvectors that are
associated with a fixed eigenvalue is closed under the vector space
operations of $\complex{n}$.  Hmmm.

The vector $\vect{y}$ is an eigenvector of $A$ for the eigenvalue
$\lambda=0$, so we can use \ref{theorem:ZSSM} to write
$A\vect{y}=0\vect{y}=\zerovector$.  But this also means that
$\vect{y}\in\nsp{A}$.  There would appear to be a connection here
also.
\end{example}

\ref{example:SEE} hints at a number of intriguing properties, and
there are many more.  We will explore the general properties of
eigenvalues and eigenvectors in \ref{section:PEE}, but in this section
we will concern ourselves with the question of actually computing
eigenvalues and eigenvectors.  First we need a bit of background
material on polynomials and matrices.

\end{document}

