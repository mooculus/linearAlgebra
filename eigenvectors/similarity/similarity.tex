\documentclass{ximera}

\input{../../preamble.tex}

\title{Similar Matrices}

\begin{document}
\begin{abstract}
Two similar matrices are not equal, but they share many important properties.
\end{abstract}
\maketitle

The notion of matrices being ``similar'' is a lot like saying two
matrices are row-equivalent.  This activity, and later activity, will
be devoted in part to discovering just what these common properties
are.  First, the main definition for this section.

\begin{definition}[Similar Matrices]
  Suppose $A$ and $B$ are two square matrices of size $n$.  Then $A$
  and $B$ are \dfn{similar} if there exists a nonsingular matrix of
  size $n$, $S$, such that $A=\similar{B}{S}$.
\end{definition}

We will say ``$A$ is similar to $B$ via $S$'' when we want to
emphasize the role of $S$ in the relationship between $A$ and $B$.
Also, it does not matter if we say $A$ is similar to $B$, or $B$ is
similar to $A$.  If one statement is true then so is the other, as can
be seen by using $\inverse{S}$ in place of $S$ (see \ref{theorem:SER}
for the careful proof).  Finally, we will refer to $\similar{B}{S}$ as
a \dfn{similarity transformation} when we want to emphasize the way
$S$ changes $B$.  OK, enough about language, let us build a few
examples.

\begin{example}[Similar matrices of size 5]

If you wondered if there are examples of similar matrices, then it will not be hard to convince you they exist.  Define
\[
B=\begin{bmatrix}
-4 & 1 & -3 & -2 & 2 \\
1 & 2 & -1 & 3 & -2 \\
-4 & 1 & 3 & 2 & 2 \\
-3 & 4 & -2 & -1 & -3 \\
3 & 1 & -1 & 1 & -4
\end{bmatrix}
\]
and define
\[
S=\begin{bmatrix}
1 & 2 & -1 & 1 & 1 \\
0 & 1 & -1 & -2 & -1 \\
1 & 3 & -1 & 1 & 1 \\
-2 & -3 & 3 & 1 & -2 \\
1 & 3 & -1 & 2 & 1\\
\end{bmatrix}.
\]

Check that $S$ is nonsingular and then compute
\begin{align*}
&A=\similar{B}{S}\\
&=
\begin{bmatrix}
10 & 1 & 0 & 2 & -5 \\
-1 & 0 & 1 & 0 & 0 \\
3 & 0 & 2 & 1 & -3 \\
0 & 0 & -1 & 0 & 1 \\
-4 & -1 & 1 & -1 & 1
\end{bmatrix}
\begin{bmatrix}
-4 & 1 & -3 & -2 & 2 \\
1 & 2 & -1 & 3 & -2 \\
-4 & 1 & 3 & 2 & 2 \\
-3 & 4 & -2 & -1 & -3 \\
3 & 1 & -1 & 1 & -4
\end{bmatrix}
\begin{bmatrix}
1 & 2 & -1 & 1 & 1 \\
0 & 1 & -1 & -2 & -1 \\
1 & 3 & -1 & 1 & 1 \\
-2 & -3 & 3 & 1 & -2 \\
1 & 3 & -1 & 2 & 1
\end{bmatrix}\\
&=
\begin{bmatrix}
-10 & -27 & -29 & -80 & -25 \\
-2 & 6 & 6 & 10 & -2 \\
-3 & 11 & -9 & -14 & -9 \\
-1 & -13 & 0 & -10 & -1 \\
11 & 35 & 6 & 49 & 19
\end{bmatrix}
\end{align*}

So by this construction, we know that $A$ and $B$ are \wordChoice{\choice[correct]{similar}\choice{not similar}}.

\end{example}

Let us do that again.

\begin{example}[Similar matrices of size 3]
  Define
  \begin{align*}
    B=\begin{bmatrix}
      -13 & -8 & -4 \\
      12 & 7 & 4 \\
      24 & 16 & 7
    \end{bmatrix}
         &&
            S=\begin{bmatrix}
              1 & 1 & 2 \\
              -2 & -1 & -3 \\
              1 & -2 & 0
            \end{bmatrix}
  \end{align*}
  
  
  Check that $S$ is nonsingular and then compute
  \begin{align*}
    A&=\similar{B}{S}\\
     &=
       \begin{bmatrix}
         -6 & -4 & -1 \\
         -3 & -2 & -1 \\
         5 & 3 & 1
       \end{bmatrix}
                 \begin{bmatrix}
                   -13 & -8 & -4 \\
                   12 & 7 & 4 \\
                   24 & 16 & 7
                 \end{bmatrix}
                             \begin{bmatrix}
                               1 & 1 & 2 \\
                               -2 & -1 & -3 \\
                               1 & -2 & 0
                             \end{bmatrix}\\
     &=
       \begin{bmatrix}
         -1 & 0 & 0 \\
         0 & 3 & 0 \\
         0 & 0 & -1
       \end{bmatrix}
  \end{align*}
  
  So by this construction, we know that $A$ and $B$ are similar.  But
  before we move on, look at how pleasing the form of $A$ is.  Not
  convinced?  Then consider that several computations related to $A$
  are especially easy.
  
  For example,
  $\detname{A}=(-1)(3)(-1)=\answer{3}$.
  Similarly, the characteristic polynomial
  is straightforward to compute by hand,
  \[
    \charpoly{A}{x}=(-1-x)(3-x)(-1-x)=-(x-3)(x+\answer{1})^2
  \]
  and since the result is already factored, the eigenvalues are
  transparently $\lambda=3,\,-1$.  Finally, the eigenvectors of $A$ are
  just the standard unit vectors (\ref{definition:SUV}).  Are similar facts true for $B$?
\end{example}

\end{document}

%%% Local Variables:
%%% mode: latex
%%% TeX-master: t
%%% End:
