\documentclass{ximera}

\input{../../preamble.tex}

\title{Properties of Similar Matrices}

\begin{document}
\begin{abstract}
  Similar matrices share many properties and it is these theorems that
  justify the choice of the word ``similar.''
\end{abstract}
\maketitle

We will show that similarity is an \dfn{equivalence relation}.
Equivalence relations are important in the study of various algebras
and can always be regarded as a kind of weak version of equality.
Sort of alike, but not quite equal.  The notion of two matrices being
row-equivalent is an example of an equivalence relation we have been
working with since the beginning of the course.  Row-equivalent
matrices are not equal, but they are a lot alike.  For example,
row-equivalent matrices have the same rank.  Formally, an equivalence
relation requires three conditions hold: reflexive, symmetric and
transitive.  We will illustrate these as we prove that similarity is
an equivalence relation.

\begin{theorem}[Similarity is an Equivalence Relation]
\label{theorem:SER}

Suppose $A$, $B$ and $C$ are square matrices of size $n$.  Then
\begin{enumerate}\item $A$ is similar to $A$.  (Reflexive)
\item If $A$ is similar to $B$, then $B$ is similar to $A$.  (Symmetric)
\item If $A$ is similar to $B$ and $B$ is similar to $C$, then $A$ is similar to $C$.  (Transitive)
\end{enumerate}

\begin{proof}
To see that $A$ is similar to $A$, we need only demonstrate a nonsingular matrix that effects a similarity transformation of $A$ to $A$.  $I_n$ is 
\wordChoice{\choice[correct]{nonsingular}\choice{singular}} (since it row-reduces to the identity matrix, \ref{theorem:NMRRI}), and
\[
\similar{A}{I_n}=I_nAI_n=A
\]

If we assume that $A$ is similar to $B$, then we know there is a nonsingular matrix $S$ so that $A=\similar{B}{S}$ by \ref{definition:SIM}.  By \ref{theorem:MIMI}, $\inverse{S}$ is invertible, and by \ref{theorem:NI} is therefore nonsingular.  So
\begin{align*}
\similar{A}{(\inverse{S})}&=SA\inverse{S}&&\ref{theorem:MIMI}\\
&=S\similar{B}{S}\inverse{S}&&\ref{definition:SIM}\\
&=\left(S\inverse{S}\right)B\left(S\inverse{S}\right)&&\ref{theorem:MMA}\\
&=I_nBI_n&&\ref{definition:MI}\\
&=B&&\ref{theorem:MMIM}
\end{align*}
and we see that $B$ is \wordChoice{\choice{not similar}\choice[correct]{similar}} to $A$.

Assume that $A$ is similar to $B$, and $B$ is similar to $C$.  This gives us the existence of two nonsingular matrices, $S$ and $R$, such that $A=\similar{B}{S}$ and $B=\similar{C}{R}$, by \ref{definition:SIM}.  (Notice how we have to assume $S\neq R$, as will usually be the case.)  Since $S$ and $R$ are invertible, so too $RS$ is invertible by \ref{theorem:SS} and then nonsingular by \ref{theorem:NI}.  Now
\begin{align*}
\similar{C}{(RS)}&=\similar{\similar{C}{R}}{S}&&\ref{theorem:SS}\\
&=\similar{\left(\similar{C}{R}\right)}{S}&&\ref{theorem:MMA}\\
&=\similar{B}{S}&&\ref{definition:SIM}\\
&=A
\end{align*}
so $A$ is similar to $C$ via the nonsingular matrix $RS$.
\end{proof}
\end{theorem}

Here is another theorem that tells us exactly what sorts of properties similar matrices share.

\begin{theorem}[Similar Matrices have Equal Eigenvalues]
\label{theorem:SMEE}

Suppose $A$ and $B$ are similar matrices.  Then the characteristic polynomials of $A$ and $B$ are equal, that is, $\charpoly{A}{x}=\charpoly{B}{x}$.

\begin{proof}
  Let $n$ denote the size of $A$ and $B$.  Since $A$ and $B$ are
  similar, there exists a nonsingular matrix $S$, such that
  $A=\similar{B}{S}$ (\ref{definition:SIM}).  Then
  \begin{align*}
    \charpoly{A}{x}&=\detname{A-xI_n}&&\ref{definition:CP}\\
                   &=\detname{\similar{B}{S}-xI_n}&&\ref{definition:SIM}\\
                   &=\detname{\similar{B}{S}-x\similar{I_n}{S}}&&\ref{theorem:MMIM}\\
                   &=\detname{\similar{B}{S}-\inverse{S}xI_nS}&&\ref{theorem:MMSMM}\\
                   &=\detname{\similar{\left(B-xI_n\right)}{S}}&&\ref{theorem:MMDAA}\\
                   &=\detname{\inverse{S}}\detname{B-xI_n}\detname{S}&&\ref{theorem:DRMM}\\
                   &=\detname{\inverse{S}}\detname{S}\detname{B-xI_n}&&\ref{property:CMCN}\\
                   &=\detname{\inverse{S}S}\detname{B-xI_n}&&\ref{theorem:DRMM}\\
                   &=\detname{I_n}\detname{B-xI_n}&&\ref{definition:MI}\\
                   &=1\detname{B-xI_n}&&\ref{definition:DM}\\
                   &=\charpoly{B}{x}&&\ref{definition:CP}
  \end{align*}
\end{proof}
\end{theorem}

So similar matrices not only have the same \textit{set} of
eigenvalues, the algebraic multiplicities of these eigenvalues will
also be the same.  However, be careful with this theorem.  It is
tempting to think the converse is true, and argue that if two matrices
have the same eigenvalues, then they are similar.  Not so, as the
following example illustrates.

\begin{example}[Equal eigenvalues, not similar]

Define
\begin{align*}
A&=\begin{bmatrix}1&1\\0&1\end{bmatrix}
&
B&=\begin{bmatrix}1&0\\0&1\end{bmatrix}
\end{align*}
and check that
\[
\charpoly{A}{x}=\charpoly{B}{x}=\answer{1-2x+x^2}
\]
and so $A$ and $B$ have equal characteristic polynomials.  If the
converse of \ref{theorem:SMEE} were true, then $A$ and $B$ would be
similar.  Suppose this is the case. More precisely, suppose there is a
nonsingular matrix $S$ so that $A=\similar{B}{S}$. Then
\[
A=\similar{B}{S}=\similar{I_2}{S}=\inverse{S}S=I_2
\]
Clearly $A\neq I_2$ and this contradiction tells us that the converse
of \ref{theorem:SMEE} is false.

\end{example}

\end{document}
