\documentclass{ximera}

\input{../../preamble.tex}

\title{Eigenvalues of Hermitian Matrices}

\begin{document}
\begin{abstract}
  Hermitian matrices have real eigenvalues and orthogonal eigenvectors,
\end{abstract}
\maketitle

Recall that a matrix is Hermitian (or self-adjoint) if $A=\adjoint{A}$ (\ref{definition:HM}).  In the case where $A$ is a matrix whose entries are all real numbers, being Hermitian is identical to being symmetric (\ref{definition:SYM}).  Keep this in mind as you read the next two theorems.  Their hypotheses could be changed to ``suppose $A$ is a real symmetric matrix.''

\begin{theorem}[Hermitian Matrices have Real Eigenvalues]
\label{theorem:HMRE}

Suppose that $A$ is a Hermitian matrix and $\lambda$ is an eigenvalue of $A$.  Then $\lambda\in{\mathbb R}$.

\begin{proof}
Let $\vect{x}\neq\zerovector$ be one eigenvector of $A$ for the eigenvalue $\lambda$.   Then by \ref{theorem:PIP} we know $\innerproduct{\vect{x}}{\vect{x}}\neq 0$.  So
\begin{align*}
\lambda
&=\frac{1}{\innerproduct{\vect{x}}{\vect{x}}}\lambda\innerproduct{\vect{x}}{\vect{x}}
&&\ref{property:MICN}\\
&=\frac{1}{\innerproduct{\vect{x}}{\vect{x}}}\innerproduct{\vect{x}}{\lambda\vect{x}}
&&\ref{theorem:IPSM}\\
&=\frac{1}{\innerproduct{\vect{x}}{\vect{x}}}\innerproduct{\vect{x}}{A\vect{x}}
&&\ref{definition:EEM}\\
&=\frac{1}{\innerproduct{\vect{x}}{\vect{x}}}\innerproduct{A\vect{x}}{\vect{x}}
&&\ref{theorem:HMIP}\\
&=\frac{1}{\innerproduct{\vect{x}}{\vect{x}}}\innerproduct{\lambda\vect{x}}{\vect{x}}
&&\ref{definition:EEM}\\
&=\frac{1}{\innerproduct{\vect{x}}{\vect{x}}}\conjugate{\lambda}\innerproduct{\vect{x}}{\vect{x}}
&&\ref{theorem:IPSM}\\
&=\conjugate{\lambda}
&&\ref{property:MICN}
\end{align*}


If a complex number is equal to its conjugate, then it has a complex
part equal to zero, and therefore is a real number.

\end{proof}
\end{theorem}

Notice the appealing symmetry to the justifications given for the
steps of this proof.  In the center is the ability to pitch a
Hermitian matrix from one side of the inner product to the other.

In many physical problems, a matrix of interest will be real and
symmetric, or Hermitian.  Then if the eigenvalues are to represent
physical quantities of interest, \ref{theorem:HMRE} guarantees that
these values will be real.

The eigenvectors of a Hermitian matrix also enjoy a pleasing property
that we will exploit later.

\begin{theorem}
\label{theorem:HMOE}
[Hermitian Matrices have Orthogonal Eigenvectors]

Suppose that $A$ is a Hermitian matrix and $\vect{x}$ and $\vect{y}$ are two eigenvectors of $A$ for different eigenvalues.  Then $\vect{x}$ and $\vect{y}$ are orthogonal vectors.

\begin{proof}
Let $\vect{x}$ be an eigenvector of $A$ for $\lambda$ and let $\vect{y}$ be an eigenvector of $A$ for a different eigenvalue $\rho$.   So we have $\lambda-\rho\neq 0$.  Then
\begin{align*}
\innerproduct{\vect{x}}{\vect{y}}
&=\frac{1}{\lambda-\rho}\left(\lambda-\rho\right)\innerproduct{\vect{x}}{\vect{y}}
&&\ref{property:MICN}\\
&=\frac{1}{\lambda-\rho}
\left(\lambda\innerproduct{\vect{x}}{\vect{y}}-\rho\innerproduct{\vect{x}}{\vect{y}}\right)
&&\ref{property:DCN}\\
&=\frac{1}{\lambda-\rho}
\left(\innerproduct{\conjugate{\lambda}\vect{x}}{\vect{y}}-\innerproduct{\vect{x}}{\rho\vect{y}}\right)
&&\ref{theorem:IPSM}\\
&=\frac{1}{\lambda-\rho}
\left(\innerproduct{\lambda\vect{x}}{\vect{y}}-\innerproduct{\vect{x}}{\rho\vect{y}}\right)
&&\ref{theorem:HMRE}\\
&=\frac{1}{\lambda-\rho}
\left(\innerproduct{A\vect{x}}{\vect{y}}-\innerproduct{\vect{x}}{A\vect{y}}\right)
&&\ref{definition:EEM}\\
&=\frac{1}{\lambda-\rho}
\left(\innerproduct{A\vect{x}}{\vect{y}}-\innerproduct{A\vect{x}}{\vect{y}}\right)
&&\ref{theorem:HMIP}\\
&=\frac{1}{\lambda-\rho}\left(0\right)
&&\ref{property:AICN}\\
&=\answer{0}.
\end{align*}

This equality says that $\vect{x}$ and $\vect{y}$ are orthogonal vectors (\ref{definition:OV}).

\end{proof}
\end{theorem}

Notice again how the key step in this proof is the fundamental property of a Hermitian matrix (\ref{theorem:HMIP}) --- the ability to swap $A$ across the two arguments of the inner product.  We will build on these results and continue to see some more interesting properties in \ref{section:OD}.

\end{document}
