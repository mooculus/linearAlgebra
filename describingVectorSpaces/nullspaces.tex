\documentclass{ximera}
\input{../preamble.tex}
\title{Nullspaces}
\author{Crichton Ogle}

\begin{document}
\begin{abstract}
  Nullspaces provide an important way of constructing subspaces of $\mathbb R^n$.
\end{abstract}
\maketitle

If $A$ is an $m\times n$ matrix with entries in $\mathbb R$, the {\it nullspace of A} is the set
\[
N(A) := \{{\bf x}\in\mathbb R^n\ |\ A*{\bf x} = {\bf 0}\}\subset \mathbb R^n
\]
In other words, $N(A)$ is the set of all solutions to the homogeneous matrix equation $A*{\bf x} = {\bf 0}$.

\begin{lemma} For any $m\times n$ matrix $A$ with entries in $\mathbb R$, $N(A)$ is a subspace of $\mathbb R^n$.
\end{lemma}

\begin{proof} Note that ${\bf 0}\in N(A)$. So we need to show that $N(A)$ satisfies the two closure axioms.
\vskip.1in

{\bf\underbar{C1}} Suppose ${\bf v}, {\bf w}\in N(A)$. Then
\[
A*({\bf v} + {\bf w}) = A*{\bf v} + A*{\bf w} = {\bf 0} + {\bf 0} = {\bf 0}
\]
Hence $N(A)$ is closed under addition.
\vskip.1in

{\bf\underbar{C2}} Suppose $\alpha\in\mathbb R$ and ${\bf v}\in N(A)$. Then
\[
A*(\alpha{\bf v}) = \alpha(A*{\bf v}) = \alpha{\bf 0} = {\bf 0} 
\]
Hence $N(A)$ is closed under scalar multiplication. This completes the proof.
\end{proof}
\vskip.2in

Nullspaces can be computed using methods that have already been discussed. If $A$ is an $m\times n$ matrix the equation $A*{\bf x} = {\bf 0}$ is consistent. Either the equation only admits the trivial solution ${\bf x} = {\bf 0}$, or we have a parametrized family of solutions. In either case we want to find a spanning set for $N(A)$.
\vskip.2in

\begin{example}
Suppose $A$ is a $5\times 5$ matrix for which $rref(A) = 
\begin{bmatrix} 1 & 0 & 0 & -1 & 2\\0 & 1 & 0 & 2 & 3\\0 & 0 & 1 & -4 & -3\\
0 & 0 & 0 & 0 & 0\\0 & 0 & 0 & 0 & 0\end{bmatrix}$. The columns which do not contain leading ones are the fourth and fifth. Thus $x_4$ and $x_5$ are the natural parameters for the parametrized solution set. We can derive a spanning set for $N(A)$ as follows:
\begin{itemize}
\item Write the all of the original set of variables $x_1, x_2, x_3, x_4, x_5$ in terms of the free variables $x_4, x_5$:
\begin{gather*}
x_1 = x_4 - 2x_5\\
x_2 = -2x_4 - 3x_5\\
x_3 = 4x_4 + 3x_5\\
x_4 = x_4\\
x_5 = x_5
\end{gather*}
\item Use this to write the general solution vector $\bf x$ in terms of $x_4$ and $x_5$:
\[
{\bf x} = \begin{bmatrix}x_1\\x_2\\x_3\\x_4\\x_5\end{bmatrix} =
\begin{bmatrix}x_4 - 2x_5\\-2x_4 -3x_5\\4x_4 + 3x_5\\x_4\\x_5\end{bmatrix}
\]
\item Separate the general solution into its homogeneous components:
\[
{\bf x} = \begin{bmatrix}x_4 - 2x_5\\-2x_4 -3x_5\\4x_4 + 3x_5\\x_4\\x_5\end{bmatrix}
= \begin{bmatrix}x_4\\-2x_4\\4x_4\\x_4\\0\end{bmatrix}
+ 
\begin{bmatrix}-2x_5\\-3x_5\\3x_5\\0\\x_5\end{bmatrix}
= x_4\begin{bmatrix}1\\-2\\4\\1\\0\end{bmatrix}
+ 
x_5\begin{bmatrix}-2\\-3\\3\\0\\1\end{bmatrix}
\]
\item Use this last expression for the general solution finally express $N(A)$ as the span of a set of vectors:
\[
N(A) = Span\left\{\begin{bmatrix}1\\-2\\4\\1\\0\end{bmatrix},\begin{bmatrix}-2\\-3\\3\\0\\1\end{bmatrix}\right\}
\]
\end{itemize}
\end{example}
\vskip.2in

\begin{exercise} Suppose $A$ is the $4\times 6$ matrix
\[
\begin{bmatrix} 3 & 4 & 7 & -1 & 1 & 5\\1 & -5 & 11 & 9 & -3 & 12\\8 & 13 & 15 & 11 & 6 & 0\\
10 & 3 & 1 & 2 & -9 & -2\\4 & 0 & 3 & 0 & 2 & 1\end{bmatrix}
\]
Using MATLAB or Octave, compute $rref(A)$ (hint: do this in rational format by entering the command "format rat" before executing the rref command). Then use the resulting output to construct a spanning set for $N(A)$. See if you can find a spanning set consisting of vectors with integer entries.
\end{exercise}

\begin{exercise} For the same matrix $A$ as in Exercise 1, in either MATLAB or Octave use the ``null'' command applied to $A$ and enter ``null(A)'' on the command line. Compare your answer to that in Exercise 1. Is it the same? Next, enter ``help null'' in the command line to to read about what that command does, and how to interpret the output.
\end{exercise}


\end{document}
