\documentclass{ximera}
\input{../preamble.tex}
\title{Shur's Theorem}
\author{Crichton Ogle}

\begin{document}
\begin{abstract}
  There are advantages to working with complex numbers.
\end{abstract}
\maketitle

If $A\in \mathbb C^{m\times n}$ is a complex $m\times n$ matrix, the {\it conjugate-transpose} of $A$ is defined entrywise by
\[
A^*(i,j) := \overline{A(j,i)}
\]
For complex matrices, this is the analogue of the transpose. A complex $n\times n$ matrix $U$ is said to be {\it unitary} if $U^* = U^{-1}$, and {\it Hermitian} if $U^* = U$ (these are the complex analogues of orthogonal and symmetric, respectively). The proof of the following theorem involves definitions and methods developed in the next module; it is included here for completeness. The theorem will allow us to determine necessary and sufficient conditions for diagonalizability.

\begin{theorem} (Shur's Theorem) Let $A$ be a complex $n\times n$ matrix. Then there is an $n\times n$ unitary matrix $U$ and an $n\times n$ upper triangular matrix $T$ such that
\[
U^**A*U = T
\]
\end{theorem}

\begin{proof} By the Fundamental Theorem of Algebra $p_A(t)$ factors completely over $\mathbb C$ into linear terms. In particular, $A$ must have at least one eigenvalue.  Let $\lambda_1$ be an eigenvalue of $A$, and ${\bf v}_1$ a corresponding eigenvector with norm 1. Let $W_1 = Span\{{\bf v}_1\}^\perp$ be the orthogonal complement of the span of ${\bf v}_1$. Write $L_1:\mathbb C^n\to\mathbb C^n$ for the linear transformation ${\bf v}\mapsto A*{\bf v}$. Now consider the linear transformation
\[
L_2:W_1\inj\mathbb C^n\xrightarrow{L_1}\mathbb C^n\surj \mathbb C^n/Span\{{\bf v}_1\}\cong W_1
\]
By induction on dimension, we can assume the subspace $W_1\subset\mathbb C^n$ admits a complex orthonormal basis $\{{\bf v}_2,\dots, {\bf v}_n\}$ such that $L_2({\bf v}_i)\subset Span\{{\bf v}_2,\dots,{\bf v}_i\}, 2\le i\le n$. In other words, for each $2\le i\le n$ we have
\[
L_2({\bf v}_i) = \sum_{j=2}^i\alpha_{i,j}{\bf v}_j
\]
This means that the composition $L_2':W_1\inj\mathbb C^n\xrightarrow{L_1}\mathbb C^n$ on the vectors $\{{\bf v}_2,\dots, {\bf v}_n\}$ is given by
\[
L_1({\bf v}_i) = L_2'({\bf v}_i) = \sum_{j=1}^i\alpha_{i,j}{\bf v}_j
\]
for some choice of complex scalars $\alpha_{1,j}$. Setting $\alpha_{1,1} = \lambda_1$, we see that
\[
A*{\bf v}_i = \sum_{j=1}^i\alpha_{i,j}{\bf v}_j,\qquad 1\le i\le n
\]
Let $U = [{\bf v}_1\ {\bf v}_2\dots {\bf v}_n]$ be the concatenation of the vectors ${\bf v}_i, 1\le i\le n$ and $T$ the upper triangular matrix with $T(i,j) = \alpha_{i,j}, 1\le i\le j\le n$. Then $U$ is unitary, and last equation can then be written in matrix form as
\[
A*U = U*T
\]
Multiplying both sides on the left by $U^* = U^{-1}$ gives the desired equation
\[
U^**A*U = T
\]
\end{proof}

The main corrolary to Shur's theorem is 

\begin{corollary} If $A$ is an $n\times n$ Hermitian matrix, then $A$ is diagonalizable (i.e., $\mathbb C^n$ has a basis consisting of eigenvectors of $A$). Moreover, all of the eigenvalues of $A$ are real.
\end{corollary}

\begin{proof} By Shur's Theorem, there exists a unitary $U$ and upper triangular $T$ with $U^**A*U = T$. But $A$ is Hermitian, so $A = A^*$. Then
\begin{align*}
T^* &= \left(U^**A*U\right)^*\\
        &= U^**A^**U\\
        &= U^**A*U = T
\end{align*}
But as $T$ is upper triangular, the identity $T = T^*$ implies not only that $T$ is diagonal, but that the diagonal elements remain invariant under complex conjugation; i.e., are real numbers. Thus Shur's equation may be rewritten in this case as $U^**A*U = D$ where $D$ is a real diagonal matrix, or alternatively,
\[
A*U = U*D
\]
But this last equation implies $A*U(i,:) = \lambda_i U(i,:), 1\le i\le n$, where $\lambda_i = D(i,i)$. As the columns of $U$ are orthonormal, they form a basis for $\mathbb C^n$, implying the result.
\end{proof}

As a special case, when $A$ is a real $n\times n$ matrix, then it is Hermitian iff it is symmetric. Thus, viewing it as a complex matrix with purely real entries, we have

\begin{corollary} If $A$ is an $n\times n$ real symmetric matrix, then it is diagonalizable and all of its eigenvalues are real.
\end{corollary}

A Hermitian matrix is said to be {\it positive definite} if all of its eignevalues are positive real numbers,
and {\it non-negative definite}, if all of its eigenvalues are non-negative. Now suppose $A$ is an arbitrary $n\times n$ complex matrix, and $B = A^**A$. Then clearly $B^* = (A^**A)^* = A^**A = B$ implying $B$ is Hermitian.

\begin{lemma} For any $n\times n$ matrix $A$, the matrix $B = A^**A$ is non-negative definite. Moreover, $B$ is positive definite iff $A$ is non-singular.
\end{lemma}

\begin{proof} Since $\mathbb C^n$ admits an orthonormal basis consisting of eigenvectors of $B$, it suffices to show that the eigenvalues of $B$ are non-negative. Let $\bf v$ be an eigenvector of $B$ with eigenvalue $\lambda$. Then
\[
\lambda\|{\bf v}\|^2 = \lambda({\bf v}^**{\bf v}) = {\bf v}^**B*{\bf v} = {\bf v}^**(A^**A)*{\bf v} = (A*{\bf v})^**(A*{\bf v}) = \|A*{\bf v}\|^2\ge 0
\]
As $\|{\bf v}\|^2 > 0$ this implies $\lambda\ge 0$. Also, as we have already seen, $B$ is non-singular iff $0$ is not an eigenvalue of $B$. But $Det(B) = |Det(A)|^2$. Hence $B$ is non-singular iff $A$ is.
\end{proof}

\end{document}
